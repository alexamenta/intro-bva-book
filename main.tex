\documentclass[a4paper,10pt,intlimits,sumlimits]{amsart}
\usepackage[utf8]{inputenc}

\usepackage{conf}

\hyphenation{pro-duct de-fi-ning}

%% Margin notes (use commands below to enable or disable notes)
% \usepackage[disable]{todonotes}
\usepackage[color=blue!30]{todonotes} 

\begin{document}

%% Title pages
\title{Introduction to Banach-valued analysis \\ (V5B7, Winter Semester 2020-2021)}
\date{\today}

\author[A. Amenta]{Alex Amenta}
\address{\noindent Mathematisches Institut \newline \indent Universit\"at Bonn, Bonn, Germany}
\email{amenta@math.uni-bonn.de}


%%% Local Variables:
%%% mode: latex
%%% TeX-master: "../main.tex"
%%% End:


\maketitle
\footnotesize
\tableofcontents
\normalsize

%% Main matter
\section{Introduction}
\label{sec:intro}
The one-sentence goal of this course is to study harmonic analysis in the context of functions $\map{f}{\R}{X}$ of a single variable, taking values in an infinite-dimensional Banach space $X$.


% {\footnotesize
% \subsection{Acknowledgements}
% }


%%% Local Variables:
%%% mode: latex
%%% TeX-master: "../main"
%%% End:



\section{The Bochner spaces \texorpdfstring{$L^p(S;X)$}{Lp(S;X)}}
\label{sec:Bochner-spaces}
\todo{add expository intro}

\subsection{Notions of measurability}

Consider a measurable space $(S,\mc{A})$,\footnote{i.e. $S$ is a set and $\mc{A}$ is a $\sigma$-algebra of subsets of $S$.} and a Banach space $X$ over a scalar field $\K$ (either $\R$ or $\C$).
The topic of these notes is the analysis of functions $\map{f}{S}{X}$, and of operators acting on such functions.
Thus it will be useful for us to gain some familiarity with the idea of $X$-valued functions.

The simplest kind of $X$-valued function arises by taking a \emph{scalar}-valued function $\map{f}{S}{\K}$ and a non-zero vector $\mb{x} \in X$, and `placing $f$ in the direction of $\mb{x}$'.
This function is denoted by $f \otimes \mb{x}$ and formally defined by
\begin{equation*}
  \map{f \otimes \mb{x}}{S}{X}, \qquad (f \otimes \mb{x})(s) := f(s)\mb{x} \quad \text{for all $s \in S$.}
\end{equation*}
The range of $f \otimes \mb{x}$ is contained in the linear span of $\mb{x}$, and is thus `one-dimensional'.

The second simplest kind of $X$-valued function are the \emph{simple functions}.
A function $\map{f}{S}{X}$ is \emph{simple} if there exists a finite collection of pairwise disjoint measurable subsets $S_1,\ldots,S_N \subset S$ and non-zero vectors $\mb{x}_1,\ldots,\mb{x}_N \in X$ such that
\begin{equation}\label{eqn:simple-function-standard-form}
  f = \sum_{n=1}^N \1_{S_n} \otimes \mb{x}_n ,
\end{equation}
where $\1_{S_n}$ is the indicator function of $S_n$.
We denote the vector space of simple functions $S \to X$ by $\Simp(S;X)$ or $\Simp_{\mc{A}}(S;X)$.
Note that the range of $f$ is contained in $\spn(\mb{x}_1,\ldots,\mb{x}_N)$, so $f$ can be thought of as `finite-dimensional'.

Of course, we need more than simple functions; measure theory tells us that the most useful class of functions are the \emph{measurable functions}.
When considering Banach-valued functions, particularly when our Banach spaces are allowed to be infinite-dimensional, there is more than one notion of measurability, and these are generally inequivalent.

\begin{defn}
  Consider a function $\map{f}{S}{X}$.
  We say that $f$ is
  \begin{itemize}
  \item
    \emph{measurable} if for every Borel set $B \subset X$, the preimage $f^{-1}(B)$ is measurable;
  \item
    \emph{strongly measurable} (or \emph{Bochner measurable}) if it is the pointwise limit of simple functions; that is, if there exists a sequence $(f_n)_{n=1}^\infty$ in $\Simp(S;X)$ such that $f = \lim_{n \to \infty} f_n$ pointwise on $S$;
  \item
    \emph{weakly measurable} if for every functional $\mb{x}^* \in X^*$, the function $\map{\langle f, \mb{x}^*\rangle}{S}{\K}$ given by $s \mapsto \langle f(s), \mb{x}^* \rangle$ is measurable.
  \end{itemize}
  All of these notions implicitly refer to the $\sigma$-algebra $\mc{A}$.

  
\end{defn}

With the convenient notation
\begin{center}
  \begin{tabular}{r|l}
    $\Meas (S;X)$  & Measurable $\map{f}{S}{X}$    \\
    $\SMeas(S;X)$  & Strongly measurable $\map{f}{S}{X}$ \\
    $\WMeas(S;X)$  & Weakly measurable $\map{f}{S}{X}$,
  \end{tabular}
\end{center}
we have the containment
\begin{equation}\label{eq:measurability-inclusions}
  \SMeas(S;X) \subset \Meas(S;X) \subset \WMeas(S;X)
\end{equation}
(Exercise \ref{ex:measurability-containments}).
When $X$ is finite-dimensional these notions coincide: the derivation of strong measurability from measurability is a standard result in measure theory,\footnote{See for example \cite[Corollary 4.2.7]{rD04} in the one-dimensional case; extending this to the finite-dimensional case can be done by summing up coordinates.} and weak measurability is just a convoluted rewriting of coordinatewise measurability.
But in the general context of Banach spaces the inclusions \eqref{eq:measurability-inclusions} are strict.

\begin{example}[A measurable function which is not strongly measurable]\todo{Can we do this with a $\sigma$-finite measure space? Maybe an uncountable product of probability spaces? Countable generation should be the issue.}
  Let $X$ be a Banach space, and consider the identity map $\map{I}{X}{X}$, which is continuous and hence measurable.
  If $I$ is strongly measurable, then there exists a sequence of simple functions $(i_n)_{n \in \N}$ converging pointwise to $I$.
  For each $\mb{x} \in X$ we then have
  \begin{equation*}
    \mb{x} = \lim_{n \to \infty} i_n(\mb{x}),
  \end{equation*}
  so that the union $U := \cup_{n \in \N} i_n(X)$ is dense in $X$.
  Since each $i_n$ is simple, $U$ is countable, which implies that $X$ is separable.
  Thus if $X$ is not separable (e.g. if $X = L^\infty(\R)$), the identity map $\map{I_X}{X}{X}$ is measurable (even continuous) but not strongly measurable.
\end{example}

\begin{rmk}
  The existence of weakly measurable functions that are not measurable is not so simple, but see \cite[Example 1.4.3]{HNVW16} for an argument which shows that the $\sigma$-algebra $\sigma(X^*)$ may be strictly smaller than the Borel $\sigma$-algebra on $X$.
  The indicator function of a Borel set which is not in $\sigma(X^*)$ is then weakly measurable, but not measurable.
\end{rmk}

It turns out that the notion of strong measurability is strongly connected with that of separability.

\begin{thm}[Pettis measurability theorem]\label{thm:Pettis-measurability}
  Let $(S,\mc{A})$ be a measurable space and $X$ a Banach space.
  Then a function $\map{f}{S}{X}$ is strongly measurable if and only if it is weakly measurable and \emph{separably valued} (i.e. there exists a separable subspace $X' \subset X$ such that $f(S) \subset X'$).
  In particular, if $X$ is separable, then
  \begin{equation*}
    \SMeas(S;X) = \Meas(S;X) = \WMeas(S;X).
  \end{equation*}
\end{thm}

\begin{proof}
  First suppose that $f$ is strongly measurable.
  Then $f$ is automatically weakly measurable, and we just need to show that $f$ is separably valued.
  Let $(f_n)_{n \in \N}$ be a sequence of simple functions converging to $f$ pointwise.
  Let $X_n \subset X$ be the closure of the range of $f_n$.
  Each $X_n$ is finite-dimensional, hence separable, and the closed subspace $X'$ of $X$ generated by the collection $(X_n)_{n \in \N}$ is also separable.
  Since $f_n \to f$ pointwise, the range of $f$ is contained in $X'$, and thus $f$ is separably valued.

  Now assume that $f$ is weakly measurable and separably valued.
  Without loss of generality we may simply assume that $X$ is separable (potentially replacing $X$ by the closure of the range of $f$).
  Let $(\mb{x}_n)_{n \in \N}$ be a dense sequence in $X$, and for each $n \in \N$ define a function $\map{\phi_n}{X}{\{\mb{x}_1,\ldots,\mb{x}_n\}}$ such that for all $\mb{x} \in X$,
  \begin{equation*}
    \|\mb{x} - \phi_n(\mb{x})\|_X = \min_{1 \leq j \leq n} \|\mb{x} - \mb{x}_j\|_X.
  \end{equation*}
  By density of $(\mb{x}_n)_{n \in \N}$ in $X$ we thus have that $\phi_n(\mb{x}) \to \mb{x}$ for all $\mb{x} \in X$.
  Now define functions $\map{f_n}{S}{X}$ by
  \begin{equation*}
    f_n(s) := \phi_n(f(s)) \qquad \forall s \in S,
  \end{equation*}
  so that $f_n \to f$ pointwise.
  Each $f_n$ has finite range, so to show that $f_n$ is simple we need only show that the preimages $f_n^{-1}(\mb{x}_k)$ are measurable.
  For all $1 \leq k \leq n$ We have
  \begin{equation*}
    f_n^{-1}(\mb{x}_k)
    = \{\phi_n(f(s)) = \mb{x}_k \}
    = \{\|f(s) - \mb{x}_k\|_X = \min_{1 \leq j \leq n} \|f(s) - \mb{x}_j\|_X \}.
  \end{equation*}
  Let $(\mb{x}_n^*)_{n \in \N}$ be a norming sequence of unit vectors on $X^*$; then since $f$ is weakly measurable, for each $j \in \{1,\ldots,n\}$ the function
  \begin{equation*}
    s \mapsto \|f(s) - \mb{x}_j\|_X = \sup_{n \in \N} |\langle f(s) - \mb{x}_j, \mb{x}_n^* \rangle|
  \end{equation*}
  is measurable (being the countable supremum of measurable functions).
  Thus the function
  \begin{equation*}
    \min_{1 \leq j \leq n} \|f(s) - \mb{x}_j\|_X
  \end{equation*}
  is also measurable, and the representation above shows that $f_n^{-1}(\mb{x}_k)$ is measurable (being the set on which two measurable functions are equal).
  Hence $f_n$ is simple, and consequently $f$ is strongly measurable.
\end{proof}

As a quick application of this result, we show that strong measurability is preserved under multiplication with measurable scalar-valued functions.\footnote{This can also be proven directly via pointwise approximation with simple functions.}

\begin{cor}\label{cor:strong-meas-meas-mult}
  Suppose that $\map{f}{S}{X}$ is strongly measurable and the scalar-valued function $\map{\phi}{S}{\K}$ is measurable.
  Then the product $\map{\phi f}{S}{X}$, $(\phi f)(s) := \phi(s)f(s)$, is strongly measurable.
\end{cor}

\begin{proof}
  By the Pettis measurability theorem, it is equivalent to show that $\phi f$ is weakly measurable and separably-valued. 
  First we show weak measurability: for each functional $\mb{x}^* \in X^*$ write for $s \in S$
  \begin{equation*}
    \langle \phi f, \mb{x}^* \rangle(s) = \phi(s) \langle f(s) , \mb{x}^* \rangle = \phi \langle f, \mb{x}^* \rangle.
  \end{equation*}
  Since $f$ is strongly measurable, it is also weakly measurable, and thus the product $\phi \langle f, \mb{x}^* \rangle$ is measurable.
  Since this is true for all $\mb{x}^* \in X^*$, we find that $\phi f$ is weakly measurable.
  To show that $\phi f$ is separably-valued, first note that since $f$ is separably-valued, there exists a separable closed subspace $X_0 \subset X$ such that $f(s) \in X_0$ for all $s \in S$.
  But then $\phi(s)f(s) \in X_0$ as well, so $\phi f$ is separably-valued.
\end{proof}

Before moving on to Bochner spaces we note that almost-everywhere equality of strongly measurable functions is equivalent to `coordinatewise' almost-everywhere equality.
This is a surprisingly useful observation; it is often used to deduce identities for vector-valued functions from corresponding identities for scalar-valued functions.

\begin{lem}\label{lem:coordinatewise-equality-test}
  Let $(S,\mc{A},\mu)$ be a measure space and $X$ a Banach space.
  Suppose that $\map{f,g}{S}{X}$ are two strongly measurable functions.
  Then $f$ and $g$ are equal almost everywhere if and only if for all functionals $\mb{x}^* \in X^*$, the scalar-valued functions $\langle f, \mb{x}^*\rangle$ and $\langle g, \mb{x}^* \rangle$ are equal almost everywhere.
\end{lem}

\begin{proof}
  The implication
  \begin{equation*}
    f \stackrel{\ae}{=}g \Longrightarrow \forall \mb{x}^* \in X^* \;  \langle f, \mb{x}^* \rangle \stackrel{\ae}{=} \langle g, \mb{x}^* \rangle
  \end{equation*}
  is straightforward, and is even true for (non-strongly) measurable functions, so we omit the proof.
  The converse direction is harder because although each of the sets
  \begin{equation*}
    N_{\mb{x}^*} := \{s \in S : \langle f(s), \mb{x}^* \rangle \neq \langle g(s), \mb{x}^* \rangle\} \qquad \mb{x}^* \in X^*
  \end{equation*}
  has measure zero, the (uncountable!) union of these sets over all $\mb{x}^* \in X^*$ does not \emph{a priori} have measure zero.
  This is where strong measurability comes into play, via the Pettis theorem.
  Since $f$ and $g$ are separably-valued, there exists a separable closed subspace $X_0 \subset X$ such that both $f$ and $g$ are $X_0$-valued.
  Since $X_0$ is separable, there is a (countable!) sequence $(\mb{x}_n^*)_{n \in \N}$ in $X^*$ which separates points of $X_0$.\footnote{That is, if $\mb{x} \neq \mb{y} \in X_0$, then there exists $n \in \N$ such that $\langle \mb{x}, \mb{x}_n^* \rangle \neq \langle \mb{y}, \mb{x}_n^* \rangle$. See \cite[Proposition B.1.11]{HNVW16}.}
  Now define
  \begin{equation*}
    N := \bigcup_{n \in \N} N_{\mb{x}_n^*};
  \end{equation*}
  this set has measure zero since it is the countable union of sets with measure zero.
  For all $s \notin N$ we then have $\langle f(s), \mb{x}_n^* \rangle = \langle g(s), \mb{x}_n^* \rangle$ for all $n \in \N$, and since $(\mb{x}_n^*)_{n \in \N}$ separates points of $X_0$, it follows that $f(s) = g(s)$.
  Thus $f \stackrel{\ae}{=} g$.  
\end{proof}



\subsection{Bochner spaces}

Given $\map{f}{S}{X}$, we let $\|f\|_X$ denote the scalar-valued function $S \to \K$ defined by $s \mapsto \|f(s)\|_X$.
If $f$ is strongly measurable, then $\|f\|_X$ is also measurable, since the function $\mb{x} \mapsto \|\mb{x}\|_X$ is continuous.

\begin{defn}
  Let $(S,\mc{A},\mu)$ be a measure space.
  For $p \in [1,\infty]$, we let $L^p(S,\mu;X)$ denote the set of \emph{strongly} measurable functions $f \in \SMeas(S;X)$ such that $\|f\|_X \in L^p(S,\mu)$, modulo $\mu$-a.e. equality, and we write
  \begin{equation*}
    \|f\|_{L^p(S,\mu;X)} := \| \|f\|_X \|_{L^p(S,\mu)}.
  \end{equation*}
  Each $L^p(S,\mu;X)$ is a Banach space: the proof is identical to the classic proof in the scalar-valued case.\footnote{For revision see \cite[Theorem 5.2.1]{rD04}.}
\end{defn}

\begin{rmk}
  It is possible for $\|f\|_X$ to be in $L^p(S)$ without $f$ itself being strongly measurable (or even measurable).
  Such a function does not qualify for membership in $L^p(S;X)$.
\end{rmk}


\begin{prop}\label{prop:simple-density}
  Let $X$ be a Banach space and $p \in [1,\infty)$.
  Then the subspace of simple functions $\Simp(S;X) \cap L^p(S;X)$ is dense in $L^p(S;X)$.
\end{prop}

\begin{proof}
  % L^infty, if needed
  % Let $f \in L^p(S;X)$ and consider the truncations
  % \begin{equation*}
  %   f_\lambda := \1_{\{s \in S : \|f_\lambda(s)\|_X \leq \lambda\}} f \qquad \forall \lambda > 0.
  % \end{equation*}
  % Then $\|(f - f_{\lambda})(s)\|_X^p \leq \|f(s)\|_X^p$ for almost all $s \in S$, so by dominated convergence
  % \begin{equation*}
  %   \begin{aligned}
  %     \lim_{\lambda \to \infty} \|f - f_{\lambda}\|_{L^p(S;X)}^p
  %     &= \lim_{\lambda \to \infty} \int_{S} \|(f - f_{\lambda})(s)\|_X^p \, \dd\mu(s) \\
  %     &= \int_{S} \lim_{\lambda \to \infty}  \|(f - f_{\lambda})(s)\|_X^p \, \dd\mu(s) = 0.
  %   \end{aligned}
  % \end{equation*}
  % Thus $f_{\lambda} \to f$ in $L^p(S;X)$ as $\lambda \to \infty$, and since each $f_{\lambda}$ is in $L^\infty(S;X) \cap L^p(S;X)$, thus subspace is dense in $L^p(S;X)$.

  Fix $f \in L^p(S;X)$.
  Since $f$ is strongly measurable, there exists a sequence of simple functions $f_n \in \Simp(S;X)$ with $\lim_{n \to \infty} f_n = f$ pointwise almost everywhere.
  Now set
  \begin{equation*}
    g_n := \1_{ \{s \in S : \|f_n(s)\|_X \leq 2\|f\|_X \} } f_n;
  \end{equation*}
  the functions $g_n$ are simple and they also converge to $f$ pointwise almost everywhere.
  Furthermore we have
  \begin{equation*}
    \|g_n\|_{L^p(S;X)}^p = \int_{\{s \in S : \|f_n(s)\|_X \leq 2\|f\|_X \} } \|f_n(s)\|_X^p \, \dd\mu(s) \leq 2^p \|f\|_{L^p(S;X)}^p,
  \end{equation*}
  so each $g_n$ is in $L^p(S;X)$.
  Since $\|f(s) - g_n(s)\|_X \leq 3\|f(s)\|_X$ for almost all $s$, dominated convergence yields
  \begin{equation*}
    \begin{aligned}
      \lim_{n \to \infty} \|f - g_n\|_{L^p(S;X)}^p &= \lim_{n \to \infty} \int_S \|f(s) - g_n(s)\|_{X}^p \, \dd\mu(s) \\
      &= \int_S \lim_{n \to \infty}  \|f(s) - g_n(s)\|_{X}^p \, \dd\mu(s) = 0,
    \end{aligned}
  \end{equation*}
  so that $g_n \to f$ in $L^p(S;X)$, completing the proof.
\end{proof}

Note that the case $p = \infty$ is not included in the proposition above, even though the simple functions are dense in $L^\infty(S)$.

\begin{prop}
  Let $X$ be a Banach space.
  Then the simple functions are dense in $\ell^\infty(\N;X)$ if and only if $X$ is finite dimensional.\footnote{This proposition can be extended to more general measure space $S$ in place of $\N$, provided $S$ contains infinitely many disjoint measurable sets of positive measure.}
\end{prop}

\begin{proof}
  First suppose $X$ is finite dimensional, and fix $f \in \ell^\infty(\N;X)$ and $\varepsilon > 0$.
  Let $C = \|f\|_{\ell^\infty(\N;X)}$, and note that the closed ball $\overline{B_C(0)} \subset X$ is compact (this uses finite dimensionality of $X$).
  Thus there exists a finite collection of vectors $(\mb{x}_i)_{i=1}^N$ in $\overline{B_C(0)}$ such that the open balls $B_{\varepsilon}(\mb{x}_i)$ cover $\overline{B_C(0)}$.
  For each $n \in \N$ we thus have that $f(n) \in B_{\varepsilon}(\mb{x}_{i(n)})$ for some $i(n) \in \{1,\ldots,N\}$.
  Define a function $\map{f_{\varepsilon}}{\N}{X}$ by
  \begin{equation*}
    f_{\varepsilon}(n) := \mb{x}_{i(n)};
  \end{equation*}
  since the range of $f_{\varepsilon}$ is finite, $f_{\varepsilon}$ is simple.
  Furthermore since $f(n) \in B_{\varepsilon}(\mb{x}_{i(n)})$ for each $n \in \N$ we have
  \begin{equation*}
    \|f - f_{\varepsilon}\|_{\ell^\infty(\N;X)} = \sup_{n \in \N} \|f(n) - \mb{x}_{i(n)}\|_{X} \leq \varepsilon.
  \end{equation*}
  Since $\varepsilon > 0$ was arbitrary, we have established density of the simple functions in $\ell^\infty(\N;X)$ when $X$ is finite dimensional.

  Now we prove the converse.
  Aiming for a contradiction, suppose that $X$ is infinite dimensional.
  Then there exists a sequence $(\mb{a}_n)_{n \in \N}$ of unit vectors in $X$ such that
  \begin{equation*}
    \|\mb{a}_n - \mb{a}_m\|_X > 1/2 \qquad \text{for all $n \neq m$.}
  \end{equation*}
  Now let $f(n) = \mb{a}_n$ for all $n \in \N$, so that $f \in \ell^\infty(\N;X)$, and suppose that there exists a simple function $g \in \Simp(\N;X)$ with $\|f - g\|_{\ell^\infty(\N;X)} < 1/4$.
  Then for all $n \neq m$ we must have
  \begin{equation*}
    \begin{aligned}
      \|\mb{a}_n - \mb{a}_m\|_X &\leq \|f(n) - g(n)\|_X + \|g(n) - g(m)\|_X + \|g(m) - f(m)\|_X \\
      &\leq \frac{1}{2} + \|g(n) - g(m)\|_X,
    \end{aligned}
  \end{equation*}
  so that
  \begin{equation*}
    \|g(n) - g(m)\|_X \geq \|\mb{a}_n - \mb{a}_m\|_X - \frac{1}{2} > 0.
  \end{equation*}
  It follows that $g$ has infinite range, contradicting the assumption that $g$ is simple.
\end{proof}

Now we present some elementary duality results.
Recall the definition of the H\"older conjugate from Section \ref{sec:conventions}.
Given a measure space $(S,\mc{A},\mu)$ and a Banach space $X$, every function $g \in L^{p'}(S;X^*)$ induces a bounded linear functional $\Phi g \in L^p(S;X)^*$ by integration of the duality pairing between $X$ and $X^*$:
\begin{equation*}
  \Phi g(f) := \int_S \langle f(s), g(s) \rangle \, \dd\mu(s) \qquad \forall f \in L^p(S;X).
\end{equation*}
H\"older's inequality implies that $\|\Phi g\|_{L^p(S;X)^*} \leq \|g\|_{L^{p'}(S;X^*)}$.
In the scalar case $X = \K$, $\Phi$ is an isometric isomorphism $L^{p'}(S) \cong L^p(S)^*$: that is, every functional $\phi \in L^p(S)^*$ is of the form $\phi = \Phi g$ for some $g \in L^{p'}(S)$, and furthermore $\|\phi\|_{L^p(S)^*} = \|g\|_{L^{p'}(S)}$.
We will see in Section \ref{sec:RNP} that for general Banach spaces $X$, $\Phi$ is an isometric isomorphism if and only if $X$ has the \emph{Radon--Nikodym property}.
For now we will establish a duality result that holds for every Banach space.

\begin{prop}\label{prop:bochner-preduality}
  Let $(S,\mc{A},\mu)$ be a $\sigma$-finite measure space\footnote{The $\sigma$-finiteness assumption is only needed for $p = 1$.} and $X$ a Banach space.
  Then for all $1 \leq p \leq \infty$, the map $\map{\Phi}{L^{p'}(S;X^*)}{L^p(S;X)^*}$ is an isometry onto a closed subspace of $L^p(S;X)^*$ which is norming for $L^p(S;X)$: that is, for every $f \in L^p(S;X)$,
  \begin{equation*}
    \|f\|_{L^p(S;X)} = \sup_{\substack{g \in L^{p'}(S;X^*) \\ g \neq 0}} \frac{| \Phi g(f) |}{\|g\|}
    = \sup_{\substack{g \in L^{p'}(S;X^*) \\ \|g\| = 1}} \Big| \int_S \langle f(s), g(s) \rangle \, \dd\mu(s) \Big|.
  \end{equation*}
\end{prop}

\begin{proof}
  To show that $\Phi$ is an isometry, it suffices to show that $\|\Phi g\|_{L^p(S;X)^*} \geq 1$ whenever $g \in L^{p'}(S;X^*)$ with $\|g\|_{L^{p'}(S;X^*)} = 1$ (the reverse estimate has already been discussed).
  
  \textbf{Mild case: $p > 1$.}
  In this case we have $p' < \infty$, so by density of the simple functions in $L^{p'}(S;X^*)$ and continuity of $\Phi$ we may assume that $g$ is simple, i.e.
  \begin{equation*}
    g = \sum_{n=1}^N \1_{S_n} \otimes \mb{x}_n^*
  \end{equation*}
  for some pairwise disjoint sets $S_n \in \mc{A}$ with $\mu(S_n) < \infty$ and some nonzero vectors $\mb{x}_n^* \in X^*$.
  Let $\varepsilon > 0$, and choose unit vectors $\mb{x}_n \in X$ (depending on $\varepsilon$) such that
  \begin{equation*}
    \langle \mb{x}_n, \mb{x}_n^* \rangle \geq (1-\varepsilon)\|\mb{x}_n^*\|_{X^*} \qquad \forall n \in \{1,\ldots,N\}.
  \end{equation*}
  Using these vectors define a test function
  \begin{equation*}
    f_\varepsilon := \sum_{n=1}^N \1_{S_n} \otimes \|\mb{x}_n^*\|_{X^*}^{p' - 1} \mb{x}_n.
  \end{equation*}
  Then we can compute
  \begin{equation*}
    \begin{aligned}
      \|f_\varepsilon\|_{L^p(S;X)}^p &= \sum_{n=1}^N \mu(S_n) \|\mb{x}_n^*\|_{X^*}^{p(p' - 1)} \|\mb{x}_n\|_{X}^p \\
      &= \sum_{n=1}^N \mu(S_n) \|\mb{x}_n^*\|_{X^*}^{p'}
      = \|g\|_{L^{p'}(S;X^*)}^{p'} = 1
    \end{aligned}
  \end{equation*}
  as the $\mb{x}_n$ are unit vectors and $p(p' - 1) = 1$.
  Testing $\Phi g$ against $f_{\varepsilon}$ yields
  \begin{equation*}
    \begin{aligned}
      \Phi g(f_\varepsilon) = \sum_{n=1}^N \mu(S_n) \|\mb{x}_n^*\|_{X^*}^{p' - 1} \langle \mb{x}_n, \mb{x}_n^* \rangle
      &\geq (1-\varepsilon) \sum_{n=1}^N \mu(S_n) \|\mb{x}_n^*\|_{X^*}^{p'} \\
      &= (1-\varepsilon) \|g\|_{L^{p'}(S;X^*)} = 1-\varepsilon.
    \end{aligned}
  \end{equation*}
  This proves that $\|\Phi g\|_{L^p(S;X)^*} \geq 1-\varepsilon$.
  Since $\varepsilon > 0$ was arbitrary, we find that $\|\Phi g\|_{L^p(S;X)^*} \geq 1$ as intended.

  \textbf{Spicy case: $p=1$.}
  Fix $\varepsilon > 0$ and define
  \begin{equation}
    A_{\varepsilon} := \{s \in S : \|g(s)\|_{X^*} > 1-\varepsilon\}.
  \end{equation}
  Then $\mu(A_\varepsilon) > 0$, but we could run into the problem that $\mu(A_\varepsilon) = \infty$.
  Since $S$ is $\sigma$-finite, we can write $S$ as an increasing union of sets of finite measure
  \begin{equation*}
    S = \bigcup_{n \in \N} S_n, \qquad S_n \subset S_{n+1}, \, \mu(S_n) < \infty \quad \forall n \in \N,
  \end{equation*}
  and thus for sufficiently large $n$ the set
  \begin{equation*}
    A_{\varepsilon}^{n} := \{s \in S_n : \|g(s)\|_{X^*} > 1 - \varepsilon\}
  \end{equation*}
  satisfies $0 < \mu(A_\varepsilon) < \infty$.\footnote{The main issue is that $A_{\varepsilon}^{n}$ could have zero measure, but if this were true for all $n$, then $\mu(A_{\varepsilon}) = \sup_{n \in \N} \mu(A_{\varepsilon}^n) = 0$, which is a contradiction.}
  Let $B_\varepsilon = A_{\varepsilon}^{n}$ for such a large $n$.

  Since $g$ is strongly measurable, the Pettis measurability theorem says that $g(B_\varepsilon)$ is separable, and thus there exists a sequence $(\mb{x}^*_{k})_{k \in \N}$ in $X^*$ such that
  \begin{equation*}
    g(B_\varepsilon) \subset \bigcup_{k \in \N} B_{\varepsilon}(\mb{x}^{*}_{k})
  \end{equation*}
  and thus
  \begin{equation*}
    B_{\varepsilon} \subset \bigcup_{k \in \N} g^{-1}(B_{\varepsilon}(\mb{x}^{*}_{k})).
  \end{equation*}
  Since $\mu(B_{\varepsilon}) > 0$, there exists a vector $\mb{x}^* \in X^*$ (i.e. one of the vectors $\mb{x}_k^{*}$) such that the set
  \begin{equation*}
     B_{\varepsilon, \mb{x}^*} := B_\varepsilon \cap g^{-1}(B_{\varepsilon}(\mb{x}^{*})) = \{s \in B_\varepsilon : \|g(s) - \mb{x}^{*}\|_{X^*} < \varepsilon\}
  \end{equation*}
  has positive measure.
  Picking a point $s_0 \in B_{\varepsilon, \mb{x}^*}$ and using the definition of $B_{\varepsilon}$, we see that
  \begin{equation*}
    \|\mb{x}^*\|_{X^*} \geq \|g(s_0)\|_{X^*} - \|g(s_0) - \mb{x}^*\|_{X^*} > 1 - 2\varepsilon.
  \end{equation*}

  Now fix a unit vector $\mb{x} \in X$ such that $\langle \mb{x}, \mb{x}^* \rangle \geq \|\mb{x}^*\|_{X^*} - \varepsilon$, and consider the test function
  \begin{equation*}
    f_{\varepsilon} := \1_{B_{\varepsilon, \mb{x}^*}} \otimes \mu(B_{\varepsilon, \mb{x}^*})^{-1} \mb{x}.
  \end{equation*}
  Then $\|f_{\varepsilon}\|_{L^1(S;X)} = 1$, and
  \begin{equation*}
    \begin{aligned}
      |\Phi g(f)| &=  \Big| \fint_{B_{\varepsilon, \mb{x}^*}} \langle \mb{x}, g(s) \rangle \, \dd\mu(s) \Big| \\
      &\geq \Big| \fint_{B_{\varepsilon, \mb{x}^*}} \langle \mb{x}, \mb{x}^* \rangle \, \dd\mu(s) \Big| - \Big| \fint_{B_{\varepsilon, \mb{x}^*}} \langle \mb{x}, g(s) - \mb{x} \rangle  \, \dd\mu(s) \Big| \\
      &\geq (\|\mb{x}^*\|_{X^*} - \varepsilon) - \varepsilon \geq 1 - 4\varepsilon.
    \end{aligned}
  \end{equation*}
  Since $\varepsilon > 0$ was arbitrary, we get $\|\Phi g\|_{L^1(S;X)^*} \geq 1$, as we wanted.

  \textbf{Norming property:}
  Since $X$ embeds isometrically into its double dual $X^{**}$, we can identify $X$-valued functions with $X^{**}$-valued functions which take values in $X$.
  We exploit this and the previous results to show
  \begin{equation*}
    \begin{aligned}
      \|f\|_{L^p(S;X)} = \|f\|_{L^p(S;(X^*)^*)} &= \|\Phi f\|_{L^{p'}(S; X^*)^*} \\
      &= \sup_{\substack{g \in L^{p'}(S;X^*) \\ \|g\| = 1}} \int_{S} |\langle g(s) , f(s) \rangle_{X^*, X^{**}}| \, \dd\mu(s) \\
      &= \sup_{\substack{g \in L^{p'}(S;X^*) \\ \|g\| = 1}} \int_{S} |\langle f(s) , g(s) \rangle_{X, X^*}| \, \dd\mu(s) \\
      &= \sup_{\substack{g \in L^{p'}(S;X^*) \\ g \neq 0}} \frac{|\Phi g(f)|}{\|g\|},
    \end{aligned}
  \end{equation*}
  completing the proof.
\end{proof}

\subsection{The Bochner integral}

We turn to defining integrals of vector-valued functions.
As in the scalar-valued theory we start with simple functions, for which there is only one reasonable definition.
Let $(S,\mc{A},\mu)$ be a measure space and $X$ a Banach space.
If $f \in \Simp(S;X)$ is a simple function represented as
\begin{equation*}
  f = \sum_{n=1}^N \1_{S_n} \otimes \mb{x}_n ,
\end{equation*}
and if in addition $f \in L^1(S,\mu;X)$, we define the \emph{Bochner integral}
\begin{equation*}
  \int_S f \, \dd\mu = \int_S f(s) \, \dd \mu(s) := \sum_{n=1}^N \mu(S_n) \mb{x}_n \in X.
\end{equation*}
Note that the assumption $f \in L^1(S,\mu;X)$ is equivalent to having $\mu(S_n) < \infty$ for all $n$.
The Bochner integral is a linear map $\Simp(S;X) \cap L^1(S,\mu;X) \to X$, and for $f$ as above it satisfies
\begin{equation*}
  \Big\| \int_S f(s) \, \dd \mu(s) \Big\|_X \leq |\mu(S_n)| \|\mb{x}_n\|_X = \|f\|_{L^1(S,\mu;X)}.
\end{equation*}
Thus by density of $\Simp(S;X) \cap L^1(S,\mu;X)$ in $L^1(S,\mu;X)$ (Proposition \ref{prop:simple-density}), the Bochner integral extends to a bounded linear map $L^1(S,\mu;X) \to X$ which we continue to call the Bochner integral and denote by the same symbol.
Thus the Bochner integral of $g \in L^1(S,\mu;X)$ is given by 
\begin{equation*}
  \int_S g \, \dd\mu := \lim_{n \to \infty} \int_S f_n \, \dd\mu \in X
\end{equation*}
where $f_n \in \Simp(S;X) \cap L^1(S,\mu;X)$ for all $n \in \N$ and $f_n \to g$ in $L^1(S,\mu;X)$.

The Bochner integral satisfies the following properties.
\begin{prop}
  Let $(S,\mc{A},\mu)$ be a measure space and $X$ a Banach space.
  The Bochner integral satisfies the following properties:
  \begin{description}
  \item[Linearity] For $f_1, f_2 \in L^1(S,\mu;X)$ and $\lambda_1, \lambda_2 \in \K$,
    \begin{equation*}
      \int_S (\lambda_1 f_1 + \lambda_2 f_2)(s) \, \dd\mu(s) = \lambda_1 \int_S f_1(s) \dd\mu(s) + \lambda_2 \int_S f_2(s) \dd\mu(s) .
    \end{equation*}
  \item[Commutation with linear maps] If $f \in L^1(S,\mu;X)$ and $T \in \Lin(X,Y)$ is a bounded linear map into a Banach space $Y$,
    \begin{equation*}
      T\Big( \int_S f \, \dd\mu \Big) = \int_S Tf(s) \, \dd\mu(s) \in Y
    \end{equation*}
    where $Tf \in L^1(S,\mu;Y)$ is given by $(Tf)(s) = T(f(s))$ for almost all $s \in S$.
    In particular, if $\mb{x}^* \in X^* = \Lin(X,\K)$, then
    \begin{equation*}
      \Big\langle \int_S f \, \dd\mu, \mb{x}^* \Big\rangle = \int_S \langle f(s), \mb{x}^* \rangle \, \dd\mu(s) \in \K.
    \end{equation*}
  \item[Closure] If $f \in L^1(S,\mu;X)$ and $X_0$ is a closed subspace of $X$ such that $f(s) \in X_0$ for almost all $s \in S$, then $\int_S f \, \dd\mu \in X_0$.
  \item[Dominated convergence] Let $(f_n)_{n \in \N}$ be a sequence in $L^1(S,\mu;X)$, $\map{f}{S}{X}$, and suppose that $\lim_{n \to \infty} f_n = f$ almost everywhere.
    Suppose that there exists a non-negative $g \in L^1(S,\mu)$ such that $\|f_n\|_X \leq g$ almost everywhere.
    Then $f \in L^1(S,\mu;X)$ and
    \begin{equation*}
      \int_S f \, \dd\mu = \lim_{n \to \infty} \int_S f_n \, \dd\mu.
    \end{equation*}
  \item[Substitution / Change of Variables]
    Let $(T,\mc{B})$ be a measurable space and $\map{\phi}{S}{T}$ a measurable function, and let $\nu = \mu \circ \phi^{-1}$ denote the pushforward measure.
    Suppose $g \in L^1(T,\nu;X)$.
    Then $g \circ \phi \in L^1(S,\mu;X)$, and
    \begin{equation*}
      \int_S g \circ \phi \, \dd\mu = \int_T g \, \dd\nu.
    \end{equation*}
    
  \end{description}
\end{prop}

\begin{proof}
  Linearity follows from the definition.
  The remaining properties are proved as follows:
  
  \begin{description}
 
    \item[Commutation with linear maps] by continuity it suffices to prove this for simple $f \in \Simp(S;X) \cap L^1(S,\mu;X)$.
    Writing $f$ as in \eqref{eqn:simple-function-standard-form} we have 
    \begin{equation*}
        T\Big(\int_S \sum_{n=1}^N \1_{S_n} \otimes \mb{x}_n \, \dd\mu \Big)
        = \sum_{n=1}^N \mu(S_n) \otimes T(\mb{x}_n) 
        = \int_S Tf \, \dd\mu.
    \end{equation*}
        
  \item[Closure] We may assume that $X_0$ is a proper subspace of $X$, otherwise there is nothing to show.
    Let $\mb{y} \in X \sm X_0$, and by Hahn--Banach\footnote{If you are philosophically opposed to Hahn--Banach, then see \cite[Corollary 1.1.22]{HNVW16} for a proof that avoids it, and promptly stop reading these notes to avoid further frustration.} choose a functional $\mb{x}^* \in X^*$ such that $\langle \mb{y}, \mb{x}^* \rangle = 1$ and $X_0 \subset \ker \mb{x}^*$.
    Then by the commutation property above we have
    \begin{equation*}
      \Big\langle \int_S f \, \dd\mu, \mb{x}^* \Big\rangle = \int_S \langle f(s), \mb{x}^* \rangle \, \dd\mu(s) = 0
    \end{equation*}
    since $f(s) \in X_0$ for almost all $s \in S$.
    Thus $\int_S f \, \dd\mu \neq y$.
    Since $y \in X \sm X_0$ was arbitrary, we conclude that $\int_S f \, \dd\mu \in X_0$.

  \item[Dominated convergence]
    By continuity of the Bochner integral it suffices to show that $f \in L^1(S,\mu;X)$ and  $f_n \to f$ in $L^1(S,\mu;X)$.
    The first fact follows from $\|f\|_{L^1(S,\mu;X)} \leq \|g\|_{L^1(S,\mu)}$.
    For the second, since $\|(f_n - f)(s)\|_X \leq 2g(s)$ almost everywhere, we have
    \begin{equation*}
      \lim_{n \to \infty} \int_S \|(f_n - f)(s)\|_X \, \dd\mu(s) = 0
    \end{equation*}
    by scalar dominated convergence.

  \item[Substitution]
    First we need to show that $g \circ \phi$ is strongly measurable.
    Let
    \begin{equation*}
      g_i = \sum_{n=1}^{N_i} \1_{S_{n,i}} \otimes \mb{x}_{n,i}
    \end{equation*}
    be a sequence of simple functions converging to $g$ $\mu$-almost everywhere.
    Then
    \begin{equation*}
      g_i \circ \phi = \sum_{n=1}^{N_i} (\1_{S_{n,i}} \circ \phi) \otimes \mb{x}_{n,i} = \sum_{n=1}^{N_i} \1_{\phi^{-1}(S_{n,i})} \otimes \mb{x}_{n,i}
    \end{equation*}
    is a sequence of simple functions converging to $g \circ \phi$ $\nu$-almost everywhere, so $g \circ \phi$ is also strongly measurable.
    The identity for scalar-valued functions
    \begin{equation*}
      \int_S \|f \circ \phi(s)\|_X \, \dd\mu(s) = \int_T \|f(s)\|_X \, \dd\nu(t)
    \end{equation*}
    implies that $f \circ \phi \in L^1(S,\mu;X)$.
    Finally, for all $\mb{x}^* \in X^*$ we have by the commutation property and the substitution identity for scalar-valued functions
    \begin{equation*}
      \begin{aligned}
        \Big\langle \int_S f \circ \phi \, \dd\mu , \mb{x}^* \Big\rangle
        = \int_S \langle f(\phi(s)), \mb{x}^* \rangle \, \dd\mu(s)
        &= \int_T \langle f(s), \mb{x}^* \rangle \, \dd\nu(t) \\
        &= \Big\langle \int_T f \, \dd\nu, \mb{x}^* \Big\rangle,
      \end{aligned}
    \end{equation*}
    which proves the result.
  \end{description}
\end{proof}

There is also a Fubini theorem for Banach-valued functions (but no Tonelli theorem, as we do not have access to the notion of a non-negative vector-valued function).

\begin{prop}[Fubini]
  Let $(S,\mc{A},\mu)$ and $(T,\mc{B},\nu)$ be $\sigma$-finite measure spaces\footnote{The scalar Fubini theorem fails for non-$\sigma$-finite spaces!} and let $f \in L^1(S \times T; X)$.
  Then
  \begin{itemize}
  \item for almost all $s \in S$ the function $f(s,\cdot)$ is in $L^1(T;X)$,
  \item for almost all $t \in T$ the function $f(\cdot,t)$ is in $L^1(S;X)$,
  \item the functions $\int_T f(\cdot,t) \, \dd\nu(t)$ and $\int_S f(s,\cdot) \, \dd\mu(s)$ are in $L^1(T;X)$ and $L^1(S;X)$ respectively, and
    \begin{equation}\label{eq:fubini}
      \int_{S \times T} f \, \dd(\mu \times \nu) = \int_T \Big(  \int_S f(s,t) \, \dd\mu(s) \Big) \, \dd\nu(t) = \int_S \Big(\int_T f(s,t) \, \dd\nu(t) \Big) \, \dd\mu(s).
    \end{equation}
  \end{itemize}
\end{prop}

\begin{proof}
  Consider an everywhere-defined representative of $f$.
  Since $f$ is strongly measurable, by the Pettis measurability theorem (Theorem \ref{thm:Pettis-measurability}), it is weakly measurable and separably valued.
  Thus the functions $f(s,\cdot)$ and $f(\cdot,t)$ are separably valued for all $s \in S$ and $t \in T$, and by the corresponding scalar-valued result, they are both weakly measurable.
  Thus $f(s,\cdot)$ and $f(\cdot,t)$ are strongly measurable.
  Now since the function $(s,t) \mapsto \|f(s,t)\|_X$ is integrable, the scalar Fubini theorem implies all of the integrability claims.
  The equalities \eqref{eq:fubini} are proven by scalarisation: for $\mb{x}^* \in X^*$ we have
  \begin{equation*}
    \begin{aligned}
      \Big\langle \int_{S \times T} f \, \dd(\mu \times \nu) , \mb{x}^* \Big\rangle
      &= \int_{S \times T} \langle f(s,t), \mb{x}^* \rangle \, \dd(\mu \times \nu) \\
      &= \int_{S} \int_{T} \langle f(s,t), \mb{x}^* \rangle \, \dd\nu(t) \, \dd\mu(s) \\
      &= \int_{S} \Big\langle \int_{T} f(s,t) \, \dd\nu(t) , \mb{x}^* \Big\rangle \, \dd\mu(s) \\
      &= \Big\langle \int_S \Big(\int_{T} f(s,t) \, \dd\nu(t) \Big)\, \dd\mu(s) , \mb{x}^* \Big\rangle
  \end{aligned}
  \end{equation*}
  by the scalar Fubini theorem, and likewise with the order of $S$ and $T$ reversed.
\end{proof}

Let's move away from the abstract stuff for a moment and define the most important operators in analysis.

\begin{defn}
  Let $X$ be a complex Banach space.
  For a Bochner integrable function $f \in L^1(\R^d;X)$ we define the \emph{Fourier transform} as the Bochner integral
  \begin{equation*}
    \hat{f}(\xi) = \mc{F}(f)(\xi) := \int_{\R^d} f(t) e^{-2\pi i t \cdot \xi} \, \dd t \qquad \forall \xi \in \R^d. 
  \end{equation*}
  Note that $\hat{f}(\xi) \in X$ for all $x \in \R^d$.
  We also define the \emph{inverse Fourier transform} on $g \in L^1(\R^d;X)$:
  \begin{equation*}
    g^{\vee}(t) = \mc{F}^{-1}(g)(x) := \int_{\R^d} g(\xi) e^{2\pi i t \cdot \xi} \, \dd \xi \qquad \forall t \in \R^d.
  \end{equation*}

  For functions $f \in L^1(\T^d ; X)$ on the $d$-torus $\T^d = [0,1]^d$, we use the same notation for the Fourier transform (and its inverse on $g \in L^1(\Z^d; X)$)
  \begin{equation*}
    \begin{aligned}
      \hat{f}(n) &= \mc{F}(f)(n) := \int_{\T^d} f(t) e^{-2\pi i t \cdot n} \, \dd t \qquad \forall n \in \Z^d, \\
      g^{\vee}(t) &= \mc{F}^{-1}(g)(t) := \sum_{n \in \Z^d} g(n) e^{2\pi i t \cdot n} X \qquad \forall t \in \R^d.
    \end{aligned}
  \end{equation*}
\end{defn}

Note that if $f \in L^1(\R^d;X)$, then the function $x \mapsto f(x)e^{-2\pi i x \cdot \xi}$ is strongly integrable for each $\xi \in \R^d$ (see Lemma \ref{cor:strong-meas-meas-mult}), so the definitions above make sense.\footnote{Analogous statements hold for $\T^d$ and $\Z^d$ of course.}
Furthermore
\begin{equation*}
  \|\hat{f}(\xi)\|_X \leq \int_{\R^d} \| f(x) e^{-2\pi i x \cdot \xi} \|_X \, \dd x = \|f\|_{L^1(\R^d;X)}.
\end{equation*}
In fact, $\hat{f}$ is itself strongly integrable, and the Fourier transform is bounded from $L^1(\R^d;X)$ to $L^\infty(\R^d;X)$ (see Exercise \ref{ex:FT-bounded-1-infty}).
Formally, the Fourier transform and inverse Fourier transform are mutually inverse operators, but to make this statement rigourous we have to restrict to appropriate classes of functions or distributions, which for now we will not do.



\subsection{Extensions of operators to Bochner spaces}

\begin{defn}
  For a measurable space $(S,\mc{A})$ and a set $V \subset \Meas(S;\K)$ of measurable scalar-valued functions on $S$, we define the \emph{algebraic tensor product}
  \begin{equation*}
    V \otimes X := \spn\{f \otimes \mb{x} : f \in V, \mb{x} \in X\} \subset \SMeas(S;X).\footnote{Functions in $V \otimes X$ are strongly measurable since they have finite-dimensional range.}
  \end{equation*}
  That is, $V \otimes X$ is the set of finite linear combinations of $X$-valued functions of the form $f \otimes \mb{x}$, where $f$ is a scalar-valued function in $V$ and $\mb{x} \in X$.
\end{defn}

For example, when $V$ is the set of characteristic functions of measurable sets, $V \otimes X = \Simp(S;X)$ is the set of $X$-valued simple functions.
Another fundamental example is $V = L^p(S)$ for some $p \in [1,\infty]$.

\begin{prop}\label{prop:ATP-density}
  Let $(S,\mc{A},\mu)$ be a measure space, $X$ a Banach space, and $p \in [1,\infty)$.
  Then $L^p(S) \otimes X$ is a dense subspace of $L^p(S;X)$.
\end{prop}

\begin{proof}
  For $f \in L^p(S)$ and $\mb{x} \in X$ we compute
  \begin{equation*}
    \|f \otimes \mb{x}\|_{L^p(S,\mu;X)}^p = \int_S \|f(s)\mb{x}\|_{X}^p \, \dd\mu(s) = \|\mb{x}\|_X^p \|f\|_{L^p(S)}^p,
  \end{equation*}
  so that $f \otimes \mb{x} \in L^p(S;X)$.
  By linearity, this implies $L^p(S) \otimes X$ is contained in $L^p(S;X)$.
  For density, note that $L^p(S) \otimes X$ contains $(\Sigma(S;\K) \cap L^p(S)) \otimes X$, and that
  \begin{equation*}
    (\Sigma(S;\K) \cap L^p(S)) \otimes X = \Sigma(S;X) \cap L^p(S;X),
  \end{equation*}
  as both spaces are equal to the set of simple functions with finite measure support.
  By Proposition \ref{prop:simple-density}, this space is dense in $L^p(S;X)$, and thus the same is true of $L^p(S) \otimes X$.
\end{proof}

\begin{defn}\label{defn:tensor-exts}
  Let $(S_i,\mc{A}_i,\mu_i)$ ($i \in \{1,2\}$) be measure spaces, $p_1 \in [1,\infty)$ and $p_2 \in [1,\infty]$, and consider a bounded linear operator
  \begin{equation*}
    \map{T}{L^{p_1}(S_1,\mu_1)}{L^{p_2}(S_2,\mu_2)}
  \end{equation*}
  acting on scalar-valued functions.
  Let $X$ be a Banach space.
  The \emph{tensor extension} of $T$ with the identity $\map{I}{X}{X}$ is the map between algebraic tensor products
  \begin{equation*}
    \map{T \otimes I}{L^{p_1}(S_1,\mu_1) \otimes X}{L^{p_2}(S_2,\mu_2) \otimes X}
  \end{equation*}
  satisfying $(T \otimes I)(f \otimes \mb{x}) = (Tf) \otimes \mb{x}$ for all $f \in L^{p_1}(S_1,\mu_1)$ and $\mb{x} \in X$.
\end{defn}

The tensor extension is a well-defined map between algebraic tensor products $L^{p_1}(S_1) \otimes X \to L^{p_2}(S_2) \otimes X$.
By Proposition \ref{prop:ATP-density}, $L^{p_1}(S_1) \otimes X$ is a dense subspace of $L^{p_1}(S_1;X)$, while $L^{p_2}(S_2) \otimes X$ is a subspace of $L^{p_2}(S_2;X)$ (possibly non-dense if $p_2 = \infty$), so if there exists $C < \infty$ such that
\begin{equation}\label{eq:tensor-ext-estimate}
  \|(T \otimes I)f\|_{L^{p_2}(S_2;X)} \leq C \|f\|_{L^{p_1}(S_1;X)} \qquad \forall f \in L^{p_1}(S_1) \otimes X,
\end{equation}
then $T \otimes I$ may be extended to a bounded linear operator $L^{p_1}(S_1;X) \to L^{p_2}(S_2;X)$.

\begin{defn}
  With the notation above, if the estimate \eqref{eq:tensor-ext-estimate} holds, we say that $T$ \emph{admits a bounded $X$-valued extension}, and we denote the continuous extension of $T \otimes I$ by $\td{T}_X$, $\td{T}$, or even just $T$.
\end{defn}


Writing out a general element $f \in L^p(S) \otimes X$ as a linear combination of elementary tensors, we see that $T$ admits a bounded $X$-valued extension if and only if there exists a constant $C < \infty$ such that
\begin{equation}\label{eq:tensor-ext-estimate-full}
  \Big\|\sum_{n=1}^N (Tf_n) \otimes \mb{x}_n\Big\|_{L^{p_2}(S_2;X)} \leq C \Big\|\sum_{n=1}^N f_n \otimes \mb{x}_n\Big\|_{L^{p_1}(S_1;X)}
\end{equation}
for all functions $f_n \in L^{p_1}(S_1)$ and vectors $\mb{x}_n \in X$.
This estimate does not simply follow from boundedness of $T$.
It turns out to rely on potentially subtle interactions between the operator $T$ and the Banach space $X$.

We have already seen one fundamental example.
\begin{example}
  Fix a measure space $(S,\mc{A},\mu)$ and let $\map{T}{L^1(S)}{\K}$ denote the Lebesgue integral.\footnote{This fits in the scope of Definition \ref{defn:tensor-exts} by considering $\C$ as a Lebesgue space $L^1(\mathrm{pt})$ over a single point, with counting measure. Then $X$ may be identified with the Bochner space $L^1(\mathrm{pt};X)$.}
  Let $X$ be a Banach space.
  Then for all $f \in \Simp(S;\K) \otimes X$ we have
  \begin{equation*}
    (T \otimes I)f = (T \otimes I)\Big(\sum_{n=1}^N \1_{S_n} \otimes \mb{x}_n \Big) = \sum_{n=1}^N T(\1_{S_n}) \otimes \mb{x}_n = \sum_{n=1}^N \mu(S_n) \otimes \mb{x}_n = \int_S f \, \dd\mu,
  \end{equation*}
  so that the tensor extension of the Lebesgue integral agrees with the Bochner integral, which we have already shown maps $L^1(S;X)$ to $X$.
  Thus the Lebesgue integral admits a bounded $X$-valued extension, namely the Bochner integral.
\end{example}

In this example, the Banach space $X$ plays no real role; we will see in Theorem \ref{thm:positive-extensions} that this phenomenon occurs for all positive operators.
Before that we record a simple observation: tensor extensions can do no better than the original operator.

\begin{prop}
  Fix measure spaces $(S_i,\mc{A}_i,\mu_i)$ ($i \in \{1,2\}$) and exponents $p_1 \in [1,\infty)$, $q \in [1,\infty]$.
  Let $T \in \Lin(L^p(S),L^q(S))$ be a bounded linear operator, and let $X$ be a Banach space.
  Then the tensor extension $T \otimes I$ satisfies
  \begin{equation*}
    \|T \otimes I\|_{L^{p_1}(S_1;X) \to L^{p_2}(S_2;X)} \geq \|T\|_{L^{p_1}(S_1) \to L^{p_2}(S_2)}.
  \end{equation*}
\end{prop}

\begin{proof}
  Fix a nonzero vector $\mb{x} \in X$.
  Then for all nonzero $f \in L^{p_1}(S_1)$ we have
  \begin{equation*}
    \begin{aligned}
      \|(T \otimes I)(f \otimes \mb{x})\|_{L^{p_2}(S_2;X)} = \|Tf \otimes \mb{x}\|_{L^{p_2}(S_2;X)} &= \|Tf\|_{L^{p_2}(S_2)} \|\mb{x}\|_X \\
      &= \frac{\|Tf\|_{L^{p_2}(S_2)}}{\|f\|_{L^{p_1}(S_1)}} \|f \otimes \mb{x}\|_{L^{p_1}(S_1;X)}.
    \end{aligned}
  \end{equation*}
  Taking the supremum over all nonzero $f \in L^p(S)$ completes the proof.
\end{proof}

\begin{thm}\label{thm:positive-extensions}
  Fix measure spaces $(S_i,\mc{A}_i,\mu_i)$ ($i \in \{1,2\}$), $p_1 \in [1,\infty)$, and $p_2 \in [1,\infty]$.
  Let $T \in \Lin(L^{p_1}(S_1),L^{p_2}(S_2))$ be a bounded linear operator which is \emph{positive}, meaning that for all a.e. non-negative $f \in L^{p_1}(S_1)$, $Tf \in L^{p_2}(S_2)$ is also a.e. non-negative.\footnote{When the scalar field $\K$ is $\C$, `non-negative' simply means `real-valued and non-negative'.}
  Then $T$ admits a bounded $X$-valued extension for every Banach space $X$, and in fact
  \begin{equation*}
    \|\td{T}\|_{L^{p_1}(S_1;X) \to L^{p_2}(S_2;X)} = \|T\|_{L^{p_1}(S_1) \to L^{p_2}(S_2)}.
  \end{equation*}
\end{thm}

\begin{proof}
  We will show the estimate
  \begin{equation}\label{eq:positive-pw-est}
    \|\td{T}f(s)\|_X \leq T(\|f\|_X)(s)
  \end{equation}
  for all $f \in \Simp(S;X) \cap L^{p_1}(S_1;X)$ and almost all $s \in S$.
  This will imply
  \begin{equation*}
    \begin{aligned}
      \|\td{T}f\|_{L^{p_2}(S_2;X)} &= \Big( \int_{S} \|\td{T}f(s)\|_X^{p_2} \, \dd\mu(s) \Big)^{1/p_2} \\
      &\leq \Big( \int_{S} T(\|f(s)\|_X)^{p_2} \, \dd\mu(s) \Big)^{1/p_2}\\
      &\leq \|T\|_{L^{p_1}(S_1) \to L^{p_2}(S_2)} \Big( \int_{S} \|f(s)\|_X^{p_1} \, \dd\mu(s) \Big)^{1/p_1} \\
      &= \|T\|_{L^{p_1}(S_1) \to L^{p_2}(S_2)} \|f\|_{L^{p_1}(S_1;X)}
    \end{aligned}
  \end{equation*}
  which implies the result by density of $\Simp(S;X) \cap L^{p_1}(S_1;X)$ in $L^{p_1}(S_1;X)$.

  Now let's prove \eqref{eq:positive-pw-est}.
  Consider a simple function
  \begin{equation*}
    f = \sum_{n=1}^N \1_{S_n} \otimes \mb{x}_n
  \end{equation*}
  and note that $|T(\1_{S_n})| = T(\1_{S_n})$ by positivity of $T$.
  Then
  \begin{equation*}
    \begin{aligned}
      \|\td{T}f(s)\|_X &= \Big\| \sum_{n=1}^N T(\1_{S_n})(s) \mb{x}_n \Big\|_X \\
      &\leq \sum_{n=1}^N |T(\1_{S_n})(s)| \|\mb{x}_n\|_X \\
      &= \sum_{n=1}^N T(\1_{S_n})(s) \|\mb{x}_n\|_X 
      = T\Big( \sum_{n=1}^N \1_{S_n} \|\mb{x}_n\|_X \Big)(s) 
      = T(\|f\|_X)(s),
    \end{aligned}
  \end{equation*}
  proving \eqref{eq:positive-pw-est} and completing the proof.
\end{proof}

Theorem \ref{thm:positive-extensions} shows that the `extension problem' for positive operators is not much of a problem: positive operators extend automatically.
Of course, most interesting operators are not positive.

\begin{example}
  \todo{Fourier transform. counterexample for $\ell^\infty$}\todo{write this up}
\end{example}



\subsection*{Exercises}

\begin{exercise}\label{ex:measurability-containments}
  Let $X$ be a Banach space and $(S,\mc{A})$ a measurable space.
  Prove the containments
  \begin{equation*}
    \SMeas(S;X) \subset \Meas(S;X) \subset \WMeas(S;X).
  \end{equation*}
\end{exercise}

\begin{exercise}
  Let $X$ be a Banach space, let $A$ be a topological space, and let $\mu$ be a Borel measure on $A$.
  \begin{itemize}
  \item If $X$ is separable or $A$ is separable, show that $C(A;X)$ is contained in $L^\infty(A, \mu;X)$.
  \item Give an example of a topological space $A$ and a Banach space $X$ such that $C(A;X)$ is not contained in $L^\infty(A,\mu;X)$.
  \end{itemize}
\end{exercise}

\begin{exercise}
  Let $(S_i, \mc{A}_i, \mu_i)$ ($i \in \{1,2\}$) be measure spaces, let $p_1 \in [1,\infty)$, and let $p_2 \in [1,\infty]$.
  Suppose that $T \in \Lin(L^{p_1}(S_1), L^{p_2}(S_2))$ is a bounded linear operator. 
  Show that $T$ admits a bounded $X$-valued extension for all finite-dimensional Banach spaces $X$.
\end{exercise}

\begin{exercise}\label{ex:tensor-adjoint}
  Let $(S,\mc{A},\mu)$ be a measure space and $p \in (1,\infty)$, let $T$ be a bounded linear operator on $L^p(S,\mu)$, and let $X$ be a Banach space.
  Let $T^*$ denote the adjoint of $T$.
  Show that $T$ admits a bounded $X$-valued extension if and only if the adjoint $T^* \in \Lin(L^{p'}(S,\mu))$ admits a bounded $X^*$-valued extension, and show that
  \begin{equation*}
    (\td{T}_X)^* \Phi f = \td{(T^*)}_{X^*} f
  \end{equation*}
  for all $f \in L^{p'}(S;X^*)$, where $\map{\Phi}{L^{p'}(S;X^*)}{L^p(S;X)^*}$ is as in Proposition \ref{prop:bochner-preduality}. 
\end{exercise}

\begin{exercise}\label{ex:FT-bounded-1-infty}
  Let $X$ be a complex Banach space.
  Show that $\hat{f} \in C(\R^d;X)$ for all $f \in L^1(\R^d;X)$.
  Conclude that the Fourier transform is bounded from $L^1(\R^d;X)$ to $L^\infty(\R^d;X)$.
\end{exercise}

\begin{exercise}
  Let $X$ and $Y$ be Banach spaces and consider an operator-valued function $\map{M}{\R^d}{\Lin(X,Y)}$, where $\Lin(X,Y)$ is the Banach space of bounded linear operators from $X$ to $Y$.
  Suppose that $M$ is continuous with respect to the strong operator topology on $\Lin(X,Y)$: that is, suppose that for all vectors $\mb{x} \in X$, the map
  \begin{equation*}
    \map{M(\cdot)\mb{x}}{\R^d}{Y}, \qquad \xi \mapsto M(\xi)\mb{x}
  \end{equation*}
  is continuous.
  \begin{itemize}
  \item
    Let $\map{g}{\R^d}{X}$ be strongly measurable.
    Show that the function $\map{Mg}{\R^d}{Y}$ defined by $(Mg)(\xi) := M(\xi)g(\xi)$ is strongly measurable.
  \item
    Suppose in addition that the function $\xi \mapsto \|M(\xi)\|_{\Lin(X,Y)}$ is measurable,\footnote{This can be proven under various assumptions, but don't worry about that now.} and that
    \begin{equation*}
      \int_{\R^d} \|M(\xi)\|_{\Lin(X,Y)} \, \dd \xi < \infty.
    \end{equation*}
    Show that the operator $T_Mf := (M\hat{f})^\vee$ is well-defined and bounded from $L^1(\R^d;X)$ to $C(\R^d;Y)$.
  \end{itemize}
\end{exercise}

\begin{exercise}
  Let $H$ be an infinite-dimensional separable Hilbert space with inner product $(\cdot , \cdot)$, and let $(S,\mc{A},\mu)$ be a measure space.
  \begin{itemize}
  \item Show that $L^2(S;H)$ is a Hilbert space with respect to the inner product
    \begin{equation*}
      (f, g) := \int_{S} (f(s), g(s)) \, \dd\mu(s) \qquad (f,g \in L^2(S;H)).
    \end{equation*}
  \item Let $(\mb{h}_n)_{n \in \N}$ be an orthonormal basis of $H$, and $(f_n)_{n \in \N}$ an orthonormal basis of $L^2(S,\mu)$.
    Show that the elementary tensors $\{f_n \otimes \mb{h}_m : n,m \in \N\}$ are an orthonormal basis of $L^2(S;H)$.
  \item Suppose that the Hilbert space $H$, as above, is defined over $\C$.
    Show that the Fourier transform on the torus, initially defined as a bounded operator $\map{\mc{F}}{L^1(\T^d;H)}{\ell^\infty(\Z^d;H)}$, extends to an isometry from $L^2(\T^d;H)$ to $\ell^2(\Z^d;H)$.
  \end{itemize}
\end{exercise}

\begin{exercise}
  Give an example of a measure space $(S, \mc{A}, \mu)$, a Banach space $X$, and a measurable function $\map{f}{S}{X}$ such that
  \begin{itemize}
  \item $\langle f, \mb{x}^* \rangle \stackrel{\ae}{=} 0$ for all $\mb{x}^* \in X^*$,
  \item $f$ is everywhere non-zero.
  \end{itemize}
  \todo{I don't even know the answer to this one, maybe too hard...}
\end{exercise}


%%% Local Variables:
%%% mode: latex
%%% TeX-master: "../main.tex"
%%% End:


\section{Probability in Banach spaces}
\label{sec:martingales} 
\subsection{Conditional expectations}


\begin{defn}\label{defn:conditional-expectation} %Dudley, p336
  Let $(\Omega, \mc{A}, \P)$ be a probability space, and let $f \in L^1(\Omega, \mc{A}; \P)$ be an integrable random variable.
  Given a $\sigma$-subalgebra $\mc{B} \subset \mc{A}$, a \emph{conditional expectation of $f$ given $\mc{B}$} is a $\mc{B}$-measurable random variable $\E^{\mc{B}}(f)$ such that
  \begin{equation}\label{eq:conditional-expectation-property}
    \qquad \int_B \E^{\mc{B}}(f) \, \dd\P = \int_B f \, \dd\P \qquad \text{for all $B \in \mc{B}$.}
  \end{equation}
\end{defn}


\begin{thm}\label{thm:conditional-expectation-EU}
  For any $f \in L^1(\Omega, \mc{A}; \P)$ and any $\sigma$-subalgebra $\mc{B} \subset \mc{A}$, a unique conditional expectation $\E^{\mc{B}}(f) \in L^1(\Omega, \mc{B}, \P)$ exists.
\end{thm}

\begin{proof} %Dudley p337 for uniqueness and 'proof by measure theory', 
  First let's handle the uniqueness.
  Suppose $\E^{\mc{B}}(f)$ and $\td{\E}^{\mc{B}}(f)$ are two conditional expectations of $f$ given $\mc{B}$.
  Then for all subsets $B \in \mc{B}$ we have
  \begin{equation*}
    \int_B \E^{\mc{B}}(f) - \td{\E}^{\mc{B}}(f) \, \dd\P = \int_B f \, \dd\P - \int_B f \, \dd\P = 0.
  \end{equation*}
  Since $\E^{\mc{B}}(f) - \td{\E}^{\mc{B}}(f)$ is $\mc{B}$-measurable, the subsets
  \begin{equation*}
    B_+ := \{\E^{\mc{B}}(f) - \td{\E}^{\mc{B}}(f) > 0\} \quad \text{and} \quad B_- := \{\E^{\mc{B}}(f) - \td{\E}^{\mc{B}}(f) < 0\}
  \end{equation*}
  are both in $\mc{B}$, so we get
  \begin{equation*}
    \int_\Omega | \E^{\mc{B}}(f) - \td{\E}^{\mc{B}}(f) | \, \dd\P
    = \int_{B_+}  \E^{\mc{B}}(f) - \td{\E}^{\mc{B}}(f) \, \dd\P -  \int_{B_-}  \E^{\mc{B}}(f) - \td{\E}^{\mc{B}}(f) \, \dd\P 
    = 0,
  \end{equation*}
  establishing that $\E^{\mc{B}}(f) = \td{\E}^{\mc{B}}(f)$ almost surely.

  As for existence: let $P^{\mc{B}}$ be the orthogonal projection of $L^2(\Omega, \mc{A}, \P)$ onto the closed subspace $L^2(\Omega, \mc{B}, \P)$.
  Then for all $f \in L^1(\Omega, \mc{A}, \P) \cap L^2(\Omega, \mc{A}, \P)$ and all $B \in \mc{B}$,
  \begin{equation}
    \int_B P^{\mc{B}} f \, \dd\P = \langle P^{\mc{B}} f, \1_{B} \rangle = \langle f, P^{\mc{B}} \1_{B} \rangle = \langle f, \1_{B} \rangle = \int_B f \, \dd\P
  \end{equation}
  where $\langle \cdot, \cdot \rangle$ denotes the inner product on $L^2(\Omega, \mc{A}, \P)$, using that $\1_{B} \in L^2(\Omega, \mc{B}, \P)$.
  Furthermore, 
  \begin{equation*}
    \|P^{\mc{B}} f \|_{L^1(\Omega, \mc{B}, \P)} = \sup_{g} |\langle P^{\mc{B}} f, g \rangle|
    = \sup_{g} |\langle f, P^{\mc{B}} g \rangle|
    = \sup_{g} |\langle f, g \rangle| 
    \leq \|f\|_{L^1(\Omega, \mc{A}, \P)}
  \end{equation*}
  where the supremum is over all $g \in L^\infty(\Omega, \mc{B}, \P)$ with $\|g\|_\infty = 1$ (note that $g \in L^2(\Omega, \mc{B}, \P)$ follows since $\Omega$ is a probability space).
  Thus $P^{\mc{B}}$ extends to a contraction $\map{\E^{\mc{B}}}{L^1(\Omega, \mc{A}, \P)}{L^1(\Omega, \mc{B}, \P)}$, and for all $f \in L^1(\Omega, \mc{A}, \P)$ and $B \in \mc{B}$ this extension satisfies
  \begin{equation*}
    \begin{aligned}
      \int_B \E^{\mc{B}}f \, \dd\P = \lim_{n \to \infty} 
    \end{aligned}
  \end{equation*}
  where $(f_n)_{n \in \N}$ is a sequence in $L^2(\Omega, \mc{A}, \P)$ converging to $f$ in $L^1$.
\end{proof}

Theorem \ref{thm:conditional-expectation-EU} lets us speak of \emph{the} conditional expectation of an integrable random variable.

\begin{example}
  (atomic $\sigma$-algebra, conditional expectation given by conditional probabilities)
\end{example}




\subsection{Definitions and key examples of martingales}

-stopped martingales
-Rademacher variables and sums

\subsection{Doob's maximal inequality}

\subsection{John--Nirenberg for adapted sequences and the Kahane--Khintchine inequalities}
% following HNVW1 section 3.2.c

Fix a Banach space $X$ and a $\sigma$-finite measure space $(S,\mc{A},\mu)$.
Consider a filtration $(\mc{F}_n)_{n \in \N}$ and a sequence $\phi = (\phi_n)_{n \in \N}$ of $X$-valued functions adapted to the filtration (i.e. $\phi_n$ is $\mc{F}_n$-measurable for all $n \in \N$).\todo{Strongly?}
For all $q \in (0,\infty)$ we consider the following measure of the oscillation of $\phi$:
\begin{equation*}
    \|\phi\|_{*,q} := \sup_{\substack{k,n \in \N \\ k \leq n}} \sup_{\substack{F \in \mc{F}_k \\ 0 < \mu(F) < \infty}} \Big( \fint_{F} \|(\phi_n - \phi_{k-1})(s)\|_{X}^{q} \, \dd\mu(s)\Big)^{1/q} .
\end{equation*}


\begin{thm}[John--Nirenberg inequality for adapted sequences]\label{thm:jn-adapted-sequences}
  With notation as above, for all $p,q \in (0,\infty)$ there exists a finite constant $c_{p,q}$ independent of $\phi$ such that
  \begin{equation*}
    \|\phi\|_{*,p} \leq c_{p,q} \|\phi\|_{*,q}.
  \end{equation*}
\end{thm}

We will prove this as a consequence of a series of lemmas, but before proving this, we demonstrate an important application to Rademacher sums.

\begin{thm}[Kahane--Khintchine inequality]\label{thm:kk}
  Let $X$ be a Banach space and let $(\varepsilon_{n})_{n \in \N}$ be a Rademacher sequence on a probability space $(\Omega,\mc{F},\P)$.
  Then for all $p,q \in (0,\infty)$ there exists a finite constant $\kappa_{p,q}$ such that for all finite sequences $(\mb{x}_n)_{n=1}^N$ in $X$,
  \begin{equation*}
    \Big\|\sum_{n=1}^{N} \varepsilon_{n} \mb{x}_{n} \Big\|_{L^p(\Omega;X)} \leq \kappa_{p,q} \Big\|\sum_{n=1}^{N} \varepsilon_{n} \mb{x}_{n} \Big\|_{L^q(\Omega;X)}.
  \end{equation*}
  That is, for all $p \in (0,\infty)$, the $L^p$-norms of a Rademacher sum are pairwise equivalent.  
\end{thm}

Since $\Omega$ is a probabibility space, H\"older's inequality yields the case $p \leq q$ with constant $\kappa_{p,q} = 1$, so we only need to consider the case $q < p$.

\begin{proof}[Proof, assuming the John--Nirenberg inequality]
  Consider the filtration and adapted sequence \todo{make sure to discuss filtrations generated by functions, particularly Rademacher variables} 
  \begin{equation*}
    \mc{F}_n := \sigma(\{\varepsilon_j : 1 \leq j \leq n\}), \qquad \phi_n := \sum_{j=1}^n \varepsilon_j \mb{x}_j.
  \end{equation*}
  We claim that for all $q \in [1,\infty)$ we have
  \begin{equation}\label{eq:kk-claim}
    \|\phi\|_{*,q} = \Big\| \sum_{j=1}^N \varepsilon_j \mb{x}_j \Big\|_{L^q(\Omega;X)}.
  \end{equation}
  Assuming this for the moment, the John--Nirenberg inequality yields a finite constant $c_{p,q}$ such that
  \begin{equation*}
    \Big\| \sum_{j=1}^N \varepsilon_j \mb{x}_j \Big\|_{L^p(\Omega;X)}
    = \|\phi\|_{*,p}
    \leq c_{p,q} \|\phi\|_{*,q}
    = \Big\| \sum_{j=1}^N \varepsilon_j \mb{x}_j \Big\|_{L^q(\Omega;X)}
  \end{equation*}
  whenever $1 \leq q < p$.
  If $q < 1 \leq p$, we fix $\theta \in (0,1)$ so that $1/p = \theta/q + (1-\theta)/2p$, and log-convexity of $L^p$-norms gives
  \begin{equation*}
    \| \phi_N \|_{L^p(\Omega;X)}
    \leq \| \phi_N \|_{L^q(\Omega;X)}^{\theta} \| \phi_N \|_{L^{2p}(\Omega;X)}^{1 - \theta}
    \leq \| \phi_N \|_{L^q(\Omega;X)}^{\theta} (c_{2p,p} \| \phi_N \|_{L^{p}(\Omega;X)})^{1 - \theta},
  \end{equation*}
  which yields
  \begin{equation*}
    \| \phi_N \|_{L^p(\Omega;X)} \leq c_{2p,p}^{(1 - \theta)/\theta} \| \phi_N \|_{L^q(\Omega;X)}.
  \end{equation*}
  Finally, if $q < p < 1$, we simply estimate
  \begin{equation*}
    \| \phi_N \|_{L^p(\Omega;X)} \leq \| \phi_N \|_{L^1(\Omega;X)} \leq c_{1,q} \| \phi_N \|_{L^q(\Omega;X)}.
  \end{equation*}
  This covers all nontrivial cases, and it remains to prove the claimed equality \eqref{eq:kk-claim}.

  First recall that
  \begin{equation*}
    \|\phi\|_{*,q} =  \sup_{\substack{k,n \in \N \\ k \leq n}} \sup_{\substack{F \in \mc{F}_k \\ \mu(F) > 0}} \Big( \fint_{F} \|(\phi_n - \phi_{k-1})(s)\|_{X}^{q} \, \dd\P(s)\Big)^{1/q}.
  \end{equation*}
  Fix $k \leq n$ and a set $F \in \mc{F}_k$ of positive measure.
  We will compute
  \begin{equation*}
    \fint_{F} \|\1_{F} (\phi_n - \phi_{k-1})(s)\|_{X}^q \, \dd\P(s) = \P(F)^{-1} \E \Big( \1_{F} \Big\| \sum_{j=k}^{n} \varepsilon_j \mb{x}_j\Big\|_{X}^q \Big).
  \end{equation*}
  Observe that for all $\omega \in \Omega$
  \begin{equation*}
    \Big\| \sum_{j=k}^{n} \varepsilon_j(\omega) \mb{x}_j\Big\|_{X} = \Big\| \varepsilon_k(\omega) \Big( \mb{x}_k + \sum_{j=k+1}^{n} \varepsilon_{j}^{\prime}(\omega) \mb{x}_j \Big) \Big\|_{X} = \Big\|\mb{x}_k + \sum_{j=k+1}^{n} \varepsilon_{j}^{\prime} \mb{x}_j \Big\|_{X}
  \end{equation*}
  where
  \begin{equation*}
    \varepsilon_{j}^{\prime} :=
    \begin{cases}
      \varepsilon_{j} & j \leq k \\
      \varepsilon_{k} \varepsilon_{j} & k+1 \leq j,
    \end{cases}
  \end{equation*}
  since $\varepsilon_j$ is $\pm 1$-valued.
  Now note that the $\sigma$-algebra $\mc{F}^\prime_{k+1,n} := \sigma(\{\varepsilon_j^\prime : k+1 \leq j \leq n\})$ is independent of $\mc{F}_k$, since for all $1 \leq j \leq k$ and $k+1 \leq j' \leq n$ we have by independence of the original Rademacher sequence
  \begin{equation*}
    \E(\varepsilon_j \varepsilon_{j'}^\prime) = \E(\varepsilon_j \varepsilon_k \varepsilon_{j'})
    =
    \begin{cases}
      \E(\varepsilon_j) \E(\varepsilon_k) \E(\varepsilon_{j'}) & \text{if $j < k$} \\
       \E(\varepsilon_{j'}) & \text{if $j = k$}
    \end{cases}
    \quad = 0.
  \end{equation*}
  Thus, since $F \in \mc{F}_k$, we have by independence of $\mc{F}_k$ and $\mc{F}^{\prime}_{k+1, n}$
  \begin{equation*}
    \begin{aligned}
      \P(F)^{-1} \E \Big( \1_{F} \Big\| \sum_{j=k}^{n} \varepsilon_j \mb{x}_j\Big\|_{X}^q \Big)
      &=  \P(F)^{-1} \E \Big( \1_{F} \Big\|\mb{x}_k + \sum_{j=k+1}^{n} \varepsilon_{j}^{\prime} \mb{x}_j \Big\|_{X}^{q} \Big) \\
      &=  \E \Big\|\mb{x}_k + \sum_{j=k+1}^{n} \varepsilon_{j}^{\prime} \mb{x}_j \Big\|_{X}^{q} 
      =  \E  \Big\|\sum_{j=k}^{n} \varepsilon_{j} \mb{x}_j \Big\|_{X}^{q}. \\
    \end{aligned}
  \end{equation*}
  Now letting $\mc{F}_{k,n} := \sigma(\{\varepsilon_j : k \leq j \leq n\})$, we have by the $L^q$-contraction property of conditional expectations
  \begin{equation*}
    \E \Big\|\sum_{j=k}^{n} \varepsilon_{j} \mb{x}_j \Big\|_{X}^{q}
    = \E \Big\|\E^{\mc{F}_{k,n}} \Big(\sum_{j=1}^{N} \varepsilon_{j} \mb{x}_j \Big)\Big\|_{X}^{q}
    \leq \E \Big\|\sum_{j=1}^{N} \varepsilon_{j} \mb{x}_j \Big\|_{X}^{q}
  \end{equation*}
  with equality when $k=1$ and $n=N$.
  Thus
  \begin{equation*}
    \|\phi\|_{*,q} =  \sup_{\substack{k,n \in \N \\ k \leq n}}  \Big(\E  \Big\|\sum_{j=k}^{n} \varepsilon_{j} \mb{x}_j \Big\|_{X}^{q}\Big)^{1/q} = \Big(\E  \Big\|\sum_{j=1}^{N} \varepsilon_{j} \mb{x}_j \Big\|_{X}^{q}\Big)^{1/q}
  \end{equation*}
  which proves the claimed equality \eqref{eq:kk-claim} and completes the proof. %mk
\end{proof}

In the setting of Hilbert-valued functions (and in particular, scalar-valued functions), the Kahane--Khintchine inequality leads to the classical Khintchine inequalities.

\begin{cor}[Khintchine's inequalities]
  Let $H$ be a Hilbert space, and let $(\varepsilon_{n})_{n \in \N}$ be a Rademacher sequence on a probability space $(\Omega, \mc{F}, \P)$.
  Then for all $p \in (0,\infty)$ there exist finite constants $A_p$ and $B_p$ such that for all finite sequences $(\mb{h}_n)_{n=1}^N$ in $H$,
  \begin{equation*}
    A_p \Big( \sum_{n=1}^N \|\mb{h}_n\|_H^2 \Big)^{1/2} \leq \Big\| \sum_{n=1}^N \varepsilon_n \mb{h}_n \Big\|_{L^p(\Omega;H)} \leq B_p \Big( \sum_{n=1}^N \|\mb{h}_n\|_H^2 \Big)^{1/2}.
  \end{equation*}
\end{cor}

\begin{proof}
  By independence of the Rademacher variables we have
  \begin{equation}
    \begin{aligned}
      \Big\| \sum_{n=1}^N \varepsilon_n \mb{h}_n \Big\|_{L^2(\Omega;H)}^2
      &= \Big\langle \sum_{n=1}^N \varepsilon_n \mb{h}_n , \sum_{m=1}^N \varepsilon_m \mb{h}_m \Big\rangle \\
      &= \sum_{n,m = 1}^N \E(\varepsilon_n \varepsilon_m) \langle \mb{h}_n, \mb{h}_m \rangle 
      = \sum_{n=1}^N \|\mb{h}_{n}\|_{H}^2,
    \end{aligned}
  \end{equation}
  so the result is true for $p=2$ with $A_2 = B_2 = 1$.
  Now use Kahane--Khintchine to extend the result to general $p \in (0,\infty)$.
\end{proof}

\subsubsection{Proof of the John--Nirenberg inequality}

We return to our analysis of a sequence $\phi = (\phi_n)_{n \in \N}$ of $X$-valued functions adapted to a filtration $(\mc{F}_n)_{n \in \N}$ on a $\sigma$-finite measure space $(S,\mc{A},\mu)$.
We prove the John--Nirenberg inequality via a series of lemmas in which we obtain increasingly fine control on the oscillation of $\phi$.

\begin{lem}\label{lem:JN-proof-1}
  For all $k \leq n$, $F \in \mc{F}_k$, and $\alpha > 0$,
  \begin{equation}\label{eq:JN-proof-1-est}
    \mu(F \cap \{ \|\phi_n - \phi_{k-1}\|_{X} > \alpha \}) \leq \Big( \frac{\|\phi\|_{*,q}}{\alpha} \Big)^{q} \mu(F).
  \end{equation}
\end{lem}

\begin{proof}
  We can assume that $0 < \mu(F) < \infty$, otherwise there is nothing to prove.
  The left hand side of \eqref{eq:JN-proof-1-est} is bounded by
  \begin{equation*}
    \int_{F} \Big( \frac{\|\phi_n - \phi_{k-1}\|_{X}}{\alpha} \Big)^{q} \, \dd\mu \leq \mu(F) \Big( \frac{\|\phi\|_{*,q}}{\alpha} \Big)^{q}
  \end{equation*}
  since $F \in \mc{F}_k$ and $k \leq n$, by the definition of $\|\phi\|_{*,q}$.
\end{proof}

Next we show that oscillation control of the form above extends to more general stopping times.

\begin{lem}
  Suppose that there exist $\alpha > 0$ and $\eta > 0$ such that
  \begin{equation*}
    \mu(F \cap \{ \| \phi_n - \phi_{k-1} \| > \alpha \} ) \leq \eta \mu(F) \qquad \forall k \leq n, F \in \mc{F}_k.
  \end{equation*}
  Then for all $k \in \N$, $F \in \mc{F}_k$, and all stopping times $\nu$ such that $\nu \geq k$ on $F$,
  \begin{equation}\label{eq:jn-stoptime}
    \mu(F \cap \{\nu < \infty\} \cap \{ \| \phi_{\nu} - \phi_{k-1} \| > 2\alpha \} ) \leq 2\eta \mu(F).
  \end{equation}
\end{lem}

\begin{proof}
  Sum over all possible values of the stopping time:
  \begin{equation*}
      \mu(F \cap \{\nu < \infty\} \cap \{ \| \phi_{\nu} - \phi_{k-1} \| > 2\alpha \} ) 
      = \lim_{N \to \infty} \sum_{n = k}^{N} \mu(F_n \cap \{ \| \phi_{n} - \phi_{k-1} \| > 2\alpha \} ),
  \end{equation*}
  where $F_{n} := F \cap \{\nu = n\}$.
  For fixed $N \geq n > k$, since $F_n \in \mc{F}_n \subset \mc{F}_{n+1}$, we have by assumption
  \begin{equation*}
    \begin{aligned}
      &\mu(F_n \cap \{ \| \phi_{n} - \phi_{k-1} \| > 2\alpha \} ) \\
      &\leq \mu(F_n \cap \{ \| \phi_{n} - \phi_{N} \| > \alpha \} )
      + \mu(F_n \cap \{ \| \phi_{k-1} - \phi_{N} \| > \alpha \} ) \\
      &\leq \eta \mu(F_n) + \mu(F_n \cap \{ \| \phi_{k-1} - \phi_{N} \| > \alpha \} ).
    \end{aligned}
  \end{equation*}
  Thus we have
  \begin{equation*}
    \begin{aligned}
      &\lim_{N \to \infty} \sum_{n = k}^{N} \mu(F_n \cap \{ \| \phi_{n} - \phi_{k-1} \| > 2\alpha \} ) \\
      &\leq \lim_{N \to \infty} \Big( \eta \sum_{n=k}^N \mu(F_n) + \sum_{n=k}^N \mu(F_n \cap \{ \| \phi_{k-1} - \phi_{N} \| > \alpha \}) \Big) \\
      &\leq \eta \mu(F) + \lim_{N \to \infty} \mu(F \cap \{ \| \phi_{k-1} - \phi_{N} \| > \alpha \})  
      \leq 2\eta\mu(F)
    \end{aligned}
  \end{equation*}
  using the assumption and $F \in \mc{F}_k$ in the last estimate.
\end{proof}
  
In the following lemma we make use of the \emph{started sequence}
\begin{equation*}
  {}^{k-1}\phi = (\phi_n - \phi_{k-1})_{n \geq k-1}
\end{equation*}
and its maximal function
\begin{equation*}
  ({}^{k-1} \phi)^{*}(s) := \sup_{n \geq k-1} \|({}^{k-1}\phi)_n(s)\|_X = \sup_{n \geq k} \|(\phi_n - \phi_{k-1})(s)\|_X.
\end{equation*}

\begin{lem}\label{lem:jn-mf}
  Suppose that $\phi$ satisfies \eqref{eq:jn-stoptime} for all $k \in \N$, $F \in \mc{F}_k$, and all stopping times $\nu$ such that $\nu \geq k$ on $F$.
  Then for all $\lambda > 0$,
  \begin{equation}\label{eq:jn-mf-eq}
    \mu(F \cap \{ ({}^{k-1} \phi)^{*} > \lambda + 2\alpha \}) \leq 2\eta \mu(F \cap \{  ({}^{k-1} \phi)^{*} > \lambda \}) \qquad \forall k \in \N, F \in \mc{F}_{k}.
  \end{equation}
\end{lem}

\begin{proof}
  Fix $k \in \N$ and consider the stopping times
  \begin{equation*}
    \begin{aligned}
      \rho &:= \inf\{n \geq k : \|\phi_n - \phi_{k-1}\| > \lambda\}, \\
      \nu &:= \inf\{n \geq k : \|\phi_n - \phi_{k-1}\| > \lambda + 2\alpha\}.
    \end{aligned}
  \end{equation*}
  Then $k \leq \rho \leq \nu$, and \eqref{eq:jn-mf-eq} can be rewritten as
  \begin{equation*}
    \mu(F \cap \{\nu < \infty\}) \leq 2\eta \mu(F \cap \{\rho < \infty\}).
  \end{equation*}
  Now fix $n \geq k$ and let $F_n := F \cap \{\rho = n\} \in \mc{F}_n$.
  On $\{F_n \cap \{\nu < \infty\}\}$ we have
  \begin{equation*}
    \|\phi_{\nu} - \phi_{n-1}\| \geq \|\phi_{\nu} - \phi_{k-1}\| - \|\phi_{n-1} - \phi_{k-1}\| > (\lambda + 2\alpha) - \lambda = 2\alpha,
  \end{equation*}
  so
  \begin{equation*}
    \mu(F_n \cap \{\nu < \infty\}) = \mu(F_n \cap \{\nu < \infty\} \cap \{\|\phi_{\nu} - \phi_{n-1}\| > 2\alpha\} )
    \leq 2\eta \mu(F_n).
  \end{equation*}
  Summing over $n \geq k$ completes the proof.
\end{proof}

\begin{lem}
  Suppose that $f$ is a non-negative function supported in $F \in \mc{A}$, satisfying
  \begin{equation*}
    \mu(f > \lambda + \alpha) \leq \eta \mu(f > \lambda) \qquad \forall \lambda > 0
  \end{equation*}
  for some $\eta \in (0,1)$ and $\alpha > 0$.
  Then for all $p \in [1,\infty)$,
  \begin{equation*}
    \|f\|_p \leq \frac{1 + \eta^{1/p}}{1 - \eta^{1/p}} \alpha \mu(F)^{1/p}.
  \end{equation*}
\end{lem}

\begin{proof}
  {\color{red} WRITE PROOF}
\end{proof}

\todo{UP TO HERE. HAVE TO COMPLETE PROOF OF JOHN-NIRENBERG.}

{\color{blue}

\begin{equation*}
  \|\phi\|_{**,q} := \sup_{k \in \N} \sup_{\substack{F \in \mc{F}_k \\ 0 < \mu(F) < \infty}} \Big( \fint_{F} ({}^{k-1} \phi)^{*}(s)^q \, \dd\mu(s) \Big)^{1/q},
\end{equation*}

}






\subsection{Gundy's decomposition}


%%% Local Variables:
%%% mode: latex
%%% TeX-master: "../main.tex"
%%% End:


\section{The Radon--Nikodym property}
\label{sec:RNP}
We now move from Banach-valued analysis and probability to Banach-valued \emph{measure theory}, and finally to the \emph{geometry} of Banach spaces.
We will tie these concepts together via the Radon--Nikodym property, which is ostensibly a measure-theoretic property but has equivalent characterisations in terms of Bochner spaces, martingales, and convex sets.

\section{Vector measures and the Radon--Nikodym property}

\begin{defn}\index{vector measures}
  Let $X$ be a Banach space and $(S,\mc{A})$ a measurable space.
  An $X$-valued \emph{vector measure} is a function $\map{\mb{\mu}}{\mc{A}}{X}$ which is countably additive, in the sense that for all sequences $(E_n)_{n \in \N}$ of pairwise disjoint sets in $\mc{A}$,
  \begin{equation*}
    \mb{\mu}\Big( \bigcup_{n \in \N} E_n \Big) = \sum_{n \in \N} \mb{\mu}(E_n).
  \end{equation*}
  Note that this condition includes the convergence in $X$ of the series on the right hand side.
\end{defn}

Vector measures are just like measures, except the measure of a set $E \subset S$ is a vector $\mb{\mu}(E) \in X$ rather than a scalar.
We are most interested in vector measures with the following boundedness condition.

\begin{defn}\index{vector measures!variation}
  Let $X$ be a Banach space and $\mb{\mu}$ an $X$-valued vector measure on a measurable space $(S,\mc{A})$.
  The \emph{variation} of $\mb{\mu}$ is the scalar-valued measure $\map{|\mb{\mu}|}{\mc{A}}{[0,\infty]}$ defined by
  \begin{equation*}
    |\mb{\mu}|(E) := \sup_{\pi} \sum_{A \in \pi} \|\mb{\mu}(A)\|_X,
  \end{equation*}
  where the supremum ranges over all partitions $\pi$ of $S$ into $\mc{A}$-measurable sets.
  We define the \emph{total variation norm} $\|\mb{\mu}\|_{\var} := |\mb{\mu}|(S)$, and we say that $\mu$ has \emph{bounded variation} if $\|\mb{\mu}\|_{\var} < \infty$.
  Equivalently, $\mu$ has bounded variation if there exists a finite scalar-valued measure $\nu$ on $\mc{A}$ such that $\|\mb{\mu}(A)\|_X \leq \nu(A)$ for all $A \in \mc{A}$ (the minimal measure with this property is $|\mb{\mu}|$).
  We let $M(S,\mc{A};X)$ denote the Banach space of all $X$-valued vector measures $\mb{\nu}$ on $\mc{A}$ with bounded variation, under the total variation norm.
\end{defn}

It is not particularly difficult to define integrals of scalar-valued functions with respect to vector measures.

\begin{prop}
  Let $X$ be a Banach space and $\mb{\mu}$ an $X$-valued vector measure of bounded variation on a measurable space $(S,\mc{A})$.
  Then there is a unique continuous linear map $\map{[\mb{\mu}]}{L^1(S,\mc{A},|\mb{\mu}|)}{X}$ such that $[\mb{\mu}](\1_{A}) = \mb{\mu}(A)$ for all $A \in \mc{A}$.
  We use integral notation to denote this map, i.e. we write
  \begin{equation*}
    \int_{S} f(s) \, \dd\mb{\mu}(s) := [\mb{\mu}](f) \qquad \forall f \in L^1(S,\mc{A},|\mb{\mu}|).
  \end{equation*}
\end{prop}

\begin{proof}
  We skip the verification that the definition $[\mb{\mu}](\1_{A}) := \mb{\mu}(A)$ extends by linearity to a well-defined map on integrable simple functions.\footnote{``It is dreadfully boring to show that this formula defines a linear map... from the space of simple functions of the above form into $X$ and we leave this as an exercise for masochists.'' \cite[pp5-6]{DU77}}
  We just need to show boundedness, and the conclusion will follow by density.
  Consider a simple function $g \in L^1(S,\mc{A},|\mb{\mu}|)$ of the form
  \begin{equation*}
    g = \sum_{n=1}^{N} c_n \1_{S_n} 
  \end{equation*}
  with scalars $c_n \in \K$.
  Then
  \begin{equation*}
    \begin{aligned}
      \|[\mb{\mu}](g)\|_X \leq \sum_{n=1}^{N} |c_n| \|\mb{\mu}(S_n)\|_X \leq \sum_{n=1}^{N} |c_n| |\mb{\mu}|(S_n) = \|g\|_{L^1(|\mb{\mu}|)}.
    \end{aligned}
  \end{equation*}
  That's all.
\end{proof}

Fundamental examples of vector measures are given by integrating vector-valued functions against scalar measures.

\begin{example}\label{eg:RN-density}
  Let $(S,\mc{A})$ be a measurable space and $X$ a Banach space.
  Suppose $\nu$ is a finite scalar-valued measure on $(S,\mc{A})$ and $\mb{f} \in L^1(S,\mc{A},\nu;X)$.
  Then we can define an $X$-valued vector measure $\mb{\mu}$ (sometimes denoted $\mb{f}\nu$) by Bochner integration:
  \begin{equation*}
    \mb{\mu}(A) = \int_A \mb{f} \, \dd\nu.
  \end{equation*}
  This vector measure has bounded variation: given a partition $S = \bigcup_{n \in \N} S_n$, we compute
  \begin{equation*}
    \sum_{n \in \N} \|\mb{\mu}(S_n)\|_{X} = \sum_{n \in \N} \Big\| \int_{S_n} \mb{f} \, \dd\nu \Big\|_X
    \leq \int_{S} \|\mb{f}\|_{X} \, \dd\nu
  \end{equation*}
  so that $\|\mb{\mu}\|_{\var} \leq \|\mb{f}\|_{L^1(\nu;X)}$.\footnote{In fact, this is an equality. See \cite[pp43]{gP16}.}
\end{example}

Now let's revise some measure theory.
Recall that if $\mu$ and $\nu$ are two scalar-valued signed measures on a measurable space $(S,\mc{A})$, then \emph{$\nu$ is absolutely continuous with respect to $\mu$},\index{absolute continuity} written $\nu \ll \mu$, if $A \in \mc{A}$ and $\mu(A) = 0$ implies $\nu(A) = 0$.

\begin{thm}[Radon--Nikodym]\index{theorem!Radon--Nikodym}
  Let $(S,\mc{A})$ be a measurable space, and let $\mu$ be a $\sigma$-finite measure on $\mc{A}$.
  Let $\nu$ be a finite signed measure on $\mc{A}$ such that $\nu \ll \mu$.
  Then there exists a unique $h \in L^1(\mu)$ such that
  \begin{equation*}
    \nu(A) = \int_{A} h(s) \, \dd\mu(s) \qquad \forall A \in \mc{A}.
  \end{equation*}
  The function $h$ is called the \emph{Radon--Nikodym derivative}\index{Radon--Nikodym derivative} of $\nu$ with respect to $\mu$, and denoted by
  \begin{equation*}
    h = \frac{\dd\nu}{\dd\mu}.
  \end{equation*}
\end{thm}

See \cite[Theorem 5.5.4]{rD04} for a proof.
One might expect that an analogous theorem holds for vector measures, but it turns out to depend on the target Banach space, and thus its validity becomes a definition.

\begin{defn}\index{Radon--Nikodym property}\index{RNP|see {Radon--Nikodym property}}
  Let $(S,\mc{A},\mu)$ be a $\sigma$-finite measure space.
  A Banach space $X$ is said to have the \emph{Radon--Nikodym property (RNP) with respect to $(S,\mc{A},\mu)$} if for every $X$-valued vector measure $\mb{\nu}$ on $(S,\mc{A})$ such that $\|\mb{\nu}\|_{\var} < \infty$ and $|\mb{\nu}| \ll \mu$, there is a function $\mb{f} \in L^1(\mu;X)$ such that $\mb{\nu} = \mb{f}\mu$ (as defined in Example \ref{eg:RN-density}).
  We say $X$ has the \emph{Radon--Nikodym property} if it has the property above with respect to every $\sigma$-finite measure space $(S,\mc{A},\mu)$.
\end{defn}

The classical Radon--Nikodym theorem says that the scalar fields $\R$ and $\C$ have the RNP.
We will investigate this property for other Banach spaces by considering its relationship with martingales and with properties of convex sets.
We will also connect it with the duality of Bochner spaces $L^p(X)$, answering a question left open in Chapter \ref{sec:Bochner-spaces}.
Before moving on we record a simple reduction.

\begin{prop}\label{prop:RNP-finite-sufficient}
  A Banach space $X$ has the Radon--Nikodym property if and only if for every finite measure space $(S,\mc{A},\mu)$ and every $X$-valued vector measure $\mb{\nu}$ on $(S,\mc{A})$ such that $\|\mb{\nu}(A)\|_{X} \leq \mu(A)$ for every $A \in \mc{A}$, there is a function $\mb{f} \in L^1(\mu;X)$ such that $\mb{\nu} = \mb{f}\mu$.
\end{prop}

\begin{proof}
  If $X$ has the RNP and $\mu$ and $\mb{\nu}$ are as hypothesised, then we have in particular that
  \begin{equation*}
    \|\mb{\nu}\|_{\var} < \|\mu\|_{\var} = \mu(S) < \infty
  \end{equation*}
  and $|\mb{\nu}| \ll \mu$, so the RNP gives us the required function $\mb{f} \in L^1(\mu;X)$.

  Conversely, suppose that $X$ has the hypothesised property, and let $(S,\mc{A},\mu)$ be a $\sigma$-finite measure space.
  Let $\mb{\nu}$ be an $X$-valued vector measure of bounded variation on $(S,\mc{A})$ such that $|\mb{\nu}| \ll \mu$.
  The measure $|\mb{\nu}|$ is finite, and for all $A \in \mc{A}$ we have
  \begin{equation*}
    \|\mb{\nu}(A)\|_{X} \leq |\mb{\nu}|(A)
  \end{equation*}
  by definition.
  Thus by hypothesis there exists a function $\mb{f} \in L^1(|\mb{\nu}|;X)$ such that $\mb{\nu} = \mb{f}|\mb{\nu}|$.
  By the scalar Radon--Nikodym theorem, there also exists a function $g \in L^1(\mu)$ such that $|\mb{\nu}| = g\mu$.
  Since $|\mb{\nu}|$ is a non-negative measure, $g$ is non-negative.
  Thus we have $\mb{\nu} = (\mb{f}g)\mu$.
  It remains to show that $\mb{f}g \in L^1(\mu;X)$: this is established by
  \begin{equation*}
    \int_{S} \|\mb{f}g\|_{X} \, \dd\mu
    \leq \int_{S} \|\mb{f}(s)\|_{X} g(s) \, \dd\mu(s) = \|\mb{f}\|_{L^1(g\mu;X)} = \|\mb{f}\|_{L^1(|\mb{\nu}|)} < \infty. 
  \end{equation*}
\end{proof}

\section{The RNP and martingale convergence}

We establish the following connection between the Radon--Nikodym property and the martingale convergence properties.

\begin{thm}\label{thm:RNP-MCP}\index{Radon--Nikodym property!implies $1$-MCP}\index{Martingale Convergence Property!implied by RNP}
  Let $X$ be a Banach space which has the Radon--Nikodym property with respect to a probability space $(\Omega,\mc{A},\P)$.
  Then $X$ has the $1$-martingale convergence property with respect to $(\Omega,\mc{A},\P)$.
\end{thm}

Applications of the Radon--Nikodym property generally involve the construction of an appropriate vector measure, from which a magical function is extracted as a Radon--Nikodym derivative.
In the setting of Theorem \ref{thm:RNP-MCP}, we are given an $X$-valued martingale, and we construct its almost-everywhere limit as the Radon--Nikodym derivative of a certain vector measure.
In the following proposition we construct the vector measure.

\begin{prop}\label{prop:martingale-measure} 
  Let $(\Omega,\mc{A},\P)$ be a probability space and $X$ a Banach space.
  Let $\mb{f}_{\bullet}$ be an $X$-valued $L^1$-bounded uniformly integrable martingale with respect to a filtration $\mc{A}_{\bullet}$.
  Then there exists an $X$-valued vector measure $\mb{\mu}$ on $\mc{A}$ with the following properties:
  \begin{itemize}
  \item $\mb{\mu}(A) = \int_{A} \mb{f}_n \, \dd\P$ for all $n \in \N$ and $A \in \mc{A}_{n}$,
  \item $\|\mb{\mu}\|_{\var} \leq \sup_{n} \|\mb{f}_n\|_{L^1(\Omega;X)}$,
  \item $|\mb{\mu}|$ is absolutely continuous with respect to $\P$.
  \end{itemize}
\end{prop}

\begin{proof}
  For all $A \in \mc{A}$ we would like to define
  \begin{equation}\label{eq:mu-stationary-limit}
    \mb{\mu}(A) := \lim_{k \to \infty} \int_A \mb{f}_k \, \dd\P,
  \end{equation}
  but it is not immediate that this limit exists.
  If $n \in \N$ and $A \in \mc{A}_{n}$, then for $k \geq n$ we have
  \begin{equation*}
    \int_A \mb{f}_k \, \dd\P = \int_A \mb{f}_n \, \dd \P
  \end{equation*}
  by the martingale property, so at least for $A \in \mc{A}_{n}$ the limit exists and equals $\int_A \mb{f}_n \, \dd\P$, establishing the first desired property.
  For a general $A \in \mc{A}$, for each $k \in \N$ we have
  \begin{equation*}
     \int_A \mb{f}_k \, \dd\P = \E( \1_{A}\mb{f}_k) = \E(\E^{\mc{A}_k}( \1_{A} \mb{f}_k)) = \E(\E^{\mc{A}_k}(\1_{A}) \mb{f}_k ).
   \end{equation*}
   Thus to show that the limit in \eqref{eq:mu-stationary-limit} exists, we need to show that the sequence $(\E(\E^{\mc{A}_k}( \1_{A}) \mb{f}_k ))_{k \in \N}$ is Cauchy.
   Let $\phi_{k,\ell} = \E^{\mc{A}_k}(\1_{A}) - \E^{\mc{A}_\ell}( \1_{A})$.
   For $k < \ell$ we have, using conditional expectation magic,
   \begin{equation*}
     \begin{aligned}
       \E(\E^{\mc{A}_k}(\1_{A}) \mb{f}_k  ) - \E(\E^{\mc{A}_\ell}( \1_{A}) \mb{f}_\ell )
       &= \E \Big( \E^{\mc{A}_k}(\1_{A}) \mb{f}_k  - \E^{\mc{A}_\ell}( \1_{A}) \mb{f}_{\ell} \Big) \\
       &= \E \Big(  \E^{\mc{A}_{k}}(\E^{\mc{A}_k}(\1_{A})\mb{f}_\ell ) - \E^{\mc{A}_{k}}( \E^{\mc{A}_\ell}( \1_{A}) \mb{f}_{\ell} )\Big) \\
       &= \E \Big(  \E^{\mc{A}_k}(\1_{A}) \mb{f}_\ell  - \E^{\mc{A}_\ell}( \1_{A}) \mb{f}_{\ell}  \Big) 
       = \E ( \phi_{k,\ell} \mb{f}_\ell ).
     \end{aligned}
   \end{equation*}
   Thus we get
   \begin{equation*}
     \|\E(\E^{\mc{A}_k}(\1_{A}) \mb{f}_k ) - \E( \E^{\mc{A}_\ell}( \1_{A}) \mb{f}_\ell)\|_{X}
     \leq \|\phi_{k,\ell} \mb{f}_\ell \|_{L^1(\Omega;X)}.
   \end{equation*}
   Since $\|\phi_{k,\ell}\|_\infty \leq 2$ for all $k$ and $\ell$, for all $t > 0$ we have
   \begin{equation*}
     \begin{aligned}
       \|  \phi_{k,\ell}  \mb{f}_\ell \|_{L^1(\Omega;X)}
       &\leq \Big( \int_{\|\mb{f}_{\ell}\|_{X} > t} + \int_{\|\mb{f}_{\ell}\|_{X} \leq t} \Big) |\phi_{k,\ell}(\omega)| \|\mb{f}_{\ell}(\omega)\|_{X} \, \dd\P(\omega) \\
       &\leq  2\int_{\|\mb{f}_{\ell}\|_{X} > t} \|\mb{f}_{\ell}(\omega)\|_{X} \dd\P(\omega) + t\E|\phi_{k,\ell}| 
     \end{aligned}
   \end{equation*}
   so that
   \begin{equation*}
     \limsup_{k,\ell \to \infty} \|  \phi_{k,\ell} \mb{f}_\ell  \|_{L^1(\Omega;X)}
     \leq 2\limsup_{\ell \to \infty} \int_{\|\mb{f}_{\ell}\|_{X} > t} \|\mb{f}_{\ell}(\omega)\|_{X} \dd\P(\omega)
   \end{equation*}
   for all $t > 0$, using that $\limsup_{k,\ell \to \infty} \|\phi_{k,\ell}\|_{1} = 0$ (i.e. $(\E^{\mc{A}_{k}} \1_{A})_{k \in \N}$ is convergent in $L^1$).
   Taking the limit as $t \to \infty$, uniform integrability of $\mb{f}_{\bullet}$ says that
   \begin{equation*}
     \limsup_{k,\ell \to \infty} \|  \phi_{k,\ell} \mb{f}_\ell  \|_{L^1(\Omega;X)}
     \leq 2\lim_{t \to \infty} \sup_{\ell \in \N} \int_{\|\mb{f}_{\ell}\|_{X} > t} \|\mb{f}_{\ell}(\omega)\|_{X} \dd\P(\omega) = 0
   \end{equation*}
   (see Exercise \ref{ex:UI-characterisation}), which establishes that the limit in \eqref{eq:mu-stationary-limit} exists.

 We still need to show that $\mb{\mu}$ is actually a vector measure.
 It is clear from the definition that it is finitely additive, but we need \emph{countable} additivity.
 Consider the submartingale $(\|\mb{f}_n\|_{X})_{n \in \N}$: this is $L^1$-bounded and uniformly integrable, so by Theorem \ref{thm:submartingale-convergence} it has an $L^1$-limit $g \in L^1(\Omega)$.
 Thus for all $A \in \mc{A}$
 \begin{equation*}
   \|\mb{\mu}(A)\|_{X} \leq \lim_{n \to \infty} \int_{A} \|\mb{f}_n(\omega)\| \, \dd\P(\omega) = \int_{A} g(\omega) \, \dd\P(\omega).
 \end{equation*}
 By Exercise \ref{ex:fa-meas-ca}, this implies that $\mb{\mu}$ is countably additive with
 \begin{equation*}
   \|\mb{\mu}\|_{\var} \leq \|g\P\|_{\var} = \|g\|_{L^1(\Omega)} = \sup_{n \in \N} \|\mb{f}_n\|_{L^1(\Omega;X)},
 \end{equation*}
 and that
 \begin{equation*}
   |\mb{\mu}| \ll g\P \ll \P,
 \end{equation*}
 as required.
\end{proof}


\begin{proof}[Proof of Theorem \ref{thm:RNP-MCP}: RNP implies $1$-MCP]
  Let $\mb{f}_{\bullet}$ be an $L^1$-bounded uniformly integrable $X$-valued martingale with respect to a filtration $\mc{A}_{\bullet}$.
  By Proposition \ref{prop:martingale-measure}, there exists an $X$-valued vector measure $\mb{\mu}$ on $\mc{A}$ of bounded variation such that
  \begin{equation*}
    \mb{\mu}(A) = \int_{A} \mb{f}_n \, \dd\P \qquad \forall A \in \mc{A}_n
  \end{equation*}
  and $\mb{\mu} \ll \P$.
  Since $X$ has the RNP with respect to $(\Omega,\mc{A},\P)$, there exists a function $\mb{f} \in L^1(\Omega;X)$ such that
  \begin{equation*}
    \int_{A} \mb{f} \, \dd\P = \mb{\mu}(A) = \int_A \mb{f}_n \, \dd\P
  \end{equation*}
  for all $A \in \mc{A}_n$.
  Equivalently stated, we have
  \begin{equation*}
    \E^{\mc{A}_n} \mb{f} = \mb{f}_n
  \end{equation*}
  for all $n \in \N$, and thus by Theorem \ref{thm:mgale-pw-conv} $\mb{f}_n$ is almost everywhere convergent to $\E^{\mc{A}_{\infty}}\mb{f}$.
  Thus $X$ has the $1$-martingale convergence property, and the proof is complete.
\end{proof}

We already have some examples of spaces which do not satisfy the $1$-MCP: it follows that these spaces cannot have the RNP either.

\begin{cor}\index{Radon--Nikodym property!failure of $c_{0}$ and $L^1$}
  The spaces $c_0$ and $L^1([0,1])$ do not have the RNP.
\end{cor}

\begin{proof}
  In Examples \ref{eg:c0-noMCP} and \ref{eg:L1-noMCP} we showed that these spaces do not have the $\infty$-MCP, and hence they do not have the $1$-MCP. 
\end{proof} 

In summary, for all $p \in (1,\infty]$ we currently have the implications
\begin{equation*}
  \mathrm{RNP} \Longrightarrow 1-\mathrm{MCP} \Longrightarrow p-\mathrm{MCP} \Longrightarrow \infty-\mathrm{MCP}
\end{equation*}
where these properties are taken either universally or with respect to a given probability space.
In the next section we will add more properties and `complete the loop'.

\section{Trees and dentability}

To connect the Radon--Nikodym property with the geometry of Banach spaces, we will use the notion of a \emph{separated tree}.

\begin{defn}
  Let $(\Omega,\mc{A},\P)$ be a probability space and $X$ a Banach space.
  Given $\delta > 0$, an $X$-valued $L^1$-bounded martingale $\mb{f}_\bullet$ on $\Omega$ is called \emph{$\delta$-separated}\index{martingales!$\delta$-separated} if the following properties hold:
  \begin{itemize}
  \item $\mb{f}_0$ is constant,
  \item each $\mb{f}_n$ has finitely many values (i.e. $\mb{f}_{n}$ is simple),
  \item for all $n \in \N$ and $\omega \in \Omega$, $\|\mb{f}_n(\omega) - \mb{f}_{n+1}(\omega)\|_X \geq \delta$.
  \end{itemize}
  The set of values $S := \{\mb{f}_n(\omega) : n \in \N, \omega \in \Omega\}$ is called a \emph{$\delta$-separated tree}.  \index{separated trees}
\end{defn}

The martingales described in Examples \ref{eg:c0-noMCP} and \ref{eg:L1-noMCP} (see also Exercise \ref{ex:L1-noMCP-var}) are $1$-separated, and thus yield $1$-separated trees in $c_0$ and $L_1$.
But when one tries to draw a $\delta$-separated tree on a piece of paper, one quickly starts to run out of space.
This is because pieces of paper model finite dimensional Banach spaces, which have good martingale convergence properties.

\begin{prop}\index{separated trees!relation with MCP} 
  If a Banach space $X$ has the $\infty$-MCP, then for all $\delta > 0$, $X$ does not contain a bounded $\delta$-separated tree.
\end{prop}

\begin{proof}
  Bounded $\delta$-separated trees correspond to $L^\infty$-bounded martingales $\mb{f}_{\bullet}$ such that
  \begin{equation*}
    \|\mb{f}_n(\omega) - \mb{f}_{n+1}(\omega)\|_{X} \geq \delta
  \end{equation*}
  for all $\omega \in \Omega$ and $n \in \N$,
  which directly obstructs convergence of $\mb{f}_\bullet$ everywhere in $\Omega$.
\end{proof}

Thus our chain of implications now has a new member:
\begin{equation*}
  \mathrm{RNP} \Longrightarrow 1-\mathrm{MCP} \Longrightarrow p-\mathrm{MCP} \Longrightarrow \infty-\mathrm{MCP}
  \Longrightarrow \mathrm{NBST}
\end{equation*}
where NBST stands for `no bounded separated trees'.\footnote{As with MCP, this is not standard terminology: ultimately it's just equivalent to RNP.}
The connection between (nonexistence of) bounded separated trees and the Radon--Nikodym property will go through the concept of \emph{dentable sets}.

\begin{defn}\index{dentable sets}
  A subset $D \subset X$ of a Banach space $X$ is called \emph{dentable} if for all $\varepsilon > 0$ there exists $\mb{x} \in D$ such that
  \begin{equation*}
    \mb{x} \notin \overline{\conv}(D \sm B_{\varepsilon}(\mb{x}))
  \end{equation*}
  where $\overline{\conv}$ denotes the closure of the convex hull.
\end{defn}

An example of a dentable set in $\R^{2}$ is shown in Figure \ref{fig:dentable}.
It is impossible to draw a bounded non-dentable set due to Theorem \ref{thm:dent-RNP}. 

\begin{figure}
    \centering
    \def\svgwidth{\columnwidth}
    \input{dentable-set.pdf_tex}
    
    \caption{A dentable set $D$ in $\R^{2}$, being `dented' at the point $\mb{x} \in D$.}
    \label{fig:dentable}
\end{figure}

Before going further we'll need a lemma which relates non-dentability at scale $\varepsilon$ to a corresponding property of an enlarged set which does not involve closures.
This will let us work directly with convex hulls rather than their closures.

\begin{lem}\label{lem:dentlem}
  Let $X$ be a Banach space.
  Fix $\varepsilon > 0$ and let $D \subset X$ be a subset such that for all $\mb{x} \in D$,
  \begin{equation}\label{eq:dentlem-ass}
    \mb{x} \in \overline{\conv}(D \sm B_{\varepsilon}(\mb{x})).
  \end{equation}
  Then for all $\mb{x} \in \tilde{D} := D + B_{\varepsilon/2}(\mb{0})$,
  \begin{equation}
    \mb{x} \in \conv(\tilde{D} \sm B_{\varepsilon/2}(\mb{x}))
  \end{equation}
  (note that no closure is taken here).
\end{lem}

\begin{proof}
  Fix $\mb{x} = \mb{x}' + \mb{y} \in \tilde{D}$, where $\mb{x}' \in D$ and $\|\mb{y}\|_{X} < \varepsilon/2$.
  Choose $\delta > 0$ so small that $\delta + \|\mb{y}\|_{X} < \varepsilon/2$.
  By \eqref{eq:dentlem-ass} we have $\mb{x}' \in \overline{\conv}(D \sm B_{\varepsilon}(\mb{x}'))$, so there exists $n \in \N$, scalars $\alpha_i \in [0,1]$ and vectors $\mb{x}_1,\ldots,\mb{x}_n \in D \sm B_{\varepsilon}(\mb{x}')$, $i = 1,\ldots,n$, with $\sum_{i} \alpha_{i} = 1$, such that
  \begin{equation*}
    \mb{x}' = \mb{z} + \sum_{i=1}^{n} \alpha_i \mb{x}_{i}
  \end{equation*}
  for some $\mb{z} \in B_{\delta}(\mb{0})$.
  We then have
  \begin{equation*}
    \mb{x} = \mb{z} + \mb{y} + \sum_{i=1}^{n} \alpha_i \mb{x}_{i} = \sum_{i=1}^{n}\alpha_i(\mb{z} + \mb{y} + \mb{x}_{i}).
  \end{equation*}
  The points $\mb{z} + \mb{y} + \mb{x}_{i}$ are in $\tilde{D} \sm B_{\varepsilon/2}(\mb{x})$: indeed, we have
  \begin{equation*}
    \|\mb{z} + \mb{y}\|_{X} \leq \delta + \|\mb{y}\|_{X} < \varepsilon/2
  \end{equation*}
  by the choice of $\delta$, and
  \begin{equation*}
    \|\mb{x} - (\mb{z} + \mb{y} + \mb{x}_i) \|_{X}
    = \|\mb{x}' - \mb{z} - \mb{x}_i\|_{X}
    \geq \|\mb{x}' - \mb{x}_i\|_{X} - \|\mb{z}\|_{X}
    > \varepsilon - \delta > \varepsilon/2,
  \end{equation*}
  finishing the job.
\end{proof}

The concept of dentability is connected with (non-existence of) bounded separated trees as follows.

\begin{thm}\index{dentable sets!relation with separated trees}\index{separated trees!relation with dentable sets}
  Let $X$ be a Banach space, and suppose that for all $\delta > 0$, $X$ does not contain a bounded $\delta$-separated tree.
  Then every bounded subset of $X$ is dentable.
\end{thm}
  
\begin{proof}
  We prove the contrapositive: we suppose that there exists a bounded non-dentable set $D \subset X$, and given $\delta > 0$ we will construct a bounded $\delta$-separated tree.
  Since $D$ is non-dentable, there exists $\varepsilon > 0$ such that for all $\mb{x} \in D$,
  \begin{equation*}
    \mb{x} \in \overline{\conv}(D \sm B_{2\varepsilon}(\mb{x})).
  \end{equation*}
  By Lemma \ref{lem:dentlem}, for all $\mb{x} \in \tilde{D} := D + B_{\varepsilon}(\mb{0})$,
  \begin{equation*}
    \mb{x} \in \conv(\tilde{D} \sm B_{\varepsilon}(\mb{x})).
  \end{equation*}
  We will construct a $\varepsilon$-separated tree in the bounded set $\tilde{D}$, and by rescaling this will yield the result.

  Let $(\Omega,\mc{A},\P)$ be the interval $[0,1]$ with Borel $\sigma$-algebra and Lebesgue measure.
  We construct a $\varepsilon$-separated martingale inductively.
  Let $\mb{x}_{0} \in \tilde{D}$ be arbitrary and $\mb{f}_0 \equiv \mb{x}_0$.
  Since $\mb{x}_0 \in \conv(\tilde{D} \sm B_{\varepsilon}(\mb{x}_0))$, there exist numbers $\alpha_1,\ldots, \alpha_n \in (0,1)$ summing to $1$ and vectors $\mb{x}_{1}, \ldots, \mb{x}_{n} \in \tilde{D}$ such that
  \begin{equation*}
    \mb{x}_0 = \sum_{i=1}^{n} \alpha_{i} \mb{x}_{i} \qquad \text{and} \qquad \|\mb{x}_{i} - \mb{x}_{0}\|_X \geq \varepsilon.
  \end{equation*}
  Partition the unit interval $[0,1]$ into intervals $(I_i)_{i=1}^{n}$ with length $|I_{i}| = \alpha_{i}$, let $\mc{A}_{0}$ be the trivial $\sigma$-algebra on $[0,1]$, and let $\mc{A}_{1}$ be the $\sigma$-algebra generated by the intervals $(I_i)_{i=1}^{n}$.
  Define
  \begin{equation*}
    \mb{f}_1 = \sum_{i=1}^{n} \1_{I_i} \otimes \mb{x}_{i}.
  \end{equation*}
  Then $\E^{\mc{A}_0} \mb{f}_1 = \mb{f}_0$, and $\|\mb{f}_1(\omega) - \mb{f}_0(\omega)\|_{X} \geq \varepsilon$ for all $\omega \in [0,1]$.
  Since each point $\mb{x}_{i}$ is in $\tilde{D}$, we can repeat this process inductively, representing each $\mb{x}_{i}$ as a convex combination of vectors in $\tilde{D} \sm B_{\varepsilon}(\mb{x}_i)$, using these vectors to define $\mb{f}_2$ on $I_{i}$, and so on, to construct a $\varepsilon$-separated martingale valued in $\tilde{D}$, and thus a bounded $\varepsilon$-separated tree.
\end{proof}

Finally we will `complete the loop' in our discussion of the Radon--Nikodym property, martingale convergence, and dentability.

\begin{thm}\label{thm:dent-RNP}\index{Radon--Nikodym property!relation with dentable sets}\index{dentable sets!relation with RNP}
  Let $X$ be a Banach space such that every bounded subset of $X$ is dentable.
  Then $X$ has the Radon--Nikodym property.
\end{thm}

\begin{proof}
  Let $(S,\mc{A},\mu)$ be a finite measure space, and let $\map{\mb{\nu}}{\mc{A}}{X}$ be an $X$-valued vector measure with $\|\mb{\nu}(A)\|_{X} \leq \mu(A)$ for all $A \in \mc{A}$.
  Our task is to find a function $\mb{f} \in L^1(\mu;X)$ such that $\mb{\nu} = \mb{f}\mu$.
  By Proposition \ref{prop:RNP-finite-sufficient} this is sufficient to prove that $X$ has the RNP.

  For all sets $A \in \mc{A}$ let
  \begin{equation*}
    \begin{aligned}
      \mc{A}_{+}(A) &:= \{B \in \mc{A} : B \subset A, \mu(B) > 0\}, \\
      \mc{A}_{+} &:= \mc{A}_{+}(S)
    \end{aligned}
  \end{equation*}
  and for all $A \in \mc{A}_{+}$ define
  \begin{equation*}
    \mb{x}_{A} := \mu(A)^{-1} \mb{\nu}(A)\in X, \qquad 
    C_{A} := \{ \mb{x}_{B} : B \in \mc{A}_{+}(A)\} \subset X.
  \end{equation*}
  Note that $\|\mb{x}_{A}\|_X \leq 1$ for all $A \in \mc{A}_{+}$ by the assumptions on $\mb{\nu}$, so every $C_{A}$ is bounded and hence (by assumption) dentable.

  \textbf{We make the following claim:} \emph{for all $\varepsilon > 0$ and $A \in \mc{A}_{+}$, there exists $A' \in \mc{A}_+(A)$ such that $\diam(C_{A'}) \leq 2\varepsilon$.}
  
  We assume this is not the case and establish a contradiction.
  Thus there exist $\varepsilon > 0$ and $A \in \mc{A}_{+}$ such that $\diam(C_{A'}) > 2\varepsilon$ for all $A' \in \mc{A}_+(A)$.
  In particular, for every $\mb{x} \in X$ and $A' \in \mc{A}_{+}(A)$, there is a subset $B \in \mc{A}_{+}(A')$ such that $\|\mb{x} - \mb{x}_{B}\|_{X} > \varepsilon$.\footnote{Otherwise there would exist a vector $\tilde{\mb{x}} \in X$ with $\|\tilde{\mb{x}} - \mb{x}_{B}\|_{X} \leq \varepsilon$ for all $B \in \mc{A}_{+}(A')$, which implies $\|\mb{x}_{B} - \mb{x}_{B'}\|_{X} \leq 2\varepsilon$ for all $B,B' \in \mc{A}_+(A')$ and hence $\diam(C_{A'}) \leq 2\varepsilon$. Contradiction.}
  
  Now consider a fixed $A' \in \mc{A}_{+}(A)$ and let $\{B_{\lambda}\}_{\lambda \in \Lambda}$ be a maximal collection of disjoint elements of $\mc{A}_+(A')$ such that $\|\mb{x}_{A'} - \mb{x}_{B_\lambda}\|_{X} > \varepsilon$, where $\Lambda$ is some indexing set.
  Since the sets $B_{\lambda}$ are disjoint and have positive measure, and since
  \begin{equation*}
    \sum_{\lambda \in \Lambda} \mu(B_{\lambda}) \leq \mu(A') < \infty,
  \end{equation*}
  the indexing set $\Lambda$ is at most countable.
  By construction we must have
  \begin{equation}\label{eq:mu-sum}
    \mu\Big(A' \sm \bigcup_{\lambda \in \Lambda} B_\lambda\Big) = 0;
  \end{equation}
  otherwise we could find a set $B_{!} \in \mc{A}_{+}(A' \sm \cup_{\lambda} B_{\lambda}) \subset \mc{A}_{+}(A')$ such that $\|\mb{x}_{A'} - \mb{x}_{B_{!}}\|_{X} > \varepsilon$, contradicting the maximality of the set $
  \{B_{\lambda}\}$.
  The assumption on $\mb{\nu}$ yields
  \begin{equation*}
    \mb{\nu}\Big(A' \sm \bigcup_{\lambda \in \Lambda} B_\lambda\Big) = 0,
  \end{equation*}
  or equivalently (using countable additivity)
  \begin{equation*}
    \mb{\nu}(A') = \sum_{\lambda \in \Lambda} \mb{\nu}(B_\lambda).
  \end{equation*}
  This lets us write
  \begin{equation*}
      \mb{x}_{A'}
      = \mu(A')^{-1}\mb{\nu}(A') 
      = \sum_{\lambda \in \Lambda} \mu(A')^{-1}\mb{\nu}(B_\lambda) 
      = \sum_{\lambda \in \Lambda} \frac{\mu(B_\lambda)}{\mu(A')} \mb{x}_{B_\lambda}.
  \end{equation*}
  By \eqref{eq:mu-sum} we have that the coefficients of this series sum to $1$, and the vectors in the series satisfy
  \begin{equation*}
    \mb{x}_{B_{\lambda}} \in C_{A'} \qquad \text{and} \qquad \| \mb{x}_{A'}- \mb{x}_{B_\lambda} \|_X > \varepsilon,
  \end{equation*}
  which tells us that
  \begin{equation*}
    \mb{x}_{A'} \in \overline{\conv}(C_{A'} \sm B_{\varepsilon}(\mb{x}_{A'})).
  \end{equation*}
  Since this is true for all $A' \in \mc{A}_{+}(A)$, we find that $C_{A}$ is not dentable.
  This is a contradiction, which implies that our claim above is true.

  Now we return to the construction of a Radon--Nikodym derivative $\mb{f}$ of $\mb{\nu}$ with respect to $\mu$.
  Fix $\varepsilon > 0$.
  Using the claim we just established, let $(A_{\lambda})_{\lambda \in \Lambda}$ be a maximal disjoint collection of sets in $\mc{A}_+$ such that $\diam(C_{A_{\lambda}}) \leq 2\varepsilon$.
  Then $\Lambda$ is at most countable (by the same argument used in the last paragraph) and
  \begin{equation*}
    \mu(S \sm \bigcup_{\lambda \in \Lambda} A_{\lambda}) = 0;
  \end{equation*}
  if this were not the case then we could select $A' \in \mc{A}_+(S \sm \cup_{\lambda \in \Lambda} A_{\lambda}) \subset \mc{A}_+$ with $\diam(C_{A'}) \leq 2\varepsilon$ (using the claim) and contradict maximality.
  Define
  \begin{equation*}
    \mb{g}_{\varepsilon} := \sum_{\lambda \in \Lambda} \1_{A_{\lambda}} \otimes \mb{x}_{A_{\lambda}}.
  \end{equation*}
  Then $\mb{g}_{\varepsilon} \in L^1(\mu;X)$ (since it is bounded and the measure is finite).
  We will show that
  \begin{equation}\label{eq:g-computation}
    \|\mb{\nu} - \mb{g}_{\varepsilon}\mu\|_{\var} \leq 2\mu(S)\varepsilon;
  \end{equation}
  since this holds for all $\varepsilon > 0$, we find that $\mb{\nu}$ is in the closure in $M(S,\mc{A};X)$ of the set of measures of the form $\mb{g}\mu$ with $\mb{g} \in L^1(\mu;X)$.
  But this set is closed in $M(S,\mc{A};X)$ (Exercise \ref{ex:closure-M}), so there exists $\mb{g} \in L^1(\mu;X)$ with $\mb{\nu} = \mb{g}\mu$, as desired.

  It remains to show \eqref{eq:g-computation}.
  To see this first note that for all $A \in \mc{A}_+$
  \begin{equation*}
    \begin{aligned}
      \mb{\nu}(A) - \mb{g}_\varepsilon\mu(A)
      &= \sum_{\lambda \in \Lambda} \Big(  \mb{\nu}(A \cap A_{\lambda}) - \int_{A \cap A_{\lambda}} \mb{g}_{\varepsilon} \, \dd\mu \Big) \\
      &= \sum_{\lambda \in \Lambda} \Big(  \mb{\nu}(A \cap A_{\lambda}) - \mu(A \cap A_{\lambda}) \mb{x}_{A_{\lambda}} \Big) \\
      &= \sum_{\lambda \in \Lambda} \mu(A \cap A_{\lambda})(\mb{x}_{A \cap A_{\lambda}} - \mb{x}_{A_{\lambda}}),
    \end{aligned}
  \end{equation*}
  and so
  \begin{equation*}
    \|\mb{\nu}(A) - \mb{g}_\varepsilon\mu(A)\|_{X}
    \leq \sum_{\lambda \in \Lambda} \mu(A \cap A_{\lambda}) \|\mb{x}_{A \cap A_{\lambda}} - \mb{x}_{A_{\lambda}}\|_{X} \leq 2\mu(A)\varepsilon
  \end{equation*}
  using that $\diam(C_{A_{\lambda}}) \leq 2\varepsilon$.
  Taking the supremum over all partitions of $S$ proves \eqref{eq:g-computation} and completes the proof.
\end{proof}

Combining this with everything else we know, we have proven the following theorem.
\begin{thm}\label{thm:RNP-characterisations}\index{Radon--Nikodym property!equivalent characterisations}
  The following properties of a Banach space $X$ are equivalent:
  \begin{itemize}
  \item $X$ has the Radon--Nikodym property;
  \item $X$ has the $p$-martingale convergence property for all $p \in [1,\infty]$;
  \item $X$ has the $\infty$-martingale convergence property;
  \item for all $\delta > 0$, $X$ does not contain a bounded $\delta$-separated tree;
  \item every bounded subset of $X$ is dentable.
  \end{itemize}
\end{thm}

\begin{rmk}
  It is possible to prove directly that the $\infty$-MCP implies the RNP, but I don't think this is as nice as arguing via dentability.
  See \cite[Proof of Theorem 2.9]{gP16}.
\end{rmk}

This set of equivalences says quite a bit.
First, it says that almost everywhere convergence of $L^p$-bounded martingales holds for some $p \in [1,\infty]$ if and only if it holds for all $p \in [1,\infty]$.
This $p$-independence of martingale-based Banach space properties turns out to be fairly typical; martingales satisfy miraculous extrapolation properties of this kind.
Second, note that the first four properties are `extrinsic': the RNP makes reference to all $\sigma$-finite measure spaces, and the MCP properties and the nonexistence of bounded separated trees make reference to martingales valued in $X$.
In contrast, the last property is an intrinsic geometric property of $X$.
It is always satisfying to find an intrinsic geometric characterisation of what seems to be an extrinsic property.
One more point: by carefully looking at the proofs of these implications, one can show that it suffices to have the RNP with respect to the unit interval $[0,1]$ in order to show that every bounded subset $X$ is dentable.
This argument shows that a Banach space has the RNP if and only if it has the RNP with respect to the unit interval (see Exercise \ref{ex:RNP-unitinterval}).

Let's rattle off some consequences of this theorem.

\begin{cor}\index{Radon--Nikodym property!of reflexive spaces and separable duals}
  Reflexive spaces and separable dual spaces have the RNP.
\end{cor}

\begin{proof}
  By Theorem \ref{thm:MCP-sepdual} and Corollary \ref{cor:MCP-reflexive}, these spaces have the $\infty$-MCP.
  Thus they also have the RNP.
\end{proof}

\begin{cor}\index{Radon--Nikodym property!is separably determined}
  The RNP is separably determined, i.e. it holds for a Banach space $X$ if and only if it holds for all separable subspaces $Y \subset X$.
\end{cor}

\begin{proof}
  By Lemma \ref{lem:MCP-sepdet}, this is true for the $p$-MCP (for all $p \in [1,\infty]$, but these properties are now known to be equivalent anyway). 
\end{proof}

Finally, we return to Bochner spaces.
Recall Proposition \ref{prop:bochner-preduality}: for every $\sigma$-finite measure space $(S,\mc{A},\mu)$ and every Banach space $X$, for all $p \in [1,\infty]$, the map $\map{\Phi}{L^{p'}(S;X^*)}{L^p(S;X)^*}$ given by
\begin{equation*}
  \Phi \mb{g}(\mb{f}) = \int_{S} \langle \mb{f}(s), \mb{g}(s) \rangle \, \dd\mu(s) \qquad \forall \mb{f} \in L^p(S;X)
\end{equation*}
is an isometry onto a closed norming subspace of $L^p(S;X)^*$.
We will now complete this result with the help of the Radon--Nikodym property.

\begin{thm}\label{thm:bochner-duality-RNP}\index{Radon--Nikodym property!and duality of Bochner spaces}
  Let $X$ be a Banach space, and let $(S,\mc{A},\mu)$ be a countably generated measure space.\footnote{This says that there is a countable (or finite) collection of sets $(E_{\lambda})_{\lambda \in \Lambda}$ which generates $\mc{A}$.}
  Then the dual space $X^*$ has the Radon--Nikodym property with respect to $(S,\mc{A},\mu)$ if and only if for all $p \in [1,\infty)$ the isometric embedding $\map{\Phi}{L^{p'}(\mu;X^*)}{L^p(\mu;X)^*}$ is an isomorphism.
\end{thm}

We leave the proof as an exercise (Exercise \ref{ex:RNP-bochner}). 

  
\begin{rmk}\index{Radon--Nikodym property!equivalent characterisations}
  Not satisfied? See \cite[\textsection VII.6]{DU77} for 29 characterisations of the Radon--Nikodym property.
\end{rmk}


  


\section*{Exercises}

\begin{exercise}
  Let $X$ be a Banach space with the Radon--Nikodym property, and suppose that $Y$ is a Banach space which is isomorphic to $X$.
  Show that $Y$ also has the Radon--Nikodym property.
\end{exercise}

\begin{exercise}
  Let $X$ be a Banach space with the Radon--Nikodym property.
  Show that every bounded linear operator $\map{T}{L^1([0,1])}{X}$ is of the form
  \begin{equation*}
    Tf = \int_0^1 f(t) \mb{g}(t) \, \dd t
  \end{equation*}
  for some $\mb{g} \in L^1([0,1];X)$.
\end{exercise}

\begin{exercise}\index{conditional expectations!existence}
  Let $X$ be a Banach space with the Radon--Nikodym property.
  Use this property to show that for every measure space $(S,\mc{A},\mu)$ and every sub-$\sigma$-algebra $\mc{B} \subset \mc{A}$, every $\mb{f} \in L^1(\mc{A};X)$ has a conditional expectation with respect to $\mc{B}$.
  (Of course this holds for \emph{every} Banach space, but your job here is to derive it in a simpler way under the RNP assumption.)
\end{exercise}

\begin{exercise}\label{ex:fa-meas-ca}
  Let $X$ be a Banach space and $(S,\mc{A})$ a measure space.
  Let $\mb{\nu}$ be a \emph{finitely additive} $X$-valued vector measure on $(S,\mc{A})$, i.e. a function $\map{\mb{\nu}}{\mc{A}}{X}$ such that
  \begin{equation*}
    \mb{\nu}\Big(\sum_{n=1}^{N} E_n\Big) = \sum_{n=1}^{N} \mb{\nu}(E_n)
  \end{equation*}
  for all $N \in \N$ and $E_1,\ldots,E_n \in \mc{A}$.
  Suppose that there is a \emph{countably additive} (scalar) measure $\mu$ on $(S,\mc{A})$ such that
  \begin{equation*}
    \|\mb{\nu}(A)\|_{X} \leq \mu(A) \qquad \forall A \in \mc{A}.
  \end{equation*}
  Show that $\mb{\nu}$ is countably additive, $\|\mb{\nu}\|_{\var} \leq \|\mu\|_{\var}$, and $|\mb{\nu}| \ll \mu$.
\end{exercise}

\begin{exercise}\label{ex:closure-M}
  Let $(S,\mc{A})$ be a measurable space and $X$ a Banach space.
  Let $\mu$ be a finite measure on $(S,\mc{A})$.
  Show that the set of $X$-valued vector measures with Radon--Nikodym derivatives with respect to $\mu$, i.e. the set
  \begin{equation*}
    \{\mb{g}\mu : \mb{g} \in L^1(\mu;X)\} \subset M(S,\mc{A};X),
  \end{equation*}
  is closed in $M(S,\mc{A};X)$.
\end{exercise}

\begin{exercise}[Rademacher's theorem and the RNP]\index{theorem!Rademacher}\index{Radon--Nikodym property!equivalence with Rademacher's theorem}
  Show that a Banach space $X$ has the Radon--Nikodym property if and only if every $X$-valued Lipschitz function on $[0,1]$ is differentiable almost everywhere.
\end{exercise}

\begin{exercise}\label{ex:RNP-unitinterval}
  Suppose that $X$ has the Radon--Nikodym property with respect to the unit interval $[0,1]$ with Borel $\sigma$-algebra and Lebesgue measure.
  Show that $X$ has the Radon--Nikodym property with respect to all $\sigma$-finite measure spaces.
\end{exercise}

\begin{exercise}\label{ex:RNP-bochner}
  Prove Theorem \ref{thm:bochner-duality-RNP} in two different ways: first, using vector measures and the Radon--Nikodym property, then using martingales and the martingale convergence properties.\footnote{You can always search for references if you get stuck!}
\end{exercise}

\begin{exercise}\label{ex:Lp-RNP}\index{Radon--Nikodym property!of Bochner spaces}\index{Bochner spaces!have RNP}
  Let $X$ be a Banach space and $(\Omega,\mc{A},\mu)$ a measure space, and $p \in (1,\infty)$.
  Show that $X$ has the Radon--Nikodym property if and only if $L^p(\Omega;X)$ has the Radon--Nikodym property.
\end{exercise}



%%% Local Variables:
%%% mode: latex
%%% TeX-master: "../main.tex"
%%% End:


\section{The class of UMD spaces}
\label{sec:UMD}
\input{main/umd.tex}

\section{Type, cotype, and Fourier type}
\label{sec:type}


\subsection{(Rademacher) type and cotype}

\begin{itemize}
\item definitions and examples
\item K-convexity: implied by UMD; implies type/cotype duality
\item STATE that K-convexity iff nontrivial type (Maurey--Pisier theorem is too hard for this course)
\item gaussian sums, orthonormal invariance/covariance domination. STATE the equivalence with rademacher for finite cotype
\item detecting Hilbert spaces through type and cotype $2$ (have to do the Lindenstrauss reduction to f.d. subspaces, and the LIndenstrauss--Pelczynski Theorem 7.3)
\end{itemize}

\subsection{Fourier type}

\begin{itemize}
\item Fourier type with respect to $\R$, $\T$, and $\Z$; examples (e.g. by interpolation or Fubini)
\item UMD implies nontrivial Fourier type (can't find an elementary proof)
\item FT with respect to arbitrary lca group, in particular $\Z/N\Z$ and Cantor/Walsh group
\item Fourier type $2$ implies type and cotype $2$
\item STATE relations with type and cotype (some too subtle to prove)
\end{itemize}


%%% Local Variables:
%%% mode: latex
%%% TeX-master: "../main.tex"
%%% End:


\section{Fourier multipliers and Littlewood--Paley theory}
\label{sec:HT}

\subsection{Fourier multipliers and transference}

\subsection{The Hilbert transform and the HT property}

\subsection{HT implies UMD}

(one-two lectures of work)

\subsection{UMD implies HT}

(use the argument in HNVW1 - two lectures of work)



%%% Local Variables:
%%% mode: latex
%%% TeX-master: "../main.tex"
%%% End:


\section{Schatten class operators}
\label{sec:schatten}
\emph{under construction}

% \section{Schatten class operators}

% \section{The UMD property of the Schatten classes}

% \section{Schur multipliers}

% \section{Operator Lipschitz functions}


%%% Local Variables:
%%% mode: latex
%%% TeX-master: "../main.tex"
%%% End:


\section{Appendices}
\label{sec:appendices}
\subsection{Results from functional analysis}

\subsection{Results from probability theory}

\subsubsection{Conditional expectation}

\subsection{Complex interpolation}

\begin{itemize}
\item defn
\item basic structural results
\item (co)retraction theorem
\item examples 
\end{itemize}


%%% Local Variables:
%%% mode: latex
%%% TeX-master: "../main.tex"
%%% End:



\section{Misc. sections to move}
\subsection{John--Nirenberg for adapted sequences and the Kahane--Khintchine inequalities}
% following HNVW1 section 3.2.c

Fix a Banach space $X$ and a $\sigma$-finite measure space $(S,\mc{A},\mu)$.\todo{put this all on a probability space}
Consider a filtration $(\mc{F}_n)_{n \in \N}$ and a sequence $\phi = (\phi_n)_{n \in \N}$ of $X$-valued functions adapted to the filtration (i.e. $\phi_n$ is $\mc{F}_n$-measurable for all $n \in \N$).\todo{Strongly?}
For all $q \in (0,\infty)$ we consider the following measure of the oscillation of $\phi$:
\begin{equation*}
    \|\phi\|_{*,q} := \sup_{\substack{k,n \in \N \\ k \leq n}} \sup_{\substack{F \in \mc{F}_k \\ 0 < \mu(F) < \infty}} \Big( \fint_{F} \|(\phi_n - \phi_{k-1})(s)\|_{X}^{q} \, \dd\mu(s)\Big)^{1/q} .
\end{equation*}


\begin{thm}[John--Nirenberg inequality for adapted sequences]\label{thm:jn-adapted-sequences}
  With notation as above, for all $p,q \in (0,\infty)$ there exists a finite constant $c_{p,q}$ independent of $\phi$ such that
  \begin{equation*}
    \|\phi\|_{*,p} \leq c_{p,q} \|\phi\|_{*,q}.
  \end{equation*}
\end{thm}

We will prove this as a consequence of a series of lemmas, but before proving this, we demonstrate an important application to Rademacher sums.

\begin{thm}[Kahane--Khintchine inequality]\label{thm:kk}
  Let $X$ be a Banach space and let $(\varepsilon_{n})_{n \in \N}$ be a Rademacher sequence on a probability space $(\Omega,\mc{F},\P)$.
  Then for all $p,q \in (0,\infty)$ there exists a finite constant $\kappa_{p,q}$ such that for all finite sequences $(\mb{x}_n)_{n=1}^N$ in $X$,
  \begin{equation*}
    \Big\|\sum_{n=1}^{N} \varepsilon_{n} \mb{x}_{n} \Big\|_{L^p(\Omega;X)} \leq \kappa_{p,q} \Big\|\sum_{n=1}^{N} \varepsilon_{n} \mb{x}_{n} \Big\|_{L^q(\Omega;X)}.
  \end{equation*}
  That is, for all $p \in (0,\infty)$, the $L^p$-norms of a Rademacher sum are pairwise equivalent.  
\end{thm}

Since $\Omega$ is a probability space, H\"older's inequality yields the case $p \leq q$ with constant $\kappa_{p,q} = 1$, so we only need to consider the case $q < p$.

\begin{proof}[Proof, assuming the John--Nirenberg inequality]
  Consider the filtration and adapted sequence \todo{make sure to discuss filtrations generated by functions, particularly Rademacher variables} 
  \begin{equation*}
    \mc{F}_n := \sigma(\{\varepsilon_j : 1 \leq j \leq n\}), \qquad \phi_n := \sum_{j=1}^n \varepsilon_j \mb{x}_j.
  \end{equation*}
  We claim that for all $q \in [1,\infty)$ we have
  \begin{equation}\label{eq:kk-claim}
    \|\phi\|_{*,q} = \Big\| \sum_{j=1}^N \varepsilon_j \mb{x}_j \Big\|_{L^q(\Omega;X)}.
  \end{equation}
  Assuming this for the moment, the John--Nirenberg inequality yields a finite constant $c_{p,q}$ such that
  \begin{equation*}
    \Big\| \sum_{j=1}^N \varepsilon_j \mb{x}_j \Big\|_{L^p(\Omega;X)}
    = \|\phi\|_{*,p}
    \leq c_{p,q} \|\phi\|_{*,q}
    = \Big\| \sum_{j=1}^N \varepsilon_j \mb{x}_j \Big\|_{L^q(\Omega;X)}
  \end{equation*}
  whenever $1 \leq q < p$.
  If $q < 1 \leq p$, we fix $\theta \in (0,1)$ so that $1/p = \theta/q + (1-\theta)/2p$, and log-convexity of $L^p$-norms gives
  \begin{equation*}
    \| \phi_N \|_{L^p(\Omega;X)}
    \leq \| \phi_N \|_{L^q(\Omega;X)}^{\theta} \| \phi_N \|_{L^{2p}(\Omega;X)}^{1 - \theta}
    \leq \| \phi_N \|_{L^q(\Omega;X)}^{\theta} (c_{2p,p} \| \phi_N \|_{L^{p}(\Omega;X)})^{1 - \theta},
  \end{equation*}
  which yields
  \begin{equation*}
    \| \phi_N \|_{L^p(\Omega;X)} \leq c_{2p,p}^{(1 - \theta)/\theta} \| \phi_N \|_{L^q(\Omega;X)}.
  \end{equation*}
  Finally, if $q < p < 1$, we simply estimate
  \begin{equation*}
    \| \phi_N \|_{L^p(\Omega;X)} \leq \| \phi_N \|_{L^1(\Omega;X)} \leq c_{1,q} \| \phi_N \|_{L^q(\Omega;X)}.
  \end{equation*}
  This covers all nontrivial cases, and it remains to prove the claimed equality \eqref{eq:kk-claim}.

  First recall that
  \begin{equation*}
    \|\phi\|_{*,q} =  \sup_{\substack{k,n \in \N \\ k \leq n}} \sup_{\substack{F \in \mc{F}_k \\ \mu(F) > 0}} \Big( \fint_{F} \|(\phi_n - \phi_{k-1})(s)\|_{X}^{q} \, \dd\P(s)\Big)^{1/q}.
  \end{equation*}
  Fix $k \leq n$ and a set $F \in \mc{F}_k$ of positive measure.
  We will compute
  \begin{equation*}
    \fint_{F} \|\1_{F} (\phi_n - \phi_{k-1})(s)\|_{X}^q \, \dd\P(s) = \P(F)^{-1} \E \Big( \1_{F} \Big\| \sum_{j=k}^{n} \varepsilon_j \mb{x}_j\Big\|_{X}^q \Big).
  \end{equation*}
  Observe that for all $\omega \in \Omega$
  \begin{equation*}
    \Big\| \sum_{j=k}^{n} \varepsilon_j(\omega) \mb{x}_j\Big\|_{X} = \Big\| \varepsilon_k(\omega) \Big( \mb{x}_k + \sum_{j=k+1}^{n} \varepsilon_{j}^{\prime}(\omega) \mb{x}_j \Big) \Big\|_{X} = \Big\|\mb{x}_k + \sum_{j=k+1}^{n} \varepsilon_{j}^{\prime} \mb{x}_j \Big\|_{X}
  \end{equation*}
  where
  \begin{equation*}
    \varepsilon_{j}^{\prime} :=
    \begin{cases}
      \varepsilon_{j} & j \leq k \\
      \varepsilon_{k} \varepsilon_{j} & k+1 \leq j,
    \end{cases}
  \end{equation*}
  since $\varepsilon_j$ is $\pm 1$-valued.
  Now note that the $\sigma$-algebra $\mc{F}^\prime_{k+1,n} := \sigma(\{\varepsilon_j^\prime : k+1 \leq j \leq n\})$ is independent of $\mc{F}_k$, since for all $1 \leq j \leq k$ and $k+1 \leq j' \leq n$ we have by independence of the original Rademacher sequence
  \begin{equation*}
    \E(\varepsilon_j \varepsilon_{j'}^\prime) = \E(\varepsilon_j \varepsilon_k \varepsilon_{j'})
    =
    \begin{cases}
      \E(\varepsilon_j) \E(\varepsilon_k) \E(\varepsilon_{j'}) & \text{if $j < k$} \\
       \E(\varepsilon_{j'}) & \text{if $j = k$}
    \end{cases}
    \quad = 0.
  \end{equation*}
  Thus, since $F \in \mc{F}_k$, we have by independence of $\mc{F}_k$ and $\mc{F}^{\prime}_{k+1, n}$
  \begin{equation*}
    \begin{aligned}
      \P(F)^{-1} \E \Big( \1_{F} \Big\| \sum_{j=k}^{n} \varepsilon_j \mb{x}_j\Big\|_{X}^q \Big)
      &=  \P(F)^{-1} \E \Big( \1_{F} \Big\|\mb{x}_k + \sum_{j=k+1}^{n} \varepsilon_{j}^{\prime} \mb{x}_j \Big\|_{X}^{q} \Big) \\
      &=  \E \Big\|\mb{x}_k + \sum_{j=k+1}^{n} \varepsilon_{j}^{\prime} \mb{x}_j \Big\|_{X}^{q} 
      =  \E  \Big\|\sum_{j=k}^{n} \varepsilon_{j} \mb{x}_j \Big\|_{X}^{q}. \\
    \end{aligned}
  \end{equation*}
  Now letting $\mc{F}_{k,n} := \sigma(\{\varepsilon_j : k \leq j \leq n\})$, we have by the $L^q$-contraction property of conditional expectations
  \begin{equation*}
    \E \Big\|\sum_{j=k}^{n} \varepsilon_{j} \mb{x}_j \Big\|_{X}^{q}
    = \E \Big\|\E^{\mc{F}_{k,n}} \Big(\sum_{j=1}^{N} \varepsilon_{j} \mb{x}_j \Big)\Big\|_{X}^{q}
    \leq \E \Big\|\sum_{j=1}^{N} \varepsilon_{j} \mb{x}_j \Big\|_{X}^{q}
  \end{equation*}
  with equality when $k=1$ and $n=N$.
  Thus
  \begin{equation*}
    \|\phi\|_{*,q} =  \sup_{\substack{k,n \in \N \\ k \leq n}}  \Big(\E  \Big\|\sum_{j=k}^{n} \varepsilon_{j} \mb{x}_j \Big\|_{X}^{q}\Big)^{1/q} = \Big(\E  \Big\|\sum_{j=1}^{N} \varepsilon_{j} \mb{x}_j \Big\|_{X}^{q}\Big)^{1/q}
  \end{equation*}
  which proves the claimed equality \eqref{eq:kk-claim} and completes the proof. 
\end{proof}

In the setting of Hilbert-valued functions (and in particular, scalar-valued functions), the Kahane--Khintchine inequality leads to the classical Khintchine inequalities.

\begin{cor}[Khintchine's inequalities]
  Let $H$ be a Hilbert space, and let $(\varepsilon_{n})_{n \in \N}$ be a Rademacher sequence on a probability space $(\Omega, \mc{F}, \P)$.
  Then for all $p \in (0,\infty)$ there exist finite constants $A_p$ and $B_p$ such that for all finite sequences $(\mb{h}_n)_{n=1}^N$ in $H$,
  \begin{equation*}
    A_p \Big( \sum_{n=1}^N \|\mb{h}_n\|_H^2 \Big)^{1/2} \leq \Big\| \sum_{n=1}^N \varepsilon_n \mb{h}_n \Big\|_{L^p(\Omega;H)} \leq B_p \Big( \sum_{n=1}^N \|\mb{h}_n\|_H^2 \Big)^{1/2}.
  \end{equation*}
\end{cor}

\begin{proof}
  By independence of the Rademacher variables we have
  \begin{equation}
    \begin{aligned}
      \Big\| \sum_{n=1}^N \varepsilon_n \mb{h}_n \Big\|_{L^2(\Omega;H)}^2
      &= \Big\langle \sum_{n=1}^N \varepsilon_n \mb{h}_n , \sum_{m=1}^N \varepsilon_m \mb{h}_m \Big\rangle \\
      &= \sum_{n,m = 1}^N \E(\varepsilon_n \varepsilon_m) \langle \mb{h}_n, \mb{h}_m \rangle 
      = \sum_{n=1}^N \|\mb{h}_{n}\|_{H}^2,
    \end{aligned}
  \end{equation}
  so the result is true for $p=2$ with $A_2 = B_2 = 1$.
  Now use Kahane--Khintchine to extend the result to general $p \in (0,\infty)$.
\end{proof}

\subsubsection{Proof of the John--Nirenberg inequality}

We return to our analysis of a sequence $\phi = (\phi_n)_{n \in \N}$ of $X$-valued functions adapted to a filtration $(\mc{F}_n)_{n \in \N}$ on a $\sigma$-finite measure space $(S,\mc{A},\mu)$.
We prove the John--Nirenberg inequality via a series of lemmas in which we obtain increasingly fine control on the oscillation of $\phi$.

\begin{lem}\label{lem:JN-proof-1}
  For all $k \leq n$, $F \in \mc{F}_k$, and $\alpha > 0$,
  \begin{equation}\label{eq:JN-proof-1-est}
    \mu(F \cap \{ \|\phi_n - \phi_{k-1}\|_{X} > \alpha \}) \leq \Big( \frac{\|\phi\|_{*,q}}{\alpha} \Big)^{q} \mu(F).
  \end{equation}
\end{lem}

\begin{proof}
  We can assume that $0 < \mu(F) < \infty$, otherwise there is nothing to prove.
  The left hand side of \eqref{eq:JN-proof-1-est} is bounded by
  \begin{equation*}
    \int_{F} \Big( \frac{\|\phi_n - \phi_{k-1}\|_{X}}{\alpha} \Big)^{q} \, \dd\mu \leq \mu(F) \Big( \frac{\|\phi\|_{*,q}}{\alpha} \Big)^{q}
  \end{equation*}
  since $F \in \mc{F}_k$ and $k \leq n$, by the definition of $\|\phi\|_{*,q}$.
\end{proof}

Next we show that oscillation control of the form above extends to more general stopping times.

\begin{lem}
  Suppose that there exist $\alpha > 0$ and $\eta > 0$ such that
  \begin{equation*}
    \mu(F \cap \{ \| \phi_n - \phi_{k-1} \| > \alpha \} ) \leq \eta \mu(F) \qquad \forall k \leq n, F \in \mc{F}_k.
  \end{equation*}
  Then for all $k \in \N$, $F \in \mc{F}_k$, and all stopping times $\nu$ such that $\nu \geq k$ on $F$,
  \begin{equation}\label{eq:jn-stoptime}
    \mu(F \cap \{\nu < \infty\} \cap \{ \| \phi_{\nu} - \phi_{k-1} \| > 2\alpha \} ) \leq 2\eta \mu(F).
  \end{equation}
\end{lem}

\begin{proof}
  Sum over all possible values of the stopping time:
  \begin{equation*}
      \mu(F \cap \{\nu < \infty\} \cap \{ \| \phi_{\nu} - \phi_{k-1} \| > 2\alpha \} ) 
      = \lim_{N \to \infty} \sum_{n = k}^{N} \mu(F_n \cap \{ \| \phi_{n} - \phi_{k-1} \| > 2\alpha \} ),
  \end{equation*}
  where $F_{n} := F \cap \{\nu = n\}$.
  For fixed $N \geq n > k$, since $F_n \in \mc{F}_n \subset \mc{F}_{n+1}$, we have by assumption
  \begin{equation*}
    \begin{aligned}
      &\mu(F_n \cap \{ \| \phi_{n} - \phi_{k-1} \| > 2\alpha \} ) \\
      &\leq \mu(F_n \cap \{ \| \phi_{n} - \phi_{N} \| > \alpha \} )
      + \mu(F_n \cap \{ \| \phi_{k-1} - \phi_{N} \| > \alpha \} ) \\
      &\leq \eta \mu(F_n) + \mu(F_n \cap \{ \| \phi_{k-1} - \phi_{N} \| > \alpha \} ).
    \end{aligned}
  \end{equation*}
  Thus we have
  \begin{equation*}
    \begin{aligned}
      &\lim_{N \to \infty} \sum_{n = k}^{N} \mu(F_n \cap \{ \| \phi_{n} - \phi_{k-1} \| > 2\alpha \} ) \\
      &\leq \lim_{N \to \infty} \Big( \eta \sum_{n=k}^N \mu(F_n) + \sum_{n=k}^N \mu(F_n \cap \{ \| \phi_{k-1} - \phi_{N} \| > \alpha \}) \Big) \\
      &\leq \eta \mu(F) + \lim_{N \to \infty} \mu(F \cap \{ \| \phi_{k-1} - \phi_{N} \| > \alpha \})  
      \leq 2\eta\mu(F)
    \end{aligned}
  \end{equation*}
  using the assumption and $F \in \mc{F}_k$ in the last estimate.
\end{proof}
  
In the following lemma we make use of the \emph{started sequence}
\begin{equation*}
  {}^{k-1}\phi = (\phi_n - \phi_{k-1})_{n \geq k-1}
\end{equation*}
and its maximal function
\begin{equation*}
  ({}^{k-1} \phi)^{*}(s) := \sup_{n \geq k-1} \|({}^{k-1}\phi)_n(s)\|_X = \sup_{n \geq k} \|(\phi_n - \phi_{k-1})(s)\|_X.
\end{equation*}

\begin{lem}\label{lem:jn-mf}
  Suppose that $\phi$ satisfies \eqref{eq:jn-stoptime} for all $k \in \N$, $F \in \mc{F}_k$, and all stopping times $\nu$ such that $\nu \geq k$ on $F$.
  Then for all $\lambda > 0$,
  \begin{equation}\label{eq:jn-mf-eq}
    \mu(F \cap \{ ({}^{k-1} \phi)^{*} > \lambda + 2\alpha \}) \leq 2\eta \mu(F \cap \{  ({}^{k-1} \phi)^{*} > \lambda \}) \qquad \forall k \in \N, F \in \mc{F}_{k}.
  \end{equation}
\end{lem}

\begin{proof}
  Fix $k \in \N$ and consider the stopping times
  \begin{equation*}
    \begin{aligned}
      \rho &:= \inf\{n \geq k : \|\phi_n - \phi_{k-1}\| > \lambda\}, \\
      \nu &:= \inf\{n \geq k : \|\phi_n - \phi_{k-1}\| > \lambda + 2\alpha\}.
    \end{aligned}
  \end{equation*}
  Then $k \leq \rho \leq \nu$, and \eqref{eq:jn-mf-eq} can be rewritten as
  \begin{equation*}
    \mu(F \cap \{\nu < \infty\}) \leq 2\eta \mu(F \cap \{\rho < \infty\}).
  \end{equation*}
  Now fix $n \geq k$ and let $F_n := F \cap \{\rho = n\} \in \mc{F}_n$.
  On $\{F_n \cap \{\nu < \infty\}\}$ we have
  \begin{equation*}
    \|\phi_{\nu} - \phi_{n-1}\| \geq \|\phi_{\nu} - \phi_{k-1}\| - \|\phi_{n-1} - \phi_{k-1}\| > (\lambda + 2\alpha) - \lambda = 2\alpha,
  \end{equation*}
  so
  \begin{equation*}
    \mu(F_n \cap \{\nu < \infty\}) = \mu(F_n \cap \{\nu < \infty\} \cap \{\|\phi_{\nu} - \phi_{n-1}\| > 2\alpha\} )
    \leq 2\eta \mu(F_n).
  \end{equation*}
  Summing over $n \geq k$ completes the proof.
\end{proof}

\begin{lem}
  Suppose that $f$ is a non-negative function supported in $F \in \mc{A}$, satisfying
  \begin{equation*}
    \mu(f > \lambda + \alpha) \leq \eta \mu(f > \lambda) \qquad \forall \lambda > 0
  \end{equation*}
  for some $\eta \in (0,1)$ and $\alpha > 0$.
  Then for all $p \in [1,\infty)$,
  \begin{equation*}
    \|f\|_p \leq \frac{1 + \eta^{1/p}}{1 - \eta^{1/p}} \alpha \mu(F)^{1/p}.
  \end{equation*}
\end{lem}

\begin{proof}
  {\color{red} WRITE PROOF}
\end{proof}

\todo{UP TO HERE. HAVE TO COMPLETE PROOF OF JOHN-NIRENBERG.}

{\color{blue}

\begin{equation*}
  \|\phi\|_{**,q} := \sup_{k \in \N} \sup_{\substack{F \in \mc{F}_k \\ 0 < \mu(F) < \infty}} \Big( \fint_{F} ({}^{k-1} \phi)^{*}(s)^q \, \dd\mu(s) \Big)^{1/q},
\end{equation*}

}





test citation \cite{HNVW16}


% Bibliography (uncomment one among biblatex and bibtex
% BibLaTeX
%\printbibliography

% %% Bibtex instead of BibLaTeX (specified in packages.tex)
% %
\footnotesize
\bibliographystyle{amsplain}
\bibliography{bibliography} 



\end{document}



%%% Local Variables: 
%%% mode: latex
%%% TeX-master: t
%%% End: 
