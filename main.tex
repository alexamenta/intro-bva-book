\documentclass[a4paper,10pt]{amsbook}
\usepackage[utf8]{inputenc}

\usepackage{conf}

\hyphenation{pro-duct de-fi-ning}

%% Margin notes (use commands below to enable or disable notes)
% \usepackage[disable]{todonotes}
\usepackage[color=blue!30]{todonotes}

%\usepackage{makeidx}
\makeindex

\begin{document}

\title[Banach-valued analysis]{Introduction to Banach-valued analysis (v0.96)}
\date{\today}

\author[A. Amenta]{Alex Amenta}
\address{\noindent Mathematisches Institut \newline \indent Universit\"at Bonn, Bonn, Germany}
\email{amenta@math.uni-bonn.de}

\maketitle
\tableofcontents

%% Main matter

\chapter*{Preface}
%\addcontentsline{toc}{chapter}{Preface}
These notes are formally the basis of a one-semester master-level course to be taught (remotely) at the University of Bonn from October 2020 to February 2021, during what I hope turns out to be the peak of the Covid-19 pandemic.
In practise, they are the basis of an \emph{international} online course, open to everybody.

Most of the material in these notes is heavily based off the textbooks of Pisier \cite{gP16} and Hyt\"onen--van Neerven--Veraar--Weis \cite{HNVW16, HNVW17}.
I do not claim any originality in the proofs, or in any of the ideas; at most, I will claim an $\varepsilon^2$ of originality in the presentation (and perhaps a $\sqrt{\varepsilon}$ of effort in putting it all together).
I have put almost no effort into finding original references, or in getting the history correct: the books cited above do a much better job of this than I possibly could.

\vspace{1cm}

I presume these notes are full of typos and mistakes.
If you find any, please contact me at \texttt{amenta@math.uni-bonn.de} and I'll fix them.
Thanks to Timothy Banova, Victor Olmos, Lennart Ronge, Aidan Schumann, and Feng Shao for comments and corrections.\footnote{These notes are still in progress, so if you want to be in this list, just point out some mistakes!}
Thanks also to Jan van Neerven and Mark Veraar for their advice and patience as I fumbled through this topic in Delft, to Christoph Thiele for supporting and `co-lecturing' the course, and to Gennady Uraltsev for pushing me into the role of Teacher of Banach-Valued Analysis.

\vspace{1cm}

Keep healthy! 




\vspace{2cm}

\textbf{Warning:} these notes are incomplete.
Empty chapters will be filled in as they are completed (if all goes well, this will be before we reach the material in the lectures).

\textbf{Changes in this version:}
\begin{itemize}
\item Revised Exercise \ref{ex:continuous-Linfty} (including the addition of a third part).
\item Fixed various typos.
\end{itemize}





%%% Local Variables:
%%% mode: latex
%%% TeX-master: "../main.tex"
%%% End:


\chapter{Introduction}
\label{sec:intro}
A good deal of real analysis is concerned with properties of scalar-valued functions
\begin{equation*}
  \map{f}{\R^{d}}{\K}, \qquad \K = \text{$\R$ or $\C$}
\end{equation*}
and with operators acting on such functions.
The one-line goal of this course is to study what happens when scalar-valued functions are replaced by \emph{Banach-valued} functions: that is, we are going to study functions
\begin{equation*}
  \map{\mb{f}}{\R^{d}}{X},
\end{equation*}
where $X$ is a Banach space (typically infinite dimensional), and we will study properties of operators acting on $X$-valued functions.
This will reveal interesting relationships between Fourier analysis, probability, measure theory, operator theory, and the geometry of Banach spaces.

\section{Case study: the Fourier transform and Plancherel's theorem}

To keep things simple, consider a finite dimensional complex Banach space $X$ with a basis $\mb{e}_{1}, \ldots, \mb{e}_{N}$.
Every vector $\mb{x} \in X$ has a basis expansion
\begin{equation*}
  \mb{x} = \sum_{n=1}^{N} x_{n} \mb{e}_{n}
\end{equation*}
for some scalars $x_{n} \in \C$, so every $X$-valued function $\map{\mb{f}}{\R}{X}$ may be written as
\begin{equation*}
  \mb{f}(x) = \sum_{n=1}^{N} f_{n}(x) \mb{e}_{n}
\end{equation*}
for some scalar-valued functions $\map{f_{n}}{\R}{\C}$.

We will investigate the Fourier transform\index{Fourier transform} on $X$-valued functions.
For integrable scalar-valued functions $f \in L^1(\R)$, the Fourier transform is defined by
\begin{equation*}
  \hat{f}(\xi) := \int_{\R} f(x) e^{-2\pi i x \xi} \, \dd x.
\end{equation*}
Plancherel's theorem\index{theorem!Plancherel}, arguably the most important result in Fourier analysis, says that the Fourier transform is an isometry on $L^2$:
\begin{equation*}
  \|\hat{f}\|_{L^2(\R)} = \|f\|_{L^2(\R)}.
\end{equation*}
Does this still hold for $X$-valued functions?
Before we can answer the question we need to say what we mean by the Fourier transform and the $L^2$-norm for $X$-valued functions.
Given $\map{\mb{f}}{\R}{X}$ as above, the Fourier transform $\map{\hat{\mb{f}}}{\R}{X}$ can be defined as in the scalar-valued case: we have
\begin{equation*}
  \begin{aligned}
    \hat{\mb{f}}(\xi)
    &:= \int_{\R} \mb{f}(x) e^{-2\pi i x \xi} \, \dd x \\
    &= \int_{\R} \Big( \sum_{n=1}^{N} f_{n}(x) \mb{e}_{n} \Big) e^{-2\pi n x \xi} \, \dd x \\
    &= \sum_{n=1}^{N} \Big(\int_{\R} f_{n}(x) e^{-2\pi i x \xi} \, \dd x \Big) \mb{e}_{n}
    &= \sum_{n=1}^{N} \hat{f_{n}}(\xi) \mb{e}_{n}
  \end{aligned}
\end{equation*}
where the Lebesgue integral is extended to $X$-valued functions by linearity.\footnote{This requires finite dimensionality of $X$. In general, the Lebesgue integral is replaced by a \emph{Bochner integral}. This is covered in Chapter \ref{sec:Bochner-spaces}.}
The $L^2$-norm on $X$-valued functions is defined by
\begin{equation*}
  \|\mb{f}\|_{L^2(\R;X)} := \Big( \int_{\R} \|\mb{f}(x)\|_{X}^{2} \, \dd x \Big)^{1/2}.
\end{equation*}
This yields a Banach space $L^2(\R;X)$, but already we start to see that something suspicious is going on: \emph{there is no natural way to turn $L^2(\R;X)$ into a Hilbert space unless $X$ itself is a Hilbert space.}
Of course, since any two norms on a finite dimensional vector space are equivalent, there exists a constant $1 \leq C_{X} < \infty$ such that
\begin{equation}\label{eq:fd-norm-equivalence}
  C_{X}^{-1} \|(x_{n})_{n=1}^{N}\|_{\ell^2_{N}} \leq \|\mb{x}\|_{X} \leq C_{X} \|(x_{n})_{n=1}^{N}\|_{\ell^2_{N}},
\end{equation}
where
\begin{equation*}
  \|(x_{n})_{i=1}^{N}\|_{\ell^2_{N}} := \Big( \sum_{n=1}^{N} |x_{n}|^{2} \Big)^{1/2}
\end{equation*}
is the familiar Euclidean norm on $\C^{N}$.
Since $\ell^{2}_{N}$ is a Hilbert space, \eqref{eq:fd-norm-equivalence} lets us treat $X$ as if it were a Hilbert space; the constant $C_{X}$ measures how close $X$ is to the Hilbert space $\ell^{2}_{N}$, with $C_{X} = 1$ if and only if $X$ is isometric to $\ell^{2}_{N}$.

We proceed with our investigation of the Plancherel theorem.
Using the basis expansion and the equivalence of norms in \eqref{eq:fd-norm-equivalence}, we can compute
\begin{equation*}
  \begin{aligned}
    \|\hat{\mb{f}}\|_{L^2(\R;X)}
    &= \Big( \int_{\R} \Big\| \sum_{n=1}^{N} \hat{f_{n}}(\xi)  \mb{e}_{n} \Big\|_{X}^{2} \, \dd \xi \Big)^{1/2} \\
    &\leq C_{X} \Big( \sum_{n=1}^{N} \int_{\R} | \hat{f_{n}}(\xi) |^{2} \, \dd \xi \Big)^{1/2} \\
    &\stackrel{(*)}{=} C_{X} \Big( \sum_{n=1}^{N} \int_{\R} | f_{n}(x) |^{2} \, \dd x \Big)^{1/2} \\
    &\leq C_{X}^{2} \Big( \int_{\R} \Big\| \sum_{n=1}^{N} f_{n}(x)  \mb{e}_{n} \Big\|_{X}^{2} \Big)^{1/2} = C_{X}^{2} \|\mb{f}\|_{L^2(\R;X)},
  \end{aligned}
\end{equation*}
using the (scalar-valued) Plancherel theorem to deduce the starred equality.
Thus we do have a kind of $X$-valued Plancherel theorem, but now instead of being an isometry, the Fourier transform is merely bounded on $L^2(\R;X)$ with norm $\leq C_{X}^{2}$.
So the boundedness of the Fourier transform on $L^2(\R;X)$ seems to have something to do with the proximity of $X$ to a Hilbert space.

Now what if $X$ is infinite dimensonal?
If a constant $C_{X}$ as in \eqref{eq:fd-norm-equivalence} exists (but with $N = \infty$), i.e. if $X$ is isomorphic to $\ell^2(\N)$, then the argument above still works,\footnote{As mentioned in an earlier footnote, the Lebesgue integral has to be replaced by the Bochner integral.} and the norm of the $X$-valued Fourier transform on $L^2(\R;X)$ is $\leq C_{X}^{2}$.
The surprising fact is that the converse is also true: we will prove this in Chapter \ref{sec:fouriertype}.

\begin{thm}[Kwapie\'n, 1972]\label{thm:Kwapien-intro}\index{theorem!Kwapie\'n}
  The $X$-valued Fourier transform is bounded on $L^2(\R;X)$ if and only if $X$ is isomorphic to a Hilbert space.
\end{thm}

This is just one (extreme) example of the following general principle: when $T$ is an operator on scalar-valued functions which is bounded on some Lebesgue space $L^p(\R)$, then $T$ can be extended to $X$-valued functions, and the boundedness of this extension on $L^p(\R;X)$ reflects geometric properties of the Banach space $X$.
Different operators can be used to reflect different geometric properties.

\section{Case study: the Hilbert transform}

Another important operator in harmonic analysis is the Hilbert transform\index{Hilbert transform}, defined on $\map{f}{\R}{\C}$ by
\begin{equation}\label{eq:HT-scalar}
  Hf(x) := \frac{1}{\pi} \mathrm{p. v.} \int_{\R} f(x-y) \, \frac{\dd y}{y} := \frac{1}{\pi} \lim_{\varepsilon \downarrow 0} \int_{|y| > \varepsilon} f(x-y) \, \frac{\dd y}{y}.
\end{equation}
The Hilbert transform is a prototypical \emph{singular integral operator}: the integral kernel $1/y$ is not integrable, so $H$ cannot be defined as a classical integral operator, but cancellation between the positive and negative values of $1/y$ allow $Hf$ to be well-defined as a \emph{principal value integral} when $f$ is sufficiently smooth (Schwartz, for example).
One can show that $H$ is a \emph{Fourier multiplier}\index{Fourier multiplier} with symbol $m(\xi) = -i\sgn(\xi)$: that is, for all Schwartz functions $f \in \Sch(\R)$, $Hf = (m\hat{f})^{\vee}$, where $\vee$ denotes the inverse Fourier transform.
Plancherel's theorem implies that the Hilbert transform is an isometry on $L^2(\R)$: indeed,
\begin{equation*}
  \|Hf\|_{L^2(\R)} = \|m\hat{f}\|_{L^2(\R)} \leq \|m\|_{L^\infty(\R)} \|\hat{f}\|_{L^2(\R)} = \|f\|_{L^{2}(\R)},
\end{equation*}
since $|m(\xi)| = 1$ for all $\xi \in \R$.
Furthermore, since the singular kernel $1/y$ has relatively nice decay and smoothness estimates (despite its singular nature), Calder\'on--Zygmund theory can be used to extrapolate the $L^2$-boundedness to $L^p$: for $p \in (1,\infty)$ there exists a constant $C_{p} < \infty$ such that
\begin{equation*}
  \|Hf\|_{L^p(\R)} \leq C_{p} \|f\|_{L^p(\R)} \qquad \forall f \in \Sch(\R).
\end{equation*}

What about vector-valued extensions?
If $X$ is a Banach space, then the $X$-valued Hilbert transform can be defined as a singular integral via the formula \eqref{eq:HT-scalar}, and it is an $X$-valued Fourier multiplier with symbol $-i\sgn(\xi)$ (using the $X$-valued Fourier transform from the previous case study).\footnote{Once more: when $X$ is infinite dimensional, Bochner integrals need to replace Lebesgue integrals!}
If $X$ is isomorphic to a Hilbert space, we can invoke the $X$-valued Plancherel theorem to estimate
\begin{equation*}
  \|H\mb{f}\|_{L^2(\R;X)} \leq K_{X} \|m\hat{\mb{f}}\|_{L^2(\R;X)} \leq K_{X} \|m\|_{L^\infty(\R)} \|\hat{\mb{f}}\|_{L^2(\R;X)}= K_{X}^{2} \|\mb{f}\|_{L^{2}(\R;X)},
\end{equation*}
where $1 \leq K_{X} < \infty$ denotes the norm of the Fourier transform on $L^2(\R;X)$.

If $X$ is not isomorphic to a Hilbert space, then we know by Kwapie\'n's theorem that the Fourier transform is \emph{not} bounded on $L^2(\R;X)$, so the previous argument does not apply.
However, it is possible to approach bounds for the Hilbert transform using probabilistic techniques, avoiding use of Plancherel's theorem, and these arguments can be carried out with values in certain (but not all) Banach spaces.
The following theorem is one of the main goals of this course.

\begin{thm}[Burkholder--Bourgain, 1983]\index{theorem!Burkholder--Bourgain}
  The $X$-valued Hilbert transform is bounded on $L^p(\R;X)$ for some (equivalently, all) $p \in (1,\infty)$ if and only if $X$ has the \emph{UMD property}.\index{UMD property}
\end{thm}

UMD stands for \emph{unconditionality of martingale differences}.
Martingales are a fundamental class of stochastic processes, and unconditionality of a sequence in a Banach space says that the elements of the sequence are, in some sense, `independent' or `orthogonal'.
The UMD property is probabilistic in nature; the Burkholder--Bourgain theorem says that it is also a harmonic-analytic property.
Thus the UMD property is a natural and often necessary assumption for Banach-valued analysis.
Hilbert spaces are UMD,\footnote{The arguments above show this as a consequence of the Kwapie\'n and Burkholder--Bourgain theorems, but this is absolute overkill. It can be proven directly from the definition in terms of martingales.} but so are most natural function spaces, in particular $L^p$ spaces and Sobolev spaces $W^{p}_{s}$ (with $p \in (1,\infty)$).
Thus the UMD property appears readily in applications to PDEs, both deterministic and stochastic.
Even spaces of operators can be UMD, for example the Schatten classes\index{Schatten class} $\mc{S}^{p} \subset B(H)$ (studied in Chapter \ref{sec:schatten}), and consequences of the UMD property for these spaces can be used to prove deep results in operator theory.


\section{Conventions and notation}\label{sec:conventions}
Throughout these notes we will deal with both real and complex Banach spaces.
If I do not explicitly specify `real' or `complex', then either choice can be made, and $\K$ denotes the scalar field (i.e. $\K = \R$ or $\K = \C$).
Every complex Banach space can be seen as a real Banach space by restricting scalar multiplication to the reals.
On the other hand, every real Banach space can be `complexified', a process which doubles the dimension over $\R$ and does what it should: for example the complexification of $\R^n$ is $\C^n$, the complexification of $L^p(S;\R)$ is $L^p(S;\C)$, and so on.
It's a good idea not to think too hard about this.

I have tried to maintain the convention of writing vectors and vector-valued functions in bold: so I write $\mb{x} \in X$ for a vector in a Banach space $X$, and $\map{\mb{f}}{\R}{X}$ for an $X$-valued function.
This convention is not standard but I believe it helps in gaining intuition.
The way I see it, vectors and vector-valued functions are intrinsically `heavier' than their scalar analogues.
This leads to some typographical paradoxes: do we consider an element of $L^2(\R)$ as a scalar-valued function (thus writing $f$) or an element of a Banach space (thus writing $\mb{f}$)?
The answer depends on the context.
I just want to point out that nobody makes this distinction in the literature (except myself, in recent papers).
Probably for good reason.

Throughout most of these notes I assume without mention that every measure space is $\sigma$-finite.
To deal with non-$\sigma$-finite measure spaces the notion of strong measurability has to be adjusted: pointwise limits of simple functions have to be replaced by almost everywhere limits of $\mu$-simple functions.
See \cite[Section 1.1.b]{HNVW16}.


For an exponent $p \in [1,\infty]$ we let $p'$ denote the \emph{H\"older conjugate}
\begin{equation*}
  p' := \begin{cases}
    \frac{p}{p-1} & p \in (1,\infty) \\
    \infty & p = 1 \\
    1 & p = \infty,
  \end{cases}
\end{equation*}
so that $p^{-1} + (p')^{-1} = 1$ (interpreting $1/\infty$ as $0$).

The natural numbers are $\N = \{0,1,2,\ldots\}$, but sometimes I will mess up and they will be $\N = \{1,2,3,\ldots\}$.


 % {\footnotesize
 %   \subsection{Acknowledgements}
 % }


%%% Local Variables:
%%% mode: latex
%%% TeX-master: "../main"
%%% End:



\chapter{Strong measurability and Bochner spaces}
\label{sec:Bochner-spaces}
Rather than considering individual functions, one by one, it is smart to consider \emph{spaces} of functions with certain properties: smooth functions, continuous functions, integrable functions, and so on.
One of the fundamental classes of functions are the Lebesgue spaces, $L^p(S)$, associated with a measure space $S$.
For vector-valued functions, the analogue of Lebesgue spaces are the \emph{Bochner spaces}.
These are spaces of measurable vector-valued functions defined up to mofication on subsets of measure zero, just like Lebesgue spaces.
In order to define them we need to make precise what we mean by `measurability', as this turns out to be more complicated than usual in the vector-valued setting.

\section{Notions of measurability}

Consider a measurable space $(S,\mc{A})$,\footnote{i.e. $S$ is a set and $\mc{A}$ is a $\sigma$-algebra of subsets of $S$.} and let $X$ be a Banach space over the scalar field $\K$ (either $\R$ or $\C$).
The simplest kind of $X$-valued function arises by taking a \emph{scalar}-valued function $\map{f}{S}{\K}$ and a non-zero vector $\mb{x} \in X$, and `placing $f$ in the direction of $\mb{x}$'.
This function is denoted using the tensor notation $f \otimes \mb{x}$, and formally defined by
\begin{equation*}
  \map{f \otimes \mb{x}}{S}{X}, \qquad (f \otimes \mb{x})(s) := f(s)\mb{x} \quad \forall s \in S.
\end{equation*}
Note that the range of $f \otimes \mb{x}$ is contained in the linear span of $\mb{x}$.

The second simplest kind of $X$-valued function are the \emph{simple functions}.
A function $\map{\mb{g}}{S}{X}$ is \emph{simple} if there exists a finite collection of pairwise disjoint measurable subsets $S_1,\ldots,S_N \subset S$ and non-zero vectors $\mb{x}_1,\ldots,\mb{x}_N \in X$ such that
\begin{equation}\label{eqn:simple-function-standard-form}
  \mb{g} = \sum_{n=1}^N \1_{S_n} \otimes \mb{x}_n ,
\end{equation}
where $\1_{S_n}$ is the indicator function of $S_n$.
We denote the vector space of simple functions $S \to X$ by $\Simp(S;X)$ or $\Simp_{\mc{A}}(S;X)$.
Note that the range of $\mb{g}$ is contained in $\spn(\mb{x}_1,\ldots,\mb{x}_N)$, which is a finite dimensional subspace of $X$.

Of course, we need more than simple functions; we need \emph{measurable functions}.
When considering Banach-valued functions, particularly when our Banach spaces are allowed to be infinite dimensional, there is more than one notion of measurability, and these are generally inequivalent.

\begin{defn}
  Let $X$ be a Banach space.
  We say that a function $\map{\mb{f}}{S}{X}$ is
  \begin{itemize}
  \item
    \emph{measurable} if for every Borel set $B \subset X$, the preimage $\mb{f}^{-1}(B)$ is measurable;
  \item
    \emph{strongly measurable} (or \emph{Bochner measurable}) if it is the pointwise limit of simple functions; that is, if there exists a sequence $(\mb{f}_n)_{n=1}^\infty$ in $\Simp(S;X)$ such that $\mb{f} = \lim_{n \to \infty} \mb{f}_n$ pointwise on $S$;
  \item
    \emph{weakly measurable} if for every functional $\mb{x}^* \in X^*$, the scalar-valued function $\map{\langle \mb{f}, \mb{x}^*\rangle}{S}{\K}$ given by $s \mapsto \langle \mb{f}(s), \mb{x}^* \rangle$ is measurable.
  \end{itemize}
  All of these notions implicitly refer to the $\sigma$-algebra $\mc{A}$.
\end{defn}

Using the notation
\begin{center}
  \begin{tabular}{r|l}
    $\Meas (S;X)$  & Measurable $\map{\mb{f}}{S}{X}$    \\
    $\SMeas(S;X)$  & Strongly measurable $\map{\mb{f}}{S}{X}$ \\
    $\WMeas(S;X)$  & Weakly measurable $\map{\mb{f}}{S}{X}$
  \end{tabular}
\end{center}
we have the containment
\begin{equation}\label{eq:measurability-inclusions}
  \SMeas(S;X) \subset \Meas(S;X) \subset \WMeas(S;X)
\end{equation}
(Exercise \ref{ex:measurability-containments}).
When $X$ is finite dimensional these notions coincide: the derivation of strong measurability from measurability is a standard result in measure theory,\footnote{See for example \cite[Corollary 4.2.7]{rD04} in the one-dimensional case; extending this to the finite dimensional case can be done by summing up coordinates.} and weak measurability is just a convoluted rewriting of coordinatewise measurability.
But in the general context of Banach spaces the inclusions \eqref{eq:measurability-inclusions} are strict.

\begin{example}[A measurable function which is not strongly measurable]\label{eg:non-sm}
  Let $X$ be a non-separable Banach space (for example, $X = L^\infty(\R)$), and consider the identity map $\map{I}{X}{X}$, which is continuous and hence measurable.
  We prove that $I$ is not strongly measurable by contradiction.
  Assuming $I$ is strongly measurable, we find that there exists a sequence of simple functions $(i_n)_{n \in \N}$ converging pointwise to $I$.
  For each $\mb{x} \in X$ we then have
  \begin{equation*}
    \mb{x} = \lim_{n \to \infty} i_n(\mb{x}),
  \end{equation*}
  so that the union $U := \cup_{n \in \N} i_n(X)$ is dense in $X$.
  Since each $i_n$ is simple, $U$ is countable, which implies that $X$ is separable.
  Thus $I$ not strongly measurable.
\end{example}

\begin{rmk}
  The existence of weakly measurable functions that are not measurable is not so simple, but see \cite[Example 1.4.3]{HNVW16} for an argument which shows that the $\sigma$-algebra $\sigma(X^*)$ (which defines the set of weakly measurable functions) may be strictly smaller than the Borel $\sigma$-algebra on $X$.
  The indicator function of a Borel set which is not in $\sigma(X^*)$ is then weakly measurable, but not measurable.
\end{rmk}

We used non-separability to construct a function which is measurable but not strongly measurable, and there is a very good reason for this.

\begin{thm}[Pettis measurability theorem]\label{thm:Pettis-measurability}
  Let $(S,\mc{A})$ be a measurable space and $X$ a Banach space.
  Then a function $\map{\mb{f}}{S}{X}$ is strongly measurable if and only if it is weakly measurable and \emph{separably valued} (i.e. there exists a separable subspace $X' \subset X$ such that $\mb{f}(S) \subset X'$).
  In particular, if $X$ is separable, then
  \begin{equation*}
    \SMeas(S;X) = \Meas(S;X) = \WMeas(S;X).
  \end{equation*}
\end{thm}

\begin{proof}
  First suppose that $\mb{f}$ is strongly measurable.
  Then $\mb{f}$ is automatically weakly measurable, and we just need to show that it is separably valued.
  This essentially follows the argument from Example \ref{eg:non-sm}.
  Let $(\mb{f}_n)_{n \in \N}$ be a sequence of simple functions converging to $\mb{f}$ pointwise, and let $X_n \subset X$ denote the span of the range of $\mb{f}_n$.
  Each $X_n$ is finite dimensional, hence separable, and thus the closed subspace $X' \subset X$ generated by the collection $(X_n)_{n \in \N}$ is also separable.
  Since $\mb{f}_n \to \mb{f}$ pointwise, the range of $\mb{f}$ is contained in $X'$, so $\mb{f}$ is separably valued.

  Now assume that $\mb{f}$ is weakly measurable and separably valued.
  By replacing $X$ with the closure of the range of $\mb{f}$, we may assume without loss of generality that $X$ is separable.
  Let $(\mb{x}_n)_{n \in \N}$ be a dense sequence in $X$, and for each $n \in \N$ define a function $\map{\phi_n}{X}{\{\mb{x}_1,\ldots,\mb{x}_n\}}$ such that for all $\mb{x} \in X$,
  \begin{equation*}
    \|\mb{x} - \phi_n(\mb{x})\|_X = \min_{1 \leq j \leq n} \|\mb{x} - \mb{x}_j\|_X.
  \end{equation*}
  By density of $(\mb{x}_n)_{n \in \N}$ in $X$ we thus have that $\phi_n(\mb{x}) \to \mb{x}$ for all $\mb{x} \in X$.
  Now define functions $\map{\mb{f}_n}{S}{X}$ by
  \begin{equation*}
    \mb{f}_n(s) := \phi_n(\mb{f}(s)) \qquad \forall s \in S,
  \end{equation*}
  so that $\mb{f}_n \to \mb{f}$ pointwise.
  Each $\mb{f}_n$ has finite range, so to show that $\mb{f}_n$ is simple we need only show that the preimages $\mb{f}_n^{-1}(\mb{x}_k)$ are measurable.
  For all $n \in \N$ and all $1 \leq k \leq n$ We have
  \begin{equation*}
    \begin{aligned}
    \mb{f}_n^{-1}(\mb{x}_k)
    &= \{s \in S : \phi_n(\mb{f}(s)) = \mb{x}_k \} \\
    &= \{s \in S : \|\mb{f}(s) - \mb{x}_k\|_X = \min_{1 \leq j \leq n} \|\mb{f}(s) - \mb{x}_j\|_X \}.
  \end{aligned}
  \end{equation*}
  Let $(\mb{x}_m^*)_{m \in \N}$ be a norming sequence of unit vectors on $X^*$.
  Since $\mb{f}$ is weakly measurable, for each $j \in \{1,\ldots,n\}$ the function
  \begin{equation*}
    s \mapsto \|\mb{f}(s) - \mb{x}_j\|_X = \sup_{m \in \N} |\langle \mb{f}(s) - \mb{x}_j, \mb{x}_m^* \rangle|
  \end{equation*}
  is measurable (being the countable supremum of measurable functions).
  Thus the function
  \begin{equation*}
    s \mapsto \min_{1 \leq j \leq n} \|\mb{f}(s) - \mb{x}_j\|_X
  \end{equation*}
  is also measurable, and the representation above shows that $\mb{f}_n^{-1}(\mb{x}_k)$ is measurable (being the set on which two measurable functions are equal).
  Hence each $\mb{f}_n$ is simple, and consequently $\mb{f}$ is strongly measurable.
\end{proof}

The Pettis measurability theorem is incredibly useful, mostly because the conditions of weak measurability and separable-valuedness are easier to work with than the existence of pointwise approximating sequences of simple functions.
As a quick application, we can show that strong measurability is preserved under multiplication with measurable scalar-valued functions.\footnote{This can of course be proven using pointwise approximation with simple functions, but proof is not as clear.}

\begin{cor}\label{cor:strong-meas-meas-mult}
  Let $(S,\mc{A})$ be a measurable space and $X$ a Banach space.
  Suppose that $\map{\mb{f}}{S}{X}$ is strongly measurable and $\map{\phi}{S}{\K}$ is measurable.
  Then the pointwise product $\map{\phi \mb{f}}{S}{X}$ is strongly measurable.
\end{cor}

\begin{proof}
  By the Pettis measurability theorem, it suffices to show that $\phi \mb{f}$ is weakly measurable and separably-valued. 
  First we show weak measurability: for each functional $\mb{x}^* \in X^*$ write for $s \in S$
  \begin{equation*}
    \langle \phi \mb{f}, \mb{x}^* \rangle(s) = \phi(s) \langle \mb{f}(s) , \mb{x}^* \rangle = \phi \langle \mb{f}, \mb{x}^* \rangle.
  \end{equation*}
  Since $f$ is weakly measurable, the product $\phi \langle \mb{f}, \mb{x}^* \rangle$ is measurable for all $\mb{x}^* \in X^*$, so $\phi \mb{f}$ is weakly measurable.
  To show that $\phi \mb{f}$ is separably-valued, first note that since $f$ is separably-valued there exists a separable closed subspace $X_0 \subset X$ such that $\mb{f}(s) \in X_0$ for all $s \in S$.
  But then $\phi(s)\mb{f}(s) \in X_0$ too, so $\phi \mb{f}$ is separably-valued.
\end{proof}

In what follows we will generally deal with equivalence classes of functions modulo almost-everywhere equivalence, i.e. gievn a measure space $(S,\mc{A},\mu)$ we will consider two measurable functions $\map{\mb{f},\mb{g}}{S}{X}$ as being equal if the set
\begin{equation*}
  \{s \in S : \mb{f}(s) \neq \mb{g}(s)\}
\end{equation*}
has measure zero.
In this case we will write $\mb{f} \aeeq \mb{g}$.
For strongly measurable functions, almost-everywhere equality is equivalent to `weak' almost-everywhere equality.
This is a surprisingly useful observation, which can be used to deduce identities for vector-valued functions from corresponding identities for scalar-valued functions.

\begin{lem}\label{lem:coordinatewise-equality-test}
  Let $(S,\mc{A},\mu)$ be a measure space and $X$ a Banach space.
  Suppose that $\map{\mb{f},\mb{g}}{S}{X}$ are strongly measurable.
  Then $\mb{f} \aeeq \mb{g}$ if and only if for all functionals $\mb{x}^* \in X^*$, $\langle \mb{f}, \mb{x}^*\rangle \aeeq \langle \mb{g}, \mb{x}^* \rangle$.
\end{lem}

\begin{proof}
  The `only if' direction is straightforward, so we omit the proof.
  The `if' direction is harder: each of the sets
  \begin{equation*}
    N_{\mb{x}^*} := \{s \in S : \langle \mb{f}(s), \mb{x}^* \rangle \neq \langle \mb{g}(s), \mb{x}^* \rangle\} \qquad \mb{x}^* \in X^*
  \end{equation*}
  has measure zero, but the (uncountable!) union of these sets over all $\mb{x}^* \in X^*$ need not.
  This is where strong measurability comes into play, via the Pettis theorem.
  Since $\mb{f}$ and $\mb{g}$ are separably-valued, there exists a separable closed subspace $X_0 \subset X$ containing the ranges of both $\mb{f}$ and $\mb{g}$.
  Since $X_0$ is separable, there is a (countable!) sequence $(\mb{x}_n^*)_{n \in \N}$ in $X^*$ which separates points of $X_0$.\footnote{That is, if $\mb{x} \neq \mb{y} \in X_0$, then there exists $n \in \N$ such that $\langle \mb{x}, \mb{x}_n^* \rangle \neq \langle \mb{y}, \mb{x}_n^* \rangle$. See Proposition \ref{prop:sep-sep-points} in the appendices.}
  Now define
  \begin{equation*}
    N := \bigcup_{n \in \N} N_{\mb{x}_n^*}.
  \end{equation*}
  This set has measure zero since it is the countable union of sets with measure zero.
  For all $s \notin N$ we then have $\langle \mb{f}(s), \mb{x}_n^* \rangle = \langle \mb{g}(s), \mb{x}_n^* \rangle$ for all $n \in \N$, and since $(\mb{x}_n^*)_{n \in \N}$ separates points of $X_0$, it follows that $\mb{f}(s) = \mb{g}(s)$.
  Thus $\mb{f} \stackrel{\ae}{=} \mb{g}$.  
\end{proof}

\section{Bochner spaces}

We are ready to define Bochner spaces, which generalise Lebesgue spaces to Banach-valued functions.
Given a Banach-valued function $\map{\mb{f}}{S}{X}$, we let $\|\mb{f}\|_X$ denote the non-negative function on $S$ defined by $s \mapsto \|\mb{f}(s)\|_X$.
If $\mb{f}$ is measurable, then $\|\mb{f}\|_X$ is also measurable, since the function $\mb{x} \mapsto \|\mb{x}\|_X$ is continuous.
\textbf{From now on, we make the standing assumption that all measure spaces are $\sigma$-finite.} This is not necessary, but it avoids a few technicalities that I don't want to deal with.

\begin{defn}
  Let $(S,\mc{A},\mu)$ be a ($\sigma$-finite) measure space and $X$ a Banach space.
  For $p \in [1,\infty]$, we let $L^p(S,\mc{A},\mu;X)$ denote the set of \emph{strongly} $\mc{A}$-measurable functions $\mb{f} \in \SMeas(S;X)$ such that $\|\mb{f}\|_X \in L^p(S,\mc{A},\mu)$, and we write
  \begin{equation*}
    \|\mb{f}\|_{L^p(S,\mc{A},\mu;X)} := \| \|\mb{f}\|_X \|_{L^p(S,\mc{A},\mu)}.
  \end{equation*}
  We consider two functions $\mb{f}, \mb{g} \in L^p(S,\mc{A},\mu;X)$ to be equal if $\mb{f} \aeeq \mb{g}$.
  Each $L^p(S,\mc{A},\mu;X)$ is a Banach space: the proof is identical to the classic proof in the scalar-valued case.\footnote{For revision see \cite[Theorem 5.2.1]{rD04}.}
\end{defn}

\begin{rmk}
  If $\mb{f}$ is strongly measurable and $\mb{g} \aeeq \mb{f}$, then it does not automatically follow that $\mb{g}$ is strongly measurable, but $\mb{g}$ nevertheless has a strongly measurable representative ($\mb{f}$).
  To be more precise, we should say that $L^p(S;X)$ consists of \emph{$\mu$-almost everywhere strongly measurable} functions $\mb{f}$.
  I won't be too careful about this distinction.
  This is discussed at length, and properly, in \cite[Section 1.1.b]{HNVW16}.
\end{rmk}

In general we won't use all of $S$, $\mc{A}$, and $\mu$ in the notation for $L^p(S,\mc{A},\mu;X)$; we will only write out the parameters that need to be emphasised (whatever combination of the set, the $\sigma$-algebra, and the measure).
If the parameters are equally unimportant we may even omit all three (as in Remark \ref{rmk:lp-issues} below).
On the other hand, we will never omit $X$ unless $X = \K$ is the scalar field.

\begin{rmk}\label{rmk:lp-issues}
  It is possible for the scalar-valued function $\|\mb{f}\|_X$ to be in $L^p$ without $\mb{f}$ itself being strongly measurable (or even measurable).
  Such a function does not qualify for membership in $L^p(X)$.
  See Exercise \ref{ex:Lp-issues}.
\end{rmk}

In the scalar-valued setting, the simple functions are dense in $L^p$-spaces, and for $p < \infty$ the same holds for Bochner spaces. 

\begin{prop}\label{prop:simple-density}
  Let $(S,\mc{A},\mu)$ be a measure space and $X$ a Banach space.
  Then for $p \in [1,\infty)$, the subspace of simple functions $\Simp(S;X) \cap L^p(S;X)$ is dense in $L^p(S;X)$.
\end{prop}

\begin{proof}
  Fix $\mb{f} \in L^p(S;X)$.
  Since $\mb{f}$ is strongly measurable, there exists a sequence of simple functions $\mb{f}_n \in \Simp(S;X)$ with $\lim_{n \to \infty} \mb{f}_n = \mb{f}$ pointwise almost everywhere.
  Now set
  \begin{equation*}
    \mb{g}_n := \1_{ \{s \in S : \|\mb{f}_n(s)\|_X \leq 2\|\mb{f}(s)\|_X\} } \mb{f}_n.
  \end{equation*}
  The functions $\mb{g}_n$ are simple, and they converge to $\mb{f}$ pointwise almost everywhere.
  Furthermore we have
  \begin{equation*}
    \|\mb{g}_n\|_{L^p(S;X)}^p = \int_{\{s \in S : \|\mb{f}_n(s)\|_X \leq 2\|\mb{f}(s)\|_X \} } \|\mb{f}_n(s)\|_X^p \, \dd\mu(s) \leq 2^p \|\mb{f}\|_{L^p(S;X)}^p,
  \end{equation*}
  so each $\mb{g}_n$ is in $L^p(S;X)$.
  Since $\|\mb{f}(s) - \mb{g}_n(s)\|_X \leq 3\|\mb{f}(s)\|_X$ for almost all $s$, dominated convergence yields
  \begin{equation*}
    \begin{aligned}
      \lim_{n \to \infty} \|\mb{f} - \mb{g}_n\|_{L^p(S;X)}^p &= \lim_{n \to \infty} \int_S \|\mb{f}(s) - \mb{g}_n(s)\|_{X}^p \, \dd\mu(s) \\
      &= \int_S \lim_{n \to \infty}  \|\mb{f}(s) - \mb{g}_n(s)\|_{X}^p \, \dd\mu(s) = 0,
    \end{aligned}
  \end{equation*}
  so $\mb{g}_n \to \mb{f}$ in $L^p(S;X)$, completing the proof.
\end{proof}


\begin{rmk}
  Proposition \ref{prop:simple-density} can be extended to more general dense subspaces of $L^p(S)$; see Exercise \ref{ex:general-density}.
\end{rmk}

Note that the case $p = \infty$ is not included in the proposition above, even though the simple functions are dense in $L^\infty$.
This is because the density of simple functions in $L^\infty(S;X)$ (for a sufficiently rich measure space) is equivalent to the compactness of the unit ball of $X$, which is equivalent to the finite dimensionality of $X$.
We will prove this in the special case where $S = \N$, but the proof can be extended to any measure space containing infinitely many disjoint measurable sets of positive measure.

\begin{prop}
  Let $X$ be a Banach space.
  Then the simple functions are dense in $\ell^\infty(\N;X)$ if and only if $X$ is finite dimensional.
\end{prop}

\begin{proof}
  First suppose $X$ is finite dimensional, and fix $\mb{f} \in \ell^\infty(\N;X)$ and $\varepsilon > 0$.
  Let $C = \|\mb{f}\|_{\ell^\infty(\N;X)}$.
  By compactness of the closed ball $\overline{B_C(0)} \subset X$, there exists a finite collection of vectors $(\mb{x}_i)_{i=1}^N$ in $\overline{B_C(0)}$ such that the open balls $B_{\varepsilon}(\mb{x}_i)$ cover $\overline{B_C(0)}$.
  For each $n \in \N$ we thus have that $\mb{f}(n) \in B_{\varepsilon}(\mb{x}_{i(n)})$ for some $i(n) \in \{1,\ldots,N\}$.
  Define a function $\map{\mb{f}_{\varepsilon}}{\N}{X}$ by
  \begin{equation*}
    \mb{f}_{\varepsilon}(n) := \mb{x}_{i(n)}.
  \end{equation*}
  Since the range of $\mb{f}_{\varepsilon}$ is finite, $\mb{f}_{\varepsilon}$ is simple.
  Furthermore, since $\mb{f}(n) \in B_{\varepsilon}(\mb{x}_{i(n)})$ for each $n \in \N$ we have
  \begin{equation*}
    \|\mb{f} - \mb{f}_{\varepsilon}\|_{\ell^\infty(\N;X)} = \sup_{n \in \N} \|\mb{f}(n) - \mb{x}_{i(n)}\|_{X} \leq \varepsilon.
  \end{equation*}
  Since $\varepsilon > 0$ was arbitrary, we have established density of the simple functions in $\ell^\infty(\N;X)$ when $X$ is finite dimensional.

  Now we prove the converse.
  Aiming for a contradiction, suppose that $X$ is infinite dimensional.
  Then there exists a sequence $(\mb{a}_n)_{n \in \N}$ of unit vectors in $X$ such that
  \begin{equation*}
    \|\mb{a}_n - \mb{a}_m\|_X > 1/2 \qquad \text{for all $n \neq m$.}
  \end{equation*}
  Now let $\mb{f}(n) = \mb{a}_n$ for all $n \in \N$, so that $\mb{f} \in \ell^\infty(\N;X)$, and suppose that there exists a simple function $\mb{g} \in \Simp(\N;X)$ with $\|\mb{f} - \mb{g}\|_{\ell^\infty(\N;X)} < 1/4$.
  Then for all $n \neq m$ we must have
  \begin{equation*}
    \begin{aligned}
      \|\mb{a}_n - \mb{a}_m\|_X &\leq \|\mb{f}(n) - \mb{g}(n)\|_X + \|\mb{g}(n) - \mb{g}(m)\|_X + \|\mb{g}(m) - \mb{f}(m)\|_X \\
      &\leq \frac{1}{2} + \|\mb{g}(n) - \mb{g}(m)\|_X,
    \end{aligned}
  \end{equation*}
  so that
  \begin{equation*}
    \|\mb{g}(n) - \mb{g}(m)\|_X \geq \|\mb{a}_n - \mb{a}_m\|_X - \frac{1}{2} > 0.
  \end{equation*}
  It follows that $\mb{g}$ has infinite range, contradicting the assumption that $\mb{g}$ is simple.
\end{proof}

Now we present some elementary duality results.
Fix an exponent $p \in [1,\infty]$, and recall the definition of the H\"older conjugate exponent $p' = p/(p-1)$, with $1' = \infty$ and $\infty' = 1$.
Given a measure space $(S,\mc{A},\mu)$ and a Banach space $X$, every function $\mb{g} \in L^{p'}(S;X^*)$ induces a bounded linear functional $\Phi \mb{g} \in L^p(S;X)^*$ by integration of the duality pairing between $X$ and $X^*$:
\begin{equation*}
  \Phi \mb{g}(\mb{f}) := \int_S \langle \mb{f}(s), \mb{g}(s) \rangle \, \dd\mu(s) \in \K \qquad \forall \mb{f} \in L^p(S;X).
\end{equation*}
H\"older's inequality implies that $\|\Phi \mb{g}\|_{L^p(S;X)^*} \leq \|\mb{g}\|_{L^{p'}(S;X^*)}$.
In the scalar case $X = \K$, for $p \in [1,\infty)$, $\Phi$ is an isometric isomorphism $L^{p'}(S) \cong L^p(S)^*$: that is, every functional $\phi \in L^p(S)^*$ is of the form $\phi = \Phi g$ for some $g \in L^{p'}(S)$, and furthermore $\|\phi\|_{L^p(S)^*} = \|g\|_{L^{p'}(S)}$.
We will see in Chapter \ref{sec:RNP} that for general Banach spaces $X$, $\Phi$ is an isometric isomorphism if and only if $X$ has the \emph{Radon--Nikodym property} with respect to the measure space $(S,\mc{A},\mu)$.
For now we will establish a duality result that holds with no additional assumptions on the Banach space.

\begin{prop}\label{prop:bochner-preduality}
  Let $(S,\mc{A},\mu)$ be a measure space and $X$ a Banach space.
  Then for all $1 \leq p \leq \infty$, the map $\map{\Phi}{L^{p'}(S;X^*)}{L^p(S;X)^*}$ is an isometry onto a closed subspace of $L^p(S;X)^*$ which is norming for $L^p(S;X)$: that is, for every $\mb{f} \in L^p(S;X)$,
  \begin{equation*}
    \|\mb{f}\|_{L^p(S;X)} = \sup_{\substack{\mb{g} \in L^{p'}(S;X^*) \\ \|\mb{g}\| = 1}} \Big| \int_S \langle \mb{f}(s), \mb{g}(s) \rangle \, \dd\mu(s) \Big|.
  \end{equation*}
\end{prop}

\begin{proof}
  To show that $\Phi$ is an isometry, it suffices to show that $\|\Phi \mb{g}\|_{L^p(S;X)^*} \geq 1$ whenever $\mb{g} \in L^{p'}(S;X^*)$ with $\|\mb{g}\|_{L^{p'}(S;X^*)} = 1$ (we've already discussed the reverse estimate).
  
  \textbf{Mild case: $p > 1$.}
  In this case we have $p' < \infty$, so by density of the simple functions in $L^{p'}(S;X^*)$ and continuity of $\Phi$ we may assume that $\mb{g}$ is simple, i.e.
  \begin{equation*}
    \mb{g} = \sum_{n=1}^N \1_{S_n} \otimes \mb{x}_n^*
  \end{equation*}
  for some pairwise disjoint sets $S_n \in \mc{A}$ with $\mu(S_n) < \infty$ and some nonzero vectors $\mb{x}_n^* \in X^*$.
  Let $\varepsilon > 0$, and choose unit vectors $\mb{x}_n \in X$ (depending on $\varepsilon$) such that
  \begin{equation*}
    \langle \mb{x}_n, \mb{x}_n^* \rangle \geq (1-\varepsilon)\|\mb{x}_n^*\|_{X^*} \qquad \forall n \in \{1,\ldots,N\}.
  \end{equation*}
  Use these vectors to define a test function
  \begin{equation*}
    \mb{f}_\varepsilon := \sum_{n=1}^N \1_{S_n} \otimes \|\mb{x}_n^*\|_{X^*}^{p' - 1} \mb{x}_n.
  \end{equation*}
  This function satisfies
  \begin{equation*}
    \begin{aligned}
      \|\mb{f}_\varepsilon\|_{L^p(S;X)}^p &= \sum_{n=1}^N \mu(S_n) \|\mb{x}_n^*\|_{X^*}^{p(p' - 1)} \|\mb{x}_n\|_{X}^p \\
      &= \sum_{n=1}^N \mu(S_n) \|\mb{x}_n^*\|_{X^*}^{p'}
      = \|\mb{g}\|_{L^{p'}(S;X^*)}^{p'} = 1
    \end{aligned}
  \end{equation*}
  as the $\mb{x}_n$ are unit vectors and $p(p' - 1) = 1$.
  Testing $\Phi \mb{g}$ against $\mb{f}_{\varepsilon}$ yields
  \begin{equation*}
    \begin{aligned}
      \Phi \mb{g}(\mb{f}_\varepsilon) = \sum_{n=1}^N \mu(S_n) \|\mb{x}_n^*\|_{X^*}^{p' - 1} \langle \mb{x}_n, \mb{x}_n^* \rangle
      &\geq (1-\varepsilon) \sum_{n=1}^N \mu(S_n) \|\mb{x}_n^*\|_{X^*}^{p'} \\
      &= (1-\varepsilon) \|\mb{g}\|_{L^{p'}(S;X^*)} = 1-\varepsilon.
    \end{aligned}
  \end{equation*}
  This proves that $\|\Phi \mb{g}\|_{L^p(S;X)^*} \geq 1-\varepsilon$.
  Since $\varepsilon > 0$ was arbitrary, we find that $\|\Phi \mb{g}\|_{L^p(S;X)^*} \geq 1$ as intended.

  \textbf{Spicy case: $p=1$.}
  Fix $\varepsilon > 0$ and define
  \begin{equation}
    A_{\varepsilon} := \{s \in S : \|\mb{g}(s)\|_{X^*} > 1-\varepsilon\},
  \end{equation}
  recalling that we assumed the normalisation $\|\mb{g}\|_{L^\infty(S;X^{*})} = 1$.
  Then $\mu(A_\varepsilon) > 0$, but we could run into the problem that $\mu(A_\varepsilon) = \infty$.
  Since $S$ is $\sigma$-finite,\footnote{Recall that we always assume this without mention.} we can write $S$ as an increasing union of sets of finite measure
  \begin{equation*}
    S = \bigcup_{n \in \N} S_n, \qquad S_n \subset S_{n+1}, \, \mu(S_n) < \infty \quad \forall n \in \N,
  \end{equation*}
  and thus for sufficiently large $n$ the set
  \begin{equation*}
    A_{\varepsilon}^{n} := \{s \in S_n : \|\mb{g}(s)\|_{X^*} > 1 - \varepsilon\}
  \end{equation*}
  satisfies $0 < \mu(A_\varepsilon) < \infty$.\footnote{The main issue is that $A_{\varepsilon}^{n}$ could have zero measure, but if this were true for all $n$, then $\mu(A_{\varepsilon}) = \sup_{n \in \N} \mu(A_{\varepsilon}^n) = 0$, which is a contradiction.}
  Let $B_\varepsilon = A_{\varepsilon}^{n}$ for such a large $n$.

  Since $\mb{g}$ is strongly measurable, the Pettis measurability theorem says that $\mb{g}(B_\varepsilon)$ is separable, and thus there exists a sequence $(\mb{x}^*_{k})_{k \in \N}$ in $X^*$ such that
  \begin{equation*}
    \mb{g}(B_\varepsilon) \subset \bigcup_{k \in \N} B_{\varepsilon}(\mb{x}^{*}_{k})
  \end{equation*}
  and thus
  \begin{equation*}
    B_{\varepsilon} \subset \bigcup_{k \in \N} \mb{g}^{-1}(B_{\varepsilon}(\mb{x}^{*}_{k})).
  \end{equation*}
  Since $\mu(B_{\varepsilon}) > 0$, there exists a vector $\mb{x}^* \in X^*$ (i.e. one of the vectors $\mb{x}_k^{*}$) such that the set
  \begin{equation*}
     B_{\varepsilon, \mb{x}^*} := B_\varepsilon \cap \mb{g}^{-1}(B_{\varepsilon}(\mb{x}^{*})) = \{s \in B_\varepsilon : \|\mb{g}(s) - \mb{x}^{*}\|_{X^*} < \varepsilon\}
  \end{equation*}
  has positive measure.
  Picking a point $s_0 \in B_{\varepsilon, \mb{x}^*}$ and using the definition of $B_{\varepsilon}$, we see that
  \begin{equation*}
    \|\mb{x}^*\|_{X^*} \geq \|\mb{g}(s_0)\|_{X^*} - \|\mb{g}(s_0) - \mb{x}^*\|_{X^*} > 1 - 2\varepsilon.
  \end{equation*}

  Now fix a unit vector $\mb{x} \in X$ such that $\langle \mb{x}, \mb{x}^* \rangle \geq \|\mb{x}^*\|_{X^*} - \varepsilon$, and consider the test function
  \begin{equation*}
    \mb{f}_{\varepsilon} := \1_{B_{\varepsilon, \mb{x}^*}} \otimes \mu(B_{\varepsilon, \mb{x}^*})^{-1} \mb{x}.
  \end{equation*}
  Then $\|\mb{f}_{\varepsilon}\|_{L^1(S;X)} = 1$, and
  \begin{equation*}
    \begin{aligned}
      |\Phi \mb{g}(\mb{f})| &=  \Big| \fint_{B_{\varepsilon, \mb{x}^*}} \langle \mb{x}, \mb{g}(s) \rangle \, \dd\mu(s) \Big| \\
      &\geq \Big| \fint_{B_{\varepsilon, \mb{x}^*}} \langle \mb{x}, \mb{x}^* \rangle \, \dd\mu(s) \Big| - \Big| \fint_{B_{\varepsilon, \mb{x}^*}} \langle \mb{x}, \mb{g}(s) - \mb{x} \rangle  \, \dd\mu(s) \Big| \\
      &\geq (\|\mb{x}^*\|_{X^*} - \varepsilon) - \varepsilon \geq 1 - 4\varepsilon.
    \end{aligned}
  \end{equation*}
  Since $\varepsilon > 0$ was arbitrary, we get $\|\Phi \mb{g}\|_{L^1(S;X)^*} \geq 1$, as we wanted.

  \textbf{Norming property:}
  This follows from the fact that $\Phi$ is an isometry, arguing via the double dual of $X$.
  Let $\map{j}{X}{X^{**}}$ denote the canonical isometric embedding of $X$ into its double dual.
  Then we can write
  \begin{equation*}
    \begin{aligned}
      \|\mb{f}\|_{L^p(S;X)} = \|j \circ \mb{f}\|_{L^p(S;(X^*)^*)} &= \|\Phi (j \circ \mb{f})\|_{L^{p'}(S; X^*)^*} \\
      &= \sup_{\substack{\mb{g} \in L^{p'}(S;X^*) \\ \|\mb{g}\| = 1}} \int_{S} |\langle \mb{g}(s) , j(\mb{f}(s)) \rangle| \, \dd\mu(s) \\
      &= \sup_{\mb{g}} \int_{S} |\langle \mb{f}(s) , \mb{g}(s) \rangle| \, \dd\mu(s),
    \end{aligned}
  \end{equation*}
  completing the proof.
\end{proof}

\section{The Bochner integral}

In the previous section we defined spaces of vector-valued functions satisfying integrability conditions.
It still remains to actually define \emph{integrals} of vector-valued functions.
In the introduction we gave a simple definition for finite dimensonal vector spaces by using basis expansions and linearity, but the world of infinite dimensional spaces is too complicated for such a simple method.
The key to our definition will be the density of the simple functions in $L^1(X)$: this will let us define our integral on simple functions (which are simple) and then extend by continuity.

Let $(S,\mc{A},\mu)$ be a measure space and $X$ a Banach space.
If $\mb{f} \in \Simp(S;X)$ is a simple function represented as
\begin{equation*}
  \mb{f} = \sum_{n=1}^N \1_{S_n} \otimes \mb{x}_n ,
\end{equation*}
and if in addition $\mb{f} \in L^1(\mu;X)$, we define the \emph{Bochner integral}
\begin{equation*}
  \int_S \mb{f} \, \dd\mu = \int_S \mb{f}(s) \, \dd \mu(s) := \sum_{n=1}^N \mu(S_n) \mb{x}_n \in X.
\end{equation*}
Note that $\mb{f} \in L^1(\mu;X)$ if and only if $\mu(S_n) < \infty$ for all $n$.
The Bochner integral is a linear map $\Simp(S;X) \cap L^1(\mu;X) \to X$, and for $\mb{f}$ as above it satisfies
\begin{equation*}
  \Big\| \int_S \mb{f} \, \dd \mu \Big\|_X \leq \sum_{n=1}^{N} |\mu(S_n)| \|\mb{x}_n\|_X = \|\mb{f}\|_{L^1(\mu;X)}.
\end{equation*}
Thus by density of $\Simp(S;X) \cap L^1(\mu;X)$ in $L^1(\mu;X)$ (Proposition \ref{prop:simple-density}), the Bochner integral extends to a bounded linear map $L^1(\mu;X) \to X$ which we continue to call the Bochner integral and denote by the same symbol.
That is, the Bochner integral of a general function $\mb{g} \in L^1(\mu;X)$ is given by 
\begin{equation*}
  \int_S \mb{g} \, \dd\mu := \lim_{n \to \infty} \int_S \mb{f}_n \, \dd\mu \in X
\end{equation*}
where $\mb{f}_n \in \Simp(S;X) \cap L^1(\mu;X)$ for all $n \in \N$ and $\mb{f}_n \to \mb{g}$ in $L^1(\mu;X)$.
We refer to functions in $L^1(\mu,X)$ as \emph{Bochner integrable}.

The Bochner integral satisfies various familiar and useful properties.
\begin{prop}
  Let $(S,\mc{A},\mu)$ be a measure space and $X$ a Banach space.
  \begin{description}
  \item[Commutation with linear maps] If $\mb{f} \in L^1(\mu;X)$ and $T \in \Lin(X,Y)$ is a bounded linear map into a Banach space $Y$,
    \begin{equation*}
      T\Big( \int_S \mb{f} \, \dd\mu \Big) = \int_S T\mb{f} \, \dd\mu \in Y,
    \end{equation*}
    where $T\mb{f} \in L^1(\mu;Y)$ is given by $(T\mb{f})(s) = T(\mb{f}(s))$ for almost all $s \in S$.
    In particular, if $\mb{x}^* \in X^* = \Lin(X,\K)$, then
    \begin{equation*}
      \Big\langle \int_S \mb{f} \, \dd\mu, \mb{x}^* \Big\rangle = \int_S \langle \mb{f}(s), \mb{x}^* \rangle \, \dd\mu(s) \in \K.
    \end{equation*}
  \item[Closure] If $\mb{f} \in L^1(\mu;X)$ and $X_0$ is a closed subspace of $X$ such that $f(s) \in X_0$ for almost all $s \in S$, then $\int_S \mb{f} \, \dd\mu \in X_0$.
  \item[Dominated convergence] Let $(\mb{f}_n)_{n \in \N}$ be a sequence in $L^1(\mu;X)$, $\map{\mb{f}}{S}{X}$, and suppose that $\lim_{n \to \infty} \mb{f}_n \aeeq \mb{f}$.
    Suppose that there exists a non-negative $g \in L^1(\mu)$ such that $\|\mb{f}_n\|_X \leq g$ almost everywhere.
    Then $\mb{f} \in L^1(\mu;X)$ and
    \begin{equation*}
      \int_S \mb{f} \, \dd\mu = \lim_{n \to \infty} \int_S \mb{f}_n \, \dd\mu.
    \end{equation*}
  \item[Substitution / Change of Variables]
    Let $(S',\mc{A}')$ be a measurable space and $\map{\phi}{S}{S'}$ a measurable function, and let $\nu = \mu \circ \phi^{-1}$ denote the pushforward measure.
    Suppose $\mb{g} \in L^1(\nu;X)$.
    Then $\mb{g} \circ \phi \in L^1(\mu;X)$, and
    \begin{equation*}
      \int_S \mb{g} \circ \phi \, \dd\mu = \int_{S'} \mb{g} \, \dd\nu.
    \end{equation*}
    
  \end{description}
\end{prop}

\begin{proof}
  \begin{description}
 
    \item[Commutation with linear maps] by continuity it suffices to prove this for simple $\mb{f} \in \Simp(S;X) \cap L^1(\mu;X)$.
    Writing $\mb{f}$ as in \eqref{eqn:simple-function-standard-form} we have 
    \begin{equation*}
        T\Big(\int_S \sum_{n=1}^N \1_{S_n} \otimes \mb{x}_n \, \dd\mu \Big)
        = \sum_{n=1}^N \mu(S_n) \otimes T(\mb{x}_n) 
        = \int_S T\mb{f} \, \dd\mu.
    \end{equation*}
        
  \item[Closure] We may assume that $X_0$ is a proper subspace of $X$, otherwise there is nothing to show.
    Let $\mb{y} \in X \sm X_0$, and by Hahn--Banach\footnote{If you are philosophically opposed to Hahn--Banach, then see \cite[Corollary 1.1.22]{HNVW16} for a proof that avoids it, and promptly stop reading these notes to avoid further frustration.} choose a functional $\mb{x}^* \in X^*$ such that $\langle \mb{y}, \mb{x}^* \rangle = 1$ and $X_0 \subset \ker \mb{x}^*$.
    Then by the commutation property above we have
    \begin{equation*}
      \Big\langle \int_S \mb{f} \, \dd\mu, \mb{x}^* \Big\rangle = \int_S \langle \mb{f}(s), \mb{x}^* \rangle \, \dd\mu(s) = 0
    \end{equation*}
    since $\mb{f}(s) \in X_0$ for almost all $s \in S$.
    Thus $\int_S \mb{f} \, \dd\mu \neq y$.
    Since $y \in X \sm X_0$ was arbitrary, we conclude that $\int_S \mb{f} \, \dd\mu \in X_0$.

  \item[Dominated convergence]
    By continuity of the Bochner integral it suffices to show that $\mb{f} \in L^1(\mu;X)$ and $\mb{f}_n \to \mb{f}$ in $L^1(\mu;X)$.
    The first fact follows from $\|\|\mb{f}\|_{X}\|_{L^1(\mu)} \leq \|g\|_{L^1(\mu)} < \infty$ and the almost-everywhere strong measurability of almost-everywhere limits of strongly measurable functions (Exercise \ref{ex:sm-limits}).
    For the second, since $\|(\mb{f}_n - \mb{f})(s)\|_X \leq 2g(s)$ almost everywhere, we have
    \begin{equation*}
      \lim_{n \to \infty} \int_S \|(\mb{f}_n - \mb{f})(s)\|_X \, \dd\mu(s) = 0
    \end{equation*}
    by dominated convergence for scalar-valued functions.
    
  \item[Substitution]
    First we need to show that $\mb{g} \circ \phi$ is strongly measurable.
    Let
    \begin{equation*}
      \mb{g}_i = \sum_{n=1}^{N_i} \1_{S_{n,i}} \otimes \mb{x}_{n,i}
    \end{equation*}
    be a sequence of simple functions converging to $\mb{g}$ $\mu$-almost everywhere.
    Then
    \begin{equation*}
      \mb{g}_i \circ \phi = \sum_{n=1}^{N_i} (\1_{S_{n,i}} \circ \phi) \otimes \mb{x}_{n,i} = \sum_{n=1}^{N_i} \1_{\phi^{-1}(S_{n,i})} \otimes \mb{x}_{n,i}
    \end{equation*}
    is a sequence of simple functions converging to $\mb{g} \circ \phi$ $\nu$-almost everywhere, so $\mb{g} \circ \phi$ is also strongly measurable.
    The identity for scalar-valued functions
    \begin{equation*}
      \int_{S'} \|\mb{g} \circ \phi(s)\|_X \, \dd\mu(s) = \int_T \|\mb{g}(s)\|_X \, \dd\nu(t)
    \end{equation*}
    implies that $\mb{g} \circ \phi \in L^1(\mu;X)$.
    Finally, for all $\mb{x}^* \in X^*$ we have by the commutation property and the substitution identity for scalar-valued functions
    \begin{equation*}
      \begin{aligned}
        \Big\langle \int_S \mb{g} \circ \phi \, \dd\mu , \mb{x}^* \Big\rangle
        = \int_S \langle \mb{g}(\phi(s)), \mb{x}^* \rangle \, \dd\mu(s)
        &= \int_{S'} \langle \mb{g}(s), \mb{x}^* \rangle \, \dd\nu(t) \\
        &= \Big\langle \int_{S'} \mb{g} \, \dd\nu, \mb{x}^* \Big\rangle,
      \end{aligned}
    \end{equation*}
    which proves the result.
  \end{description}
\end{proof}

There is also a Fubini theorem for Banach-valued functions (but no Tonelli theorem, as we have no concept of a non-negative vector-valued function at this level of generality).\footnote{Recall that we assume every measure space is $\sigma$-finite. Fubini's theorem fails for non-$\sigma$-finite measure spaces, even for scalar-valued functions.}

\begin{prop}[Fubini]
  Let $(S,\mc{A},\mu)$ and $(S',\mc{A}',\mu')$ be measure spaces, and consider the product measure space $(S \times S', \mc{A} \times \mc{A}', \mu \times \mu')$.
  Let $\mb{f} \in L^1(S \times S'; X)$.
  Then
  \begin{itemize}
  \item for almost all $s \in S$ the function $\mb{f}(s,\cdot)$ is in $L^1(S';X)$,
  \item for almost all $s' \in S'$ the function $\mb{f}(\cdot,s')$ is in $L^1(S;X)$,
  \item the functions $\int_{S'} \mb{f}(\cdot,s') \, \dd\mu'(s')$ and $\int_{S} \mb{f}(s,\cdot) \, \dd\mu(s)$ are in $L^1(S;X)$ and $L^1(S';X)$ respectively, and
    \begin{equation}\label{eq:fubini}
      \int_{S \times S'} \mb{f} \, \dd(\mu \times \mu') = \int_{S'} \Big(  \int_S \mb{f}(s,s') \, \dd\mu(s) \Big) \, \dd\mu'(s') = \int_S \Big(\int_{S'} \mb{f}(s,s') \, \dd\mu'(s') \Big) \, \dd\mu(s).
    \end{equation}
  \end{itemize}
\end{prop}

\begin{proof}
  Consider an everywhere-defined representative of $\mb{f}$.
  Since $\mb{f}$ is strongly measurable, by the Pettis measurability theorem (Theorem \ref{thm:Pettis-measurability}), it is weakly measurable and separably valued.
  Thus the functions $\mb{f}(s,\cdot)$ and $\mb{f}(\cdot,s')$ are separably valued for all $s \in S$ and $s' \in S'$, and by the scalar-valued Fubini theorem, they are both weakly measurable.
  Thus $\mb{f}(s,\cdot)$ and $\mb{f}(\cdot,s')$ are strongly measurable.
  Now since the function $(s,s') \mapsto \|\mb{f}(s,s')\|_X$ is integrable, the scalar-valued Fubini theorem implies all of the integrability claims.
  The equalities \eqref{eq:fubini} are proven by scalarisation: for $\mb{x}^* \in X^*$ we have
  \begin{equation*}
    \begin{aligned}
      \Big\langle \int_{S \times S'} \mb{f} \, \dd(\mu \times \mu') , \mb{x}^* \Big\rangle
      &= \int_{S \times S'} \langle \mb{f}(s,s'), \mb{x}^* \rangle \, \dd(\mu \times \mu') \\
      &= \int_{S} \int_{S'} \langle \mb{f}(s,s'), \mb{x}^* \rangle \, \dd\mu'(s') \, \dd\mu(s) \\
      &= \int_{S} \Big\langle \int_{S'} \mb{f}(s,s') \, \dd\mu'(s') , \mb{x}^* \Big\rangle \, \dd\mu(s) \\
      &= \Big\langle \int_S \Big(\int_{S'} \mb{f}(s,s') \, \dd\mu'(s') \Big)\, \dd\mu(s) , \mb{x}^* \Big\rangle
  \end{aligned}
  \end{equation*}
  by the scalar-valued Fubini theorem, and likewise with the roles of $S$ and $S'$ reversed.
\end{proof}

Let's move away from the abstract stuff for a moment and define vector-valued Fourier transforms.
As we described in the introduction, these are defined just like scalar-valued Fourier transforms, but with Bochner integrals replacing Lebesgue integrals.

\begin{defn}
  Let $X$ be a complex Banach space.
  For a Bochner integrable function $\mb{f} \in L^1(\R^d;X)$ we define the \emph{Fourier transform} as the Bochner integral
  \begin{equation*}
    \hat{\mb{f}}(\xi) = \mc{F}(\mb{f})(\xi) := \int_{\R^d} e^{-2\pi i t \cdot \xi} \mb{f}(t)  \, \dd t \in X \qquad \forall \xi \in \R^d. 
  \end{equation*}
  We also define the \emph{inverse Fourier transform} on $\mb{g} \in L^1(\R^d;X)$:
  \begin{equation*}
    \mb{g}^{\vee}(x) = \mc{F}^{-1}(\mb{g})(x) := \int_{\R^d} e^{2\pi i x \cdot \xi} \mb{g}(\xi) \, \dd \xi \in X \qquad \forall x \in \R^d.
  \end{equation*}
  For functions $\mb{f} \in L^1(\T^d ; X)$ on the $d$-torus $\T^d = [0,1]^d$, we use the same notation for the Fourier transform (and its inverse on $\mb{g} \in L^1(\Z^d; X)$)
  \begin{equation*}
    \begin{aligned}
      \hat{\mb{f}}(n) &= \mc{F}(\mb{f})(n) := \int_{\T^d} e^{-2\pi i t \cdot n} \mb{f}(t)  \, \dd t \in X \qquad \forall n \in \Z^d, \\
      \mb{g}^{\vee}(t) &= \mc{F}^{-1}(\mb{g})(t) := \sum_{n \in \Z^d} e^{2\pi i t \cdot n}  \mb{g}(n) \in X \qquad \forall t \in \T^d.
    \end{aligned}
  \end{equation*}
\end{defn}

Note that if $\mb{f} \in L^1(\R^d;X)$, then the function $x \mapsto \mb{f}(x)e^{-2\pi i x \cdot \xi}$ is Bochner integrable for each $\xi \in \R^d$ (see Lemma \ref{cor:strong-meas-meas-mult}), so the definitions above make sense.\footnote{Analogous statements hold for $\T^d$ and $\Z^d$.}
Furthermore
\begin{equation*}
  \|\hat{\mb{f}}(\xi)\|_X \leq \int_{\R^d} \| \mb{f}(x) e^{-2\pi i x \cdot \xi} \|_X \, \dd x = \|\mb{f}\|_{L^1(\R^d;X)}.
\end{equation*}
In fact, $\hat{\mb{f}}$ is strongly measurable, and so the Fourier transform is bounded from $L^1(\R^d;X)$ to $L^\infty(\R^d;X)$ (see Exercise \ref{ex:FT-bounded-1-infty}).
Formally, the Fourier transform and inverse Fourier transform are mutually inverse operators, but to make this statement rigourous we have to restrict to appropriate classes of functions or distributions, which for now we will not do.

\section{Extensions of operators to Bochner spaces}

In the introduction we showed that extending bounded operators on scalar-valued functions to bounded operators on vector-valued functions is a potentially difficult task, and depends strongly on the operators and the Banach spaces under consideration.
Before we can talk about the boundedness of such extensions we need to define the extensions themselves.
This can be done by basis expansions (particularly in the finite dimensional setting, where this suffices), but the `right' definition is through tensor extensions.

\begin{defn}
  For a measurable space $(S,\mc{A})$ and a set $V \subset \Meas(S;\K)$ of $\mc{A}$-measurable scalar-valued functions on $S$, we define the \emph{algebraic tensor product}
  \begin{equation*}
    V \otimes X := \spn\{f \otimes \mb{x} : f \in V, \mb{x} \in X\} \subset \SMeas(S;X).
  \end{equation*}
  That is, $V \otimes X$ is the set of \emph{finite} linear combinations of $X$-valued functions of the form $f \otimes \mb{x}$, where $f$ is a scalar-valued function in $V$ and $\mb{x} \in X$.
  The function $f \otimes \mb{x}$ is called an \emph{elementary tensor}.
  Functions in $V \otimes X$, having finite dimensional range, are automatically strongly measurable.
\end{defn}

For example, when $V$ is the set of characteristic functions of measurable sets, $V \otimes X = \Simp(S;X)$ is the set of $X$-valued simple functions.
Another fundamental example is $V = L^p(S)$ for some $p \in [1,\infty]$.

\begin{prop}\label{prop:ATP-density}
  Let $(S,\mc{A},\mu)$ be a measure space, $X$ a Banach space, and $p \in [1,\infty)$.
  Then $L^p(S) \otimes X$ is a dense subspace of $L^p(S;X)$.
\end{prop}

\begin{proof}
  For $f \in L^p(S)$ and $\mb{x} \in X$ we compute
  \begin{equation*}
    \|f \otimes \mb{x}\|_{L^p(S;X)}^p = \int_S \|f(s)\mb{x}\|_{X}^p \, \dd\mu(s) = \|\mb{x}\|_X^p \|f\|_{L^p(S)}^p,
  \end{equation*}
  so that $f \otimes \mb{x} \in L^p(S;X)$.
  By linearity, this implies $L^p(S) \otimes X$ is contained in $L^p(S;X)$.
  For density, note that $L^p(S) \otimes X$ contains $(\Sigma(S;\K) \cap L^p(S)) \otimes X$, and that
  \begin{equation*}
    (\Sigma(S;\K) \cap L^p(S)) \otimes X = \Sigma(S;X) \cap L^p(S;X),
  \end{equation*}
  as both spaces are equal to the set of simple functions supported on sets of finite measure.
  By Proposition \ref{prop:simple-density}, this space is dense in $L^p(S;X)$, and thus the same is true of $L^p(S) \otimes X$.
\end{proof}

\begin{defn}\label{defn:tensor-exts}
  Let $(S_i,\mc{A}_i,\mu_i)$ ($i \in \{1,2\}$) be measure spaces, $p_1 \in [1,\infty)$ and $p_2 \in [1,\infty]$ (note that $p_{2} = \infty$ is allowed), and consider a bounded linear operator
  \begin{equation*}
    \map{T}{L^{p_1}(S_{1})}{L^{p_2}(S_{2})}
  \end{equation*}
  acting on scalar-valued functions.
  Let $X$ be a Banach space.
  The \emph{tensor extension of $T$ by the identity map $\map{I}{X}{X}$} is the linear map between algebraic tensor products
  \begin{equation*}
    \map{T \otimes I}{L^{p_1}(S_1) \otimes X}{L^{p_2}(S_2) \otimes X}
  \end{equation*}
  satisfying $(T \otimes I)(f \otimes \mb{x}) = (Tf) \otimes \mb{x}$ for all $f \in L^{p_1}(S_1)$ and $\mb{x} \in X$.
\end{defn}

The tensor extension is a well-defined map between algebraic tensor products $L^{p_1}(S_1) \otimes X \to L^{p_2}(S_2) \otimes X$.
By Proposition \ref{prop:ATP-density}, $L^{p_1}(S_1) \otimes X$ is a dense subspace of $L^{p_1}(S_1;X)$, while $L^{p_2}(S_2) \otimes X$ is a subspace of $L^{p_2}(S_2;X)$ (possibly non-dense if $p_2 = \infty$), so if there exists $C < \infty$ such that
\begin{equation}\label{eq:tensor-ext-estimate}
  \|(T \otimes I)\mb{f}\|_{L^{p_2}(S_2;X)} \leq C \|\mb{f}\|_{L^{p_1}(S_1;X)} \qquad \forall \mb{f} \in L^{p_1}(S_1) \otimes X,
\end{equation}
then $T \otimes I$ may be extended to a bounded linear operator $L^{p_1}(S_1;X) \to L^{p_2}(S_2;X)$.

\begin{defn}
  With the notation above, if the estimate \eqref{eq:tensor-ext-estimate} holds, we say that $T$ \emph{admits a bounded $X$-valued extension}, and we denote the continuous extension of $T \otimes I$ by $\td{T}_X$, $\td{T}$, or even just $T$.
\end{defn}
%mk
Writing out a general element $\mb{f} \in L^p(S) \otimes X$ as a linear combination of elementary tensors, we see that $T$ admits a bounded $X$-valued extension if and only if there exists a constant $C < \infty$ such that
\begin{equation}\label{eq:tensor-ext-estimate-full}
  \Big\|\sum_{n=1}^N (Tf_n) \otimes \mb{x}_n\Big\|_{L^{p_2}(S_2;X)} \leq C \Big\|\sum_{n=1}^N f_n \otimes \mb{x}_n\Big\|_{L^{p_1}(S_1;X)}
\end{equation}
for all functions $f_n \in L^{p_1}(S_1)$ and vectors $\mb{x}_n \in X$.
\emph{This estimate does not simply follow from boundedness of $T$.}
It turns out to rely on potentially subtle interactions between the operator $T$ and the Banach space $X$.

\begin{example}
  Fix a measure space $(S,\mc{A},\mu)$ and let $\map{T}{L^1(S)}{\K}$ denote the Lebesgue integral.\footnote{This fits in the scope of Definition \ref{defn:tensor-exts} by considering $\K$ as a Lebesgue space $L^1(\mathrm{pt})$ over a single point. Then $X$ may be identified with the Bochner space $L^1(\mathrm{pt};X)$.}
  Let $X$ be a Banach space.
  Then for all $\mb{f} \in \Simp(S;\K) \otimes X$ we have
  \begin{equation*}
    (T \otimes I)\mb{f} = (T \otimes I)\Big(\sum_{n=1}^N \1_{S_n} \otimes \mb{x}_n \Big) = \sum_{n=1}^N T(\1_{S_n}) \otimes \mb{x}_n = \sum_{n=1}^N \mu(S_n) \otimes \mb{x}_n = \int_S \mb{f} \, \dd\mu,
  \end{equation*}
  so that the tensor extension of the Lebesgue integral agrees with the Bochner integral, which we have already shown maps $L^1(S;X)$ to $X$.
  Thus the Lebesgue integral admits a bounded $X$-valued extension, namely the Bochner integral.
\end{example}

In the example of the Lebesgue integral the Banach space $X$ plays no real role; we will see in Theorem \ref{thm:positive-extensions} that this phenomenon occurs for all \emph{positive} operators.
Before that we record a simple observation: bounds for a tensor extension can be no better than bounds for the original operator.

\begin{prop}\label{prop:lb-ext}
  Fix measure spaces $(S_i,\mc{A}_i,\mu_i)$ ($i \in \{1,2\}$) and exponents $p_1 \in [1,\infty)$, $p_2 \in [1,\infty]$.
  Let $T \in \Lin(L^{p_1}(S),L^{p_2}(S))$ be a bounded linear operator, and let $X$ be a Banach space.
  Then the tensor extension $T \otimes I$ satisfies
  \begin{equation*}
    \|T \otimes I\|_{L^{p_1}(S_1;X) \to L^{p_2}(S_2;X)} \geq \|T\|_{L^{p_1}(S_1) \to L^{p_2}(S_2)}.
  \end{equation*}
\end{prop}

\begin{proof}
  Fix a nonzero vector $\mb{x} \in X$.
  Then for all nonzero $f \in L^{p_1}(S_1)$ we have
  \begin{equation*}
    \begin{aligned}
      \|(T \otimes I)(f \otimes \mb{x})\|_{L^{p_2}(S_2;X)} = \|Tf \otimes \mb{x}\|_{L^{p_2}(S_2;X)} &= \|Tf\|_{L^{p_2}(S_2)} \|\mb{x}\|_X \\
      &= \frac{\|Tf\|_{L^{p_2}(S_2)}}{\|f\|_{L^{p_1}(S_1)}} \|f \otimes \mb{x}\|_{L^{p_1}(S_1;X)}.
    \end{aligned}
  \end{equation*}
  Taking the supremum over all nonzero $f \in L^{p_1}(S_1)$ completes the proof.
\end{proof}

\begin{thm}\label{thm:positive-extensions}
  Fix measure spaces $(S_i,\mc{A}_i,\mu_i)$ ($i \in \{1,2\}$), $p_1 \in [1,\infty)$, and $p_2 \in [1,\infty]$.
  Let $T \in \Lin(L^{p_1}(S_1),L^{p_2}(S_2))$ be a bounded linear operator which is \emph{positive}, i.e. for all non-negative $f \in L^{p_1}(S_1)$, $Tf \in L^{p_2}(S_2)$ is also non-negative.\footnote{When the scalar field $\K$ is $\C$, `non-negative' simply means `real-valued and non-negative'.}
  Then $T$ admits a bounded $X$-valued extension for every Banach space $X$: in fact, we have
  \begin{equation*}
    \|\td{T}\|_{L^{p_1}(S_1;X) \to L^{p_2}(S_2;X)} = \|T\|_{L^{p_1}(S_1) \to L^{p_2}(S_2)}
  \end{equation*}
  and the pointwise estimate
  \begin{equation}\label{eq:positive-pw-est}
    \|\td{T}\mb{f}\|_X \stackrel{\mathrm{a.e.}}{\leq} T(\|\mb{f}\|_X) \qquad \forall \mb{f} \in L^{p_1}(S_1;X).
  \end{equation}
  
\end{thm}

\begin{proof}
  To ease notation we will assume $(S_1, \mc{A}_1, \mu_1) = (S_2, \mc{A}_2, \mu_2)$ and $p_1 = p_2$, and omit the subscripts. The same proof holds in general.
  
  We will show the pointwise estimate in \eqref{eq:positive-pw-est} for all simple functions $\mb{f} \in \Simp(S;X) \cap L^{p}(S;X)$.
  This will imply
  \begin{equation*}
    \begin{aligned}
      \|\td{T}\mb{f}\|_{L^{p}(S;X)} &= \Big( \int_{S} \|\td{T}\mb{f}(s)\|_X^{p} \, \dd\mu(s) \Big)^{1/p} \\
      &\leq \Big( \int_{S} T(\|\mb{f}(s)\|_X)^{p} \, \dd\mu(s) \Big)^{1/p}\\
      &\leq \|T\|_{\Lin(L^{p}(S))} \Big( \int_{S} \|\mb{f}(s)\|_X^{p} \, \dd\mu(s) \Big)^{1/p} \\
      &= \|T\|_{\Lin(L^{p}(S))} \|\mb{f}\|_{L^{p}(S;X)}
    \end{aligned}
  \end{equation*}
  which implies the result by density of $\Simp(S;X) \cap L^{p}(S;X)$ in $L^{p}(S;X)$ (noting that the reverse estimate is shown in Proposition \ref{prop:lb-ext}).

  Now let's prove \eqref{eq:positive-pw-est}.
  Consider a simple function
  \begin{equation*}
    \mb{f} = \sum_{n=1}^N \1_{E_n} \otimes \mb{x}_n
  \end{equation*}
  and note that $|T(\1_{E_n})| \aeeq T(\1_{E_n})$ by positivity of $T$.
  Then
  \begin{equation*}
    \begin{aligned}
      \|\td{T}\mb{f}(s)\|_X &= \Big\| \sum_{n=1}^N T(\1_{E_n})(s) \mb{x}_n \Big\|_X \\
      &\leq \sum_{n=1}^N |T(\1_{E_n})(s)| \|\mb{x}_n\|_X \\
      &\aeeq \sum_{n=1}^N T(\1_{E_n})(s) \|\mb{x}_n\|_X 
      = T\Big( \sum_{n=1}^N \1_{E_n} \|\mb{x}_n\|_X \Big)(s) 
      = T(\|\mb{f}\|_X)(s),
    \end{aligned}
  \end{equation*}
  proving \eqref{eq:positive-pw-est} and completing the proof.
\end{proof}

Theorem \ref{thm:positive-extensions} shows that the `extension problem' for positive operators is not much of a problem: positive operators extend automatically.
Of course, most interesting operators are not positive.

\begin{example}
  The \emph{Hausdorff-Young inequality} says that the Fourier transform $\mc{F}$ on scalar-valued functions is bounded from $L^p(\R)$ to $L^{p'}(\R)$ for all $p \in [1,2]$.\footnote{For $p=2$ this is Plancherel's theorem, and for $p=1$ it is straightforward, and as shown above it even holds for Banach-valued functions. The intermediate result can be proven by interpolation, e.g. by the Riesz--Thorin theorem.}
  Fix $p \in [1,2)$, and consider the Banach space $\ell^p := \ell^{p}(\N)$.
  We will show that for all $r \in (p,2]$, the bound
  \begin{equation}\label{eq:FT-bound-lp}
    \map{\mc{F}}{L^r(\R;\ell^p)}{L^{r'}(\R;\ell^p)}
  \end{equation}
  does not hold.
  Thus the Fourier transform \emph{from $L^r(\R)$ to $L^{r'}(\R)$} does not have a bounded extension to $\ell^p$ for any $1 \leq p < r$.\footnote{On the other hand, it has a bounded extension to $\ell^p$ for $r \leq p \leq 2$. What we are saying is that \emph{$\ell^p$ has Fourier type $r$ for all $r \in [1,p]$}: we will discuss this in more depth in Chapter \ref{sec:fouriertype}.}

  Fix a Schwartz function $f \in \mc{S}(\R)$ supported in the unit interval $(0,1)$, normalised such that $\|f\|_r = 1$.
  For each $N \in \{1,2,\ldots\}$ define a function $\map{\mb{f}_N}{\R}{\ell^p}$ by
  \begin{equation*}
      \mb{f}_N := \sum_{n=0}^{N-1} \Tr_{n}(f) \otimes \mb{e}_n
      = (f, \Tr_{1}(f), \Tr_{2}(f), \ldots, \Tr_{N}(f), 0, 0, \ldots).
  \end{equation*}
  where $(\mb{e}_{n})_{n \in \N}$ are the standard basis elements of $\ell^p$ and $\Tr_{n}f(x) = f(x-n)$ denotes the operator of translation by $n$.
  We can write
  \begin{equation*}
    \begin{aligned}
      \|\mb{f}_N\|_{L^r(\R;\ell^p)}^{r}
      &= \int_{\R} \|\mb{f}_N(x)\|_{\ell^p}^r \, \dd x  \\
      &= \sum_{m=0}^{N-1} \int_{m}^{m+1} \Big(\sum_{n=0}^{N-1} |f(x-n)|^p \Big)^{r/p} \, \dd x \\
      &= \sum_{m=0}^{N-1} \int_{m}^{m+1} |f(x-m)|^r \, \dd x 
      = \sum_{m=0}^{N-1} \int_0^1 |f(x)|^r \, \dd x = N
    \end{aligned}
  \end{equation*}
  using that for $x \in (m,m+1)$ we have $f(x-n) = 0$ unless $n=m$.
  Now using that the Fourier transform of a translated function is a modulation of the Fourier transform, for all $\xi \in \R$ we have
  \begin{equation*}
      \widehat{\mb{f}_N}(\xi) = \sum_{n=0}^{N-1} (\widehat{\Tr_{n}(f)} \otimes \mb{e}_n)(\xi) 
      = \sum_{n=0}^{N-1} e^{-in\xi} \hat{f}(\xi) \mb{e}_n 
      = \hat{f}(\xi) \sum_{n=0}^{N-1} e^{-in\xi} \mb{e}_n,
  \end{equation*}
  so
  \begin{equation*}
    \|\widehat{\mb{f}_N}\|_{L^{r^\prime}(\R;\ell^p)}^{r'}
    = \int_\R |\hat{f}(\xi)|^{r^\prime} \bigg( \sum_{n=0}^{N-1} |e^{-in\xi}|^p \bigg)^{r^\prime/p} \, d\xi 
    = N^{r'/p} \|\hat{f}\|_{L^{r^\prime}(\R)}^{r'}.
  \end{equation*}
  If the bound \eqref{eq:FT-bound-lp} holds, there is a constant $C$ (independent of $N$) such that
  \begin{equation*}
    \|\widehat{\mb{f}_N}\|_{L^{r^\prime}(\R;\ell^p)} \leq C \|\mb{f}_N\|_{L^r(\R;\ell^p)}
  \end{equation*}
  for all $N \geq 1$, but our estimates then imply
  \begin{equation*}
    N^{1/p} \|\hat{f}\|_{r^\prime} \leq C N^{1/r},
  \end{equation*}
  or equivalently $N^{\frac{1}{p} - \frac{1}{r}} \leq C/\|\hat{f}\|_{r^\prime}$ (using that $\|\hat{f}\|_{r^\prime} \neq 0$).
  But since $r > p$, the left hand side is unbounded, while the right hand side is constant.
  This contradiction shows that the bound \eqref{eq:FT-bound-lp} does not hold.
\end{example}


\section*{Exercises}

\begin{exercise}\label{ex:measurability-containments}
  Let $X$ be a Banach space and $(S,\mc{A})$ a measurable space.
  Prove the containments
  \begin{equation*}
    \SMeas(S;X) \subset \Meas(S;X) \subset \WMeas(S;X).
  \end{equation*}
\end{exercise}

\begin{exercise}
  Let $X$ be a Banach space, let $A$ be a topological space, and let $\mu$ be a Borel measure on $A$.
  \begin{itemize}
  \item If $X$ is separable or $A$ is separable, show that $C(A;X)$ is contained in $L^\infty(A, \mu;X)$.
  \item Give an example of a topological space $A$ and a Banach space $X$ such that $C(A;X)$ is not contained in $L^\infty(A,\mu;X)$.
  \end{itemize}
\end{exercise}

\begin{exercise}\label{ex:Lp-issues}
  Give an example of a Banach space $X$, a measure space $(S,\mc{A},\mu)$, and a function $\map{\mb{f}}{S}{X}$ such that $\|\mb{f}\|_{X} \in L^p(S,\mu)$, but $\mb{f} \notin L^p(S,\mu;X)$.
\end{exercise}

\begin{exercise}\label{ex:general-density}
  Let $X$ be a Banach space and $(S,\mc{A},\mu)$ a measure space, and let $p \in [1,\infty)$.
  Let $V$ be a dense subspace of $L^p(S)$.
  Show that $V \otimes X$ is dense in $L^p(S;X)$.
\end{exercise}

\begin{exercise}\label{ex:finite-sigma-alg-duality}
  Let $X$ be a Banach space and $(S,\mc{A},\mu)$ a measure space such that the $\sigma$-algebra $\mc{A}$ is finite.
  Show that the isometric embedding
  \begin{equation*}
    \map{\Phi}{L^{p'}(S,\mc{A},\mu;X^*)}{L^p(S,\mc{A},\mu;X)^*}, \qquad \Phi \mb{g}(\mb{f}) = \int_S \langle \mb{f}(x), \mb{g}(x) \rangle \, \dd\mu(x)
  \end{equation*}
  is an isomorphism for all $p \in [1,\infty]$.
\end{exercise}

\begin{exercise}\label{ex:sm-limits}
  Let $(S,\mc{A},\mu)$ be a measure space and $X$ a Banach space.
  Let $(\mb{f}_{n})_{n \in \N}$ be a sequence of $X$-valued functions and $\map{\mb{f}}{S}{X}$, and suppose that $\mb{f}_{n} \to \mb{f}$ almost everywhere (with respect to $\mu$).
  Suppose that for each $n \in \N$ there exists a set $N_{n} \subset S$ with $\mu(N_{n}) = 0$ such that $\mb{f}_{n}$ is strongly measurable on $S \sm N_{n}$.
  Show that there exists a set $N \subset S$ with $\mu(N) = 0$ such that $\mb{f}$ is strongly measurable on $S \sm N$.
  (That is: show that the a.e. limit of a.e. strongly measurable functions is a.e. strongly measurable.)
\end{exercise}

\begin{exercise}\label{ex:tensor-extension-basic-props}
  Let $(S_i, \mc{A}_i, \mu_i)$ ($i \in \{1,2\}$) be measure spaces, let $p_1 \in [1,\infty)$, and let $p_2 \in [1,\infty]$.
  Suppose that $T \in \Lin(L^{p_1}(S_1), L^{p_2}(S_2))$ is a bounded linear operator.
  \begin{itemize}
  \item
    Show that $T$ admits a bounded $X$-valued extension for all finite dimensional Banach spaces $X$.
  \item
    Let $X$ be any Banach space and suppose $\mb{x}^* \in X^*$.
    Show that for all $f \in L^{p_1}(S_1) \otimes X$,
    \begin{equation*}
      \langle (T \otimes I)f, \mb{x}^* \rangle = T(\langle f, \mb{x}^* \rangle).
    \end{equation*}
  \end{itemize}

\end{exercise}

\begin{exercise}\label{ex:tensor-adjoint}
  Let $(S,\mc{A},\mu)$ be a measure space and $p \in (1,\infty)$, let $T$ be a bounded linear operator on $L^p(S,\mu)$, and let $X$ be a Banach space.
  Let $T^*  \in \Lin(L^{p'}(S,\mu))$ denote the adjoint of $T$.
  Show that $T$ admits a bounded $X$-valued extension if and only if $T^*$ admits a bounded $X^*$-valued extension, and show that
  \begin{equation*}
    (\td{T}_X)^* \Phi \mb{g} = \td{(T^*)}_{X^*} \mb{g}
  \end{equation*}
  for all $\mb{g} \in L^{p'}(S;X^*)$, where $\map{\Phi}{L^{p'}(S;X^*)}{L^p(S;X)^*}$ is as in Proposition \ref{prop:bochner-preduality}.
  Assuming $T$ admits a bounded $X$-valued extension, conclude that for all $\mb{f} \in L^p(S;X)$ and $\mb{g} \in L^{p'}(S;X)$,
  \begin{equation}\label{eq:tensor-adjoint-identity}
    \langle \td{T}_X \mb{f}, \mb{g} \rangle = \langle \mb{f}, \td{(T^*)}_{X^*} \mb{g} \rangle.
  \end{equation}
\end{exercise}

\begin{exercise}\label{ex:FT-bounded-1-infty}
  Let $X$ be a complex Banach space.
  Show that $\hat{\mb{f}} \in C(\R^d;X)$ for all $\mb{f} \in L^1(\R^d;X)$.
  Conclude that the Fourier transform is bounded from $L^1(\R^d;X)$ to $L^\infty(\R^d;X)$.
\end{exercise}

\begin{exercise}
  Let $X$ and $Y$ be Banach spaces and consider an operator-valued function $\map{M}{\R^d}{\Lin(X,Y)}$, where $\Lin(X,Y)$ is the Banach space of bounded linear operators from $X$ to $Y$.
  Suppose that $M$ is continuous with respect to the strong operator topology on $\Lin(X,Y)$: that is, suppose that for all vectors $\mb{x} \in X$, the map
  \begin{equation*}
    \map{M(\cdot)\mb{x}}{\R^d}{Y}, \qquad \xi \mapsto M(\xi)\mb{x}
  \end{equation*}
  is continuous.
  \begin{itemize}
  \item
    Let $\map{\mb{g}}{\R^d}{X}$ be strongly measurable.
    Show that the function $\map{M\mb{g}}{\R^d}{Y}$ defined by $(M\mb{g})(\xi) := M(\xi)\mb{g}(\xi)$ is strongly measurable.
  \item
    Suppose in addition that the function $\xi \mapsto \|M(\xi)\|_{\Lin(X,Y)}$ is measurable, and that
    \begin{equation*}
      \int_{\R^d} \|M(\xi)\|_{\Lin(X,Y)} \, \dd \xi < \infty.
    \end{equation*}
    Show that the operator $T_M\mb{f} := (M\hat{\mb{f}})^\vee$ is well-defined and bounded from $L^1(\R^d;X)$ to $C(\R^d;Y)$.
  \end{itemize}
\end{exercise}

\begin{exercise}
  Let $H$ be an infinite dimensional separable Hilbert space with inner product $(\cdot , \cdot)$, and let $(S,\mc{A},\mu)$ be a measure space.
  \begin{itemize}
  \item Show that $L^2(\mu;H)$ is a Hilbert space with respect to the inner product
    \begin{equation*}
      (\mb{f}, \mb{g}) := \int_{S} (\mb{f}(s), \mb{g}(s)) \, \dd\mu(s) \qquad (\mb{f},\mb{g} \in L^2(\mu;H)).
    \end{equation*}
  \item Let $(\mb{e}_n)_{n \in \N}$ be an orthonormal basis of $H$ and $(f_n)_{n \in \N}$ an orthonormal basis of $L^2(\mu)$.
    Show that the elementary tensors $\{f_n \otimes \mb{e}_m : n,m \in \N\}$ are an orthonormal basis of $L^2(\mu;H)$.
  \item Suppose that the Hilbert space $H$ is complex.
    Show that the Fourier transform on the torus, initially defined as a bounded operator $\map{\mc{F}}{L^2(\T^d;H)}{\ell^2(\Z^d;H)}$, extends to an isometry from $L^2(\T^d;H)$ to $\ell^2(\Z^d;H)$.
  \end{itemize}
\end{exercise}


%%% Local Variables:
%%% mode: latex
%%% TeX-master: "../main.tex"
%%% End:


\chapter{Probability in Banach spaces}
\label{sec:martingales} 

\todo{Write expository introduction}

\subsection{Gambling in Banach spaces}\label{sec:gambling}

To motivate the theory in this section, we're going to imagine a betting game.
At each turn, you bet on the outcome of a coin toss.
The quantities that you can bet are taken from a Banach space $X$.
The initial state of your wallet, $\mb{s}_{-1}$, is the zero vector
\begin{equation*}
  \mb{s}_{-1} = 0 \in X.
\end{equation*}
At each time $n \in \N = \{0,1,\ldots\}$, you choose a vector $\mb{x}_n \in X$ to wager.
I then flip a fair coin, which shows either Heads or Tails, and the state of your wallet becomes
\begin{equation*}
  \mb{s}_n =
  \begin{cases}
    \mb{s}_{n-1} + \mb{x}_n & \text{if the coin shows Heads} \\
    \mb{s}_{n-1} - \mb{x}_n & \text{if the coin shows Tails.}
  \end{cases}
\end{equation*}
The Banach space $X$ is not ordered, so there is no canonical notion of $\mb{s}_n$ being `more' or `less' than $\mb{s}_{n-1}$. Thus the game is not about winning or losing (the true winner of the game is Functional Analysis).
In this chapter we will discuss various probabilistic concepts that can be well-understood in the context of this game.


\subsection{Filtrations and stochastic processes}

\begin{defn}
  A \emph{filtration} on a probability space $(\Omega,\mc{A},\P)$ is a monotone increasing (i.e. nondecreasing) sequence of $\sigma$-subalgebras
  \begin{equation*}
    \mc{A}_0 \subseteq \mc{A}_1 \subseteq \mc{A}_2 \subseteq \cdots \subseteq \mc{A}.
  \end{equation*}
\end{defn}

Filtrations are closely linked with stochastic processes.
While we don't plan on saying anything really serious about these in this course, it will be useful to keep the core concept in mind, as it guides a lot of probabilistic intuition.

\begin{defn}
  Let $(\Omega,\mc{A},\P)$ be a probability space and $X$ a Banach space.
  A \emph{(discrete-time) $X$-valued stochastic process} on $(\Omega,\mc{A},\P)$ is a sequence of $\mc{A}$-measurable random variables $\map{f_n}{\Omega}{X}$, $n \in \N$.
  Given a filtration $(\mc{A}_n)_{n \in \N}$, a stochastic process $(f_n)_{n \in \N}$ is called \emph{predictable} (with respect to the filtration) if each $f_n$ is $\mc{A}_{n-1}$-measurable (with the convention that $\mc{A}_{-1} = \{\varnothing, \Omega\}$).
\end{defn}

\begin{rmk}
  With obvious modifications one can talk about filtrations and stochastic processes starting at an arbitrary index, finite filtrations/processes, or filtrations/processes with respect to arbitrary (total or partial) orders, for example with a continuous time index.
  In this course we will only consider discrete indexing sets contained in $\N$.
\end{rmk}

One should think of a filtration $(\mc{A}_n)_{n=0}^\infty$ as representing the progression of available information over time, usually in relation to a stochastic process.
Each $\sigma$-subalgebra $\mc{A}_n \subset \mc{A}$ represents the information available at time $n$.
There are two equivalent ways of thinking about the availability of information: one is that at time $n$ one has access to all $\mc{A}_n$-measurable subsets; the other is that at time $n$ one has access to all $\mc{A}_n$-measurable functions.
The monotonicity assumption says that no information is lost as time progresses.
Predictability of a stochastic process $(f_n)_{n \in \N}$ with respect to the filtration $(\mc{A}_n)_{n \in \N}$ thus says the following: if the available information is represented by $(\mc{A}_n)_{n \in \N}$, then at each time $n$, one already has access to the $\mc{A}_n$-measurable function $f_{n+1}$.

\begin{example}\label{eg:filtration-generated-by-process}
  Let $(\Omega,\mc{A},\P)$ be a probability space, $X$ a Banach space, and let $(f_n)_{n \in \N}$ be an $X$-valued stochastic process on $(\Omega,\mc{A},\P)$.
  The \emph{filtration generated by the process $(f_n)_{n \in \N}$} is given by
  \begin{equation*}
    \mc{F}_n := \sigma(f_0,f_1,\ldots,f_n) \qquad \forall n \in \N.
  \end{equation*}
  The information-theoretic intuition says that at time $n \in \N$, one `knows' the functions $f_0, f_1, \ldots f_n$, as these are in $\mc{F}_n$, and one also knows all functions of the form
  \begin{equation*}
    g \circ (f_0, f_1, \ldots, f_n) \colon \omega \mapsto g(f_0(\omega),f_1(\omega),\ldots,f_n(\omega))
  \end{equation*}
  where $\map{g}{X^{n+1}}{\C}$ is measurable (as such compositions are automatically measurable).
  In fact, all $\mc{F}_n$-measurable functions $\Omega \to \C$ are of this form.\todo{cite a reference} %mk
\end{example}

\begin{example}\label{eg:gambling-filtrations}
  Consider the game we introduced in Section \ref{sec:gambling}.
  At each time $n \in \N$ I flip a fair coin, which comes up Heads ($H$) or Tails $(T)$ with equal probability.
  The natural probability space on which to base this game is the infinite product $\Omega = \{-1,+1\}^{\N}$, where each factor has the uniform probability measure, and the product has the natural product $\sigma$-algebra and probability measure.
  The value $-1$ represents Tails, while $+1$ represents Heads.
  For each $n \in \N$ let $\map{\pi_n}{\Omega}{\{-1,+1\}}$ be the $n$-th coordinate function, which represents the outcome of the $n$-th coin toss.
  The sequence of functions $(\pi_n)_{n \in \N}$ is then a scalar-valued stochastic process.
  Let $\mc{F}_n = \sigma(\pi_0,\pi_1,\ldots,\pi_n)$ be the filtration generated by this process.

  
  Your bet at time $n$, the vector $\mb{x}_n \in X$, is allowed to depend on the outcomes $\pi_0, \pi_1, \ldots, \pi_{n-1}$: you do not need to register all your bets in advance.
  In probabilistic language, $\map{\mb{x}_n}{\Omega}{X}$ is $\mc{F}_{n-1}$-measurable, i.e. the sequence $(\mb{x}_n)_{n \in \N}$ is a stochastic process which is predictable with respect to the filtration $(\mc{F}_n)_{n \in \N}$.

  Now consider the stochastic process $(\mb{s}_n)_{n \in \N}$, representing the evolution of the state of your wallet.
  By definition we have
  \begin{equation*}
    \mb{s}_{n+1} = \mb{s}_n + \pi_{n+1} \mb{x}_{n+1} \qquad \forall n \in \N;
  \end{equation*}
  keep in mind that this is an equality of $X$-valued random variables, i.e. functions $\Omega \to X$.
  Since $\mb{s}_n$, $\pi_{n+1}$, and $\mb{x}_{n+1}$ are all $\mc{F}_{n+1}$-measurable, we find that $\mb{s}_{n+1}$ is $\mc{F}_{n+1}$-measurable (i.e. we know the state of our wallet $\mb{s}_{n+1}$ at time $n+1$).
  Heuristically, $\mb{s}_{n+1}$ should not be $\mc{F}_n$-measurable unless $\mb{x}_{n+1} \equiv 0$, as this would amount to predicting the future (which can only be done by wagering nothing).
  You should prove this rigourously (Exercise \ref{ex:winnings-unpredictability}).
\end{example}


\begin{defn}
  Given a filtration $(\mc{A}_n)_{n=0}^\infty$ on a probability space $(\Omega, \mc{A}, \P)$, a random variable $\map{T}{\Omega}{\N \cup \{\infty\}}$ is called a \emph{stopping time} (with respect to $(\mc{A}_n)$) if 
  \begin{equation*}
    \{\omega \in \Omega: T(\omega) \leq n\} \in \mc{A}_n \qquad \forall n \geq 0.
  \end{equation*}
  The stopping time $T$ is \emph{finite} if $T$ is almost surely finite.
\end{defn}

Generally stopping times $T$ are defined in terms of some kind of stochastic \emph{stopping condition}.
Interpreting the filtration $(\mc{A}_n)_{n \in \N}$ as modelling the available information at time $n$, $T$ being a stopping time says precisely that at time $n$, one `knows' the set of points $\omega \in \Omega$ for which $T(\omega) \leq n$.
Said less precisely, if $T$ is a stopping time, then at time $n$, one can determine whether or not $T \leq n$.

\begin{example}\label{eg:gambling-stoppingtimes}
  We return to the betting game of Section \ref{sec:gambling}, elaborated upon in Example \ref{eg:gambling-filtrations}.
  Let's suppose that our goal is to get the state of our wallet $\mb{s} \in X$ into a fixed Borel measurable set $K \subset X$, and that we intend to stop betting once this condition holds (i.e. from that point on we only wager the zero vector).
  
  Let
  \begin{equation*}
    T_K(\omega) := \inf\{n \in \N : \mb{s}_n(\omega) \in K \}
  \end{equation*}
  with the usual convention that $T_K(\omega) = \infty$ if $\mb{s}_n(\omega) \notin K$ for all $n \in \N$.
  That is, $T_K$ is the first time $n$ at which $\mb{s}_n \in K$.
  At time $n$ we heuristically know whether or not our wallet satisfied $\mb{s}_m \in K$ for some $m \leq n$, which indicates that $T_K$ should be a stopping time with respect to the filtration $(\mc{F}_n)_{n \in \N}$ associated with the stochastic process $(\pi_n)_{n \in \N}$.
  Rigourously, one shows this by writing for all $n \in \N$
  \begin{equation*}
    \begin{aligned}
      \{\omega \in \Omega : T_K(\omega) \leq n\}
      &= \big\{\omega : \inf\{m : \mb{s}_m(\omega) \in K\} \leq n\big\} \\
      &= \{\omega  : \text{$\mb{s}_m(\omega) \in K$ for some $m \leq n$}\} \\
      &= \bigcup_{m = 0}^n \mb{s}_m^{-1}(K),
    \end{aligned}
  \end{equation*}
  and noting that since each $\mb{s}_m$ is $\mc{F}_m$-measurable, the set above is $\mc{F}_n$-measurable.
  Thus $T_K$ is a stopping time.
  Of course, whether $T_K$ is a finite stopping time depends on the set $K \subset X$, the wager vectors $(\mb{x}_n)_{n \in \N}$, and potentially even the geometry of $X$ (see Exercise \ref{ex:gambling-in-linfty}).
\end{example}

The proof of the following proposition was already done in the previous exercise for the a particular stochastic process, but the proof is identical for a general stochastic process.

\begin{prop}
  Let $(\Omega, \mc{A}, \P)$ be a probability space and $X$ a Banach space.
  Let $(f_n)_{n \in \N}$ be an $X$-valued stochastic process and $K \subset X$ a Borel measurable set.
  Then the function $\map{T_K}{\Omega}{\N \cup \{\infty\}}$ defined by
  \begin{equation*}
    T_K(\omega) := \inf\{n \in \N : f_n(\omega) \in K \}
  \end{equation*}
  is a stopping time.
\end{prop}

The stopping time $T_K$ defined above is called the \emph{first hitting time of $K$}.

\subsection{Conditional expectations}

\todo{a bit of exposition; include defn of $\E$ and talk about `best approximation with given information'}

\begin{defn}\label{defn:conditional-expectation} %Dudley, p336
  Let $(\Omega, \mc{A}, \P)$ be a probability space and $X$ a Banach space.
  Let $f \in L^1(\Omega, \mc{A}; X)$ be a integrable $X$-valued random variable.
  Given a $\sigma$-subalgebra $\mc{B} \subset \mc{A}$, a \emph{conditional expectation of $f$ given $\mc{B}$} is a $\mc{B}$-measurable random variable $\E^{\mc{B}}f \in L^1(\Omega, \mc{B}; X) \subset L^1(\Omega,\mc{A};X)$ such that
  \begin{equation}\label{eq:conditional-expectation-property}
    \qquad \int_B \E^{\mc{B}}f \, \dd\P = \int_B f \, \dd\P \qquad \text{for all $B \in \mc{B}$.}
  \end{equation}
\end{defn}

\begin{example}
  Let $(\Omega,\mc{A},\P)$ be a probability space and let $\mc{B} \subset \mc{A}$ be a sub-$\sigma$-algebra which is \emph{atomic}, in the sense that there is a collection of pairwise disjoint subsets $(B_\lambda)_{\lambda \in \Lambda}$ of $\mc{B}$ which generate $\mc{B}$, such that $\P(B_{\lambda}) > 0$ for all $\lambda$, and such that if $B_\lambda$ can be written as a disjoint union $B_\lambda = C \cup D$ for some sets $C,D \in \mc{B}$, then $\P(C) = 0$ or $\P(D) = 0$ (i.e. the sets $B_\lambda$ are \emph{atoms}).
  Let's compute \emph{the} conditional expectation $\E^{\mc{B}}f$ of an integrable random variable $f \in L^1(\mc{A};X)$ (it turns out there is only one).
  Since the atoms $(B_\lambda)_{\lambda}$ generate $\mc{B}$ and are pairwise disjoint, and since $\E^{\mc{B}}f$ is $\mc{B}$-measurable, $\E^{\mc{B}}f$ must be constant on each $B_\lambda$, so that
  \begin{equation*}
    \E^{\mc{B}} f = \sum_{\lambda \in \Lambda} \1_{B_\lambda} \otimes \mb{x}_\lambda
  \end{equation*}
  for some vectors $\mb{x}_{\lambda} \in X$.
  Averaging over one of the atoms $B_{\lambda}$ and using \eqref{eq:conditional-expectation-property} tells us that
  \begin{equation*}
    \begin{aligned}
      \mb{x}_\lambda = \frac{1}{\P(B_\lambda)} \int_{B_\lambda} \E^{\mc{B}} f \, \dd\P = \frac{1}{\P(B_\lambda)} \int_{B_\lambda} f \, \dd\P.
    \end{aligned}
  \end{equation*}
  In probabilistic terms, the quantity on the right hand side is the conditional expectation of $f$ given $B_\lambda$, which exists since $\P(B_\lambda) > 0$.
\end{example}

The previous example shows that conditional expectations with respect to atomic $\sigma$-algebras exist and are unique.
The same is true for general $\sigma$-algebras, but proving this will take a few steps.
First we will establish a foundational result in the scalar-valued case.

\begin{thm}\label{thm:conditional-expectation-EU}
  Let $(\Omega,\mc{A},\P)$ be a probability space.
  For any $f \in L^1(\mc{A})$ and any $\sigma$-subalgebra $\mc{B} \subset \mc{A}$, a unique (up to almost sure equality) conditional expectation $\E^{\mc{B}}f$ exists.
  Furthermore, for all $p \in [1,\infty]$, $\E^{\mc{B}}$ is a positive contraction on $L^p(\mc{A})$: that is, if $f \in L^p(\mc{A})$, then
  \begin{equation*}
    \|\E^{\mc{B}}f\|_{p} \leq \|f\|_{p},
  \end{equation*}
  and if $f$ is a.s. nonnegative then so is $\E^{\mc{B}}f$.
\end{thm}

\begin{proof}
  First we handle uniqueness.
  For this it suffices to consider the case $\K = \R$, as the case $\K = \C$ follows by considering real and imaginary parts.
  Suppose that $\E^{\mc{B}}f$ and $\td{E}^{\mc{B}}f$ are two conditional expectations of $f$ given $\mc{B}$.
  For all subsets $B \in \mc{B}$ we have
  \begin{equation*}
    \int_B \E^{\mc{B}}f - \td{\E}^{\mc{B}}f \, \dd\P = \int_B f \, \dd\P - \int_B f \, \dd\P = 0.
  \end{equation*}
  Since $\E^{\mc{B}}f - \td{\E}^{\mc{B}}f$ is $\mc{B}$-measurable, the subsets
  \begin{equation*}
    B_+ := \{ \E^{\mc{B}}f - \td{\E}^{\mc{B}}f > 0\} \quad \text{and} \quad B_- := \{\E^{\mc{B}}f - \td{\E}^{\mc{B}}f < 0\}
  \end{equation*}
  are both in $\mc{B}$, so we get
  \begin{equation*}
    \int_\Omega |\E^{\mc{B}}f - \td{\E}^{\mc{B}}f | \, \dd\P
    = \int_{B_+}  \E^{\mc{B}}f - \td{\E}^{\mc{B}}f \, \dd\P
    -  \int_{B_-}  \E^{\mc{B}}f - \td{\E}^{\mc{B}}f \, \dd\P
    = 0,
  \end{equation*}
  establishing that $\E^{\mc{B}}f = \td{\E}^{\mc{B}}f$ almost surely.

  We can prove positivity from the defining property \eqref{eq:conditional-expectation-property}, before we establish existence.
  Suppose $f \in L^1(\mc{A})$ is a.s. nonnegative.
  Then for all $B \in \mc{B}$ we have
  \begin{equation*}
    \int_B \E^{\mc{B}} f \, \dd\P = \int_B f \, \dd\P \geq 0,
  \end{equation*}
  which implies that the $\mc{B}$-measurable function $\E^{\mc{B}}f$ is a.s. nonnegative.\footnote{This uses an exercise from measure theory: if $g$ is $\mc{B}$-measurable and $\int_B g \geq 0$ for all $\mc{B}$-measurable sets, then $g \geq 0$ a.e.. Proof: the set $N := \{g(\omega) < 0\}$ is $\mc{B}$-measurable, and assuming it has positive measure leads to the contradiction $0 \leq \int_B g < 0$.}


  Now we prove existence.
  Fix $p \in [1,\infty]$ and let $f \in L^p(\mc{A})$; we will construct a contractive conditional expectation $\E^{\mc{B}}f \in L^p(\mc{B};\K) \subset L^p(\mc{A};\K)$ directly.

  \textbf{Mild case: $p > 1$}, so most importantly $p' < \infty$.
  The inclusion map $\map{\iota}{L^{p'}(\mc{B})}{L^{p'}(\mc{A})}$ is contractive, so (using that $L^p$ is the dual of $(L^{p'})^*$, which requires $p' < \infty$) its adjoint $\map{\iota^*}{L^p(\mc{A})}{L^p(\mc{B})}$ is also contractive.
  For all $f \in L^p(\mc{A})$ and $B \in \mc{B}$ we then have
  \begin{equation*}
    \int_B \iota^* f \, \dd\P = \langle \iota^* f, \1_{B} \rangle = \langle f, \iota \1_{B} \rangle = \langle f, \1_{B} \rangle = \int_B f \, \dd\P,
  \end{equation*}
  so $\iota^* f \in L^p(\mc{B}) \subset L^1(\mc{B})$ is a conditional expectation of $f$ given $\mc{B}$.

  \textbf{(German) spicy case: $p = 1$.} The difficulty here is that $L^1$ is \emph{strictly} contained in the dual of $L^\infty$, so we can't just take an adjoint of the inclusion $L^\infty(\mc{B}) \to L^\infty(\mc{A})$.
  Instead we argue by density.
  We have that $L^2(\mc{A})$ is dense in $L^1(\mc{A})$, so we aim to extend the conditional expectation defined above (in the case $p=2$) by continuity.
  For $f \in L^2(\mc{A})$ and $g \in L^\infty(\mc{B})$ we have
  \begin{equation*}
    |\langle \iota^* f, g \rangle| = |\langle f, \iota g \rangle| = |\langle f, g \rangle| \leq \|f\|_1 \|g\|_\infty
  \end{equation*}
  using that $L^\infty(\mc{B}) \subset L^2(\mc{B})$.
  Taking the supremum over all nonzero $g \in L^\infty(\mc{B})$ proves that
  \begin{equation*}
    \|\iota^* f\|_1 \leq \|f\|_1,
  \end{equation*}
  so $\iota^*$ extends to a contraction $L^1(\mc{A}) \to L^1(\mc{B})$.
  For $f \in L^1(\mc{A})$ and $B \in \mc{B}$, using that integration on $B$ is a continuous linear functional on $L^1$, we have
  \begin{equation*}
    \int_B \iota^* f \, \dd\P
    = \lim_{n \to \infty} \int_B \iota^* f_n \, \dd\P
    = \lim_{n \to \infty} \int_B  f_n \, \dd\P
    = \int_B f \, \dd\P
  \end{equation*}
  where $f_n$ is a sequence in $L^2(\mc{A})$ converging to $f$ in $L^1(\mc{A})$.
  Thus the continuous extension of $\iota^*$ is a conditional expectation of $f$, and we are done.
\end{proof}

Note that the proof of the previous result also establishes the following adjoint relation (the case $p=\infty$ involves a few lines of work which are left as an exercise).\todo{include exercise}

\begin{prop}
  Let $(\Omega,\mc{A},\P)$ be a probability space and $\mc{B}$ a $\sigma$-subalgebra of $\mc{A}$.
  For all $p \in (1,\infty]$, the conditional expectation $\E^{\mc{B}}$ on $L^p(\mc{A})$ is the adjoint of the corresponding conditional expectation on $L^{p'}(\mc{A})$. 
\end{prop}

Now we can use the extension theorem for positive operators to show the existence of conditional expectations of Banach-valued random variables.

\begin{prop}
  Let $(\Omega,\mc{A},\P)$ be a probability space, let $\mc{B}$ be a $\sigma$-subalgebra of $\mc{A}$, and let $X$ be a Banach space.
  Then for any $f \in L^1(\mc{A};X)$, a unique (up to a.s. equality) conditional expectation $\E_X^{\mc{B}}f$ exists.
  Furthermore, for all $p \in [1,\infty]$, $\E^{\mc{B}}$ is a contraction on $L^p(\mc{A};X)$.
\end{prop}

\begin{proof}
  \todo{do this after sorting out real vs. complex Banach spaces}
\end{proof}


\todo{up to here re: checking real vs. complex BS issue}

{\color{blue}
\begin{defn}
  Let $(\Omega,\mc{A},\P)$ be a probability space and $X$ a Banach space.
  Given a $\sigma$-subalgebra $\mc{B} \subset \mc{A}$ and an exponent $p \in [1,\infty)$, we define the \emph{conditional expectation operator}
  \begin{equation*}
    \map{\E_X^{\mc{B}}}{L^p(\Omega,\mc{A},\P;X)}{L^p(\Omega,\mc{B};\P;X)} 
  \end{equation*}
  as the unique bounded linear extension of
  \begin{equation*}
    \map{\E^{\mc{B}} \otimes I}{L^p(\Omega,\mc{A},\P) \otimes X}{L^p(\Omega,\mc{B},\P) \otimes X}.
  \end{equation*}
  We will often overload notation and write $\E^{\mc{B}}$ to denote $\E_X^{\mc{B}}$.
\end{defn}
\todo{up to here}
Since the conditional expectation $\E^{\mc{B}}$ on scalar-valued functions is positive and contractive on $L^p$ for all $p \in [1,\infty)$, Theorem \ref{thm:positive-extensions} tells us that $\E_X^{\mc{B}}$ is well-defined and is in fact a contraction on $L^p(\Omega,\mc{A},\P;X)$ for all $p \in [1,\infty)$.
For $p = \infty$, since $L^\infty(\Omega;X)$ is lacking in convenient dense subspaces, we can't argue so abstractly.
Here is a somewhat ad-hoc argument for contractivity on $L^\infty$.

\begin{prop}
  Let $(\Omega,\mc{A},\P)$ be a probability space and $X$ a Banach space.
  Then the conditional expectation $\E_X^{\mc{B}}$, which \emph{a priori} maps
  \begin{equation*}
    \map{\E_X^{\mc{B}}}{L^\infty(\Omega,\mc{A},\P;X) \subset L^1(\Omega,\mc{A},\P;X)}{L^1(\Omega,\mc{B},\P;X)},
  \end{equation*}
  is a contraction on $L^\infty(\Omega,\mc{A},\P;X)$.
\end{prop}

\begin{proof}
  Let $f \in L^\infty(\Omega,\mc{A},P;X)$.
  Since $L^\infty(\Omega,\mc{A},\P;X) \subset L^2(\Omega,\mc{A},\P;X)$, $E^{\mc{B}} f = \td{P^{\mc{B}}} f$ can be seen as the tensor extension of the orthogonal projection of $L^2(\Omega,\mc{A},\P)$ onto $L^2(\Omega, \mc{B},\P)$ (as in the proof of Theorem \ref{thm:conditional-expectation-EU}).
  By Proposition \ref{prop:bochner-preduality}, we can test $\|\E_X^{\mc{B}} f\|_\infty$ by duality, using that $L^2(\Omega,\mc{B}; X^*)$ is dense in $L^1(\Omega, \mc{B}; X^*)$.
  For all nonzero $g \in L^2(\Omega, \mc{B}; X^*)$ we have
\begin{equation*}
  \begin{aligned}
    | \langle \E_X^{\mc{B}} f, g \rangle |
    =  | \langle \td{P^{\mc{B}}} f, g \rangle | 
    \stackrel{(*)}{=} | \langle f, \td{P^{\mc{B}}} g \rangle | 
    &=  | \langle f, \E_X^{\mc{B}} g \rangle | \\
    &\leq  \|f\|_{L^\infty(\Omega,\mc{A},\P;X)} \|\E_X^{\mc{B}} g\|_{L^1(\Omega,\mc{A},\P;X^*)} \\
    &\leq  \|f\|_{L^\infty(\Omega,\mc{A},\P;X)} \| g\|_{L^1(\Omega,\mc{A},\P;X^*)}.
  \end{aligned}
\end{equation*}
For the starred equality see Exercise \ref{ex:tensor-adjoint} in the previous chapter.
Taking the supremum over all such $g$ completes the proof.
\end{proof}

}


\todo{further properties of vector-valued conditional expectations}

\begin{rmk}
  We have only considered conditional expectations $\E^{\mc{B}}$ on probability spaces, but the concept can be extended to general measure spaces $(S,\mc{A},\mu)$ provided that the measure $\mu$ is $\sigma$-finite on the $\sigma$-subalgebra $\mc{B} \subset \mc{A}$ (although the arguments require a fair bit of modification).
  This approach is taken in \cite{HNVW16}.
\end{rmk}


\subsection{Definitions and key examples of martingales}

\begin{defn}
  Let $(\Omega, \mc{A}, \P)$ be a probability space and let $X$ be a Banach space.
  A sequence of integrable $X$-valued random variables $(M_n)_{n \in \N}$ in $L^1(\Omega, \mc{A}, \P; X)$ is called a \emph{martingale} with respect to a filtration $(\mc{A}_n)_{n \in \N}$ if
  \begin{itemize}
  \item $M_n$ is $\mc{A}_n$-measurable for all $n \in \N$,
  \item for all $n \in \N$,
    \begin{equation}\label{eq:mgale-defining-property}
      M_n = \E^{\mc{A}_n} M_{n+1}.
    \end{equation}
  \end{itemize}
\end{defn}

\begin{rmk}
  Now is a good time to complete Exercise \ref{ex:martingale-elementary-properties}, establishing a few elementary properties of martingales.
\end{rmk}

-stopped martingales
-Rademacher variables and sums

\begin{defn}
  limiting $\sigma$-algebra 
\end{defn}

\begin{thm}[$L^p$-convergence of martingales]
  %Pisier Theorem 1.14
  
\end{thm}

\subsection{Maximal inequalities and pointwise convergence}
% proofs from Pisier with little modifications to allow for infinite martingales


\begin{defn}
  definition of $\R$-valued submartingale
\end{defn}

As an example, consider a martingale $(f_n)_{n \in \N}$ taking values in a Banach space $X$.
Then by {\color{red} CITE PART OF POSITIVE OP EXTENSION HERE} we have for all $m \geq n$
\begin{equation*}
  \|f_n\|_X = \|\E^{\mc{F}_n} f_m\|_X \leq \E^{\mc{F}_n} \|f_m\|_X \qquad \text{almost surely},
\end{equation*}
so that the sequence $(\|f_n\|_X)_{n \in \N}$ is an $\R$-valued submartingale.

\begin{thm}[Doob's maximal inequalties]
  
\end{thm}

% i think we can do without this
% but include it if necessary
% \begin{thm}[Burkholder--Davis--Gundy inequality]
% \end{thm}

\begin{thm}[Martingale convergence theorem]
  
\end{thm}



\subsection{John--Nirenberg for adapted sequences and the Kahane--Khintchine inequalities}
% following HNVW1 section 3.2.c

Fix a Banach space $X$ and a $\sigma$-finite measure space $(S,\mc{A},\mu)$.
Consider a filtration $(\mc{F}_n)_{n \in \N}$ and a sequence $\phi = (\phi_n)_{n \in \N}$ of $X$-valued functions adapted to the filtration (i.e. $\phi_n$ is $\mc{F}_n$-measurable for all $n \in \N$).\todo{Strongly?}
For all $q \in (0,\infty)$ we consider the following measure of the oscillation of $\phi$:
\begin{equation*}
    \|\phi\|_{*,q} := \sup_{\substack{k,n \in \N \\ k \leq n}} \sup_{\substack{F \in \mc{F}_k \\ 0 < \mu(F) < \infty}} \Big( \fint_{F} \|(\phi_n - \phi_{k-1})(s)\|_{X}^{q} \, \dd\mu(s)\Big)^{1/q} .
\end{equation*}


\begin{thm}[John--Nirenberg inequality for adapted sequences]\label{thm:jn-adapted-sequences}
  With notation as above, for all $p,q \in (0,\infty)$ there exists a finite constant $c_{p,q}$ independent of $\phi$ such that
  \begin{equation*}
    \|\phi\|_{*,p} \leq c_{p,q} \|\phi\|_{*,q}.
  \end{equation*}
\end{thm}

We will prove this as a consequence of a series of lemmas, but before proving this, we demonstrate an important application to Rademacher sums.

\begin{thm}[Kahane--Khintchine inequality]\label{thm:kk}
  Let $X$ be a Banach space and let $(\varepsilon_{n})_{n \in \N}$ be a Rademacher sequence on a probability space $(\Omega,\mc{F},\P)$.
  Then for all $p,q \in (0,\infty)$ there exists a finite constant $\kappa_{p,q}$ such that for all finite sequences $(\mb{x}_n)_{n=1}^N$ in $X$,
  \begin{equation*}
    \Big\|\sum_{n=1}^{N} \varepsilon_{n} \mb{x}_{n} \Big\|_{L^p(\Omega;X)} \leq \kappa_{p,q} \Big\|\sum_{n=1}^{N} \varepsilon_{n} \mb{x}_{n} \Big\|_{L^q(\Omega;X)}.
  \end{equation*}
  That is, for all $p \in (0,\infty)$, the $L^p$-norms of a Rademacher sum are pairwise equivalent.  
\end{thm}

Since $\Omega$ is a probabibility space, H\"older's inequality yields the case $p \leq q$ with constant $\kappa_{p,q} = 1$, so we only need to consider the case $q < p$.

\begin{proof}[Proof, assuming the John--Nirenberg inequality]
  Consider the filtration and adapted sequence \todo{make sure to discuss filtrations generated by functions, particularly Rademacher variables} 
  \begin{equation*}
    \mc{F}_n := \sigma(\{\varepsilon_j : 1 \leq j \leq n\}), \qquad \phi_n := \sum_{j=1}^n \varepsilon_j \mb{x}_j.
  \end{equation*}
  We claim that for all $q \in [1,\infty)$ we have
  \begin{equation}\label{eq:kk-claim}
    \|\phi\|_{*,q} = \Big\| \sum_{j=1}^N \varepsilon_j \mb{x}_j \Big\|_{L^q(\Omega;X)}.
  \end{equation}
  Assuming this for the moment, the John--Nirenberg inequality yields a finite constant $c_{p,q}$ such that
  \begin{equation*}
    \Big\| \sum_{j=1}^N \varepsilon_j \mb{x}_j \Big\|_{L^p(\Omega;X)}
    = \|\phi\|_{*,p}
    \leq c_{p,q} \|\phi\|_{*,q}
    = \Big\| \sum_{j=1}^N \varepsilon_j \mb{x}_j \Big\|_{L^q(\Omega;X)}
  \end{equation*}
  whenever $1 \leq q < p$.
  If $q < 1 \leq p$, we fix $\theta \in (0,1)$ so that $1/p = \theta/q + (1-\theta)/2p$, and log-convexity of $L^p$-norms gives
  \begin{equation*}
    \| \phi_N \|_{L^p(\Omega;X)}
    \leq \| \phi_N \|_{L^q(\Omega;X)}^{\theta} \| \phi_N \|_{L^{2p}(\Omega;X)}^{1 - \theta}
    \leq \| \phi_N \|_{L^q(\Omega;X)}^{\theta} (c_{2p,p} \| \phi_N \|_{L^{p}(\Omega;X)})^{1 - \theta},
  \end{equation*}
  which yields
  \begin{equation*}
    \| \phi_N \|_{L^p(\Omega;X)} \leq c_{2p,p}^{(1 - \theta)/\theta} \| \phi_N \|_{L^q(\Omega;X)}.
  \end{equation*}
  Finally, if $q < p < 1$, we simply estimate
  \begin{equation*}
    \| \phi_N \|_{L^p(\Omega;X)} \leq \| \phi_N \|_{L^1(\Omega;X)} \leq c_{1,q} \| \phi_N \|_{L^q(\Omega;X)}.
  \end{equation*}
  This covers all nontrivial cases, and it remains to prove the claimed equality \eqref{eq:kk-claim}.

  First recall that
  \begin{equation*}
    \|\phi\|_{*,q} =  \sup_{\substack{k,n \in \N \\ k \leq n}} \sup_{\substack{F \in \mc{F}_k \\ \mu(F) > 0}} \Big( \fint_{F} \|(\phi_n - \phi_{k-1})(s)\|_{X}^{q} \, \dd\P(s)\Big)^{1/q}.
  \end{equation*}
  Fix $k \leq n$ and a set $F \in \mc{F}_k$ of positive measure.
  We will compute
  \begin{equation*}
    \fint_{F} \|\1_{F} (\phi_n - \phi_{k-1})(s)\|_{X}^q \, \dd\P(s) = \P(F)^{-1} \E \Big( \1_{F} \Big\| \sum_{j=k}^{n} \varepsilon_j \mb{x}_j\Big\|_{X}^q \Big).
  \end{equation*}
  Observe that for all $\omega \in \Omega$
  \begin{equation*}
    \Big\| \sum_{j=k}^{n} \varepsilon_j(\omega) \mb{x}_j\Big\|_{X} = \Big\| \varepsilon_k(\omega) \Big( \mb{x}_k + \sum_{j=k+1}^{n} \varepsilon_{j}^{\prime}(\omega) \mb{x}_j \Big) \Big\|_{X} = \Big\|\mb{x}_k + \sum_{j=k+1}^{n} \varepsilon_{j}^{\prime} \mb{x}_j \Big\|_{X}
  \end{equation*}
  where
  \begin{equation*}
    \varepsilon_{j}^{\prime} :=
    \begin{cases}
      \varepsilon_{j} & j \leq k \\
      \varepsilon_{k} \varepsilon_{j} & k+1 \leq j,
    \end{cases}
  \end{equation*}
  since $\varepsilon_j$ is $\pm 1$-valued.
  Now note that the $\sigma$-algebra $\mc{F}^\prime_{k+1,n} := \sigma(\{\varepsilon_j^\prime : k+1 \leq j \leq n\})$ is independent of $\mc{F}_k$, since for all $1 \leq j \leq k$ and $k+1 \leq j' \leq n$ we have by independence of the original Rademacher sequence
  \begin{equation*}
    \E(\varepsilon_j \varepsilon_{j'}^\prime) = \E(\varepsilon_j \varepsilon_k \varepsilon_{j'})
    =
    \begin{cases}
      \E(\varepsilon_j) \E(\varepsilon_k) \E(\varepsilon_{j'}) & \text{if $j < k$} \\
       \E(\varepsilon_{j'}) & \text{if $j = k$}
    \end{cases}
    \quad = 0.
  \end{equation*}
  Thus, since $F \in \mc{F}_k$, we have by independence of $\mc{F}_k$ and $\mc{F}^{\prime}_{k+1, n}$
  \begin{equation*}
    \begin{aligned}
      \P(F)^{-1} \E \Big( \1_{F} \Big\| \sum_{j=k}^{n} \varepsilon_j \mb{x}_j\Big\|_{X}^q \Big)
      &=  \P(F)^{-1} \E \Big( \1_{F} \Big\|\mb{x}_k + \sum_{j=k+1}^{n} \varepsilon_{j}^{\prime} \mb{x}_j \Big\|_{X}^{q} \Big) \\
      &=  \E \Big\|\mb{x}_k + \sum_{j=k+1}^{n} \varepsilon_{j}^{\prime} \mb{x}_j \Big\|_{X}^{q} 
      =  \E  \Big\|\sum_{j=k}^{n} \varepsilon_{j} \mb{x}_j \Big\|_{X}^{q}. \\
    \end{aligned}
  \end{equation*}
  Now letting $\mc{F}_{k,n} := \sigma(\{\varepsilon_j : k \leq j \leq n\})$, we have by the $L^q$-contraction property of conditional expectations
  \begin{equation*}
    \E \Big\|\sum_{j=k}^{n} \varepsilon_{j} \mb{x}_j \Big\|_{X}^{q}
    = \E \Big\|\E^{\mc{F}_{k,n}} \Big(\sum_{j=1}^{N} \varepsilon_{j} \mb{x}_j \Big)\Big\|_{X}^{q}
    \leq \E \Big\|\sum_{j=1}^{N} \varepsilon_{j} \mb{x}_j \Big\|_{X}^{q}
  \end{equation*}
  with equality when $k=1$ and $n=N$.
  Thus
  \begin{equation*}
    \|\phi\|_{*,q} =  \sup_{\substack{k,n \in \N \\ k \leq n}}  \Big(\E  \Big\|\sum_{j=k}^{n} \varepsilon_{j} \mb{x}_j \Big\|_{X}^{q}\Big)^{1/q} = \Big(\E  \Big\|\sum_{j=1}^{N} \varepsilon_{j} \mb{x}_j \Big\|_{X}^{q}\Big)^{1/q}
  \end{equation*}
  which proves the claimed equality \eqref{eq:kk-claim} and completes the proof. 
\end{proof}

In the setting of Hilbert-valued functions (and in particular, scalar-valued functions), the Kahane--Khintchine inequality leads to the classical Khintchine inequalities.

\begin{cor}[Khintchine's inequalities]
  Let $H$ be a Hilbert space, and let $(\varepsilon_{n})_{n \in \N}$ be a Rademacher sequence on a probability space $(\Omega, \mc{F}, \P)$.
  Then for all $p \in (0,\infty)$ there exist finite constants $A_p$ and $B_p$ such that for all finite sequences $(\mb{h}_n)_{n=1}^N$ in $H$,
  \begin{equation*}
    A_p \Big( \sum_{n=1}^N \|\mb{h}_n\|_H^2 \Big)^{1/2} \leq \Big\| \sum_{n=1}^N \varepsilon_n \mb{h}_n \Big\|_{L^p(\Omega;H)} \leq B_p \Big( \sum_{n=1}^N \|\mb{h}_n\|_H^2 \Big)^{1/2}.
  \end{equation*}
\end{cor}

\begin{proof}
  By independence of the Rademacher variables we have
  \begin{equation}
    \begin{aligned}
      \Big\| \sum_{n=1}^N \varepsilon_n \mb{h}_n \Big\|_{L^2(\Omega;H)}^2
      &= \Big\langle \sum_{n=1}^N \varepsilon_n \mb{h}_n , \sum_{m=1}^N \varepsilon_m \mb{h}_m \Big\rangle \\
      &= \sum_{n,m = 1}^N \E(\varepsilon_n \varepsilon_m) \langle \mb{h}_n, \mb{h}_m \rangle 
      = \sum_{n=1}^N \|\mb{h}_{n}\|_{H}^2,
    \end{aligned}
  \end{equation}
  so the result is true for $p=2$ with $A_2 = B_2 = 1$.
  Now use Kahane--Khintchine to extend the result to general $p \in (0,\infty)$.
\end{proof}

\subsubsection{Proof of the John--Nirenberg inequality}

We return to our analysis of a sequence $\phi = (\phi_n)_{n \in \N}$ of $X$-valued functions adapted to a filtration $(\mc{F}_n)_{n \in \N}$ on a $\sigma$-finite measure space $(S,\mc{A},\mu)$.
We prove the John--Nirenberg inequality via a series of lemmas in which we obtain increasingly fine control on the oscillation of $\phi$.

\begin{lem}\label{lem:JN-proof-1}
  For all $k \leq n$, $F \in \mc{F}_k$, and $\alpha > 0$,
  \begin{equation}\label{eq:JN-proof-1-est}
    \mu(F \cap \{ \|\phi_n - \phi_{k-1}\|_{X} > \alpha \}) \leq \Big( \frac{\|\phi\|_{*,q}}{\alpha} \Big)^{q} \mu(F).
  \end{equation}
\end{lem}

\begin{proof}
  We can assume that $0 < \mu(F) < \infty$, otherwise there is nothing to prove.
  The left hand side of \eqref{eq:JN-proof-1-est} is bounded by
  \begin{equation*}
    \int_{F} \Big( \frac{\|\phi_n - \phi_{k-1}\|_{X}}{\alpha} \Big)^{q} \, \dd\mu \leq \mu(F) \Big( \frac{\|\phi\|_{*,q}}{\alpha} \Big)^{q}
  \end{equation*}
  since $F \in \mc{F}_k$ and $k \leq n$, by the definition of $\|\phi\|_{*,q}$.
\end{proof}

Next we show that oscillation control of the form above extends to more general stopping times.

\begin{lem}
  Suppose that there exist $\alpha > 0$ and $\eta > 0$ such that
  \begin{equation*}
    \mu(F \cap \{ \| \phi_n - \phi_{k-1} \| > \alpha \} ) \leq \eta \mu(F) \qquad \forall k \leq n, F \in \mc{F}_k.
  \end{equation*}
  Then for all $k \in \N$, $F \in \mc{F}_k$, and all stopping times $\nu$ such that $\nu \geq k$ on $F$,
  \begin{equation}\label{eq:jn-stoptime}
    \mu(F \cap \{\nu < \infty\} \cap \{ \| \phi_{\nu} - \phi_{k-1} \| > 2\alpha \} ) \leq 2\eta \mu(F).
  \end{equation}
\end{lem}

\begin{proof}
  Sum over all possible values of the stopping time:
  \begin{equation*}
      \mu(F \cap \{\nu < \infty\} \cap \{ \| \phi_{\nu} - \phi_{k-1} \| > 2\alpha \} ) 
      = \lim_{N \to \infty} \sum_{n = k}^{N} \mu(F_n \cap \{ \| \phi_{n} - \phi_{k-1} \| > 2\alpha \} ),
  \end{equation*}
  where $F_{n} := F \cap \{\nu = n\}$.
  For fixed $N \geq n > k$, since $F_n \in \mc{F}_n \subset \mc{F}_{n+1}$, we have by assumption
  \begin{equation*}
    \begin{aligned}
      &\mu(F_n \cap \{ \| \phi_{n} - \phi_{k-1} \| > 2\alpha \} ) \\
      &\leq \mu(F_n \cap \{ \| \phi_{n} - \phi_{N} \| > \alpha \} )
      + \mu(F_n \cap \{ \| \phi_{k-1} - \phi_{N} \| > \alpha \} ) \\
      &\leq \eta \mu(F_n) + \mu(F_n \cap \{ \| \phi_{k-1} - \phi_{N} \| > \alpha \} ).
    \end{aligned}
  \end{equation*}
  Thus we have
  \begin{equation*}
    \begin{aligned}
      &\lim_{N \to \infty} \sum_{n = k}^{N} \mu(F_n \cap \{ \| \phi_{n} - \phi_{k-1} \| > 2\alpha \} ) \\
      &\leq \lim_{N \to \infty} \Big( \eta \sum_{n=k}^N \mu(F_n) + \sum_{n=k}^N \mu(F_n \cap \{ \| \phi_{k-1} - \phi_{N} \| > \alpha \}) \Big) \\
      &\leq \eta \mu(F) + \lim_{N \to \infty} \mu(F \cap \{ \| \phi_{k-1} - \phi_{N} \| > \alpha \})  
      \leq 2\eta\mu(F)
    \end{aligned}
  \end{equation*}
  using the assumption and $F \in \mc{F}_k$ in the last estimate.
\end{proof}
  
In the following lemma we make use of the \emph{started sequence}
\begin{equation*}
  {}^{k-1}\phi = (\phi_n - \phi_{k-1})_{n \geq k-1}
\end{equation*}
and its maximal function
\begin{equation*}
  ({}^{k-1} \phi)^{*}(s) := \sup_{n \geq k-1} \|({}^{k-1}\phi)_n(s)\|_X = \sup_{n \geq k} \|(\phi_n - \phi_{k-1})(s)\|_X.
\end{equation*}

\begin{lem}\label{lem:jn-mf}
  Suppose that $\phi$ satisfies \eqref{eq:jn-stoptime} for all $k \in \N$, $F \in \mc{F}_k$, and all stopping times $\nu$ such that $\nu \geq k$ on $F$.
  Then for all $\lambda > 0$,
  \begin{equation}\label{eq:jn-mf-eq}
    \mu(F \cap \{ ({}^{k-1} \phi)^{*} > \lambda + 2\alpha \}) \leq 2\eta \mu(F \cap \{  ({}^{k-1} \phi)^{*} > \lambda \}) \qquad \forall k \in \N, F \in \mc{F}_{k}.
  \end{equation}
\end{lem}

\begin{proof}
  Fix $k \in \N$ and consider the stopping times
  \begin{equation*}
    \begin{aligned}
      \rho &:= \inf\{n \geq k : \|\phi_n - \phi_{k-1}\| > \lambda\}, \\
      \nu &:= \inf\{n \geq k : \|\phi_n - \phi_{k-1}\| > \lambda + 2\alpha\}.
    \end{aligned}
  \end{equation*}
  Then $k \leq \rho \leq \nu$, and \eqref{eq:jn-mf-eq} can be rewritten as
  \begin{equation*}
    \mu(F \cap \{\nu < \infty\}) \leq 2\eta \mu(F \cap \{\rho < \infty\}).
  \end{equation*}
  Now fix $n \geq k$ and let $F_n := F \cap \{\rho = n\} \in \mc{F}_n$.
  On $\{F_n \cap \{\nu < \infty\}\}$ we have
  \begin{equation*}
    \|\phi_{\nu} - \phi_{n-1}\| \geq \|\phi_{\nu} - \phi_{k-1}\| - \|\phi_{n-1} - \phi_{k-1}\| > (\lambda + 2\alpha) - \lambda = 2\alpha,
  \end{equation*}
  so
  \begin{equation*}
    \mu(F_n \cap \{\nu < \infty\}) = \mu(F_n \cap \{\nu < \infty\} \cap \{\|\phi_{\nu} - \phi_{n-1}\| > 2\alpha\} )
    \leq 2\eta \mu(F_n).
  \end{equation*}
  Summing over $n \geq k$ completes the proof.
\end{proof}

\begin{lem}
  Suppose that $f$ is a non-negative function supported in $F \in \mc{A}$, satisfying
  \begin{equation*}
    \mu(f > \lambda + \alpha) \leq \eta \mu(f > \lambda) \qquad \forall \lambda > 0
  \end{equation*}
  for some $\eta \in (0,1)$ and $\alpha > 0$.
  Then for all $p \in [1,\infty)$,
  \begin{equation*}
    \|f\|_p \leq \frac{1 + \eta^{1/p}}{1 - \eta^{1/p}} \alpha \mu(F)^{1/p}.
  \end{equation*}
\end{lem}

\begin{proof}
  {\color{red} WRITE PROOF}
\end{proof}

\todo{UP TO HERE. HAVE TO COMPLETE PROOF OF JOHN-NIRENBERG.}

{\color{blue}

\begin{equation*}
  \|\phi\|_{**,q} := \sup_{k \in \N} \sup_{\substack{F \in \mc{F}_k \\ 0 < \mu(F) < \infty}} \Big( \fint_{F} ({}^{k-1} \phi)^{*}(s)^q \, \dd\mu(s) \Big)^{1/q},
\end{equation*}

}



\subsection{Gundy's decomposition}

\subsection{The Radon--Nikodym property, martingale convergence, and Bochner space duality}\label{sec:RNP}

\subsection*{Exercises}

\begin{exercise}
  Let $(f_n)_{n \in \N}$ be a stochastic process on a probability space $(\Omega,\mc{A},\P)$.
  Suppose that $(f_n)$ is predictable with respect to the filtration generated by $(f_n)$ (see Example \ref{eg:filtration-generated-by-process}).
  Show that the process is deterministic, in the sense that each $f_n$ is constant.
\end{exercise}

\begin{exercise}\label{ex:winnings-unpredictability}
  In the setting of Example \ref{eg:gambling-filtrations}, show that the random variable $\mb{s}_{n+1}$ is $\mc{F}_n$-measurable if and only if $\mb{x}_{n+1} \equiv 0$.
\end{exercise}

\begin{exercise}\label{ex:gambling-in-linfty}
  This exercise takes place in the setting of Example \ref{eg:gambling-stoppingtimes}.
  \begin{itemize}
  \item
    Let $X = \ell^\infty(\N)$.
    Suppose that the wager vectors $\map{\mb{x}_n}{\Omega}{\ell^\infty(\N)}$ are such that for all $\omega \in \Omega$, the vectors $(\mb{x}_n(\omega))_{n \in \N}$ are pairwise distinct standard basis vectors (i.e. $\{0,1\}$-valued sequences, zero for all but one index).
    Fix $\lambda > 0$ and let $K = \{\mb{a} \in \ell^\infty(\N) : \|\mb{a}\|_\infty \geq \lambda\} = \ell^\infty(\N) \sm B_\lambda(0)$.
    Show that the stopping time
    \begin{equation*}
      T_K(\omega) := \inf\{n \in \N : \mb{s}_n(\omega) \in K\} 
    \end{equation*}
    is finite if and only if $\lambda \leq 1$.
  \item
    As above, but now let $X = \ell^2(\N)$, and show that the stopping time $T_K$ is finite for all $\lambda > 0$.
  \end{itemize}
\end{exercise}

\begin{exercise}\label{ex:martingale-elementary-properties}
  Let $(\Omega,\mc{A},\P)$ be a probability space, $X$ a Banach space, and let $(M_n)_{n \in \N}$ be a martingale with respect to some filtration $(\mc{A}_n)_{n \in \N}$.
  \begin{itemize}
  \item
    Show that $M_n  \E^{\mc{A}_n} M_m$ for all $n,m \in \N$ with $m > n$,
  \item
    Let $(\mc{F}_n)_{n \in \N}$ be the filtration generated by $(M_n)_{n \in \N}$, i.e.
    \begin{equation*}
      \mc{F}_n := \sigma(M_0, M_1, \ldots, M_n).
    \end{equation*}
    Show that $(M_n)_{n \in \N}$ is a martingale with respect to $(\mc{F}_n)_{n \in \N}$.
  \item
    For all $p \in [1,\infty]$, show that the sequence $\|M_n\|_{L^p(\Omega,\P;X)}$ is monotonically increasing in $n$.
  \end{itemize}

\end{exercise}



%%% Local Variables:
%%% mode: latex
%%% TeX-master: "../main.tex"
%%% End:


\chapter{The Radon--Nikodym property}
\label{sec:RNP}
We now move from Banach-valued analysis and probability to Banach-valued \emph{measure theory}, and finally to the \emph{geometry} of Banach spaces.
We will tie these concepts together via the Radon--Nikodym property, which is ostensibly a measure-theoretic property but has equivalent characterisations in terms of Bochner spaces, martingales, and convex sets.

\section{Vector measures and the Radon--Nikodym property}

\begin{defn}\index{vector measures}
  Let $X$ be a Banach space and $(S,\mc{A})$ a measurable space.
  An $X$-valued \emph{vector measure} is a function $\map{\mb{\mu}}{\mc{A}}{X}$ which is countably additive, in the sense that for all sequences $(E_n)_{n \in \N}$ of pairwise disjoint sets in $\mc{A}$,
  \begin{equation*}
    \mb{\mu}\Big( \bigcup_{n \in \N} E_n \Big) = \sum_{n \in \N} \mb{\mu}(E_n).
  \end{equation*}
  Note that this condition includes the convergence in $X$ of the series on the right hand side.
\end{defn}

Vector measures are just like measures, except the measure of a set $E \subset S$ is a vector $\mb{\mu}(E) \in X$ rather than a scalar.
We are most interested in vector measures with the following boundedness condition.

\begin{defn}\index{vector measures!variation}
  Let $X$ be a Banach space and $\mb{\mu}$ an $X$-valued vector measure on a measurable space $(S,\mc{A})$.
  The \emph{variation} of $\mb{\mu}$ is the scalar-valued measure $\map{|\mb{\mu}|}{\mc{A}}{[0,\infty]}$ defined by
  \begin{equation*}
    |\mb{\mu}|(E) := \sup_{\pi} \sum_{A \in \pi} \|\mb{\mu}(A)\|_X,
  \end{equation*}
  where the supremum ranges over all partitions $\pi$ of $S$ into $\mc{A}$-measurable sets.
  We define the \emph{total variation norm} $\|\mb{\mu}\|_{\var} := |\mb{\mu}|(S)$, and we say that $\mu$ has \emph{bounded variation} if $\|\mb{\mu}\|_{\var} < \infty$.
  Equivalently, $\mu$ has bounded variation if there exists a finite scalar-valued measure $\nu$ on $\mc{A}$ such that $\|\mb{\mu}(A)\|_X \leq \nu(A)$ for all $A \in \mc{A}$ (the minimal measure with this property is $|\mb{\mu}|$).
  We let $M(S,\mc{A};X)$ denote the Banach space of all $X$-valued vector measures $\mb{\nu}$ on $\mc{A}$ with bounded variation, under the total variation norm.
\end{defn}

It is not particularly difficult to define integrals of scalar-valued functions with respect to vector measures.

\begin{prop}
  Let $X$ be a Banach space and $\mb{\mu}$ an $X$-valued vector measure of bounded variation on a measurable space $(S,\mc{A})$.
  Then there is a unique continuous linear map $\map{[\mb{\mu}]}{L^1(S,\mc{A},|\mb{\mu}|)}{X}$ such that $[\mb{\mu}](\1_{A}) = \mb{\mu}(A)$ for all $A \in \mc{A}$.
  We use integral notation to denote this map, i.e. we write
  \begin{equation*}
    \int_{S} f(s) \, \dd\mb{\mu}(s) := [\mb{\mu}](f) \qquad \forall f \in L^1(S,\mc{A},|\mb{\mu}|).
  \end{equation*}
\end{prop}

\begin{proof}
  We skip the verification that the definition $[\mb{\mu}](\1_{A}) := \mb{\mu}(A)$ extends by linearity to a well-defined map on integrable simple functions.\footnote{``It is dreadfully boring to show that this formula defines a linear map... from the space of simple functions of the above form into $X$ and we leave this as an exercise for masochists.'' \cite[pp5-6]{DU77}}
  We just need to show boundedness, and the conclusion will follow by density.
  Consider a simple function $g \in L^1(S,\mc{A},|\mb{\mu}|)$ of the form
  \begin{equation*}
    g = \sum_{n=1}^{N} c_n \1_{S_n} 
  \end{equation*}
  with scalars $c_n \in \K$.
  Then
  \begin{equation*}
    \begin{aligned}
      \|[\mb{\mu}](g)\|_X \leq \sum_{n=1}^{N} |c_n| \|\mb{\mu}(S_n)\|_X \leq \sum_{n=1}^{N} |c_n| |\mb{\mu}|(S_n) = \|g\|_{L^1(|\mb{\mu}|)}.
    \end{aligned}
  \end{equation*}
  That's all.
\end{proof}

Fundamental examples of vector measures are given by integrating vector-valued functions against scalar measures.

\begin{example}\label{eg:RN-density}
  Let $(S,\mc{A})$ be a measurable space and $X$ a Banach space.
  Suppose $\nu$ is a finite scalar-valued measure on $(S,\mc{A})$ and $\mb{f} \in L^1(S,\mc{A},\nu;X)$.
  Then we can define an $X$-valued vector measure $\mb{\mu}$ (sometimes denoted $\mb{f}\nu$) by Bochner integration:
  \begin{equation*}
    \mb{\mu}(A) = \int_A \mb{f} \, \dd\nu.
  \end{equation*}
  This vector measure has bounded variation: given a partition $S = \bigcup_{n \in \N} S_n$, we compute
  \begin{equation*}
    \sum_{n \in \N} \|\mb{\mu}(S_n)\|_{X} = \sum_{n \in \N} \Big\| \int_{S_n} \mb{f} \, \dd\nu \Big\|_X
    \leq \int_{S} \|\mb{f}\|_{X} \, \dd\nu
  \end{equation*}
  so that $\|\mb{\mu}\|_{\var} \leq \|\mb{f}\|_{L^1(\nu;X)}$.\footnote{In fact, this is an equality. See \cite[pp43]{gP16}.}
\end{example}

Now let's revise some measure theory.
Recall that if $\mu$ and $\nu$ are two scalar-valued signed measures on a measurable space $(S,\mc{A})$, then \emph{$\nu$ is absolutely continuous with respect to $\mu$},\index{absolute continuity} written $\nu \ll \mu$, if $A \in \mc{A}$ and $\mu(A) = 0$ implies $\nu(A) = 0$.

\begin{thm}[Radon--Nikodym]\index{theorem!Radon--Nikodym}
  Let $(S,\mc{A})$ be a measurable space, and let $\mu$ be a $\sigma$-finite measure on $\mc{A}$.
  Let $\nu$ be a finite signed measure on $\mc{A}$ such that $\nu \ll \mu$.
  Then there exists a unique $h \in L^1(\mu)$ such that
  \begin{equation*}
    \nu(A) = \int_{A} h(s) \, \dd\mu(s) \qquad \forall A \in \mc{A}.
  \end{equation*}
  The function $h$ is called the \emph{Radon--Nikodym derivative}\index{Radon--Nikodym derivative} of $\nu$ with respect to $\mu$, and denoted by
  \begin{equation*}
    h = \frac{\dd\nu}{\dd\mu}.
  \end{equation*}
\end{thm}

See \cite[Theorem 5.5.4]{rD04} for a proof.
One might expect that an analogous theorem holds for vector measures, but it turns out to depend on the target Banach space, and thus its validity becomes a definition.

\begin{defn}\index{Radon--Nikodym property}\index{RNP|see {Radon--Nikodym property}}
  Let $(S,\mc{A},\mu)$ be a $\sigma$-finite measure space.
  A Banach space $X$ is said to have the \emph{Radon--Nikodym property (RNP) with respect to $(S,\mc{A},\mu)$} if for every $X$-valued vector measure $\mb{\nu}$ on $(S,\mc{A})$ such that $\|\mb{\nu}\|_{\var} < \infty$ and $|\mb{\nu}| \ll \mu$, there is a function $\mb{f} \in L^1(\mu;X)$ such that $\mb{\nu} = \mb{f}\mu$ (as defined in Example \ref{eg:RN-density}).
  We say $X$ has the \emph{Radon--Nikodym property} if it has the property above with respect to every $\sigma$-finite measure space $(S,\mc{A},\mu)$.
\end{defn}

The classical Radon--Nikodym theorem says that the scalar fields $\R$ and $\C$ have the RNP.
We will investigate this property for other Banach spaces by considering its relationship with martingales and with properties of convex sets.
We will also connect it with the duality of Bochner spaces $L^p(X)$, answering a question left open in Chapter \ref{sec:Bochner-spaces}.
Before moving on we record a simple reduction.

\begin{prop}\label{prop:RNP-finite-sufficient}
  A Banach space $X$ has the Radon--Nikodym property if and only if for every finite measure space $(S,\mc{A},\mu)$ and every $X$-valued vector measure $\mb{\nu}$ on $(S,\mc{A})$ such that $\|\mb{\nu}(A)\|_{X} \leq \mu(A)$ for every $A \in \mc{A}$, there is a function $\mb{f} \in L^1(\mu;X)$ such that $\mb{\nu} = \mb{f}\mu$.
\end{prop}

\begin{proof}
  If $X$ has the RNP and $\mu$ and $\mb{\nu}$ are as hypothesised, then we have in particular that
  \begin{equation*}
    \|\mb{\nu}\|_{\var} < \|\mu\|_{\var} = \mu(S) < \infty
  \end{equation*}
  and $|\mb{\nu}| \ll \mu$, so the RNP gives us the required function $\mb{f} \in L^1(\mu;X)$.

  Conversely, suppose that $X$ has the hypothesised property, and let $(S,\mc{A},\mu)$ be a $\sigma$-finite measure space.
  Let $\mb{\nu}$ be an $X$-valued vector measure of bounded variation on $(S,\mc{A})$ such that $|\mb{\nu}| \ll \mu$.
  The measure $|\mb{\nu}|$ is finite, and for all $A \in \mc{A}$ we have
  \begin{equation*}
    \|\mb{\nu}(A)\|_{X} \leq |\mb{\nu}|(A)
  \end{equation*}
  by definition.
  Thus by hypothesis there exists a function $\mb{f} \in L^1(|\mb{\nu}|;X)$ such that $\mb{\nu} = \mb{f}|\mb{\nu}|$.
  By the scalar Radon--Nikodym theorem, there also exists a function $g \in L^1(\mu)$ such that $|\mb{\nu}| = g\mu$.
  Since $|\mb{\nu}|$ is a non-negative measure, $g$ is non-negative.
  Thus we have $\mb{\nu} = (\mb{f}g)\mu$.
  It remains to show that $\mb{f}g \in L^1(\mu;X)$: this is established by
  \begin{equation*}
    \int_{S} \|\mb{f}g\|_{X} \, \dd\mu
    \leq \int_{S} \|\mb{f}(s)\|_{X} g(s) \, \dd\mu(s) = \|\mb{f}\|_{L^1(g\mu;X)} = \|\mb{f}\|_{L^1(|\mb{\nu}|)} < \infty. 
  \end{equation*}
\end{proof}

\section{The RNP and martingale convergence}

We establish the following connection between the Radon--Nikodym property and the martingale convergence properties.

\begin{thm}\label{thm:RNP-MCP}\index{Radon--Nikodym property!implies $1$-MCP}\index{Martingale Convergence Property!implied by RNP}
  Let $X$ be a Banach space which has the Radon--Nikodym property with respect to a probability space $(\Omega,\mc{A},\P)$.
  Then $X$ has the $1$-martingale convergence property with respect to $(\Omega,\mc{A},\P)$.
\end{thm}

Applications of the Radon--Nikodym property generally involve the construction of an appropriate vector measure, from which a magical function is extracted as a Radon--Nikodym derivative.
In the setting of Theorem \ref{thm:RNP-MCP}, we are given an $X$-valued martingale, and we construct its almost-everywhere limit as the Radon--Nikodym derivative of a certain vector measure.
In the following proposition we construct the vector measure.

\begin{prop}\label{prop:martingale-measure} 
  Let $(\Omega,\mc{A},\P)$ be a probability space and $X$ a Banach space.
  Let $\mb{f}_{\bullet}$ be an $X$-valued $L^1$-bounded uniformly integrable martingale with respect to a filtration $\mc{A}_{\bullet}$.
  Then there exists an $X$-valued vector measure $\mb{\mu}$ on $\mc{A}$ with the following properties:
  \begin{itemize}
  \item $\mb{\mu}(A) = \int_{A} \mb{f}_n \, \dd\P$ for all $n \in \N$ and $A \in \mc{A}_{n}$,
  \item $\|\mb{\mu}\|_{\var} \leq \sup_{n} \|\mb{f}_n\|_{L^1(\Omega;X)}$,
  \item $|\mb{\mu}|$ is absolutely continuous with respect to $\P$.
  \end{itemize}
\end{prop}

\begin{proof}
  For all $A \in \mc{A}$ we would like to define
  \begin{equation}\label{eq:mu-stationary-limit}
    \mb{\mu}(A) := \lim_{k \to \infty} \int_A \mb{f}_k \, \dd\P,
  \end{equation}
  but it is not immediate that this limit exists.
  If $n \in \N$ and $A \in \mc{A}_{n}$, then for $k \geq n$ we have
  \begin{equation*}
    \int_A \mb{f}_k \, \dd\P = \int_A \mb{f}_n \, \dd \P
  \end{equation*}
  by the martingale property, so at least for $A \in \mc{A}_{n}$ the limit exists and equals $\int_A \mb{f}_n \, \dd\P$, establishing the first desired property.
  For a general $A \in \mc{A}$, for each $k \in \N$ we have
  \begin{equation*}
     \int_A \mb{f}_k \, \dd\P = \E( \1_{A}\mb{f}_k) = \E(\E^{\mc{A}_k}( \1_{A} \mb{f}_k)) = \E(\E^{\mc{A}_k}(\1_{A}) \mb{f}_k ).
   \end{equation*}
   Thus to show that the limit in \eqref{eq:mu-stationary-limit} exists, we need to show that the sequence $(\E(\E^{\mc{A}_k}( \1_{A}) \mb{f}_k ))_{k \in \N}$ is Cauchy.
   Let $\phi_{k,\ell} = \E^{\mc{A}_k}(\1_{A}) - \E^{\mc{A}_\ell}( \1_{A})$.
   For $k < \ell$ we have, using conditional expectation magic,
   \begin{equation*}
     \begin{aligned}
       \E(\E^{\mc{A}_k}(\1_{A}) \mb{f}_k  ) - \E(\E^{\mc{A}_\ell}( \1_{A}) \mb{f}_\ell )
       &= \E \Big( \E^{\mc{A}_k}(\1_{A}) \mb{f}_k  - \E^{\mc{A}_\ell}( \1_{A}) \mb{f}_{\ell} \Big) \\
       &= \E \Big(  \E^{\mc{A}_{k}}(\E^{\mc{A}_k}(\1_{A})\mb{f}_\ell ) - \E^{\mc{A}_{k}}( \E^{\mc{A}_\ell}( \1_{A}) \mb{f}_{\ell} )\Big) \\
       &= \E \Big(  \E^{\mc{A}_k}(\1_{A}) \mb{f}_\ell  - \E^{\mc{A}_\ell}( \1_{A}) \mb{f}_{\ell}  \Big) 
       = \E ( \phi_{k,\ell} \mb{f}_\ell ).
     \end{aligned}
   \end{equation*}
   Thus we get
   \begin{equation*}
     \|\E(\E^{\mc{A}_k}(\1_{A}) \mb{f}_k ) - \E( \E^{\mc{A}_\ell}( \1_{A}) \mb{f}_\ell)\|_{X}
     \leq \|\phi_{k,\ell} \mb{f}_\ell \|_{L^1(\Omega;X)}.
   \end{equation*}
   Since $\|\phi_{k,\ell}\|_\infty \leq 2$ for all $k$ and $\ell$, for all $t > 0$ we have
   \begin{equation*}
     \begin{aligned}
       \|  \phi_{k,\ell}  \mb{f}_\ell \|_{L^1(\Omega;X)}
       &\leq \Big( \int_{\|\mb{f}_{\ell}\|_{X} > t} + \int_{\|\mb{f}_{\ell}\|_{X} \leq t} \Big) |\phi_{k,\ell}(\omega)| \|\mb{f}_{\ell}(\omega)\|_{X} \, \dd\P(\omega) \\
       &\leq  2\int_{\|\mb{f}_{\ell}\|_{X} > t} \|\mb{f}_{\ell}(\omega)\|_{X} \dd\P(\omega) + t\E|\phi_{k,\ell}| 
     \end{aligned}
   \end{equation*}
   so that
   \begin{equation*}
     \limsup_{k,\ell \to \infty} \|  \phi_{k,\ell} \mb{f}_\ell  \|_{L^1(\Omega;X)}
     \leq 2\limsup_{\ell \to \infty} \int_{\|\mb{f}_{\ell}\|_{X} > t} \|\mb{f}_{\ell}(\omega)\|_{X} \dd\P(\omega)
   \end{equation*}
   for all $t > 0$, using that $\limsup_{k,\ell \to \infty} \|\phi_{k,\ell}\|_{1} = 0$ (i.e. $(\E^{\mc{A}_{k}} \1_{A})_{k \in \N}$ is convergent in $L^1$).
   Taking the limit as $t \to \infty$, uniform integrability of $\mb{f}_{\bullet}$ says that
   \begin{equation*}
     \limsup_{k,\ell \to \infty} \|  \phi_{k,\ell} \mb{f}_\ell  \|_{L^1(\Omega;X)}
     \leq 2\lim_{t \to \infty} \sup_{\ell \in \N} \int_{\|\mb{f}_{\ell}\|_{X} > t} \|\mb{f}_{\ell}(\omega)\|_{X} \dd\P(\omega) = 0
   \end{equation*}
   (see Exercise \ref{ex:UI-characterisation}), which establishes that the limit in \eqref{eq:mu-stationary-limit} exists.

 We still need to show that $\mb{\mu}$ is actually a vector measure.
 It is clear from the definition that it is finitely additive, but we need \emph{countable} additivity.
 Consider the submartingale $(\|\mb{f}_n\|_{X})_{n \in \N}$: this is $L^1$-bounded and uniformly integrable, so by Theorem \ref{thm:submartingale-convergence} it has an $L^1$-limit $g \in L^1(\Omega)$.
 Thus for all $A \in \mc{A}$
 \begin{equation*}
   \|\mb{\mu}(A)\|_{X} \leq \lim_{n \to \infty} \int_{A} \|\mb{f}_n(\omega)\| \, \dd\P(\omega) = \int_{A} g(\omega) \, \dd\P(\omega).
 \end{equation*}
 By Exercise \ref{ex:fa-meas-ca}, this implies that $\mb{\mu}$ is countably additive with
 \begin{equation*}
   \|\mb{\mu}\|_{\var} \leq \|g\P\|_{\var} = \|g\|_{L^1(\Omega)} = \sup_{n \in \N} \|\mb{f}_n\|_{L^1(\Omega;X)},
 \end{equation*}
 and that
 \begin{equation*}
   |\mb{\mu}| \ll g\P \ll \P,
 \end{equation*}
 as required.
\end{proof}


\begin{proof}[Proof of Theorem \ref{thm:RNP-MCP}: RNP implies $1$-MCP]
  Let $\mb{f}_{\bullet}$ be an $L^1$-bounded uniformly integrable $X$-valued martingale with respect to a filtration $\mc{A}_{\bullet}$.
  By Proposition \ref{prop:martingale-measure}, there exists an $X$-valued vector measure $\mb{\mu}$ on $\mc{A}$ of bounded variation such that
  \begin{equation*}
    \mb{\mu}(A) = \int_{A} \mb{f}_n \, \dd\P \qquad \forall A \in \mc{A}_n
  \end{equation*}
  and $\mb{\mu} \ll \P$.
  Since $X$ has the RNP with respect to $(\Omega,\mc{A},\P)$, there exists a function $\mb{f} \in L^1(\Omega;X)$ such that
  \begin{equation*}
    \int_{A} \mb{f} \, \dd\P = \mb{\mu}(A) = \int_A \mb{f}_n \, \dd\P
  \end{equation*}
  for all $A \in \mc{A}_n$.
  Equivalently stated, we have
  \begin{equation*}
    \E^{\mc{A}_n} \mb{f} = \mb{f}_n
  \end{equation*}
  for all $n \in \N$, and thus by Theorem \ref{thm:mgale-pw-conv} $\mb{f}_n$ is almost everywhere convergent to $\E^{\mc{A}_{\infty}}\mb{f}$.
  Thus $X$ has the $1$-martingale convergence property, and the proof is complete.
\end{proof}

We already have some examples of spaces which do not satisfy the $1$-MCP: it follows that these spaces cannot have the RNP either.

\begin{cor}\index{Radon--Nikodym property!failure of $c_{0}$ and $L^1$}
  The spaces $c_0$ and $L^1([0,1])$ do not have the RNP.
\end{cor}

\begin{proof}
  In Examples \ref{eg:c0-noMCP} and \ref{eg:L1-noMCP} we showed that these spaces do not have the $\infty$-MCP, and hence they do not have the $1$-MCP. 
\end{proof} 

In summary, for all $p \in (1,\infty]$ we currently have the implications
\begin{equation*}
  \mathrm{RNP} \Longrightarrow 1-\mathrm{MCP} \Longrightarrow p-\mathrm{MCP} \Longrightarrow \infty-\mathrm{MCP}
\end{equation*}
where these properties are taken either universally or with respect to a given probability space.
In the next section we will add more properties and `complete the loop'.

\section{Trees and dentability}

To connect the Radon--Nikodym property with the geometry of Banach spaces, we will use the notion of a \emph{separated tree}.

\begin{defn}
  Let $(\Omega,\mc{A},\P)$ be a probability space and $X$ a Banach space.
  Given $\delta > 0$, an $X$-valued $L^1$-bounded martingale $\mb{f}_\bullet$ on $\Omega$ is called \emph{$\delta$-separated}\index{martingales!$\delta$-separated} if the following properties hold:
  \begin{itemize}
  \item $\mb{f}_0$ is constant,
  \item each $\mb{f}_n$ has finitely many values (i.e. $\mb{f}_{n}$ is simple),
  \item for all $n \in \N$ and $\omega \in \Omega$, $\|\mb{f}_n(\omega) - \mb{f}_{n+1}(\omega)\|_X \geq \delta$.
  \end{itemize}
  The set of values $S := \{\mb{f}_n(\omega) : n \in \N, \omega \in \Omega\}$ is called a \emph{$\delta$-separated tree}.  \index{separated trees}
\end{defn}

The martingales described in Examples \ref{eg:c0-noMCP} and \ref{eg:L1-noMCP} (see also Exercise \ref{ex:L1-noMCP-var}) are $1$-separated, and thus yield $1$-separated trees in $c_0$ and $L_1$.
But when one tries to draw a $\delta$-separated tree on a piece of paper, one quickly starts to run out of space.
This is because pieces of paper model finite dimensional Banach spaces, which have good martingale convergence properties.

\begin{prop}\index{separated trees!relation with MCP} 
  If a Banach space $X$ has the $\infty$-MCP, then for all $\delta > 0$, $X$ does not contain a bounded $\delta$-separated tree.
\end{prop}

\begin{proof}
  Bounded $\delta$-separated trees correspond to $L^\infty$-bounded martingales $\mb{f}_{\bullet}$ such that
  \begin{equation*}
    \|\mb{f}_n(\omega) - \mb{f}_{n+1}(\omega)\|_{X} \geq \delta
  \end{equation*}
  for all $\omega \in \Omega$ and $n \in \N$,
  which directly obstructs convergence of $\mb{f}_\bullet$ everywhere in $\Omega$.
\end{proof}

Thus our chain of implications now has a new member:
\begin{equation*}
  \mathrm{RNP} \Longrightarrow 1-\mathrm{MCP} \Longrightarrow p-\mathrm{MCP} \Longrightarrow \infty-\mathrm{MCP}
  \Longrightarrow \mathrm{NBST}
\end{equation*}
where NBST stands for `no bounded separated trees'.\footnote{As with MCP, this is not standard terminology: ultimately it's just equivalent to RNP.}
The connection between (nonexistence of) bounded separated trees and the Radon--Nikodym property will go through the concept of \emph{dentable sets}.

\begin{defn}\index{dentable sets}
  A subset $D \subset X$ of a Banach space $X$ is called \emph{dentable} if for all $\varepsilon > 0$ there exists $\mb{x} \in D$ such that
  \begin{equation*}
    \mb{x} \notin \overline{\conv}(D \sm B_{\varepsilon}(\mb{x}))
  \end{equation*}
  where $\overline{\conv}$ denotes the closure of the convex hull.
\end{defn}

An example of a dentable set in $\R^{2}$ is shown in Figure \ref{fig:dentable}.
It is impossible to draw a bounded non-dentable set due to Theorem \ref{thm:dent-RNP}. 

\begin{figure}
    \centering
    \def\svgwidth{\columnwidth}
    \input{dentable-set.pdf_tex}
    
    \caption{A dentable set $D$ in $\R^{2}$, being `dented' at the point $\mb{x} \in D$.}
    \label{fig:dentable}
\end{figure}

Before going further we'll need a lemma which relates non-dentability at scale $\varepsilon$ to a corresponding property of an enlarged set which does not involve closures.
This will let us work directly with convex hulls rather than their closures.

\begin{lem}\label{lem:dentlem}
  Let $X$ be a Banach space.
  Fix $\varepsilon > 0$ and let $D \subset X$ be a subset such that for all $\mb{x} \in D$,
  \begin{equation}\label{eq:dentlem-ass}
    \mb{x} \in \overline{\conv}(D \sm B_{\varepsilon}(\mb{x})).
  \end{equation}
  Then for all $\mb{x} \in \tilde{D} := D + B_{\varepsilon/2}(\mb{0})$,
  \begin{equation}
    \mb{x} \in \conv(\tilde{D} \sm B_{\varepsilon/2}(\mb{x}))
  \end{equation}
  (note that no closure is taken here).
\end{lem}

\begin{proof}
  Fix $\mb{x} = \mb{x}' + \mb{y} \in \tilde{D}$, where $\mb{x}' \in D$ and $\|\mb{y}\|_{X} < \varepsilon/2$.
  Choose $\delta > 0$ so small that $\delta + \|\mb{y}\|_{X} < \varepsilon/2$.
  By \eqref{eq:dentlem-ass} we have $\mb{x}' \in \overline{\conv}(D \sm B_{\varepsilon}(\mb{x}'))$, so there exists $n \in \N$, scalars $\alpha_i \in [0,1]$ and vectors $\mb{x}_1,\ldots,\mb{x}_n \in D \sm B_{\varepsilon}(\mb{x}')$, $i = 1,\ldots,n$, with $\sum_{i} \alpha_{i} = 1$, such that
  \begin{equation*}
    \mb{x}' = \mb{z} + \sum_{i=1}^{n} \alpha_i \mb{x}_{i}
  \end{equation*}
  for some $\mb{z} \in B_{\delta}(\mb{0})$.
  We then have
  \begin{equation*}
    \mb{x} = \mb{z} + \mb{y} + \sum_{i=1}^{n} \alpha_i \mb{x}_{i} = \sum_{i=1}^{n}\alpha_i(\mb{z} + \mb{y} + \mb{x}_{i}).
  \end{equation*}
  The points $\mb{z} + \mb{y} + \mb{x}_{i}$ are in $\tilde{D} \sm B_{\varepsilon/2}(\mb{x})$: indeed, we have
  \begin{equation*}
    \|\mb{z} + \mb{y}\|_{X} \leq \delta + \|\mb{y}\|_{X} < \varepsilon/2
  \end{equation*}
  by the choice of $\delta$, and
  \begin{equation*}
    \|\mb{x} - (\mb{z} + \mb{y} + \mb{x}_i) \|_{X}
    = \|\mb{x}' - \mb{z} - \mb{x}_i\|_{X}
    \geq \|\mb{x}' - \mb{x}_i\|_{X} - \|\mb{z}\|_{X}
    > \varepsilon - \delta > \varepsilon/2,
  \end{equation*}
  finishing the job.
\end{proof}

The concept of dentability is connected with (non-existence of) bounded separated trees as follows.

\begin{thm}\index{dentable sets!relation with separated trees}\index{separated trees!relation with dentable sets}
  Let $X$ be a Banach space, and suppose that for all $\delta > 0$, $X$ does not contain a bounded $\delta$-separated tree.
  Then every bounded subset of $X$ is dentable.
\end{thm}
  
\begin{proof}
  We prove the contrapositive: we suppose that there exists a bounded non-dentable set $D \subset X$, and given $\delta > 0$ we will construct a bounded $\delta$-separated tree.
  Since $D$ is non-dentable, there exists $\varepsilon > 0$ such that for all $\mb{x} \in D$,
  \begin{equation*}
    \mb{x} \in \overline{\conv}(D \sm B_{2\varepsilon}(\mb{x})).
  \end{equation*}
  By Lemma \ref{lem:dentlem}, for all $\mb{x} \in \tilde{D} := D + B_{\varepsilon}(\mb{0})$,
  \begin{equation*}
    \mb{x} \in \conv(\tilde{D} \sm B_{\varepsilon}(\mb{x})).
  \end{equation*}
  We will construct a $\varepsilon$-separated tree in the bounded set $\tilde{D}$, and by rescaling this will yield the result.

  Let $(\Omega,\mc{A},\P)$ be the interval $[0,1]$ with Borel $\sigma$-algebra and Lebesgue measure.
  We construct a $\varepsilon$-separated martingale inductively.
  Let $\mb{x}_{0} \in \tilde{D}$ be arbitrary and $\mb{f}_0 \equiv \mb{x}_0$.
  Since $\mb{x}_0 \in \conv(\tilde{D} \sm B_{\varepsilon}(\mb{x}_0))$, there exist numbers $\alpha_1,\ldots, \alpha_n \in (0,1)$ summing to $1$ and vectors $\mb{x}_{1}, \ldots, \mb{x}_{n} \in \tilde{D}$ such that
  \begin{equation*}
    \mb{x}_0 = \sum_{i=1}^{n} \alpha_{i} \mb{x}_{i} \qquad \text{and} \qquad \|\mb{x}_{i} - \mb{x}_{0}\|_X \geq \varepsilon.
  \end{equation*}
  Partition the unit interval $[0,1]$ into intervals $(I_i)_{i=1}^{n}$ with length $|I_{i}| = \alpha_{i}$, let $\mc{A}_{0}$ be the trivial $\sigma$-algebra on $[0,1]$, and let $\mc{A}_{1}$ be the $\sigma$-algebra generated by the intervals $(I_i)_{i=1}^{n}$.
  Define
  \begin{equation*}
    \mb{f}_1 = \sum_{i=1}^{n} \1_{I_i} \otimes \mb{x}_{i}.
  \end{equation*}
  Then $\E^{\mc{A}_0} \mb{f}_1 = \mb{f}_0$, and $\|\mb{f}_1(\omega) - \mb{f}_0(\omega)\|_{X} \geq \varepsilon$ for all $\omega \in [0,1]$.
  Since each point $\mb{x}_{i}$ is in $\tilde{D}$, we can repeat this process inductively, representing each $\mb{x}_{i}$ as a convex combination of vectors in $\tilde{D} \sm B_{\varepsilon}(\mb{x}_i)$, using these vectors to define $\mb{f}_2$ on $I_{i}$, and so on, to construct a $\varepsilon$-separated martingale valued in $\tilde{D}$, and thus a bounded $\varepsilon$-separated tree.
\end{proof}

Finally we will `complete the loop' in our discussion of the Radon--Nikodym property, martingale convergence, and dentability.

\begin{thm}\label{thm:dent-RNP}\index{Radon--Nikodym property!relation with dentable sets}\index{dentable sets!relation with RNP}
  Let $X$ be a Banach space such that every bounded subset of $X$ is dentable.
  Then $X$ has the Radon--Nikodym property.
\end{thm}

\begin{proof}
  Let $(S,\mc{A},\mu)$ be a finite measure space, and let $\map{\mb{\nu}}{\mc{A}}{X}$ be an $X$-valued vector measure with $\|\mb{\nu}(A)\|_{X} \leq \mu(A)$ for all $A \in \mc{A}$.
  Our task is to find a function $\mb{f} \in L^1(\mu;X)$ such that $\mb{\nu} = \mb{f}\mu$.
  By Proposition \ref{prop:RNP-finite-sufficient} this is sufficient to prove that $X$ has the RNP.

  For all sets $A \in \mc{A}$ let
  \begin{equation*}
    \begin{aligned}
      \mc{A}_{+}(A) &:= \{B \in \mc{A} : B \subset A, \mu(B) > 0\}, \\
      \mc{A}_{+} &:= \mc{A}_{+}(S)
    \end{aligned}
  \end{equation*}
  and for all $A \in \mc{A}_{+}$ define
  \begin{equation*}
    \mb{x}_{A} := \mu(A)^{-1} \mb{\nu}(A)\in X, \qquad 
    C_{A} := \{ \mb{x}_{B} : B \in \mc{A}_{+}(A)\} \subset X.
  \end{equation*}
  Note that $\|\mb{x}_{A}\|_X \leq 1$ for all $A \in \mc{A}_{+}$ by the assumptions on $\mb{\nu}$, so every $C_{A}$ is bounded and hence (by assumption) dentable.

  \textbf{We make the following claim:} \emph{for all $\varepsilon > 0$ and $A \in \mc{A}_{+}$, there exists $A' \in \mc{A}_+(A)$ such that $\diam(C_{A'}) \leq 2\varepsilon$.}
  
  We assume this is not the case and establish a contradiction.
  Thus there exist $\varepsilon > 0$ and $A \in \mc{A}_{+}$ such that $\diam(C_{A'}) > 2\varepsilon$ for all $A' \in \mc{A}_+(A)$.
  In particular, for every $\mb{x} \in X$ and $A' \in \mc{A}_{+}(A)$, there is a subset $B \in \mc{A}_{+}(A')$ such that $\|\mb{x} - \mb{x}_{B}\|_{X} > \varepsilon$.\footnote{Otherwise there would exist a vector $\tilde{\mb{x}} \in X$ with $\|\tilde{\mb{x}} - \mb{x}_{B}\|_{X} \leq \varepsilon$ for all $B \in \mc{A}_{+}(A')$, which implies $\|\mb{x}_{B} - \mb{x}_{B'}\|_{X} \leq 2\varepsilon$ for all $B,B' \in \mc{A}_+(A')$ and hence $\diam(C_{A'}) \leq 2\varepsilon$. Contradiction.}
  
  Now consider a fixed $A' \in \mc{A}_{+}(A)$ and let $\{B_{\lambda}\}_{\lambda \in \Lambda}$ be a maximal collection of disjoint elements of $\mc{A}_+(A')$ such that $\|\mb{x}_{A'} - \mb{x}_{B_\lambda}\|_{X} > \varepsilon$, where $\Lambda$ is some indexing set.
  Since the sets $B_{\lambda}$ are disjoint and have positive measure, and since
  \begin{equation*}
    \sum_{\lambda \in \Lambda} \mu(B_{\lambda}) \leq \mu(A') < \infty,
  \end{equation*}
  the indexing set $\Lambda$ is at most countable.
  By construction we must have
  \begin{equation}\label{eq:mu-sum}
    \mu\Big(A' \sm \bigcup_{\lambda \in \Lambda} B_\lambda\Big) = 0;
  \end{equation}
  otherwise we could find a set $B_{!} \in \mc{A}_{+}(A' \sm \cup_{\lambda} B_{\lambda}) \subset \mc{A}_{+}(A')$ such that $\|\mb{x}_{A'} - \mb{x}_{B_{!}}\|_{X} > \varepsilon$, contradicting the maximality of the set $
  \{B_{\lambda}\}$.
  The assumption on $\mb{\nu}$ yields
  \begin{equation*}
    \mb{\nu}\Big(A' \sm \bigcup_{\lambda \in \Lambda} B_\lambda\Big) = 0,
  \end{equation*}
  or equivalently (using countable additivity)
  \begin{equation*}
    \mb{\nu}(A') = \sum_{\lambda \in \Lambda} \mb{\nu}(B_\lambda).
  \end{equation*}
  This lets us write
  \begin{equation*}
      \mb{x}_{A'}
      = \mu(A')^{-1}\mb{\nu}(A') 
      = \sum_{\lambda \in \Lambda} \mu(A')^{-1}\mb{\nu}(B_\lambda) 
      = \sum_{\lambda \in \Lambda} \frac{\mu(B_\lambda)}{\mu(A')} \mb{x}_{B_\lambda}.
  \end{equation*}
  By \eqref{eq:mu-sum} we have that the coefficients of this series sum to $1$, and the vectors in the series satisfy
  \begin{equation*}
    \mb{x}_{B_{\lambda}} \in C_{A'} \qquad \text{and} \qquad \| \mb{x}_{A'}- \mb{x}_{B_\lambda} \|_X > \varepsilon,
  \end{equation*}
  which tells us that
  \begin{equation*}
    \mb{x}_{A'} \in \overline{\conv}(C_{A'} \sm B_{\varepsilon}(\mb{x}_{A'})).
  \end{equation*}
  Since this is true for all $A' \in \mc{A}_{+}(A)$, we find that $C_{A}$ is not dentable.
  This is a contradiction, which implies that our claim above is true.

  Now we return to the construction of a Radon--Nikodym derivative $\mb{f}$ of $\mb{\nu}$ with respect to $\mu$.
  Fix $\varepsilon > 0$.
  Using the claim we just established, let $(A_{\lambda})_{\lambda \in \Lambda}$ be a maximal disjoint collection of sets in $\mc{A}_+$ such that $\diam(C_{A_{\lambda}}) \leq 2\varepsilon$.
  Then $\Lambda$ is at most countable (by the same argument used in the last paragraph) and
  \begin{equation*}
    \mu(S \sm \bigcup_{\lambda \in \Lambda} A_{\lambda}) = 0;
  \end{equation*}
  if this were not the case then we could select $A' \in \mc{A}_+(S \sm \cup_{\lambda \in \Lambda} A_{\lambda}) \subset \mc{A}_+$ with $\diam(C_{A'}) \leq 2\varepsilon$ (using the claim) and contradict maximality.
  Define
  \begin{equation*}
    \mb{g}_{\varepsilon} := \sum_{\lambda \in \Lambda} \1_{A_{\lambda}} \otimes \mb{x}_{A_{\lambda}}.
  \end{equation*}
  Then $\mb{g}_{\varepsilon} \in L^1(\mu;X)$ (since it is bounded and the measure is finite).
  We will show that
  \begin{equation}\label{eq:g-computation}
    \|\mb{\nu} - \mb{g}_{\varepsilon}\mu\|_{\var} \leq 2\mu(S)\varepsilon;
  \end{equation}
  since this holds for all $\varepsilon > 0$, we find that $\mb{\nu}$ is in the closure in $M(S,\mc{A};X)$ of the set of measures of the form $\mb{g}\mu$ with $\mb{g} \in L^1(\mu;X)$.
  But this set is closed in $M(S,\mc{A};X)$ (Exercise \ref{ex:closure-M}), so there exists $\mb{g} \in L^1(\mu;X)$ with $\mb{\nu} = \mb{g}\mu$, as desired.

  It remains to show \eqref{eq:g-computation}.
  To see this first note that for all $A \in \mc{A}_+$
  \begin{equation*}
    \begin{aligned}
      \mb{\nu}(A) - \mb{g}_\varepsilon\mu(A)
      &= \sum_{\lambda \in \Lambda} \Big(  \mb{\nu}(A \cap A_{\lambda}) - \int_{A \cap A_{\lambda}} \mb{g}_{\varepsilon} \, \dd\mu \Big) \\
      &= \sum_{\lambda \in \Lambda} \Big(  \mb{\nu}(A \cap A_{\lambda}) - \mu(A \cap A_{\lambda}) \mb{x}_{A_{\lambda}} \Big) \\
      &= \sum_{\lambda \in \Lambda} \mu(A \cap A_{\lambda})(\mb{x}_{A \cap A_{\lambda}} - \mb{x}_{A_{\lambda}}),
    \end{aligned}
  \end{equation*}
  and so
  \begin{equation*}
    \|\mb{\nu}(A) - \mb{g}_\varepsilon\mu(A)\|_{X}
    \leq \sum_{\lambda \in \Lambda} \mu(A \cap A_{\lambda}) \|\mb{x}_{A \cap A_{\lambda}} - \mb{x}_{A_{\lambda}}\|_{X} \leq 2\mu(A)\varepsilon
  \end{equation*}
  using that $\diam(C_{A_{\lambda}}) \leq 2\varepsilon$.
  Taking the supremum over all partitions of $S$ proves \eqref{eq:g-computation} and completes the proof.
\end{proof}

Combining this with everything else we know, we have proven the following theorem.
\begin{thm}\label{thm:RNP-characterisations}\index{Radon--Nikodym property!equivalent characterisations}
  The following properties of a Banach space $X$ are equivalent:
  \begin{itemize}
  \item $X$ has the Radon--Nikodym property;
  \item $X$ has the $p$-martingale convergence property for all $p \in [1,\infty]$;
  \item $X$ has the $\infty$-martingale convergence property;
  \item for all $\delta > 0$, $X$ does not contain a bounded $\delta$-separated tree;
  \item every bounded subset of $X$ is dentable.
  \end{itemize}
\end{thm}

\begin{rmk}
  It is possible to prove directly that the $\infty$-MCP implies the RNP, but I don't think this is as nice as arguing via dentability.
  See \cite[Proof of Theorem 2.9]{gP16}.
\end{rmk}

This set of equivalences says quite a bit.
First, it says that almost everywhere convergence of $L^p$-bounded martingales holds for some $p \in [1,\infty]$ if and only if it holds for all $p \in [1,\infty]$.
This $p$-independence of martingale-based Banach space properties turns out to be fairly typical; martingales satisfy miraculous extrapolation properties of this kind.
Second, note that the first four properties are `extrinsic': the RNP makes reference to all $\sigma$-finite measure spaces, and the MCP properties and the nonexistence of bounded separated trees make reference to martingales valued in $X$.
In contrast, the last property is an intrinsic geometric property of $X$.
It is always satisfying to find an intrinsic geometric characterisation of what seems to be an extrinsic property.
One more point: by carefully looking at the proofs of these implications, one can show that it suffices to have the RNP with respect to the unit interval $[0,1]$ in order to show that every bounded subset $X$ is dentable.
This argument shows that a Banach space has the RNP if and only if it has the RNP with respect to the unit interval (see Exercise \ref{ex:RNP-unitinterval}).

Let's rattle off some consequences of this theorem.

\begin{cor}\index{Radon--Nikodym property!of reflexive spaces and separable duals}
  Reflexive spaces and separable dual spaces have the RNP.
\end{cor}

\begin{proof}
  By Theorem \ref{thm:MCP-sepdual} and Corollary \ref{cor:MCP-reflexive}, these spaces have the $\infty$-MCP.
  Thus they also have the RNP.
\end{proof}

\begin{cor}\index{Radon--Nikodym property!is separably determined}
  The RNP is separably determined, i.e. it holds for a Banach space $X$ if and only if it holds for all separable subspaces $Y \subset X$.
\end{cor}

\begin{proof}
  By Lemma \ref{lem:MCP-sepdet}, this is true for the $p$-MCP (for all $p \in [1,\infty]$, but these properties are now known to be equivalent anyway). 
\end{proof}

Finally, we return to Bochner spaces.
Recall Proposition \ref{prop:bochner-preduality}: for every $\sigma$-finite measure space $(S,\mc{A},\mu)$ and every Banach space $X$, for all $p \in [1,\infty]$, the map $\map{\Phi}{L^{p'}(S;X^*)}{L^p(S;X)^*}$ given by
\begin{equation*}
  \Phi \mb{g}(\mb{f}) = \int_{S} \langle \mb{f}(s), \mb{g}(s) \rangle \, \dd\mu(s) \qquad \forall \mb{f} \in L^p(S;X)
\end{equation*}
is an isometry onto a closed norming subspace of $L^p(S;X)^*$.
We will now complete this result with the help of the Radon--Nikodym property.

\begin{thm}\label{thm:bochner-duality-RNP}\index{Radon--Nikodym property!and duality of Bochner spaces}
  Let $X$ be a Banach space, and let $(S,\mc{A},\mu)$ be a countably generated measure space.\footnote{This says that there is a countable (or finite) collection of sets $(E_{\lambda})_{\lambda \in \Lambda}$ which generates $\mc{A}$.}
  Then the dual space $X^*$ has the Radon--Nikodym property with respect to $(S,\mc{A},\mu)$ if and only if for all $p \in [1,\infty)$ the isometric embedding $\map{\Phi}{L^{p'}(\mu;X^*)}{L^p(\mu;X)^*}$ is an isomorphism.
\end{thm}

We leave the proof as an exercise (Exercise \ref{ex:RNP-bochner}). 

  
\begin{rmk}\index{Radon--Nikodym property!equivalent characterisations}
  Not satisfied? See \cite[\textsection VII.6]{DU77} for 29 characterisations of the Radon--Nikodym property.
\end{rmk}


  


\section*{Exercises}

\begin{exercise}
  Let $X$ be a Banach space with the Radon--Nikodym property, and suppose that $Y$ is a Banach space which is isomorphic to $X$.
  Show that $Y$ also has the Radon--Nikodym property.
\end{exercise}

\begin{exercise}
  Let $X$ be a Banach space with the Radon--Nikodym property.
  Show that every bounded linear operator $\map{T}{L^1([0,1])}{X}$ is of the form
  \begin{equation*}
    Tf = \int_0^1 f(t) \mb{g}(t) \, \dd t
  \end{equation*}
  for some $\mb{g} \in L^1([0,1];X)$.
\end{exercise}

\begin{exercise}\index{conditional expectations!existence}
  Let $X$ be a Banach space with the Radon--Nikodym property.
  Use this property to show that for every measure space $(S,\mc{A},\mu)$ and every sub-$\sigma$-algebra $\mc{B} \subset \mc{A}$, every $\mb{f} \in L^1(\mc{A};X)$ has a conditional expectation with respect to $\mc{B}$.
  (Of course this holds for \emph{every} Banach space, but your job here is to derive it in a simpler way under the RNP assumption.)
\end{exercise}

\begin{exercise}\label{ex:fa-meas-ca}
  Let $X$ be a Banach space and $(S,\mc{A})$ a measure space.
  Let $\mb{\nu}$ be a \emph{finitely additive} $X$-valued vector measure on $(S,\mc{A})$, i.e. a function $\map{\mb{\nu}}{\mc{A}}{X}$ such that
  \begin{equation*}
    \mb{\nu}\Big(\sum_{n=1}^{N} E_n\Big) = \sum_{n=1}^{N} \mb{\nu}(E_n)
  \end{equation*}
  for all $N \in \N$ and $E_1,\ldots,E_n \in \mc{A}$.
  Suppose that there is a \emph{countably additive} (scalar) measure $\mu$ on $(S,\mc{A})$ such that
  \begin{equation*}
    \|\mb{\nu}(A)\|_{X} \leq \mu(A) \qquad \forall A \in \mc{A}.
  \end{equation*}
  Show that $\mb{\nu}$ is countably additive, $\|\mb{\nu}\|_{\var} \leq \|\mu\|_{\var}$, and $|\mb{\nu}| \ll \mu$.
\end{exercise}

\begin{exercise}\label{ex:closure-M}
  Let $(S,\mc{A})$ be a measurable space and $X$ a Banach space.
  Let $\mu$ be a finite measure on $(S,\mc{A})$.
  Show that the set of $X$-valued vector measures with Radon--Nikodym derivatives with respect to $\mu$, i.e. the set
  \begin{equation*}
    \{\mb{g}\mu : \mb{g} \in L^1(\mu;X)\} \subset M(S,\mc{A};X),
  \end{equation*}
  is closed in $M(S,\mc{A};X)$.
\end{exercise}

\begin{exercise}[Rademacher's theorem and the RNP]\index{theorem!Rademacher}\index{Radon--Nikodym property!equivalence with Rademacher's theorem}
  Show that a Banach space $X$ has the Radon--Nikodym property if and only if every $X$-valued Lipschitz function on $[0,1]$ is differentiable almost everywhere.
\end{exercise}

\begin{exercise}\label{ex:RNP-unitinterval}
  Suppose that $X$ has the Radon--Nikodym property with respect to the unit interval $[0,1]$ with Borel $\sigma$-algebra and Lebesgue measure.
  Show that $X$ has the Radon--Nikodym property with respect to all $\sigma$-finite measure spaces.
\end{exercise}

\begin{exercise}\label{ex:RNP-bochner}
  Prove Theorem \ref{thm:bochner-duality-RNP} in two different ways: first, using vector measures and the Radon--Nikodym property, then using martingales and the martingale convergence properties.\footnote{Hint: when proving RNP, you can reduce to the case of countably generated filtrations by using strong measurability. You can always search for references if you get stuck!}
\end{exercise}

\begin{exercise}\label{ex:Lp-RNP}\index{Radon--Nikodym property!of Bochner spaces}\index{Bochner spaces!have RNP}
  Let $X$ be a Banach space and $(\Omega,\mc{A},\mu)$ a measure space, and $p \in (1,\infty)$.
  Show that $X$ has the Radon--Nikodym property if and only if $L^p(\Omega;X)$ has the Radon--Nikodym property.
\end{exercise}



%%% Local Variables:
%%% mode: latex
%%% TeX-master: "../main.tex"
%%% End:


\chapter{Introduction to Rademacher sums}
\label{sec:rademacher}
\subsection{Random sums and Rademacher spaces}

\begin{defn}
  A \emph{Rademacher variable} is a random variable $\map{\varepsilon}{\Omega}{\{-1,+1\}}$ defined on a probability space $(\Omega,\mc{A},\P)$ such that
  \begin{equation*}
    \P(\varepsilon = -1) = \P(\varepsilon = +1) = \frac{1}{2}.
  \end{equation*}
  A \emph{Rademacher sequence} is a sequence of mutually independent Rademacher variables $(\varepsilon_{n})_{n \in \N}$ on a probability space $(\Omega,\mc{A},\P)$.
\end{defn}

\begin{rmk}
  We will also consider Rademacher sequences $(\varepsilon_{\lambda})_{\lambda \in \Lambda}$ on different countable (or finite) indexing sets $\Lambda$.
\end{rmk}

When we discuss Rademacher sequences we generally discuss properties with only depend on the (joint) distributions of the random variables, rather than on the random variables themselves.
Thus we can exploit properties or intuition coming from different realisations of Rademacher sequences.
Two standard choices are:

\begin{example}[The probabilist's Rademachers]
  As in Example \ref{eg:gambling-filtrations}, consider the probability space
  \begin{equation*}
    \Omega := \prod_{n \in \N} \{-1,1\} = \{-1,1\}^{\N}
  \end{equation*}
  with the product $\sigma$-algebra and measure, where the factors are equipped with the uniform probability measure, and let $\map{\pi_{n}}{\Omega}{\{-1,1\}}$ be the $n$-th coordinate function.
  Then $(\pi_{n})_{n \in \N}$ is a Rademacher sequence.
\end{example}

\begin{example}[The analyst's Rademachers]
  Let $\Omega = [0,1]$ with the Lebesgue measure and Borel $\sigma$-algebra, and consider the \emph{Rademacher functions}
  \begin{equation*}
    r_n(t) := \sgn(\sin(2^n\pi t)) \qquad \forall t \in [0,1], n = 1,2,\cdots.
  \end{equation*}
  These are square waves with period $2^{-n}$, and form a Rademacher sequence.
\end{example}

A \emph{(finite) Rademacher sum} in a Banach space $X$ is an $X$-valued random variable of the form
\begin{equation}\label{eq:rademacher-sum}
  \sum_{n = 0}^{N} \varepsilon_n \mb{x}_{n} \colon \Omega \to X,
\end{equation}
where $(\mb{x}_{n})_{n = 0}^{N}$ is a finite sequence of vectors in $X$ and $(\varepsilon_{n})_{n = 0}^{N}$ is a finite Rademacher sequence on $\Omega$.
Although the random variable in \eqref{eq:rademacher-sum} depends on the choice of Rademacher sequence, its distribution is independent of this choice (Exercise \ref{ex:rad-sum-dist}).\footnote{Recall that the distribution of a random variable $\map{f}{\Omega}{X}$ is the pushforward measure $f_{*}(\P)$ on $X$, given by $f_{*}(\P)(A) := \P(f^{-1}(A))$ for all Borel sets $A \subset X$.}
As a consequence, the \emph{(finite) Rademacher average}
\begin{equation*}
  \E \Big\| \sum_{n = 0}^{N} \varepsilon_{n} \mb{x}_{n} \Big\|_{X} = \int_{\Omega}  \Big\| \sum_{n = 0}^{N} \varepsilon_{n}(\omega) \mb{x}_{n} \Big\|_{X} \, \dd\P(\omega)
\end{equation*}
is independent of the choice of Rademacher sequence.

Rademacher averages show up \emph{all the time} in Banach-valued analysis, so we better get used to them.
There are two fundamental properties that they satisfy: the \emph{contraction principle} (to be proven in a moment), and the \emph{Kahane--Khintchine inequalities} (the subject of the next section).

\begin{thm}[Contraction principle]
  There exists a constant $C \leq 2$ with the following property: for all Banach spaces $X$, all $N \in \N$, and all finite sequences $(\mb{x}_{n})_{n = 0}^{N}$ in $X$ and $(a_{n})_{n=0}^{N}$ in the scalar field $\K$,
  \begin{equation*}
    \E \Big\| \sum_{n = 0}^{N} \varepsilon_{n} a_{n} \mb{x}_{n} \Big\|_{X} \leq C \E \Big\| \sum_{n = 0}^{N} \varepsilon_{n} \mb{x}_{n} \Big\|_{X}.
  \end{equation*}
\end{thm}

\begin{proof}
  First suppose that $a_n \in \{-1,+1\}$ for all $n = 0, \ldots, N$.
  Then $(a_n \varepsilon_{n})_{n=0}^{N}$ is a Rademacher sequence, and since Rademacher averages do not depend on the choice of Rademacher sequence we have
  \begin{equation*}
    \E \Big\| \sum_{n = 0}^{N} \varepsilon_{n} a_{n} \mb{x}_{n} \Big\|_{X} = \E \Big\| \sum_{n = 0}^{N} \varepsilon_{n} \mb{x}_{n} \Big\|_{X}.
  \end{equation*}
  Now for a general \emph{real} sequence $a = (a_n)_{n=0}^{N}$ with $-1 \leq a_n \leq 1$ for each $n$, note that the point $a \in \R^{N+1}$ lies in the hypercube with vertices
  \begin{equation*}
    V = \{v = (v_{0}, v_{1}\ldots, v_{N}) : v_{n} \in \{-1,+1\} \quad \forall n \}.
  \end{equation*}
  Since the hypercube is the convex hull of its vertices, there exist numbers $(\lambda_{v})_{v \in V}$ such that
  \begin{equation*}
    a = \sum_{v \in V} \lambda_{v} v \qquad \text{and} \qquad \sum_{v \in V} \lambda_{v} = 1.
  \end{equation*}
  Thus we have
  \begin{equation*}
    \begin{aligned}
    \E \Big\| \sum_{n = 0}^{N} \varepsilon_{n} a_{n} \mb{x}_{n} \Big\|_{X}
    &= \E \Big\| \sum_{n = 0}^{N} \varepsilon_{n} \big( \sum_{v \in V} \lambda_{v} v_{n} \big) \mb{x}_{n} \Big\|_{X} \\
    &\leq \sum_{v \in V} \lambda_{v} \E \Big\| \sum_{n = 0}^{N} \varepsilon_{n} v_{n} \mb{x}_{n} \Big\|_{X} \\
    &= \sum_{v \in V} \lambda_{v} \E \Big\| \sum_{n = 0}^{N} \varepsilon_{n} \mb{x}_{n} \Big\|_{X} 
    = \E \Big\| \sum_{n = 0}^{N} \varepsilon_{n} \mb{x}_{n} \Big\|_{X}
  \end{aligned}
\end{equation*}
using the equality for $\pm 1$-valued sequences.
The result for general real-valued sequences (with $C=1$) follows by scaling.
For complex Banach spaces and $\C$-valued sequences, the result (with $C=2$) follows by considering real and imaginary parts of the sequence $(a_n)_{n = 0}^{N}$ separately. 
\end{proof}

\begin{rmk}
  The proof shows that $C = 1$ suffices for real Banach spaces.
  For complex spaces the optimal constant is $C = \pi/2$, but this needs a different proof.
  See \cite[Proposition 3.2.10]{HNVW16}.
\end{rmk}

\subsection{The Kahane--Khintchine inequality and its consequences}

We formulated our Rademacher averages simply as the expectation of a Rademacher sum.
One could instead consider the $p$-th moments (or `$L^p$-averages')
\begin{equation*}
  \Big( \E \Big\| \sum_{n=0}^{N} \varepsilon_{n} \mb{x}_{n} \Big\|_{X}^{p} \Big)^{1/p} = \Big\| \sum_{n=0}^{N} \varepsilon_{n} \mb{x}_{n} \Big\|_{L^p(\Omega;X)}.
\end{equation*}
It turns out that these are independent of $p$ up to a constant \emph{uniformly in $N$}.

\begin{thm}[Kahane--Khintchine inequalitiy]\label{thm:kk}
  Let $X$ be a Banach space and let $(\varepsilon_{n})_{n \in \N}$ be a Rademacher sequence on a probability space $(\Omega,\mc{F},\P)$.
  Then for all $p,q \in (0,\infty)$ there exists a finite constant $\kappa_{p,q}$ such that for all finite sequences $(\mb{x}_n)_{n=1}^N$ in $X$,
  \begin{equation*}
    \Big\|\sum_{n=1}^{N} \varepsilon_{n} \mb{x}_{n} \Big\|_{L^p(\Omega;X)} \leq \kappa_{p,q} \Big\|\sum_{n=1}^{N} \varepsilon_{n} \mb{x}_{n} \Big\|_{L^q(\Omega;X)}.
  \end{equation*}
  That is, for all $p \in (0,\infty)$, the $L^p$-norms of a Rademacher sum are pairwise equivalent.  
\end{thm}

Since $\Omega$ is a probability space, H\"older's inequality yields the case $p \leq q$ with constant $\kappa_{p,q} = 1$, so the real difficulty is when $p > q$.

The Kahane--Khintchine inequality is so useful that it is often used without comment in the literature.
One well-known proof is a reduction to \emph{hypercontractivity of the heat semigroup on the discrete cube}, providing an interesting link with the analysis of Boolean functions (and of course this link goes much deeper).
We will follow \cite{HNVW16} and prove it via the \emph{John--Nirenberg theorem for adapted sequences}, which is quite useful in itself.
But before that, let's look at some consequences of Kahane--Khintchine.

\begin{prop}
  Let $X$ be a Banach space and consider a sequence $(\mb{x}_{n})_{n \in \N}$ in $X$.
    For all $p,q \in (0,\infty)$, the Rademacher sum
    \begin{equation}\label{eq:rad-sum-example}
      \sum_{n \in \N} \varepsilon_{n} \mb{x}_{n}
    \end{equation}
    converges in $L^p(\Omega;X)$ if and only if it converges in $L^q(\Omega;X)$.
\end{prop}

\begin{proof}
  Suppose that the Rademacher sum converges in $L^q(\Omega;X)$.
  Convergence of the Rademacher sum in $L^p(\Omega;X)$ is equivalent to the limit
  \begin{equation*}
    \limsup_{N,M \to \infty} \Big\| \sum_{n=N}^{M} \varepsilon_{n} \mb{x}_{n} \Big\|_{L^p(\Omega;X)} = 0.
  \end{equation*}
  But of course by Kahane--Khintchine we have
  \begin{equation*}
    \limsup_{N,M \to \infty} \Big\| \sum_{n=N}^{M} \varepsilon_{n} \mb{x}_{n} \Big\|_{L^p(\Omega;X)}
    \leq \kappa_{p,q} \limsup_{N,M \to \infty} \Big\| \sum_{n=N}^{M} \varepsilon_{n} \mb{x}_{n} \Big\|_{L^q(\Omega;X)} = 0.
  \end{equation*}
\end{proof}

When the Banach space under consideration is a Hilbert space, Rademacher averages are equivalent to $\ell^2$ sums.

\begin{thm}[Khintchine's inequality]\label{thm:khintchine}
  Let $H$ be a Hilbert space and $(\mb{h}_{n})_{n \in \N}$ a sequence in $H$.
  Then for all $p \in (0,\infty)$,
  \begin{equation*}
    \Big(\E \Big\| \sum_{n \in \N} \varepsilon_{n} \mb{h}_{n} \Big\|_{H}^{p} \Big)^{1/p} \simeq_{p} \|\mb{h}\|_{\ell^2(H)}.
  \end{equation*}
  In particular, for all scalar sequences $(a_{n})_{n \in \N}$,
  \begin{equation*}
    \Big( \E \Big| \sum_{n \in \N} \varepsilon_{n} a_{n} \Big|^{p} \Big)^{1/p} \simeq_{p} \Big( \sum_{n \in \N} |a_{n}|^{2} \Big)^{1/2}.
  \end{equation*}
\end{thm}

\begin{proof}
  Using Kahane--Khintchine, independence of the Rademacher variables, and $\E(\varepsilon_{n}^{2}) = 1$,
  \begin{equation*}
    \begin{aligned}
      \Big( \E \Big\| \sum_{n \in \N} \varepsilon_{n} \mb{h}_{n} \Big\|_{H}^{p} \Big)^{2/p}
      &\simeq_{p} \Big\| \sum_{n \in \N} \varepsilon_{n} \mb{h}_{n} \Big\|_{L^2(\Omega;H)}^{2} \\
      &=  \int_{\Omega} \big\langle \sum_{n \in \N} \varepsilon_{n}(\omega) \mb{h}_{n}, \sum_{m \in \N} \varepsilon_{m}(\omega) \mb{h}_{m} \big\rangle \, \dd \P(\omega) \\
      &  \sum_{n,m \in \N} \E(\varepsilon_n \varepsilon_m) \langle \mb{h}_{n}, \mb{h}_{m} \rangle 
      =  \sum_{n \in \N} \langle \mb{h}_{n}, \mb{h}_{n} \rangle  = \|\mb{h}\|_{\ell^2(H)}^{2}.
    \end{aligned}
  \end{equation*}
\end{proof}

When the Banach space is a Lebesgue space, the Kahane--Khintchine inequality also lets us conveniently identify Rademacher averages with \emph{square functions}.

\begin{thm}
  Let $(S,\mc{A},\mu)$ be a $\sigma$-finite measure space.
  Fix $p \in (0,\infty)$ and suppose $(f_{n})_{n \in \N}$ is a sequence of (scalar-valued) functions in $L^p(S,\mu)$.
  Then
  \begin{equation*}
    \E \Big\| \sum_{n \in \N} \varepsilon_{n} f_{n} \Big\|_{L^p(S,\mu)} \simeq_{p} \Big\| \Big( \sum_{n \in \N} |f_{n}|^{2} \Big)^{1/2} \Big\|_{L^p(S,\mu)}.
  \end{equation*}
\end{thm}

In this equality, the sum inside the $L^p$-norm is the function
\begin{equation*}
  \Big( \sum_{n \in \N} |f_{n}|^{2} \Big)^{1/2}(s) := \Big( \sum_{n \in \N} |f_{n}(s)|^{2} \Big)^{1/2} \qquad \forall s \in S.
\end{equation*}

\begin{proof}
  By Kahane--Khintchine, along with Fubini and Theorem \ref{thm:khintchine}, we have
  \begin{equation*}
    \begin{aligned}
      \Big(\E \Big\| \sum_{n \in \N} \varepsilon_{n} f_{n} \Big\|_{L^p(S,\mu)}\Big)^{p}
      &\simeq_{p} \E \Big\| \sum_{n \in \N} \varepsilon_{n} f_{n} \Big\|_{L^p(S,\mu)}^{p} \\
      &=  \int_{\Omega} \int_{S} \Big| \sum_{n \in \N} \varepsilon_{n}(\omega) f_{n}(s) \Big|^{p} \, \dd\mu(s) \, \dd\P(\omega)  \\
      &=  \int_{S} \int_{\Omega} \Big| \sum_{n \in \N} \varepsilon_{n}(\omega) f_{n}(s) \Big|^{p} \, \dd\P(\omega) \, \dd\mu(s)  \\
      &=  \int_{S} \E \Big| \sum_{n \in \N} \varepsilon_{n} f_{n}(s) \Big|^{p}  \, \dd\mu(s)  \\
      &\simeq \int_{S} \Big( \sum_{n \in \N} |f_{n}(s)|^{2} \Big)^{p/2}  \, \dd\mu(s) \\
      &= \Big\| \Big( \sum_{n \in \N} |f_{n}|^{2} \Big)^{1/2} \Big\|_{L^p(S,\mu)}^{p}.
    \end{aligned}
\end{equation*}
\end{proof}

\subsection{The John--Nirenberg inequality for stochastic processes}




Fix a Banach space $X$ and a probability space $(\Omega,\mc{A},\P)$.
Consider a filtration $(\mc{A}_n)_{n \in \N}$ and an adapted $X$-valued stochastic process $f = (f_n)_{n \in \N}$ (i.e. $f_n$ is $\mc{A}_n$-measurable for all $n \in \N$).
For all $q \in (0,\infty)$ we consider the following measure of the oscillation of $f$:
\begin{equation*}
    \|f\|_{*,q} := \sup_{\substack{k,n \in \N \\ k \leq n}} \sup_{\substack{A \in \mc{A}_k \\ 0 < \mu(A) < \infty}} \Big( \fint_{A} \|(f_n - f_{k-1})(\omega)\|_{X}^{q} \, \dd\P(\omega)\Big)^{1/q}
\end{equation*}

\begin{thm}[John--Nirenberg inequality for adapted sequences]\label{thm:jn-adapted-sequences}
  For all $p,q \in (0,\infty)$ there exists a finite constant $c_{p,q}$ such that, for all $f$ as above,
  \begin{equation*}
    \|f\|_{*,p} \leq c_{p,q} \|f\|_{*,q}.
  \end{equation*}
\end{thm}

We will prove this as a consequence of a series of lemmas, but before that, let's derive the Kahane--Khintchine inequality from it.

\begin{proof}[Proof of Theorem \ref{thm:kk}, assuming the John--Nirenberg inequality]
  Consider the filtration and adapted sequence\todo{have to unify all the notation... UP TO HERE}
  \begin{equation*}
    \mc{A}_n := \sigma(\{\varepsilon_j : 1 \leq j \leq n\}), \qquad \phi_n := \sum_{j=1}^n \varepsilon_j \mb{x}_j.
  \end{equation*}
  We claim that for all $q \in [1,\infty)$ we have
  \begin{equation}\label{eq:kk-claim}
    \|\phi\|_{*,q} = \Big\| \sum_{j=1}^N \varepsilon_j \mb{x}_j \Big\|_{L^q(\Omega;X)}.
  \end{equation}
  Assuming this for the moment, the John--Nirenberg inequality yields a finite constant $c_{p,q}$ such that
  \begin{equation*}
    \Big\| \sum_{j=1}^N \varepsilon_j \mb{x}_j \Big\|_{L^p(\Omega;X)}
    = \|\phi\|_{*,p}
    \leq c_{p,q} \|\phi\|_{*,q}
    = \Big\| \sum_{j=1}^N \varepsilon_j \mb{x}_j \Big\|_{L^q(\Omega;X)}
  \end{equation*}
  whenever $1 \leq q < p$.
  If $q < 1 \leq p$, we fix $\theta \in (0,1)$ so that $1/p = \theta/q + (1-\theta)/2p$, and log-convexity of $L^p$-norms gives
  \begin{equation*}
    \| \phi_N \|_{L^p(\Omega;X)}
    \leq \| \phi_N \|_{L^q(\Omega;X)}^{\theta} \| \phi_N \|_{L^{2p}(\Omega;X)}^{1 - \theta}
    \leq \| \phi_N \|_{L^q(\Omega;X)}^{\theta} (c_{2p,p} \| \phi_N \|_{L^{p}(\Omega;X)})^{1 - \theta},
  \end{equation*}
  which yields
  \begin{equation*}
    \| \phi_N \|_{L^p(\Omega;X)} \leq c_{2p,p}^{(1 - \theta)/\theta} \| \phi_N \|_{L^q(\Omega;X)}.
  \end{equation*}
  Finally, if $q < p < 1$, we simply estimate
  \begin{equation*}
    \| \phi_N \|_{L^p(\Omega;X)} \leq \| \phi_N \|_{L^1(\Omega;X)} \leq c_{1,q} \| \phi_N \|_{L^q(\Omega;X)}.
  \end{equation*}
  This covers all nontrivial cases, and it remains to prove the claimed equality \eqref{eq:kk-claim}.

  First recall that
  \begin{equation*}
    \|\phi\|_{*,q} =  \sup_{\substack{k,n \in \N \\ k \leq n}} \sup_{\substack{F \in \mc{F}_k \\ \mu(F) > 0}} \Big( \fint_{F} \|(\phi_n - \phi_{k-1})(s)\|_{X}^{q} \, \dd\P(s)\Big)^{1/q}.
  \end{equation*}
  Fix $k \leq n$ and a set $F \in \mc{F}_k$ of positive measure.
  We will compute
  \begin{equation*}
    \fint_{F} \|\1_{F} (\phi_n - \phi_{k-1})(s)\|_{X}^q \, \dd\P(s) = \P(F)^{-1} \E \Big( \1_{F} \Big\| \sum_{j=k}^{n} \varepsilon_j \mb{x}_j\Big\|_{X}^q \Big).
  \end{equation*}
  Observe that for all $\omega \in \Omega$
  \begin{equation*}
    \Big\| \sum_{j=k}^{n} \varepsilon_j(\omega) \mb{x}_j\Big\|_{X} = \Big\| \varepsilon_k(\omega) \Big( \mb{x}_k + \sum_{j=k+1}^{n} \varepsilon_{j}^{\prime}(\omega) \mb{x}_j \Big) \Big\|_{X} = \Big\|\mb{x}_k + \sum_{j=k+1}^{n} \varepsilon_{j}^{\prime} \mb{x}_j \Big\|_{X}
  \end{equation*}
  where
  \begin{equation*}
    \varepsilon_{j}^{\prime} :=
    \begin{cases}
      \varepsilon_{j} & j \leq k \\
      \varepsilon_{k} \varepsilon_{j} & k+1 \leq j,
    \end{cases}
  \end{equation*}
  since $\varepsilon_j$ is $\pm 1$-valued.
  Now note that the $\sigma$-algebra $\mc{F}^\prime_{k+1,n} := \sigma(\{\varepsilon_j^\prime : k+1 \leq j \leq n\})$ is independent of $\mc{F}_k$, since for all $1 \leq j \leq k$ and $k+1 \leq j' \leq n$ we have by independence of the original Rademacher sequence
  \begin{equation*}
    \E(\varepsilon_j \varepsilon_{j'}^\prime) = \E(\varepsilon_j \varepsilon_k \varepsilon_{j'})
    =
    \begin{cases}
      \E(\varepsilon_j) \E(\varepsilon_k) \E(\varepsilon_{j'}) & \text{if $j < k$} \\
       \E(\varepsilon_{j'}) & \text{if $j = k$}
    \end{cases}
    \quad = 0.
  \end{equation*}
  Thus, since $F \in \mc{F}_k$, we have by independence of $\mc{F}_k$ and $\mc{F}^{\prime}_{k+1, n}$
  \begin{equation*}
    \begin{aligned}
      \P(F)^{-1} \E \Big( \1_{F} \Big\| \sum_{j=k}^{n} \varepsilon_j \mb{x}_j\Big\|_{X}^q \Big)
      &=  \P(F)^{-1} \E \Big( \1_{F} \Big\|\mb{x}_k + \sum_{j=k+1}^{n} \varepsilon_{j}^{\prime} \mb{x}_j \Big\|_{X}^{q} \Big) \\
      &=  \E \Big\|\mb{x}_k + \sum_{j=k+1}^{n} \varepsilon_{j}^{\prime} \mb{x}_j \Big\|_{X}^{q} 
      =  \E  \Big\|\sum_{j=k}^{n} \varepsilon_{j} \mb{x}_j \Big\|_{X}^{q}. \\
    \end{aligned}
  \end{equation*}
  Now letting $\mc{F}_{k,n} := \sigma(\{\varepsilon_j : k \leq j \leq n\})$, we have by the $L^q$-contraction property of conditional expectations
  \begin{equation*}
    \E \Big\|\sum_{j=k}^{n} \varepsilon_{j} \mb{x}_j \Big\|_{X}^{q}
    = \E \Big\|\E^{\mc{F}_{k,n}} \Big(\sum_{j=1}^{N} \varepsilon_{j} \mb{x}_j \Big)\Big\|_{X}^{q}
    \leq \E \Big\|\sum_{j=1}^{N} \varepsilon_{j} \mb{x}_j \Big\|_{X}^{q}
  \end{equation*}
  with equality when $k=1$ and $n=N$.
  Thus
  \begin{equation*}
    \|\phi\|_{*,q} =  \sup_{\substack{k,n \in \N \\ k \leq n}}  \Big(\E  \Big\|\sum_{j=k}^{n} \varepsilon_{j} \mb{x}_j \Big\|_{X}^{q}\Big)^{1/q} = \Big(\E  \Big\|\sum_{j=1}^{N} \varepsilon_{j} \mb{x}_j \Big\|_{X}^{q}\Big)^{1/q}
  \end{equation*}
  which proves the claimed equality \eqref{eq:kk-claim} and completes the proof. 
\end{proof}

In the setting of Hilbert-valued functions (and in particular, scalar-valued functions), the Kahane--Khintchine inequality leads to the classical Khintchine inequalities.

\begin{cor}[Khintchine's inequalities]
  Let $H$ be a Hilbert space, and let $(\varepsilon_{n})_{n \in \N}$ be a Rademacher sequence on a probability space $(\Omega, \mc{F}, \P)$.
  Then for all $p \in (0,\infty)$ there exist finite constants $A_p$ and $B_p$ such that for all finite sequences $(\mb{h}_n)_{n=1}^N$ in $H$,
  \begin{equation*}
    A_p \Big( \sum_{n=1}^N \|\mb{h}_n\|_H^2 \Big)^{1/2} \leq \Big\| \sum_{n=1}^N \varepsilon_n \mb{h}_n \Big\|_{L^p(\Omega;H)} \leq B_p \Big( \sum_{n=1}^N \|\mb{h}_n\|_H^2 \Big)^{1/2}.
  \end{equation*}
\end{cor}

\begin{proof}
  By independence of the Rademacher variables we have
  \begin{equation}
    \begin{aligned}
      \Big\| \sum_{n=1}^N \varepsilon_n \mb{h}_n \Big\|_{L^2(\Omega;H)}^2
      &= \Big\langle \sum_{n=1}^N \varepsilon_n \mb{h}_n , \sum_{m=1}^N \varepsilon_m \mb{h}_m \Big\rangle \\
      &= \sum_{n,m = 1}^N \E(\varepsilon_n \varepsilon_m) \langle \mb{h}_n, \mb{h}_m \rangle 
      = \sum_{n=1}^N \|\mb{h}_{n}\|_{H}^2,
    \end{aligned}
  \end{equation}
  so the result is true for $p=2$ with $A_2 = B_2 = 1$.
  Now use Kahane--Khintchine to extend the result to general $p \in (0,\infty)$.
\end{proof}

\subsubsection{Proof of the John--Nirenberg inequality}

We return to our analysis of a sequence $\phi = (\phi_n)_{n \in \N}$ of $X$-valued functions adapted to a filtration $(\mc{F}_n)_{n \in \N}$ on a $\sigma$-finite measure space $(S,\mc{A},\mu)$.
We prove the John--Nirenberg inequality via a series of lemmas in which we obtain increasingly fine control on the oscillation of $\phi$.

\begin{lem}\label{lem:JN-proof-1}
  For all $k \leq n$, $F \in \mc{F}_k$, and $\alpha > 0$,
  \begin{equation}\label{eq:JN-proof-1-est}
    \mu(F \cap \{ \|\phi_n - \phi_{k-1}\|_{X} > \alpha \}) \leq \Big( \frac{\|\phi\|_{*,q}}{\alpha} \Big)^{q} \mu(F).
  \end{equation}
\end{lem}

\begin{proof}
  We can assume that $0 < \mu(F) < \infty$, otherwise there is nothing to prove.
  The left hand side of \eqref{eq:JN-proof-1-est} is bounded by
  \begin{equation*}
    \int_{F} \Big( \frac{\|\phi_n - \phi_{k-1}\|_{X}}{\alpha} \Big)^{q} \, \dd\mu \leq \mu(F) \Big( \frac{\|\phi\|_{*,q}}{\alpha} \Big)^{q}
  \end{equation*}
  since $F \in \mc{F}_k$ and $k \leq n$, by the definition of $\|\phi\|_{*,q}$.
\end{proof}

Next we show that oscillation control of the form above extends to more general stopping times.

\begin{lem}
  Suppose that there exist $\alpha > 0$ and $\eta > 0$ such that
  \begin{equation*}
    \mu(F \cap \{ \| \phi_n - \phi_{k-1} \| > \alpha \} ) \leq \eta \mu(F) \qquad \forall k \leq n, F \in \mc{F}_k.
  \end{equation*}
  Then for all $k \in \N$, $F \in \mc{F}_k$, and all stopping times $\nu$ such that $\nu \geq k$ on $F$,
  \begin{equation}\label{eq:jn-stoptime}
    \mu(F \cap \{\nu < \infty\} \cap \{ \| \phi_{\nu} - \phi_{k-1} \| > 2\alpha \} ) \leq 2\eta \mu(F).
  \end{equation}
\end{lem}

\begin{proof}
  Sum over all possible values of the stopping time:
  \begin{equation*}
      \mu(F \cap \{\nu < \infty\} \cap \{ \| \phi_{\nu} - \phi_{k-1} \| > 2\alpha \} ) 
      = \lim_{N \to \infty} \sum_{n = k}^{N} \mu(F_n \cap \{ \| \phi_{n} - \phi_{k-1} \| > 2\alpha \} ),
  \end{equation*}
  where $F_{n} := F \cap \{\nu = n\}$.
  For fixed $N \geq n > k$, since $F_n \in \mc{F}_n \subset \mc{F}_{n+1}$, we have by assumption
  \begin{equation*}
    \begin{aligned}
      &\mu(F_n \cap \{ \| \phi_{n} - \phi_{k-1} \| > 2\alpha \} ) \\
      &\leq \mu(F_n \cap \{ \| \phi_{n} - \phi_{N} \| > \alpha \} )
      + \mu(F_n \cap \{ \| \phi_{k-1} - \phi_{N} \| > \alpha \} ) \\
      &\leq \eta \mu(F_n) + \mu(F_n \cap \{ \| \phi_{k-1} - \phi_{N} \| > \alpha \} ).
    \end{aligned}
  \end{equation*}
  Thus we have
  \begin{equation*}
    \begin{aligned}
      &\lim_{N \to \infty} \sum_{n = k}^{N} \mu(F_n \cap \{ \| \phi_{n} - \phi_{k-1} \| > 2\alpha \} ) \\
      &\leq \lim_{N \to \infty} \Big( \eta \sum_{n=k}^N \mu(F_n) + \sum_{n=k}^N \mu(F_n \cap \{ \| \phi_{k-1} - \phi_{N} \| > \alpha \}) \Big) \\
      &\leq \eta \mu(F) + \lim_{N \to \infty} \mu(F \cap \{ \| \phi_{k-1} - \phi_{N} \| > \alpha \})  
      \leq 2\eta\mu(F)
    \end{aligned}
  \end{equation*}
  using the assumption and $F \in \mc{F}_k$ in the last estimate.
\end{proof}
  
In the following lemma we make use of the \emph{started sequence}
\begin{equation*}
  {}^{k-1}\phi = (\phi_n - \phi_{k-1})_{n \geq k-1}
\end{equation*}
and its maximal function
\begin{equation*}
  ({}^{k-1} \phi)^{*}(s) := \sup_{n \geq k-1} \|({}^{k-1}\phi)_n(s)\|_X = \sup_{n \geq k} \|(\phi_n - \phi_{k-1})(s)\|_X.
\end{equation*}

\begin{lem}\label{lem:jn-mf}
  Suppose that $\phi$ satisfies \eqref{eq:jn-stoptime} for all $k \in \N$, $F \in \mc{F}_k$, and all stopping times $\nu$ such that $\nu \geq k$ on $F$.
  Then for all $\lambda > 0$,
  \begin{equation}\label{eq:jn-mf-eq}
    \mu(F \cap \{ ({}^{k-1} \phi)^{*} > \lambda + 2\alpha \}) \leq 2\eta \mu(F \cap \{  ({}^{k-1} \phi)^{*} > \lambda \}) \qquad \forall k \in \N, F \in \mc{F}_{k}.
  \end{equation}
\end{lem}

\begin{proof}
  Fix $k \in \N$ and consider the stopping times
  \begin{equation*}
    \begin{aligned}
      \rho &:= \inf\{n \geq k : \|\phi_n - \phi_{k-1}\| > \lambda\}, \\
      \nu &:= \inf\{n \geq k : \|\phi_n - \phi_{k-1}\| > \lambda + 2\alpha\}.
    \end{aligned}
  \end{equation*}
  Then $k \leq \rho \leq \nu$, and \eqref{eq:jn-mf-eq} can be rewritten as
  \begin{equation*}
    \mu(F \cap \{\nu < \infty\}) \leq 2\eta \mu(F \cap \{\rho < \infty\}).
  \end{equation*}
  Now fix $n \geq k$ and let $F_n := F \cap \{\rho = n\} \in \mc{F}_n$.
  On $\{F_n \cap \{\nu < \infty\}\}$ we have
  \begin{equation*}
    \|\phi_{\nu} - \phi_{n-1}\| \geq \|\phi_{\nu} - \phi_{k-1}\| - \|\phi_{n-1} - \phi_{k-1}\| > (\lambda + 2\alpha) - \lambda = 2\alpha,
  \end{equation*}
  so
  \begin{equation*}
    \mu(F_n \cap \{\nu < \infty\}) = \mu(F_n \cap \{\nu < \infty\} \cap \{\|\phi_{\nu} - \phi_{n-1}\| > 2\alpha\} )
    \leq 2\eta \mu(F_n).
  \end{equation*}
  Summing over $n \geq k$ completes the proof.
\end{proof}

\begin{lem}
  Suppose that $f$ is a non-negative function supported in $F \in \mc{A}$, satisfying
  \begin{equation*}
    \mu(f > \lambda + \alpha) \leq \eta \mu(f > \lambda) \qquad \forall \lambda > 0
  \end{equation*}
  for some $\eta \in (0,1)$ and $\alpha > 0$.
  Then for all $p \in [1,\infty)$,
  \begin{equation*}
    \|f\|_p \leq \frac{1 + \eta^{1/p}}{1 - \eta^{1/p}} \alpha \mu(F)^{1/p}.
  \end{equation*}
\end{lem}

\begin{proof}
  {\color{red} WRITE PROOF}
\end{proof}

\todo{UP TO HERE. HAVE TO COMPLETE PROOF OF JOHN-NIRENBERG.}

{\color{blue}

\begin{equation*}
  \|\phi\|_{**,q} := \sup_{k \in \N} \sup_{\substack{F \in \mc{F}_k \\ 0 < \mu(F) < \infty}} \Big( \fint_{F} ({}^{k-1} \phi)^{*}(s)^q \, \dd\mu(s) \Big)^{1/q},
\end{equation*}

}




\subsection{Type and cotype}

\begin{itemize}
\item definitions and examples
\item K-convexity: implied by UMD; implies type/cotype duality
\item STATE that K-convexity iff nontrivial type (Maurey--Pisier theorem is too hard for this course)
\end{itemize}

\subsection{\texorpdfstring{$R$-boundedness of sets of operators}{R-boundedness of sets of operators}}

\subsection*{Exercises}

\begin{exercise}\label{ex:rad-sum-dist}
  Fix $N \in \N$, and let $(\varepsilon_{n})_{n=0}^{N}$ and $(\varepsilon_{n}')_{n=0}^{N}$ be two finite Rademacher sequences on probability spaces $\Omega$ and $\Omega'$ respectively.
  Let $(\mb{x}_{n})_{n=0}^{N}$ be a finite sequence in a Banach space $X$.
  Show that the Rademacher sums
  \begin{equation*}
    \sum_{n = 0}^{N} \varepsilon_n \mb{x}_{n} \colon \Omega \to X \qquad \text{and} \qquad
    \sum_{n = 0}^{N} \varepsilon_n' \mb{x}_{n} \colon \Omega' \to X
  \end{equation*}
  have the same distribution.
  (Hint: use Theorem \ref{thm:characteristic-function-uniqueness}.) 
\end{exercise}


%%% Local Variables:
%%% mode: latex
%%% TeX-master: "../main.tex"
%%% End:


\chapter{The UMD property}
\label{sec:UMD}
\input{main/umd.tex}

\chapter{Fourier multipliers on UMD-valued functions}
\label{sec:Fourier-multipliers}
\section{The Hilbert transform}

Recall from the introduction that the Hilbert transform of a scalar-valued function $\map{f}{\R}{\C}$ is defined by
\begin{equation*}
  Hf(x) := \frac{1}{\pi} \mathrm{p. v.} \int_{\R} f(x-y) \, \frac{\dd y}{y} := \frac{1}{\pi} \lim_{\substack{\varepsilon \downarrow 0 \\ R \uparrow \infty}} \int_{\varepsilon < |y| < R} f(x-y) \, \frac{\dd y}{y}
\end{equation*}
(the truncation of the integral can be done in different ways; this is not too important).
This operator arises in complex analysis as follows.
Given a real-valued function $\map{f}{\R}{\R}$, there is a unique holomorphic function $F$ on the upper half-space $\C_{+} = \{z \in \C : \Im(z) > 0\}$ such that for almost all $x \in \R = \partial \C_{+}$, the real part $\Re(F)$ satisfies
\begin{equation*}
  \lim_{z \to x} \Re(F(z)) = f(x) \qquad \ae \,  x \in \R,
\end{equation*}
provided the limit is taken within an appropriate region.
It turns out that
\begin{equation*}
  \lim_{z \to x} \Im(F(z)) = Hf(x)
\end{equation*}
for almost all $x \in \R$.
Thus the Hilbert transform computes what is called the \emph{conjugate function} of a real-valued $\map{f}{\R}{\R}$.
See \cite[\textsection 5.1.2]{grafakos} for more details.
There are plenty of other applications of this operator (and its generalisations, notably the Riesz transforms) outside of complex analysis, but we will not be too concerned with these.

Classical methods in harmonic analysis (most notably Calder\'on--Zygmund theory) imply that the Hilbert transform is bounded on $L^p(\R)$ for all $p \in (1,\infty)$, while explicit computation of $Hf$ when $f = \1_{[0,1]}$ shows that $H$ is not bounded on $L^1(\R)$ or $L^\infty(\R)$.
In this chapter (and the next) we will consider Banach-valued extensions of the Hilbert transform.
In particular, we will show that these extensions are bounded on $L^p(\R;X)$ for $p \in (1,\infty)$ if and only if the Banach space $X$ has the UMD property.

\begin{defn}
  Let $X$ be a complex Banach space.\footnote{All of this theory can be formulated for real Banach spaces too, but once we get to Fourier multipliers it will be much nicer to have restricted ourselves to the complex setting.}
  Given a function $\mb{f} \in L^p(\R;X)$ for some $p \in [1,\infty]$, for all $0 < \varepsilon < R < \infty$ we define the \emph{truncated Hilbert transform}
  \begin{equation*}
    H_{\varepsilon,R} \mb{f}(x) :=  \frac{1}{\pi} \int_{\varepsilon < |y| < R} \mb{f}(x-y) \, \frac{\dd y}{y} =  \frac{1}{\pi} \int_{\varepsilon < |x-y| < E} \mb{f}(y) \, \frac{\dd y}{x-y} \qquad \forall x \in \R.
  \end{equation*}
  We also define the \emph{Hilbert transform}
  \begin{equation*}
    H\mb{f}(x) := \lim_{\substack{\varepsilon \downarrow 0 \\ E \uparrow \infty}} H_{\varepsilon,E}\mb{f}(x)
  \end{equation*}
  provided that the limit (in $X$) exists.
  To distinguish these operator from the classical scalar-valued Hilbert transform (or its truncations) we may write $H_{X}$ or $H_{X, \varepsilon, E}$.
\end{defn}

Note that the truncated Hilbert transforms always exist, and they are bounded on $L^p(\R;X)$ for all $p \in [1,\infty]$: indeed, by Young's convolution inequality,
\begin{equation*}
  \|H_{\varepsilon,E} \mb{f}\|_{L^p(\R;X)} \leq \Big( \int_{\varepsilon < |y| < E} \frac{\dd y}{|y|} \Big) \|\mb{f}\|_{L^p(\R;X)} = C_{\varepsilon,E}  \|\mb{f}\|_{L^p(\R;X)}.
\end{equation*}
The constants $C_{\varepsilon,E}$ are always finite, but they blows up as $\varepsilon \downarrow 0$ and $E \uparrow \infty$.
To show the existence and boundedness of the Hilbert transform, we need to use methods that exploit the cancellation between positive and negative values of the kernel $1/y$.
For example, calculus can be used to show that $H\mb{f}$ exists when $\mb{f} \in C_{c}^1(\R;X)$ is continuously differentiable and compactly supported (Exercise \ref{ex:HT-C1}).

The Hilbert transform, at least when it exists, commutes with translations and dilations.
That is, if the input function $\mb{f}$ is replaced by a translation by $s \in \R$ or a dilation\footnote{The definition of $\Dil_{\lambda}$ below is such that $\|\Dil_{\lambda} \mb{f}\|_{2} = \|\mb{f}\|_{2}$: it is $L^2$-normalised.} by $\lambda > 0$,
\begin{equation*}
  \Tr_{s} \mb{f} (x) := \mb{f}(x - s), \qquad \Dil_{\lambda} \mb{f}(x) := \lambda^{-1/2}\mb{f}(x/\lambda), 
\end{equation*}
then the output of the Hilbert transform is also replaced by the same translation or dilation.
It also anticommutes with reflections, i.e. if the input function is replaced by
\begin{equation*}
  \Refl \mb{f}(x) := \mb{f}(-x),
\end{equation*}
then the output is replaced by the negative of its reflection.
These results are summed up as follows.

\begin{prop}
  Let $X$ be a complex Banach space and $\map{\mb{f}}{\R}{X}$, and suppose that $H\mb{f}(x)$ exists for almost all $x \in \R$.
  Then for all $s \in \R$ and $\lambda > 0$, the Hilbert transforms $H(\Tr_{s} \mb{f})$, $H(\Dil_{\lambda} \mb{f})$, and $H(\Refl \mb{f})$  exist almost everywhere, and
  \begin{equation*}
    H(\Tr_{s} \mb{f}) = \Tr_{s} (H\mb{f}), \qquad H(\Dil_{\lambda} \mb{f}) = \Dil_{\lambda} H(\mb{f}), \quad H(\Refl(\mb{f})) = -\Refl(H\mb{f}).
  \end{equation*}
\end{prop}

\begin{proof}
  For all $0 < \varepsilon < E < \infty$ the truncated Hilbert transforms, being convolution operators, satisfy
  \begin{equation*}
    H_{\varepsilon,E}(\Tr_{s} \mb{f}) = \Tr_{s} (H_{\varepsilon,E} \mb{f}),
  \end{equation*}
  and the result follows by taking $\varepsilon \downarrow 0$ and $E \uparrow \infty$.
  The reflection is handled in the same way, using that the convolution kernel $1/y$ is odd.
  As for the dilation, we have
  \begin{equation*}
    \begin{aligned}
      H_{\varepsilon,E}(\Dil_{\lambda} \mb{f})(x)
      &= \frac{1}{\lambda^{1/2}\pi} \int_{\varepsilon < |x-y| < E}  \mb{f}\Big(\frac{y}{\lambda}\Big) \, \frac{\dd y}{x-y} \\
      &= \frac{\lambda^{1/2}}{\pi} \int_{\varepsilon < |x - \lambda z| < E} \mb{f}(z) \, \frac{\dd z}{x - \lambda z} \\
      &= \frac{\lambda^{1/2}}{\pi} \int_{\varepsilon < |\lambda(\lambda^{-1}x - z)| < E} \mb{f}(z) \, \frac{\dd z}{\lambda(\lambda^{-1}x - z)} \\
      &= \Dil_{\lambda}(H_{\varepsilon/\lambda, E/\lambda} \mb{f})(x)
    \end{aligned}
  \end{equation*}
  and again the result follows by taking limits in $\varepsilon$ and $E$.
\end{proof}

In fact, these properties characterise the Hilbert transform among all bounded linear operators on $L^2(\R)$ (and in fact on $L^p(\R)$ for all $p$), up to a scalar multiple.
The following result is `well-known' in the scalar case: it appears as \cite[Exercise 5.1.11]{grafakos}.
The proof for geenral complex Banach spaces is Exercise \ref{ex:HT-char}.%mk\

\begin{prop}\label{prop:HT-char}
  Let $X$ be a complex Banach space, and suppose that $T$ is a bounded linear operator on $L^2(\R;X)$ which commutes with all translations and dilations, and which anticommutes with reflections.
  Then $T = cH_{X}$ for some $c \in \C$.
\end{prop}


\section{Dyadic systems}

To establish boundedness of the Hilbert transform associated with a UMD space, we will need to relate it to martingale transforms.
The transforms we will need are related to the Haar multipliers we considered in Section \ref{sec:haar-decomp}.
However, we will need not only the dyadic filtration on the unit interval $[0,1)$, but more general \emph{dyadic systems} on $\R$.

\begin{defn}
  For all $j \in \Z$ let
  \begin{equation*}
    \mc{D}_{j}^{0} = \{2^{-j}[k, k+1) : k \in \Z\}
  \end{equation*}
  denote the set of all dyadic intervals in $\R$ of length $2^{-j}$, and let
  \begin{equation*}
    \mc{D}^{0} := \bigcup_{j \in \Z} \mc{D}_{j}^{0}
  \end{equation*}
  denote the \emph{standard dyadic system}---i.e. the set of all dyadic intervals in $\R$.
  For every two-sided (i.e. $\Z$-indexed) sequence $\omega \in \{0,1\}^{\Z}$, define for all $j \in \Z$
  \begin{equation*}
    \mc{D}_{j}^{\omega} := \mc{D}_{j}^{0} + \sum_{i > j} 2^{-i} \omega_{i}:
  \end{equation*}
  this is the set of all dyadic intervals of length $2^{-j}$ as before, but with left endpoints shifted by the number $\sum_{i > j} 2^{-i} \omega_{i} \in [0,2^{-j})$: this number has the binary representation
  \begin{equation*}
    2^{-j}(0.\omega_{j+1}\omega_{j+2}\ldots)_{2} .
  \end{equation*}
  Finally, define the \emph{$\omega$-shifted dyadic system}
  \begin{equation*}
   \mc{D}^{\omega} := \bigcup_{j \in \Z} \mc{D}_{j}^{\omega}.
  \end{equation*}
\end{defn}

We call $\mc{D}^{\omega}$ a `dyadic system' because it satisfies the following property.\footnote{This is technically the \emph{definition} of a dyadic system of intervals in $\R$. \cite[Lemma 5.1.7]{HNVW16} then says that every such dyadic system is given by $\mc{D}^\omega$ for some $\omega$. More general notions of dyadic systems (not necessarily of intervals, not necessarily in $\R$) exist. A good reference on this topic is \cite{LN18}.}

\begin{prop}
  Suppose $\omega \in \{0,1\}^{\Z}$ and $j \in \Z$.
  Then every interval $I \in \mc{D}_{j}^{\omega}$ has length $2^{-j}$, and there exist intervals $I_{-}, I_{+} \in \mc{D}^{\omega}_{j+1}$ such that $I = I_{-} \cup I_{+}$.
\end{prop}

\begin{proof}
  By construction every interval $I \in \mc{D}_{j}^{\omega}$ has length $2^{-j}$.
  We have
  \begin{equation*}
    I = 2^{-j}[k,k+1) + \sum_{i > j} 2^{-i} \omega_{i}
  \end{equation*}
  for some $k \in \Z$.
  Let $k' := k - \omega_{j}$, and consider the intervals
  \begin{equation*}
    \begin{aligned}
      I_{-} &= 2^{-{j+1}}[2k', 2k'+1) + \sum_{i > j-1} 2^{-i} \omega_{i}, \\
      I_{+} &= 2^{-{j+1}}[2k'+1, 2k'+2) + \sum_{i > j-1} 2^{-i} \omega_{i}. 
    \end{aligned}
  \end{equation*}
  Both of these intervals are in $\mc{D}^{\omega}_{j-1}$, both have length $2^{-(j-1)}$, and they are adjacent.
  Thus their union is an interval of length $2^{-j}$.
  The left endpoint of $I_{-}$ is
  \begin{equation*}
    \begin{aligned}
    2^{-j+1}(2k') + \sum_{i > j-1} 2^{-i} \omega_{i}
    &= 2^{-j}(k - \omega_{j}) + 2^{-j} \omega_{j} + \sum_{i > j} 2^{-i} \omega_{i} \\
    &= 2^{-j}k + \sum_{i > j} 2^{-i} \omega_{i},
  \end{aligned}
  \end{equation*}
  which equals the left endpoint of $I$.
  Similarly, the right endpoint of $I_{+}$ is equal to the right endpoint of $I$.
  Thus $I = I_{-} \cup I_{+}$.
\end{proof}

\begin{rmk}
The intervals in a shifted dyadic system $\mc{D}^{\omega}$ also satisfy the \emph{dyadic dichotomy}: if $I$ and $J$ are intervals in $\mc{D}^{\omega}$, then they are either comparable ($I \subseteq J$ or $J \subseteq I$) or disjoint ($I \cap J = \varnothing$).
Prove this yourself in Exercise \ref{ex:dyadic-dichotomy}.
\end{rmk}

Dyadic systems are not translation invariant: given an interval $I \in \mc{D}^{\omega}$ and a parameter $x \in \R$, it is generally not true that $I + x \in \mc{D}^{\omega}$.
However, the \emph{set of dyadic systems} is itself translation invariant.

\begin{prop}
  Let $\omega \in \{0,1\}^{\Z}$ and $x \in \R$.
  Then $\mc{D}^{\omega} + x = \mc{D}^{\omega'}$ for some $\omega' \in \{0,1\}^{\Z}$.
\end{prop}

\begin{proof}
  Suppose $I \in \mc{D}^{\omega}$, so that
  \begin{equation*}
    I = 2^{-j}[k, k+1) + \sum_{i > j} 2^{-i} \omega_{j}
  \end{equation*}
  for some $k \in \Z$.
  The parameter $x$ has a binary expansion
  \begin{equation*}
    x = \sum_{i \in \Z} 2^{-i} \eta_{i}
    = 2^{-j} \sum_{i \in \Z} 2^{j-i}\eta_{i}
    = 2^{-j} \big( \sum_{i \leq j} 2^{j-i}\eta_{i} + \sum_{i > j} 2^{j-i}\eta_{i} \Big) 
  \end{equation*}
  for some finitely supported $\eta \in \{0,1\}^{\Z}$.
  Letting $k_{j} = k + \sum_{i \leq j} 2^{j-i} \eta_{i} \in \Z$, we thus have
  \begin{equation*}
    \begin{aligned}
      I + x &= 2^{-j}[k_{j}, k_{j} + 1) + \sum_{i > j} 2^{-i} (\omega_{j} + \eta_{i}) \\
      &= 2^{-j}[k_{j}, k_{j} + 1) + \sum_{i > j} 2^{-i} \omega_{j}'
    \end{aligned}
  \end{equation*}
  where $\omega'$ is the (formal) binary expansion of the sum of the numbers with (formal) binary expansions $\omega$ and $\eta$.
  Thus $\mc{D}^{\omega} + x \subseteq \mc{D}^{\omega'}$, and a symmetric argument shows that $\mc{D}^{\omega'} \subseteq \mc{D}^{\omega} + x$.
  Therefore we have $\mc{D}^{\omega} + x = \mc{D}^{\omega'}$.
\end{proof}

We are not content with shifted dyadic systems: we also need dilations.

\begin{defn}
  For $\omega \in \{0,1\}^{\Z}$ and $t > 0$, define the \emph{dilated dyadic system}
  \begin{equation*}
    t\mc{D}^{\omega} := \{tI : I \in \mc{D}^{\omega}\}.
  \end{equation*}
\end{defn}

Of course, if we take $t$ to be a dyadic power $t = 2^{j}$ for some $j \in \Z$, then $t\mc{D}^{\omega} = \mc{D}^{\omega'}$ where $\omega'$ is a shift of $\omega$.
Thus as long as we allow for changes in the parameter $\omega$ we only need to consider dilated dyadic systems $t\mc{D}^{\omega}$ for $t \in [1,2)$.
In the following sections we will refer to a translated shifted dyadic system $t\mc{D}^{\omega}$, for some $t \in [1,2)$ and $\omega \in \{0,1\}^{\Z}$, simply as a \emph{generalised dyadic system}, and we may suppress reference to $t$ and $\omega$ and denote it by $\mc{D}$.

Why do we consider generalised dyadic systems?
Consider the probability space $\Omega := [1,2) \times \{0,1\}^{\Z}$, where $[1,2)$ is equipped with a probability measure $\nu$ (not necessarily the Lebesgue measure) and each factor $\{0,1\}$ has the uniform probability measure.
Since our set of generalised dyadic systems is parametrised by $\Omega$, we are now in a position to consider a `random dyadic system', and to consider the associated `random martingale transforms'.
We are also able to take \emph{expectations} of such random martingale transforms.
By the construction of our shifted and dilated dyadic systems, these expectations will satisfy the same translation, dilation, and reflection invariances that characterise the Hilbert transform.\footnote{For reflection invariance of shifted dyadic systems see Exercise \ref{ex:dyadic-refln-invariance}.}
We will thus be able to recover the Hilbert transform as an expectation of random martingale transforms.

All of this is easier said than done: these invariances do not characterise the Hilbert transform itself, but rather $cH$ for some scalar $c \in \C$.
We will need to make sure that $c \neq 0$.

\section{Generalised Haar expansions and shift operators}

Recall from Proposition \ref{prop:haar-multipliers} that for all $\mb{f} \in L^p([0,1);X)$ with $p \in (1,\infty)$ and $X$ a UMD space, we have
\begin{equation*}
  \Big\| \sum_{I \in \mc{D}} a_{I} h_{I} \otimes \langle \mb{f}, h_{I} \rangle \Big\|_{L^p([0,1);X)} \lesssim_{p,X} \Big(\sup_{I \in \mc{D}} |a_{I}| \Big) \|\mb{f}\|_{L^p([0,1);X)}
\end{equation*}
for all scalar sequences $(a_{I})_{I \in \mc{D}}$ indexed by the set of dyadic subintervals of $[0,1)$.
In particular, Haar decompositions are unconditional in $L^p([0,1);X)$.
\todo{replace by two-sided randomised estimate via Burkholder's inequality}
This can be extended in a straightforward way to general dyadic systems on $\R$, and the $L^p$-unconditionality can also be supplemented with a.e. pointwise convergence.\todo{change notaation of generalised dyadic system}\todo{we need the reverse estimate but randomised}

\begin{thm}\label{thm:general-haar-decompositions}
  Let $X$ be a complex UMD Banach space, $p \in (1,\infty)$, and let $\mc{D}^{\omega}$ be a shifted dyadic system.
  Then for all scalar sequences $(a_{I})_{I \in \mc{D}^{\omega}}$ indexed by $\mc{D}^{\omega}$ and all $\mb{f} \in L^p(\R;X)$,
  \begin{equation}\label{eq:generalised-haar}
    \Big\| \sum_{I \in \mc{D}^{\omega}} a_{I} h_{I} \otimes \langle \mb{f}, h_{I} \rangle \Big\|_{L^p(\R;X)} \lesssim_{p,X} \Big(\sup_{I \in \mc{D}} |a_{I}| \Big) \|\mb{f}\|_{L^p([0,1);X)}
  \end{equation}
  (this statement includes the unconditional summability of the series in $L^p(\R;X)$).
  Furthermore, for almost every $x \in \R$ we have the pointwise limit
  \begin{equation*}
    \lim_{\substack{n \uparrow \infty \\ m \downarrow -\infty}} \sum_{\substack{I \in \mc{D}^{\omega} \\ 2^{m} < |I| < 2^{n}}} a_{I} h_{I} \otimes \langle \mb{f}, h_{I} \rangle = \sum_{I \in \mc{D}^{\omega}} a_{I} h_{I} \otimes \langle \mb{f}, h_{I} \rangle.
  \end{equation*}
  
\end{thm}

\begin{rmk}
  Previously we only defined the Haar functions $h_{I}$ for dyadic subintervals of $[0,1)$, but the same definition applies for all intervals in $\R$:
  we have
  \begin{equation*}
    h_{I} := |I|^{-1/2}(\1_{I_{-}} - \1_{I_{+}})
  \end{equation*}
  where $I_{-}$ and $I_{+}$ are the left and right halves of the interval $I$.
\end{rmk}



\begin{proof}[Proof of Theorem \ref{thm:general-haar-decompositions}]
  We begin with the $L^p$-unconditionality statement.
  By density, it suffices to prove the result for compactly supported $\mb{f}$.
  Under this assumption, there exists a scale $j_{0} \in \Z$ such that the support of $\mb{f}$ is covered by the union of two adjacent intervals $I_{j}^{0}, I_{j}^{1} \in \mc{D}^{\omega}_{j}$ for all $j \geq j_{0}$.\footnote{The union of these intervals need not be an element of $\mc{D}^{\omega}_{j-1}$: consider for example the standard dyadic system and the function $\1_{[-1/4,1/4)}$.
    The support of this function is covered by the standard dyadic intervals $[-1/2,0)$ and $[0,1/2)$, but $[-1/2,1/2)$ is not a standard dyadic interval.}
  Let $I_{j} := I_{j}^{0} \cup I_{j}^{1}$ be the union of these two intervals.
  Then for all $j \geq j_{0}$ we can write
  \begin{equation*}
    \begin{aligned}
      \Big\| \sum_{I \in \mc{D}^{\omega}} a_{I} h_{I} \otimes \langle \mb{f}, h_{I} \rangle \Big\|_{L^p(\R;X)}
      &\leq \Big\| \sum_{k \geq j} \sum_{\substack{I \in \mc{D}_{k}^{\omega} \\ I \subseteq I_{j}}} a_{I} h_{I} \otimes \langle \mb{f}, h_{I} \rangle \Big\|_{L^p(\R;X)} \\
      &+ \Big\| \sum_{k < j} \sum_{\substack{I \in \mc{D}_{k}^{\omega} \\ I \supset I_{j}}} a_{I} h_{I} \otimes \langle \mb{f}, h_{I} \rangle \Big\|_{L^p(\R;X)},
    \end{aligned}
  \end{equation*}
  exploiting the dyadic dichotomy to restrict to intervals $I \supset I_{j}$ in the second term.
  This term can be estimated by
  \begin{equation*}
    \begin{aligned}
      \Big\| \sum_{k < j} \sum_{\substack{I \in \mc{D}_{k}^{\omega} \\ I \supset I_{j}}} a_{I} h_{I} \otimes \langle \mb{f}, h_{I} \rangle \Big\|_{L^p(\R;X)}
      &\leq \sum_{k < j} \sum_{\sigma = 0,1} \big\| a_{I^{\sigma}_{k}} h_{I^{\sigma}_{k}} \otimes \langle \mb{f}, h_{I^{\sigma}_{j}} \rangle \big\|_{L^p(\R;X)} \\
      &\leq \|a_{\bullet}\|_{\infty} \sum_{k < j} \sum_{\sigma = 0,1} \|h_{I^{\sigma}_{k}}\|_{p} \| \langle \mb{f}, h_{I^{\sigma}_{j}} \rangle \|_{X} \\
      &\leq \|a_{\bullet}\|_{\infty} \sum_{k < j} \sum_{\sigma = 0,1} |I_{k}^\sigma|^{\frac{1}{p}-\frac{1}{2}} \|\mb{f}\|_{L^1(\R;X)} |I_{k}^{\sigma}|^{-\frac{1}{2}} \\
      &= 2\|a_{\bullet}\|_{\infty} |I_{j_{0}}|^{1/p'} \|\mb{f}\|_{L^p(\R;X)} \sum_{k < j} 2^{-k(\frac{1}{p} - 1)} \\
      &\lesssim 2\|a_{\bullet}\|_{\infty} |I_{j_{0}}|^{1/p'} \|\mb{f}\|_{L^p(\R;X)} 2^{j(1 - \frac{1}{p})}
    \end{aligned}
  \end{equation*}
  using that $\mb{f}$ is supported in $I_{j_{0}}$ to estimate the $L^1$ norm.
  To control the first term, let $\P_{j}$ be the renormalised Lebesgue measure $|I_{j}|^{-1} \dd x$ on $I_{j}$.
  Then $(I_{j},\P_{j})$ is a probability space (with the Borel $\sigma$-algebra).
  Define a filtration $(\mc{F}_{n}^{j})_{n \in \N}$ on $I_{j}$ by
  \begin{equation*}
    \mc{F}_{n}^{j} := \sigma\{I \in \mc{D}^{\omega}_{j+n} : I \subset I_{j}\}.
  \end{equation*}
  Then by the same argument we used to prove Proposition \ref{prop:haar-multipliers}---really we are just looking at a rescaled, shifted version of that result---we identify the generalised Haar multiplier with a martingale transform associated with the filtration $\mc{F}^{j}_{\bullet}$, with consequence that
  \begin{equation*}
    \begin{aligned}
      \Big\| \sum_{k \geq j} \sum_{\substack{I \in \mc{D}_{k}^{\omega} \\ I \subseteq I_{j}}} a_{I} h_{I} \otimes \langle \mb{f}, h_{I} \rangle \Big\|_{L^p(\R;X)}
      &=  |I_{j}|^{1/p} \Big\| \sum_{k \geq j} \sum_{\substack{I \in \mc{D}_{k}^{\omega} \\ I \subseteq I_{j}}} a_{I} h_{I} \otimes \langle \mb{f}, h_{I} \rangle \Big\|_{L^p(I_{j}, \P_{j};X)} \\
      &\lesssim_{p,X} |I_{j}|^{1/p} \|a_{\bullet}\|_{\infty} \|\mb{f}\|_{L^p(I_{j},\P_{j};X)} \\
      &\lesssim_{p,X} \|a_{\bullet}\|_{\infty} \|\mb{f}\|_{L^p(\R;X)}
    \end{aligned}
  \end{equation*}
  using that $X$ has the UMD property and that $\mb{f}$ is supported in $I_{j}$.

  In the end, for all $j \leq j_{0}$ we have
  \begin{equation*}
    \Big\| \sum_{I \in \mc{D}^{\omega}} a_{I} h_{I} \otimes \langle \mb{f}, h_{I} \rangle \Big\|_{L^p(\R;X)}
    \lesssim_{p,X} \|a_{\bullet}\|_{\infty} \|\mb{f}\|_{L^p(\R;X)} \Big(|I_{j_{0}}|^{1/p'}  2^{j(1 - \frac{1}{p})} + 1\Big),
  \end{equation*}
  so taking $j \downarrow -\infty$ yields
    \begin{equation*}
    \Big\| \sum_{I \in \mc{D}^{\omega}} a_{I} h_{I} \otimes \langle \mb{f}, h_{I} \rangle \Big\|_{L^p(\R;X)}
    \lesssim_{p,X} \|a_{\bullet}\|_{\infty}\|\mb{f}\|_{L^p(\R;X)},
  \end{equation*}
  completing the proof of $L^p$-uncondionality.

  Now we prove the a.e. pointwise convergence.
  Let
  \begin{equation*}
    \mb{g} := \sum_{I \in \mc{D}^{\omega}} a_{I} h_{I} \otimes \langle \mb{f}, h_{I} \rangle;
  \end{equation*}
  we know from the previous statement that this sum exists in $L^p(\R;X)$.
  For almost all $x \in \R$, fixing $m,n \in \Z$, we can write
  \begin{equation*}
    \begin{aligned}
    \mb{g}(x) - \sum_{\substack{x \in I \in \mc{D}^{\omega} \\ 2^{m} < |I| < 2^{n} \\ I \ni x}} a_{I} h_{I}(x) \otimes \langle \mb{f}, h_{I} \rangle 
    = \Big( \sum_{k \leq m} + \sum_{k \geq n} \Big) a_{I_{k}(x)} h_{I_{k}(x)}(x) \otimes \langle \mb{f}, h_{I_{k}(x)} \rangle
  \end{aligned}
\end{equation*}
where $I_{k}(x)$ is the unique interval in $\mc{D}^\omega_{k}$ containing $x$.
The first summand satisfies
\begin{equation*}
  \begin{aligned}
  \Big\| \sum_{k \leq m} a_{I_{k}(x)} h_{I_{k}(x)}(x) \otimes \langle \mb{f}, h_{I_{k}(x)} \rangle \Big\|
  &\leq \sum_{k \leq m} |I_{k}(x)|^{-1/2} \|\mb{f}\|_{L^p(\R;X)} \|h_{I_k(x)}\|_{p'} \\
  &=  \|\mb{f}\|_{L^p(\R;X)} \sum_{k \leq m}  2^{k(1 - 1/p')} \\
  &\lesssim \|\mb{f}\|_{L^p(\R;X)} 2^{m(1 - 1/p')}.
\end{aligned}
\end{equation*}
Now fix $\varepsilon > 0$, and let $m \in \Z$ be so small that this term is less than $\varepsilon$.
We get
\begin{equation*}
  \Big\|  \mb{g}(x) - \sum_{\substack{x \in I \in \mc{D}^{\omega} \\ 2^{m} < |I| < 2^{n} \\ I \ni x}} a_{I} h_{I}(x) \otimes \langle \mb{f}, h_{I} \rangle  \Big\|_{X} \leq \varepsilon + \Big\| \sum_{k \geq n} a_{I_{k}(x)} h_{I_{k}(x)}(x) \otimes \langle \mb{f}, h_{I_{k}(x)} \rangle \Big\|_{X}.
\end{equation*}
By considering only the subintervals of $I_{m}(x)$ and rescaling as in the proof of the unconditionality statement, the a.e. convergence of martingales implies that
\begin{equation*}
  \sum_{k \geq m} a_{I_{k}(x)} h_{I_{k}(x)}(x) \otimes \langle \mb{f}, h_{I_{k}(x)} \rangle
\end{equation*}
converges for almost all $x \in I_{m}$ (thus for almost all $x \in \R$, since varying $x \in \R$ leads to only countably many intervals $I_{m}(x)$).
Thus for $n \in \Z$ sufficiently large we have
\begin{equation*}
  \Big\| \sum_{k \geq m} a_{I_{k}(x)} h_{I_{k}(x)}(x) \otimes \langle \mb{f}, h_{I_{k}(x)} \rangle \Big\|_{X} < \varepsilon.
\end{equation*}
It follows that for small $m$ and large $n$ we have
\begin{equation*}
  \Big\|  \mb{g}(x) - \sum_{\substack{x \in I \in \mc{D}^{\omega} \\ 2^{m} < |I| < 2^{n} \\ I \ni x}} a_{I} h_{I}(x) \otimes \langle \mb{f}, h_{I} \rangle  \Big\|_{X} \leq 2\varepsilon,
\end{equation*}
and since $\varepsilon > 0$ was arbitrary we get our result.
\end{proof}

\begin{rmk}
  This argument could have been avoided by considering conditional expectations and martingales on general $\sigma$-finite measure spaces, indexed over $\Z$ rather than $\N$.
  The result would then follow from the `extended' UMD property defined via such martingales (which is, of course, equivalent to our UMD property).
\end{rmk}

\todo{exposition}

Notice that for every interval $I \subset \R$, the Haar function $h_{I}$ is just a translation and ($L^2$-normalised) dilation of the Haar function of the unit interval:%mk
\begin{equation*}
  h_{I} = \Dil_{|I|} \Tr_{\ell(I)} h,
\end{equation*}
where $\ell(I) := \inf(I)$ is the left endpoint of $I$ and $h = h_{[0,1)}$.
This motivates a more general definition: given a function $\map{k}{\R}{\C}$ and an interval $I \subset \R$, define
\begin{equation*}
  k_{I} := \Dil_{|I|} \Tr_{\ell(I)} k.
\end{equation*}

\begin{defn}
  Given a generalised dyadic system $\mc{D}$, we say a function $\map{k}{\R}{\C}$ is \emph{$\mc{D}$-admissible} if it is a finite linear combination of Haar functions $h_{J}$ with $J \in \mc{D}_{j}$ for some fixed $j \in \Z$ and $J \subset [0,1)$,
  and $\|k\|_{\infty} \leq 1$.
\end{defn}

\todo{exposition}

\begin{defn}
  Let $\mc{D}$ be a generalised dyadic system, $X$ a Banach space, and $k$ a $\mc{D}$-admissible function.
  The \emph{shift operator} $S_{k}$ associated with $k$ is defined by the formal sum
  \begin{equation*}
    S_{k} \mb{f} := \sum_{I \in \mc{D}} k_{I} \otimes \langle \mb{f}, h_{I} \rangle \qquad \forall \map{\mb{f}}{\R}{X}.
  \end{equation*}
\end{defn}

When the $\mc{D}$-admissible function $k$ is the base Haar function $h$, $S_{h}$ is just the identity operator (ignoring issues of well-definedness of the formal sum), as it sends a function to its own Haar expansion.
In general, since $k$ is a finite linear combination of Haar functions localised to equal-length subintervals of $[0,1)$, $S_{k}$ takes the Haar expansion of $\mb{f}$ and `shifts' each entry onto multiple Haar functions at the same scale.
This is most easily seen by considering the action of $S_{k}$ on $h$ (in the case $X = \C$):
\begin{equation*}
  S_{k} h = k_{[0,1)} = \sum_{n=1}^{N} c_{n} h_{I_{n}}
\end{equation*}
for some scalars $c_{n} \in \C$ and intervals $I_{n} \subset [0,1)$, $I_{n} \in \mc{D}_{j}$ for some $j \in \Z$.

\begin{thm}
  
\end{thm}

\section{Existence and boundedness of the Hilbert transform}

\section{Fourier multipliers}

\section{The Mikhlin and Littlewood--Paley  theorems}

\section{Exercises}

\begin{exercise}\label{ex:HT-C1}
  Let $X$ be a complex Banach space and suppose that $\mb{f} \in C_{c}^1(\R;X)$ is continuously differentiable and compactly supported.
  Show that the Hilbert transform $H\mb{f}(x)$ exists for all $x \in \R$.
\end{exercise}

\begin{exercise}
  Let $X$ be a complex Banach space such that the Hilbert transform $H_{X}$ is bounded on $L^p(\R;X)$ for some $p \in (1,\infty)$,
  Show that the scalar-valued Hilbert transform $H$ admits a bounded $X$-valued extension, in the sense of Definition \ref{defn:X-val-extn}.
\end{exercise}

\begin{exercise}\label{ex:HT-char}
  Prove Proposition \ref{prop:HT-char}, assuming the result in the scalar case.
\end{exercise}

\begin{exercise}\label{ex:dyadic-dichotomy}
  Prove the dyadic dichotomy for shifted dyadic systems: if $\omega \in \{0,1\}^{\Z}$ and $I,J \in \mc{D}^{\omega}$, then either $I$ and $J$ are comparable (i.e. $I \subseteq J$ or $J \subseteq I$) or disjoint ($I \cap J = \varnothing$).
\end{exercise}

\begin{exercise}\label{ex:dyadic-refln-invariance}
  Show that the set of shifted dyadic systems is reflection invariant, i.e. that $-\mc{D}^{\omega} = \mc{D}^{\omega'}$ for some $\omega' \in \{0,1\}^{\Z}$.
\end{exercise}




%%% Local Variables:
%%% mode: latex
%%% TeX-master: "../main.tex"
%%% End:


\chapter{Necessity of the UMD property}
\label{sec:UMD-necessity}
\emph{under construction}

%\section{Paley--Walsh martingales and the dyadic UMD property}

% \section{Transference of Fourier multipliers}

% \section{Bourgain's theorem (might need a few sections)}


%%% Local Variables:
%%% mode: latex
%%% TeX-master: "../main.tex"
%%% End:


\chapter{Applications to Schatten class operators}
\label{sec:schatten}


\section{Schatten class operators}
Let $H$ be a Hilbert space (finite or infinite dimensional) and consider a bounded linear operator $u \in \Lin(H)$.
The \emph{approximation numbers} (or \emph{singular values}) of $u$ are the numbers
\begin{equation*}
  a_{n}(u) := \inf\{\|u - v\|_{\Lin(H)} : \mathrm{rk}(v) < n\}
\end{equation*}
i.e. $a_{n}(u)$ is the distance from $u$ to the set of operators of rank less than $n$ in $\Lin(H)$.
We have $\|u\|_{\Lin(H)} = a_{1}(u)$, and the sequence $(a_{n}(u))_{n \geq 1}$ is monotonically decreasing, with $a_{n}(u) \to 0$ if and only if $u$ is compact.
By putting integrability conditions on the sequence $a_{\bullet}(u)$, we obtain useful classes of compact operators.

\begin{defn}
  For a Hilbert space $H$ and $p \in [1,\infty)$, we define the \emph{Schatten class} $\mc{C}^{p}(H) \subset \Lin(H)$ by
  \begin{equation*}
    \mc{C}^{p}(H) := \{u \in \Lin(H) : a_{\bullet}(u) \in \ell^{p}\}.
  \end{equation*}
  Equipped with the norm $\|u\|_{\mc{C}^{p}(H)} := \|a_{\bullet}(u)\|_{\ell^{p}}$, $\mc{C}^{p}(H)$ is a Banach space.
\end{defn}

One can equivalently write
\begin{equation*}
  \|u\|_{\mc{C}^{p}(H)} = \tr(|u|^{p})^{1/p},
\end{equation*}
where $|u| = u^{*}u$ and $|u|^{p}$ is defined via the spectral theorem.
From this representation it is clear that the Schatten class $\mc{C}^{1}(H)$ is exactly the set of trace class operators.
On the other hand, $\mc{C}^{2}(H)$ is the set of Hilbert--Schmidt operators---this is a Hilbert space under the inner product
\begin{equation*}
  \langle u, v \rangle := \sum_{i \in I} \langle u(h_{i}), v(h_{i}) \rangle_{H}
\end{equation*}
with $(h_{i})_{i \in I}$ being any orthonormal basis of $H$.
We have not defined $\mc{C}^\infty(H)$, but there are two conventions: it is generally defined as either the space $\mc{K}(H)$ of all compact operators, or simply as $\Lin(H)$.
In general, the Schatten class $\mc{C}^{p}(H)$ behaves like a non-commutative version of the sequence space $\ell^{p}$.
Indeed, since the norm of $\mc{C}^{p}(H)$ is directly modeled on that of $\ell^{p}$, we immediatley have the continuous inclusion
\begin{equation*}
  \mc{C}^{p_{0}}(H) \hookrightarrow \mc{C}^{p_{1}}(H) \qquad 1 \leq p_0 \leq p_1 < \infty.
\end{equation*}
We will not prove the basic properties of Schatten classes here; these can be found in \cite[Appendix D]{HNVW16}.
Below we collect the fundamental properties of Schatten classes that we will use without proof.

\todo{insert properties}

\begin{prop}[Duality]\label{prop:Schatten-duality}
  For any Hilbert space $H$ and $p \in (1,\infty)$, the map $\map{\phi}{\mc{C}^{p'}(H)}{\mc{C}^{p}(H)^*}$ given by
  \begin{equation*}
    \phi(u)(v) := \tr(uv)
  \end{equation*}
  is an isometric isomorphism.
  In particular we have the `non-commmutative H\"older inequality'
  \begin{equation*}
    |\tr(uv)| \leq \|u\|_{\mc{C}^{p}(H)} \|v\|_{\mc{C}^{p'}(H)},
  \end{equation*}
  and the dual of $\mc{C}^{p}(H)$ is naturally identified with $\mc{C}^{p'}(H)$.
\end{prop}

\begin{prop}[Interpolation]\label{prop:Schatten-interpolation}
  Fix a Hilbert space $H$, let $1 \leq p_0 < p_1 < \infty$.
  \begin{itemize}
  \item Suppose that $T \in \Lin(\mc{C}^{p_{1}}(H))$ is a bounded linear operator from $\mc{C}^{p_{1}}(H)$ to itself such that for all $u \in \mc{C}^{p_{0}}(H)$,
  \begin{equation*}
    \|Tu\|_{\mc{C}^{p_{0}}(H)} \lesssim \|u\|_{\mc{C}^{p_{0}}(H)}
  \end{equation*}
  (i.e. $T$ is bounded on $\mc{C}^{p_0}(H)$ as well as on the larger space $\mc{C}^{p_1}(H)$).
  Then for all $p \in [p_0,p_1]$, $T$ extends to a bounded linear operator on $\mc{C}^{p}(H)$, i.e.
  \begin{equation*}
    \|Tu\|_{\mc{C}^{p}(H)} \lesssim \|u\|_{\mc{C}^{p}(H)} \qquad \forall u \in \mc{C}^{p}(H).
  \end{equation*}

\item Let $(S,\mc{A},\mu)$ be a $\sigma$-finite measure space.
  Suppose that $T \in \Lin(L^{p_1}(S;\mc{C}^{p_{1}}(H)))$, and that for all $\mb{f} \in L^{p_0}(S;\mc{C}^{p_{0}}(H))$,
  \begin{equation*}
    \|T\mb{f}\|_{L^{p_0}(S;\mc{C}^{p_{0}}(H))} \lesssim \|\mb{f}\|_{L^{p_0}(S;\mc{C}^{p_{0}}(H))}.
  \end{equation*}
  Then for all $p \in [p_0,p_1]$, $T$ extends to a bounded linear operator on $L^{p}(S;\mc{C}^{p}(H)$).
  \end{itemize}
\end{prop}

\begin{rmk}[For those who know about interpolation of Banach spaces]  In the language of complex interpolation spaces (which we have not introduced), we have that $[\mc{C}^{p_0}(H), \mc{C}^{p_1}(H)]_{\theta} = \mc{C}^{p}(H)$, where $\frac{1}{p} = \frac{1-\theta}{p_0} + \frac{\theta}{p_1}$, for all $\theta \in [0,1]$: this implies the first part of Proposition \ref{prop:Schatten-interpolation}.
  The second part is then implied by the identity $[L^{p_0}(S;X_{0}), L^{p_1}(S; X_{1})]_{\theta} = L^{p}(S;[X_0,X_1]_{\theta})$, which holds for general $\sigma$-finite measure spaces $(S,\mc{A},\mu)$ and interpolation couples $(X_0,X_1)$.
\end{rmk}

\section{The UMD property of the Schatten classes}

\begin{thm}\label{thm:Schatten-UMD}
  Let $H$ be a Hilbert space and $p \in (1,\infty)$.
  Then the Schatten class $\mc{C}^{p}(H)$ is UMD.\footnote{We won't stress this point, but the UMD constants are bounded independently of the Hilbert space $H$.
  Naturally, they are smaller when $H$ is finite dimensional.}
\end{thm}

Of course, we won't prove the UMD property directly: instead, we'll show that the Hilbert transform $H_{\T}$ is bounded on $L^p(\T; \mc{C}^{p}(H))$, which by Remark \ref{rmk:HT-UMD-T} implies that $\mc{C}^{p}(H)$ is UMD.
This argument relies on the \emph{Cotlar identity} for the Hilbert transform on the torus: for all trigonometric polynomials $\map{f,g}{\T}{\C}$,
\begin{equation}\label{eq:Cotlar}
  H_{\T}\big(f \cdot (H_{\T} g) + (H_{\T} f) \cdot g) = (H_{\T} f)(H_{\T} g) - fg.
\end{equation}
(Exercise \ref{ex:Cotlar}).
This can be extended to Banach-valued functions as follows.
\begin{lem}
  Let $X$, $Y$, and $Z$ be complex Banach spaces, and let $\map{B}{X \times Y}{Z}$ be a bilinear operator.
  For all functions $\mb{F} \colon \T \to X$ and $\mb{G} \colon \T \to Y$, define the lifting $\map{\tilde{B}(\mb{F},\mb{G})}{\T}{Z}$ by
  \begin{equation*}
    \tilde{B}(\mb{F},\mb{G})(t) := B(\mb{F}(t), \mb{G}(t)).
  \end{equation*}
  Then for all trigonometric polynomials $\map{\mb{f}}{\T}{X}$ and $\map{\mb{g}}{\T}{Y}$,
  \begin{equation*}
    H_{\T}^{Z}\big(\tilde{B}(\mb{f}, H_{\T}^{Y} \mb{g}) + \tilde{B}(H_{\T}^{X} \mb{f}, \mb{g})) = \tilde{B}(H_{\T}^{X} \mb{f}, H_{\T}^{Y} \mb{g}) - \tilde{B}(\mb{f}, \mb{g})
  \end{equation*}
  using superscripts to make clear which Banach-valued $H_{\T}$ is being used.
\end{lem}

\begin{proof}
  By linearity, it suffices to check this for elementary tensors $\mb{f} = e_{n} \otimes \mb{x}$ and $\mb{g} = e_{m} \otimes \mb{y}$, with $n,m \in \Z$, $\mb{x} \in X$, and $\mb{y} \in Y$.
  Since $H_{\T}^{X}$ acts as the tensor extension $H_{\T} \otimes I$ on $X$-valued trigonometric polynomials, and likewise with $Y$ and $Z$ in place of $X$, we have
  \begin{equation*}
    \begin{aligned}
      &H_{\T}^{Z}\big(\tilde{B}(\mb{f}, H_{\T}^{Y} \mb{g}) + \tilde{B}(H_{\T}^{X} \mb{f}, \mb{g})) \\
      &= H_{\T}^{Z} \big( \tilde{B}(e_{n} \otimes \mb{x}, H_{\T}^{Y} (e_{m} \otimes \mb{y}) ) + \tilde{B}(H_{\T}^{X} (e_{n} \otimes \mb{x}), e_{m} \otimes \mb{y})) \\
      &= H_{\T}^{Z} \big( \tilde{B}(e_{n} \otimes \mb{x}, H_{\T} e_{m} \otimes \mb{y} ) + \tilde{B}(H_{\T} e_{n} \otimes \mb{x}, e_{m} \otimes \mb{y})) \\
      &= H_{\T}^{Z} \big( (e_{n} \cdot (H_{\T} e_{m}) + (H_{\T} e_{n}) \cdot e_{m} )\otimes B(\mb{x},  \mb{y})) \\
      &= H_{\T}(e_{n} \cdot (H_{\T} e_{m}) + (H_{\T} e_{n})) B(\mb{x}, \mb{y})
    \end{aligned}
  \end{equation*}
  and similarly
  \begin{equation*}
    \tilde{B}(H_{\T}^{X} \mb{f}, H_{\T}^{Y} \mb{g}) - \tilde{B}(\mb{f}, \mb{g})
    = ((H_{\T} e_n)(H_{\T} e_m) - e_n e_m) B(\mb{x},\mb{y}).
  \end{equation*}
  Thus the result follows from the complex-valued case.
\end{proof}

\begin{proof}[Proof of Theoem \ref{thm:Schatten-UMD}].
  By the duality relation $\mc{C}^{p}(H)^* = \mc{C}^{p'}(H)$ (Proposition \ref{prop:Schatten-duality}) and the fact that a Banach space is UMD if and only if its dual is UMD (Proposition \ref{prop:UMD-duality}), it suffices to prove the result for $p \in [2,\infty)$.
  We will show that the Hilbert transform $H_{\T}$ is bounded on $L^p(\T;\mc{C}^{p}(H))$ for all $p \in [2,\infty)$: let
  \begin{equation*}
    A_{p} := \|H_{\T}\|_{\Lin(L^p(\T;\mc{C}^{p}(H)))}.
  \end{equation*}
  By interpolation (Proposition \ref{prop:Schatten-interpolation}), it suffices to prove that $A_{2^{n}}$ is finite for $n \in \{1,2,\ldots\}$.
  We will prove this by induction, with base case $n=1$ being true since $\mc{C}^{2}(H)$ is a Hilbert space.

  \todo{UP TO HERE}
\end{proof}

\section{Schur multipliers}

\section{Operator Lipschitz functions}

\begin{thm}[Potapov--Sukochev]
  Let $H$ be a Hilbert space and $p \in (1,\infty)$, and let $u$ and $v$ be compact self-adjoint operators on $H$ such that $u - v \in \mc{C}^{p}$.
  Then for all Lipschitz functions $\map{f}{\R}{\R}$,
  \begin{equation*}
    \|f(u) - f(v)\|_{\mc{C}^{p}(H)} \lesssim_{p} \|f\|_{\Lip} \|u - v\|_{\mc{C}^{p}(H)}.
  \end{equation*}
\end{thm}

\section{Exercises}

\begin{exercise}\label{ex:Cotlar}
  Prove the Cotlar identity \eqref{eq:Cotlar} for complex-valued functions.\footnote{See \cite[\textsection 5.4.a]{HNVW16} if you're stuck.}
\end{exercise}

%%% Local Variables:
%%% mode: latex
%%% TeX-master: "../main.tex"
%%% End:


\chapter{Fourier type and Kwapie\'n's theorem}
\label{sec:fouriertype}
\section{Fourier type}

We have already discussed the Fourier transform on functions $\R \to X$ (Definition \ref{defn:FT}), and on functions $\T \to X$ (Section \ref{sec:HT-torus}); we can also define it on functions $\Z \to X$.
We compile these definitions below.
\begin{defn}
  Let $X$ be a complex Banach space.
  For integrable functions $\mb{f} \in L^1(\R;X)$, $\mb{g} \in L^1(\T;X)$, and $\mb{h} \in \ell^{1}(\Z;X)$, we define the Fourier transforms
  \begin{equation*}
    \begin{aligned}
      \mc{F}_{\R} \mb{f} &\in L^{\infty}(\R;X), \qquad \mc{F}_{\R} \mb{f}(\xi) := \int_{\R} e^{-2\pi i x \xi} \mb{f}(x) \, \dd x, \\
      \mc{F}_{\T} \mb{g} &\in \ell^{\infty}(\Z;X), \qquad \mc{F}_{\T} \mb{g}(n) := \int_{\T} e^{-2\pi i tn} \mb{g}(t) \, \dd t, \\
      \mc{F}_{\Z} \mb{h} &\in L^\infty(\T;X), \qquad \mc{F}_{\Z} \mb{h}(t) := \sum_{n \in \Z} e^{-2\pi i nt} \mb{h}(n). \\
    \end{aligned}
  \end{equation*}
\end{defn}
In all the definitions above, it is straightforward to show that the Fourier transform maps $L^1(G;X)$ to $L^\infty(\widehat{G};X)$, where $\widehat{\R} = \R$, $\widehat{\T} = \Z$, and $\widehat{\Z} = \T$.\footnote{More generally, if $G$ is a locally compact Abelian group equipped with the Haar measure, then this statement still holds, with $\widehat{G}$ the \emph{dual group} of $G$--also equipped with its Haar measure. We will not go into this level of generality here, but for more details on Fourier analysis on topological groups, see for example ...}\todo{reference to Rudin perhaps in the footnote}
When $X=H$ is a complex Hilbert space, Plancherel's theorem holds: for all $\mb{f} \in L^1(G; H)$ (with $G$ and $\widehat{G}$ as above), we have
\begin{equation*}
  \|\mc{F}_{G} \mb{f}\|_{L^2(\widehat{G};H)} = \|\mb{f}\|_{L^2(G;H)}.
\end{equation*}
By interpolation (e.g. by the Riesz--Thorin interpolation theorem) with the $L^1$-$L^\infty$ bound which holds for all Banach spaces, we deduce the Hilbert-valued \emph{Hausdorff--Young inequality}: for all $p \in [1,2]$ and $\mb{f} \in L^1(G;H) \cap L^p(G;H)$, we have
\begin{equation*}
  \|\mc{F}_{G} \mb{f}\|_{L^{p'}(\widehat{G};H)} \leq \|\mb{f}\|_{L^{p}(G;H)}.
\end{equation*}
It turns out that these estimates do not hold for general Banach spaces, so we turn the question of their validity into a definition.

\begin{defn}
  Let $X$ be a complex Banach space and $p \in [1,2]$.
  We say that $X$ has \emph{$\R$-Fourier type $p$} if for all $\mb{f} \in L^1(\R;X) \cap L^p(\R;X)$,
  \begin{equation*}
    \|\mc{F}_{\R} \mb{f}\|_{L^{p'}(\R;X)} \lesssim \|\mb{f}\|_{L^{p}(\R;X)}.
  \end{equation*}
  Likewise, we say that $X$ has \emph{$\T$-Fourier type $p$} if for all $\mb{g} \in L^p(\T;X) \subset L^1(\T;X)$,
  \begin{equation*}
    \|\mc{F}_{\T} \mb{g}\|_{\ell^{p'}(\Z;X)} \lesssim \|\mb{g}\|_{L^{p}(\T;X)},
  \end{equation*}
  and that $X$ has \emph{$\Z$-Fourier type $p$} if for all $\mb{h} \in \ell^{1}(\T;X) \subset \ell^{p}(\T;X)$,
  \begin{equation*}
    \|\mb{F}_{\Z} \mb{h}\|_{L^{p'}(\T;X)} \lesssim \|\mb{h}\|_{\ell^{p}(\Z;X)}.
  \end{equation*}
  That is, $X$ has \emph{$G$-Fourier type $p$} if the Fourier transform $\mc{F}_{G}$ extends to a bounded linear operator $L^p(G;X) \to L^{p'}(\widehat{G};X)$.
\end{defn}

\begin{example}
  In Example \ref{eg:FT} we showed that the Banach space $\ell^p$ (with $p \in [1,2)$) does not have $\R$-Fourier type $r$ for any $r \in (p,2]$.
  We will show that it has $G$-Fourier type $p$ for $G \in \{\R,\T,\Z\}$.\footnote{The same argument works for any locally compact Abelian group $G$.}
  Fix $\mb{f} \in L^1(G;\ell^{p}) \cap L^p(G;\ell^{p})$, and identify functions into $\ell^{p}$ with sequences of $\C$-valued functions.
  Under this identification, linearity of the Fourier transform yields that $(\mc{F}_{G}\mb{f})_{n} = \mc{F}_{G} (\mb{f}_{n})$ for all $n \in \N$; the Fourier transform $\mc{F}_{G}$ on the right hand side is acting on the $\C$-valued function $\mb{f}_{n}$.
  This lets us estimate, writing $\mu$ for the appropriate measure on $\widehat{G}$ (Lebesgue or counting),
  \begin{equation*}
    \begin{aligned}
      \|\mc{F}_{G} \mb{f}\|_{L^{p'}(\widehat{G};\ell^{p})}
      &= \Big( \int_{\widehat{G}} \|\mc{F}_{G} \mb{f}(\xi)\|_{\ell^{p}}^{p'} \, \dd\mu(\xi) \Big)^{1/p'} \\
      &= \Big( \int_{\widehat{G}} \Big( \sum_{n \in \N} |\mc{F}_{G} (\mb{f}_{n})(\xi)|^{p}  \Big)^{p'/p} \, \dd\mu(\xi) \Big)^{1/p'} \\
      &\leq \Big( \sum_{n \in \N}   \Big( \int_{\widehat{G}} |\mc{F}_{G} (\mb{f}_{n})(\xi)|^{p'} \dd\mu(\xi)  \Big)^{p/p'} \Big)^{1/p} \\
      &\leq \Big( \sum_{n \in \N}  \Big( \int_{G} |\mb{f}_{n}(x)|^{p} \dd\mu(\xi) \Big)^{p/p} \Big)^{1/p} = \|\mb{f}\|_{L^{p}(G;\ell^{p})},
  \end{aligned}
  \end{equation*}
  applying Minkowski's inequality on the third line, and the scalar-valued Hausdorff--Young inequality in the last line.
\end{example}

We will need a simple duality result for $\R$-Fourier type.
Analogous statements hold for $\T$- and $\Z$-Fourier type (and for more general locally compact Abelian groups $G$), but we will not need them explicitly.

\begin{prop}
  Let $X$ be a complex Banach space and $p \in [1,2]$.
  Then $X$ has $\R$-Fourier type $p$ if and only if $X^{*}$ has $\R$-Fourier type $p$.
\end{prop}

\begin{proof}
  First suppose $X$ has $\R$-Fourier type $p$, and let $\mb{g} \in L^1(\R;X^{*}) \cap L^{p}(\R;X^{*})$.
  We estimate $\|\mc{F}_{\R} \mb{g}\|_{L^{p'}(\R;X^{*})}$ by duality: for all $\mb{f} \in L^1(\R;X)} \cap L^p(\R;X)$, using Fubini, we have
  \begin{equation*}
    \begin{aligned}
      \Big| \int_{\R} \langle \mb{f}(\xi), \mc{F}_{\R} \mb{g}(\xi) \rangle \, \dd\xi \Big|
      &= \Big| \int_{\R} \Big\langle \mb{f}(\xi), \int_{\R} e^{-2\pi i x\xi} \mb{g}(x) \, \dd x \Big\rangle \, \dd\xi \Big| \\
      &= \Big| \int_{\R} \Big\langle \int_{\R} e^{-2\pi i x\xi} \mb{f}(\xi) \, \dd\xi, \mb{g}(x) \Big\rangle \, \dd x \Big| \\
      &= \Big| \int_{\R} \langle \mc{F}_{\R} \mb{f}(x), \mb{g}(x) \rangle \, \dd x\Big| \\
      &\leq \|\mc{F}_{\R} \mb{f} \|_{L^{p'}(\R;X)} \|\mb{g}\|_{L^p(\R;X^{*})} 
      \lesssim  \|\mb{f} \|_{L^{p}(\R;X)} \|\mb{g}\|_{L^p(\R;X^{*})}. 
    \end{aligned}
  \end{equation*}
  Since $L^1(\R;X)} \cap L^p(\R;X)$ is dense in $L^p(\R;X)$, we deduce that $\|\mc{F}_{\R} \mb{g}\|_{L^{p'}(\R;X)} \lesssim \|\mb{g}\|_{L^p(\R;X^{*})}$.
The same argument can be used to show that $X$ has $\R$-Fourier type $p$, assuming that $X^{*}$ has $\R$-Fourier type $p$.\footnote{Alternatively, starting from the assumption that $X^{*}$ has $\R$-Fourier type $p$, deduce that $X^{**}$ has $\R$-Fourier type $p$, and use that $X$ is isometric to a closed subspace of $X^{**}$. Naturally if a Banach space has $\R$-Fourier type $p$, then so do its closed subspaces.}
\end{proof}

\todo{remark earlier that $\R$, $\T$, $\Z$ fourier type are the same and that our first goal is to prove them equivalent}

% \begin{itemize}
% \item UMD implies nontrivial Fourier type (can't find an elementary proof)
% \item Fourier type $2$ implies type and cotype $2$
% \end{itemize}

\section{Kwapien's theorem}



%%% Local Variables:
%%% mode: latex
%%% TeX-master: "../main.tex"
%%% End:


\chapter{Appendix: Review of `assumed' topics}
\label{sec:appendices}
\subsection{Results from functional analysis}

\begin{itemize}
\item Hahn-Banach
\item dual, double dual, reflexivity
\item separability
\end{itemize}

\subsection{Results from probability theory}

\begin{itemize}
\item independence of random variables and $\sigma$-algebras
\item mutual independence of sequences of RVs
\end{itemize}

\subsection{Complex interpolation?}

\begin{itemize}
\item defn
\item basic structural results
\item (co)retraction theorem
\item examples 
\end{itemize}


%%% Local Variables:
%%% mode: latex
%%% TeX-master: "../main.tex"
%%% End:




% Bibliography (uncomment one among biblatex and bibtex
% BibLaTeX
%\printbibliography

% %% Bibtex instead of BibLaTeX (specified in packages.tex)
% %
%\footnotesize
\bibliographystyle{amsplain}
\bibliography{bibliography} 

% index
\printindex


\end{document}



%%% Local Variables: 
%%% mode: latex
%%% TeX-master: t
%%% End: 
