\section{Schatten class operators}
Let $H$ be a Hilbert space (finite or infinite dimensional) and consider a bounded linear operator $u \in \Lin(H)$.
The \emph{approximation numbers} (or \emph{singular values}) of $u$ are the numbers
\begin{equation*}
  a_{n}(u) := \inf\{\|u - v\|_{\Lin(H)} : \mathrm{rk}(v) < n\}
\end{equation*}
i.e. $a_{n}(u)$ is the distance from $u$ to the set of operators of rank less than $n$ in $\Lin(H)$.
We have $\|u\|_{\Lin(H)} = a_{1}(u)$, and the sequence $(a_{n}(u))_{n \geq 1}$ is monotonically decreasing, with $a_{n}(u) \to 0$ if and only if $u$ is compact.
By putting integrability conditions on the sequence $a_{\bullet}(u)$, we obtain useful classes of compact operators.

\begin{defn}
  For a Hilbert space $H$ and $p \in [1,\infty)$, we define the \emph{Schatten class} $\mc{C}^{p}(H) \subset \Lin(H)$ by
  \begin{equation*}
    \mc{C}^{p}(H) := \{u \in \Lin(H) : a_{\bullet}(u) \in \ell^{p}\}.
  \end{equation*}
  Equipped with the norm $\|u\|_{\mc{C}^{p}(H)} := \|a_{\bullet}(u)\|_{\ell^{p}}$, $\mc{C}^{p}(H)$ is a Banach space.
\end{defn}

One can equivalently write
\begin{equation*}
  \|u\|_{\mc{C}^{p}(H)} = \tr(|u|^{p})^{1/p},
\end{equation*}
where $|u| = (u^{*}u)^{1/2}$ (the square root of the positive operator $u^{*} u$) and $|u|^{p}$ is defined via the spectral theorem.
From this representation it is clear that the Schatten class $\mc{C}^{1}(H)$ is exactly the set of trace class operators.
On the other hand, $\mc{C}^{2}(H)$ is the set of Hilbert--Schmidt operators---this is a Hilbert space with inner product
\begin{equation*}
  \langle u, v \rangle := \tr(uv^*).
\end{equation*}
For general $p \in [1,\infty)$ we have
\begin{equation*}
  \|u\|_{\mc{C}^{p}(H)} = \|u^* \|_{\mc{C}^{p}(H)},
\end{equation*}
as well as the useful identity
\begin{equation}\label{eq:Schatten-convexity}
  \|u\|_{\mc{C}^{2p}(H)} = \tr(((u^*u)^{1/2})^{2p})^{1/2p} = \tr((u^* u)^{p})^{1/2p} = \|u^* u\|_{\mc{C}^{p}}^{1/2}
\end{equation}
which follows from $u^{*} u = |u^{*} u|$.
This will let us translate properties of the Banach spaces $\mc{C}^{p}(H)$, most importantly the UMD property, between each other.



We have not defined $\mc{C}^\infty(H)$ (and we will not need to), but there are two reasonable definitions: it is either the subspace $\mc{K}(H) \subset \Lin(H)$ of all compact operators, or $\Lin(H)$ itself.
In general, the Schatten class $\mc{C}^{p}(H)$ behaves like a non-commutative version of the sequence space $\ell^{p}$.
Indeed, since the norm of $\mc{C}^{p}(H)$ is directly modeled on that of $\ell^{p}$, we immediatley have the continuous inclusion
\begin{equation*}
  \mc{C}^{p_{0}}(H) \hookrightarrow \mc{C}^{p_{1}}(H) \qquad 1 \leq p_0 \leq p_1 < \infty.
\end{equation*}
We will not prove the basic properties of Schatten classes here; these can be found in \cite[Appendix D]{HNVW16}.
Below we collect the fundamental properties of Schatten classes that we will use without proof.

\begin{prop}
  For any Hilbert space $H$ and $p \in [1,\infty)$, the finite-rank operators form a dense subspace of $\mc{C}^{p}(H)$.
\end{prop}

\begin{prop}[Duality]\label{prop:Schatten-duality}
  For any Hilbert space $H$ and $p \in (1,\infty)$, the map $\map{\phi}{\mc{C}^{p'}(H)}{\mc{C}^{p}(H)^*}$ given by
  \begin{equation*}
    \phi(u)(v) := \tr(uv)
  \end{equation*}
  is an isometric isomorphism.
  Thus the dual of $\mc{C}^{p}(H)$ is naturally identified with $\mc{C}^{p'}(H)$.
  Using this along with \ref{eq:Schatten-convexity}, we can deduce the `non-commmutative H\"older inequality'
  \begin{equation}\label{eq:NC-holder}
    \|uv\|_{\mc{C}^{r}(H)} \leq \|u\|_{\mc{C}^{p}(H)} \|v\|_{\mc{C}^{q}(H)}
  \end{equation}
  whenever $p,q,r \in [1,\infty)$ satisfy $\frac{1}{p} + \frac{1}{q} = \frac{1}{r}$.
\end{prop}

\begin{prop}[Interpolation]\label{prop:Schatten-interpolation}
  Fix a Hilbert space $H$, let $1 \leq p_0 < p_1 < \infty$.
  \begin{itemize}
  \item Suppose that $T \in \Lin(\mc{C}^{p_{1}}(H))$ is a bounded linear operator from $\mc{C}^{p_{1}}(H)$ to itself such that for all $u \in \mc{C}^{p_{0}}(H)$,
  \begin{equation*}
    \|Tu\|_{\mc{C}^{p_{0}}(H)} \lesssim \|u\|_{\mc{C}^{p_{0}}(H)}
  \end{equation*}
  (i.e. $T$ is bounded on $\mc{C}^{p_0}(H)$ as well as on the larger space $\mc{C}^{p_1}(H)$).
  Then for all $p \in [p_0,p_1]$, $T$ extends to a bounded linear operator on $\mc{C}^{p}(H)$, i.e.
  \begin{equation*}
    \|Tu\|_{\mc{C}^{p}(H)} \lesssim \|u\|_{\mc{C}^{p}(H)} \qquad \forall u \in \mc{C}^{p}(H).
  \end{equation*}

\item Let $(S,\mc{A},\mu)$ be a $\sigma$-finite measure space.
  Suppose that $T \in \Lin(L^{p_1}(S;\mc{C}^{p_{1}}(H)))$, and that for all $\mb{f} \in L^{p_0}(S;\mc{C}^{p_{0}}(H))$,
  \begin{equation*}
    \|T\mb{f}\|_{L^{p_0}(S;\mc{C}^{p_{0}}(H))} \lesssim \|\mb{f}\|_{L^{p_0}(S;\mc{C}^{p_{0}}(H))}.
  \end{equation*}
  Then for all $p \in [p_0,p_1]$, $T$ extends to a bounded linear operator on $L^{p}(S;\mc{C}^{p}(H)$).
  \end{itemize}
\end{prop}

\begin{rmk}[For those who know about interpolation of Banach spaces]  In the language of complex interpolation spaces (which we have not introduced), we have that $[\mc{C}^{p_0}(H), \mc{C}^{p_1}(H)]_{\theta} = \mc{C}^{p}(H)$, where $\frac{1}{p} = \frac{1-\theta}{p_0} + \frac{\theta}{p_1}$, for all $\theta \in [0,1]$: this implies the first part of Proposition \ref{prop:Schatten-interpolation}.
  The second part is then implied by the identity $[L^{p_0}(S;X_{0}), L^{p_1}(S; X_{1})]_{\theta} = L^{p}(S;[X_0,X_1]_{\theta})$, which holds for general $\sigma$-finite measure spaces $(S,\mc{A},\mu)$ and interpolation couples $(X_0,X_1)$.
\end{rmk}

\section{The UMD property of the Schatten classes}

\begin{thm}\label{thm:Schatten-UMD}
  Let $H$ be a Hilbert space and $p \in (1,\infty)$.
  Then the Schatten class $\mc{C}^{p}(H)$ is UMD.\footnote{We won't stress this point, but the UMD constants are bounded independently of the Hilbert space $H$.
  Naturally, they are smaller when $H$ is finite dimensional.}
\end{thm}

Of course, we won't prove the UMD property directly: instead, we'll show that the Hilbert transform $H_{\T}$ is bounded on $L^p(\T; \mc{C}^{p}(H))$, which by Remark \ref{rmk:HT-UMD-T} implies that $\mc{C}^{p}(H)$ is UMD.
This argument relies on the \emph{Cotlar identity} for the Hilbert transform on the torus: for all trigonometric polynomials $\map{f,g}{\T}{\C}$,
\begin{equation}\label{eq:Cotlar}
  H_{\T}\big(f \cdot (H_{\T} g) + (H_{\T} f) \cdot g) = (H_{\T} f)(H_{\T} g) - fg.
\end{equation}
(Exercise \ref{ex:Cotlar}).
This can be extended to Banach-valued functions as follows.

\begin{lem}\label{lem:Cotlar-BV}
  Let $X$, $Y$, and $Z$ be complex Banach spaces, and let $\map{B}{X \times Y}{Z}$ be a bilinear operator.
  For all functions $\mb{F} \colon \T \to X$ and $\mb{G} \colon \T \to Y$, define the lifting $\map{\tilde{B}(\mb{F},\mb{G})}{\T}{Z}$ by
  \begin{equation*}
    \tilde{B}(\mb{F},\mb{G})(t) := B(\mb{F}(t), \mb{G}(t)).
  \end{equation*}
  Then for all trigonometric polynomials $\map{\mb{f}}{\T}{X}$ and $\map{\mb{g}}{\T}{Y}$,
  \begin{equation*}
    H_{\T}^{Z}\big(\tilde{B}(\mb{f}, H_{\T}^{Y} \mb{g}) + \tilde{B}(H_{\T}^{X} \mb{f}, \mb{g})) = \tilde{B}(H_{\T}^{X} \mb{f}, H_{\T}^{Y} \mb{g}) - \tilde{B}(\mb{f}, \mb{g})
  \end{equation*}
  using superscripts to make clear which Banach-valued $H_{\T}$ is being used.
\end{lem}

\begin{proof}
  By linearity, it suffices to check this for elementary tensors $\mb{f} = e_{n} \otimes \mb{x}$ and $\mb{g} = e_{m} \otimes \mb{y}$, with $n,m \in \Z$, $\mb{x} \in X$, and $\mb{y} \in Y$.
  Since $H_{\T}^{X}$ acts as the tensor extension $H_{\T} \otimes I$ on $X$-valued trigonometric polynomials, and likewise with $Y$ and $Z$ in place of $X$, we have
  \begin{equation*}
    \begin{aligned}
      &H_{\T}^{Z}\big(\tilde{B}(\mb{f}, H_{\T}^{Y} \mb{g}) + \tilde{B}(H_{\T}^{X} \mb{f}, \mb{g})) \\
      &= H_{\T}^{Z} \big( \tilde{B}(e_{n} \otimes \mb{x}, H_{\T}^{Y} (e_{m} \otimes \mb{y}) ) + \tilde{B}(H_{\T}^{X} (e_{n} \otimes \mb{x}), e_{m} \otimes \mb{y})) \\
      &= H_{\T}^{Z} \big( \tilde{B}(e_{n} \otimes \mb{x}, H_{\T} e_{m} \otimes \mb{y} ) + \tilde{B}(H_{\T} e_{n} \otimes \mb{x}, e_{m} \otimes \mb{y})) \\
      &= H_{\T}^{Z} \big( (e_{n} \cdot (H_{\T} e_{m}) + (H_{\T} e_{n}) \cdot e_{m} )\otimes B(\mb{x},  \mb{y})) \\
      &= H_{\T}(e_{n} \cdot (H_{\T} e_{m}) + (H_{\T} e_{n})) B(\mb{x}, \mb{y})
    \end{aligned}
  \end{equation*}
  and similarly
  \begin{equation*}
    \tilde{B}(H_{\T}^{X} \mb{f}, H_{\T}^{Y} \mb{g}) - \tilde{B}(\mb{f}, \mb{g})
    = ((H_{\T} e_n)(H_{\T} e_m) - e_n e_m) B(\mb{x},\mb{y}).
  \end{equation*}
  Thus the result follows from the complex-valued case.
\end{proof}

We will use this result with the bilinear form $\map{B}{\mc{C}^{2p}(H) \times \mc{C}^{2p}(H)}{\mc{C}^{p}(H)}$ given by $B(u,v) := uv$, whose boundedness follows by the non-commutative H\"older inequality \eqref{eq:NC-holder}.

\begin{proof}[Proof of Theorem \ref{thm:Schatten-UMD}]
  Write $\mc{C}^{p} = \mc{C}^{p}(H)$ to ease notation.
  By the duality relation $(\mc{C}^{p})^* = \mc{C}^{p'}$ (Proposition \ref{prop:Schatten-duality}) and the fact that a Banach space is UMD if and only if its dual is UMD (Proposition \ref{prop:UMD-duality}), it suffices to prove the result for $p \in [2,\infty)$.
  We will show that the Hilbert transform $H_{\T}$ is bounded on $L^p(\T;\mc{C}^{p})$ for all $p \in [2,\infty)$: let
  \begin{equation*}
    A_{p} := \|H_{\T}\|_{\Lin(L^p(\T;\mc{C}^{p}))}.
  \end{equation*}
  By interpolation (Proposition \ref{prop:Schatten-interpolation}), it suffices to prove that $A_{2^{n}}$ is finite for $n \in \{1,2,\ldots\}$.
  We will prove this by induction, with base case $n=1$ being true since $\mc{C}^{2}(H)$ is a Hilbert space.
  To simply notation further, we will fix $p \geq 2$, assume that $A_{p}$ is finite, and deduce that $A_{2p}$ is also finite.

  For all $N \in \N$, let $P_{N}(\mc{C}^{2p})$ denote the set of $\mc{C}^{2p}$-valued trigonometric polynomials $\mb{f}$ of degree $\leq N$ (i.e. $\hat{\mb{f}}(m) = \mb{0}$ whenever $|m| > N$), and define
  \begin{equation*}
    A_{2p,N} := \sup_{\mb{f} \in P_{N}, \mb{f} \neq 0} \frac{\|H_{\T} \mb{f}\|_{L^p(\T;\mc{C}^{2p})}}{\|\mb{f}\|_{L^p(\T;\mc{C}^{2p})}},
  \end{equation*}
  i.e. $A_{2p,N}$ is the norm of the Hilbert transform $H_{\T}$ restricted to the subspace $P_{N} \subset L^p(\T;\mc{C}^{2p})$.
  Then by density of the trigonometric polynomials
  \begin{equation*}
    A_{2p} = \lim_{N \to \infty} A_{2p,N},
  \end{equation*}
  and by Exercise \ref{ex:HT-FD}, the numbers $A_{2p,N}$ are all finite.\footnote{This could be thought of as a `noncommutative truncation argument'.}

  For every function $\mb{F} \colon \T \to \Lin(H)$, let $\mb{F}^*$ denote the pointwise adjoint, given by $\mb{F}^{*}(t) := \mb{F}(t)^*$.
  Fix a trigonometric polynomial $\mb{f} \in P_{N}$.
  We have the identity
  \begin{equation*}
    (H_{\T} \mb{F})^{*} = H_{\T} (\mb{F}^*)
  \end{equation*}
  (Exercise \ref{ex:HT-adjoint-Schatten}), so by \eqref{eq:Schatten-convexity}, we can write
  \begin{equation*}
    \begin{aligned}
      \|H_{\T} \mb{f}\|_{L^{2p}(\T;\mc{C}^{2p})}^{2}
      &= \Big( \int_{\T} \|H_{\T} \mb{f}(t) \|_{\mc{C}^{2p}}^{2p} \, \dd t \Big)^{2/2p} \\
      &= \Big( \int_{\T} \|H_{\T} \mb{f}(t)^* H_{\T} \mb{f}(t) \|_{\mc{C}^{p}}^{p} \, \dd t \Big)^{1/p} \\
      &= \Big( \int_{\T} \|H_{\T} (\mb{f}^{*})(t) H_{\T} \mb{f}(t) \|_{\mc{C}^{p}}^{p} \, \dd t \Big)^{1/p} \\
      &= \|H_{\T} (\mb{f}^*) H_{\T} \mb{f}\|_{L^p(\T;\mc{C}^{p})}.
    \end{aligned}
  \end{equation*}
  By Cotlar's identity (more precisely, by the Banach-valued extension \eqref{lem:Cotlar-BV}), the definition of $A_{p}$ and $A_{2p,N}$, and the classical and non-commutative H\"older inequalities, we can estimate
  \begin{equation*}
    \begin{aligned}
      &\|H_{\T} (\mb{f}^*) H_{\T} \mb{f}\|_{L^p(\T;\mc{C}^{p})} \\
      &\leq \Big\|H_{\T}\Big(\mb{f}^* (H_{\T} \mb{f}) + (H_{\T} \mb{f}^*)\mb{f} \Big) \Big\|_{L^p(\T;\mc{C}^{p})} + \|\mb{f}^{*} \mb{f}\|_{L^p(\T;\mc{C}^{p})} \\
      &\leq A_{p} \big( \|\mb{f}^* (H_{\T} \mb{f})\|_{L^p(\T;\mc{C}^{p})} + \| (H_{\T} \mb{f}^*)\mb{f} \|_{L^p(\T;\mc{C}^{p})} \big) + \|\mb{f}^{*} \mb{f}\|_{L^p(\T;\mc{C}^{p})} \\
      &\leq A_{p} \big( \|\mb{f}^*\| \|H_{\T} \mb{f}\| + \| H_{\T} \mb{f}^*\| \| \mb{f} \| \big)   + \|\mb{f}^{*}\| \| \mb{f}\| 
      \leq (2A_{p} A_{2p,N} + 1) \|\mb{f}\|^{2}.
    \end{aligned}
  \end{equation*}
  where all the unlabeled norms are in $L^{2p}(\T;\mc{C}^{2p})$.
  This tells us that
  \begin{equation*}
    A_{2p,N}^{2} \leq 2A_{p} A_{2p,N} + 1,
  \end{equation*}
  and thus by solving a quadratic equation
  \begin{equation*}
    A_{2p,N} \leq A_{p} + \sqrt{A_{p}^{2} - 1}.
  \end{equation*}
  Taking the limit as $N \to \infty$ and using that $A_{p} < \infty$, we find that $A_{2p} < \infty$ also.
  This completes the proof.
\end{proof}

\begin{rmk}
  With a small bit of extra work, this argument can be extended to show that
  \begin{equation*}
    \|H_{\T}\|_{L^p(\T;\mc{C}^{p})} \leq \frac{8}{\pi} \max(p,p')
  \end{equation*}
  for all $p \in (1,\infty)$, but we don't need this degree of control.
  See \cite[Proposition 5.4.2]{HNVW16}.
\end{rmk}

\section{Schur multipliers}

Let $H$ be a Hilbert space.
A \emph{countable spectral resolution} of $H$ is a sequence of pairwise orthogonal projections $(e_{\lambda})_{\lambda \in \Lambda}$ on $H$, where $\Lambda \subset \R$ is countable, such that $\mb{h} = \sum_{\lambda \in \Lambda} e_{\lambda} \mb{h}$ for all $\mb{h} \in H$.\footnote{Note that the vectors $e_{\lambda} \mb{h}$ are orthogonal, so convergence of the series is independent of the ordering.}

\begin{defn}
  Given a countable spectral resolution $(e_{\lambda})_{\lambda \in \Lambda}$ on a Hilbert space $H$, and given an infinite matrix $(m_{\lambda,\mu})_{\lambda,\mu \in \Lambda}$ indexed by $\Lambda$, the \emph{Schur multiplier} $M_{m}^{e}$ is the operator acting on $\Lin(H)$ (note that this is an operator \emph{on} $\Lin(H)$, not \emph{in} $\Lin(H)$) defined formally by
  \begin{equation*}
    M_{m}^{e} u := \sum_{\lambda, \mu \in \Lambda} m_{\lambda,\mu} e_{\lambda} u e_{\mu} \qquad \forall u \in \Lin(H).
  \end{equation*}
  Note that when $u$ has finite rank, $M_{m}^{e} u$ is well-defined, as the sum is finite.
\end{defn}

In essence, $M_{m}^{e}$ takes the representation of $u \in \Lin(H)$ as an infinite matrix with respect to the spectral resolution $e_{\bullet}$, and multiplies this \emph{componentwise} by the infinite matrix $m$.
There is no guarantee that this operation is well-defined on all $u \in \Lin(H)$, but when $m_{\lambda,\mu}$ is given by the differences of a scalar-valued Mikhlin symbol (recall Definition \ref{defn:mikhlin-symbol}), we can deduce the boundedness of the Schur multiplier $M_{m}^{e}$ on $\mc{C}^{p}(H)$ from the UMD property.

\begin{thm}
  Let $H$ be a Hilbert space with countable spectral resolution $(e_{\lambda})_{\lambda \in \Lambda}$, and fix $p \in (1,\infty)$.
  Let $m \in L^\infty(\R)$ be a scalar-valued Mikhlin symbol, and consider the infinite matrix
  \begin{equation*}
    m_{\lambda,\mu} := m(\lambda - \mu) \qquad \lambda,\mu \in \Lambda \subset \R.
  \end{equation*}
  Then for all finite rank operators $v \in \Lin(H)$ we have the estimate
  \begin{equation*}
    \|M_{m}^{e} v\|_{\mc{C}^{p}(H)} \lesssim_{p} \|m\|_{\mf{M}(\R)} \|v\|_{\mc{C}^{p}(H)},
  \end{equation*}
  and thus by density we can extend $M_{m}^{e}$ to a bounded linear operator on $\mc{C}^{p}(H)$.
\end{thm}

\begin{proof}
  For each $t \in \R$, consider the unitary operator
  \begin{equation*}
    u_{t} := \sum_{\lambda \in \Lambda} e^{2\pi i t} e_{\lambda}, 
  \end{equation*}
  so that $u_{t}^{*} = u_{-t}$.
  Then for all $t \in \R$ we have
  \begin{equation*}
    \| M_{m}^{e} v \|_{\mc{C}^{p}(H)} = 
    \|u_{t}  (M_{m}^{e} v) u_{-t} \|_{\mc{C}^{p}(H)}
  \end{equation*}
  since conjugating by a unitary preserves all approximation numbers, hence also the Schatten norms.
  \todo{UP TO HERE}
\end{proof}




\section{Operator Lipschitz functions}

\begin{thm}[Potapov--Sukochev]
  Let $H$ be a Hilbert space and $p \in (1,\infty)$, and let $u$ and $v$ be compact self-adjoint operators on $H$ such that $u - v \in \mc{C}^{p}$.
  Then for all Lipschitz functions $\map{f}{\R}{\R}$,
  \begin{equation*}
    \|f(u) - f(v)\|_{\mc{C}^{p}(H)} \lesssim_{p} \|f\|_{\Lip} \|u - v\|_{\mc{C}^{p}(H)}.
  \end{equation*}
\end{thm}

\section{Exercises}

\begin{exercise}\label{ex:Cotlar}
  Prove the Cotlar identity \eqref{eq:Cotlar} for complex-valued functions.\footnote{See \cite[\textsection 5.4.a]{HNVW16} if you're stuck.}
\end{exercise}

\begin{exercise}\label{ex:HT-FD}
  Let $X$ be a complex Banach space and let $P_{N}(X)$ denote the set of trigonometric polynomials $\map{\mb{f}}{\T}{X}$ of degree $\leq N$ (meaning that $\hat{\mb{f}}$ is supported in $\{-N,-N+1,\cdots,N-1,N\}$).
  Show that 
  \begin{equation*}
    \|H_{\T} \mb{f}\|_{L^p(\T;X)} \lesssim_{p,X,N} \|\mb{f}\|_{L^p(\T;X)} \qquad \forall \mb{f} \in P_{N}(X).
  \end{equation*}
\end{exercise}

\begin{exercise}\label{ex:HT-adjoint-Schatten}
  Let $H$ be a Hilbert space and  $\mb{f} \colon \T \to \Lin(H)$ a $\Lin(H)$-valued trigonometric polynomial.
  Show that $(H_{\T} \mb{f})(t))^* = (H_{\T} (\mb{f}^*))(t)$, where $\mb{f}^*$ is the pointwise adjoint of $\mb{f}$.
\end{exercise}


%%% Local Variables:
%%% mode: latex
%%% TeX-master: "../main.tex"
%%% End:
