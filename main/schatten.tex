\section{Schatten class operators}
Let $H$ be a Hilbert space (finite or infinite dimensional) and consider a bounded linear operator $u \in \Lin(H)$.
The \emph{approximation numbers} (or \emph{singular values}) of $u$ are the numbers
\begin{equation*}
  a_{n}(u) := \inf\{\|u - v\|_{\Lin(H)} : \mathrm{rk}(v) < n\}
\end{equation*}
i.e. $a_{n}(u)$ is the distance from $u$ to the set of operators of rank less than $n$ in $\Lin(H)$.
We have $\|u\|_{\Lin(H)} = a_{1}(u)$, and the sequence $(a_{n}(u))_{n \geq 1}$ is monotonically decreasing, with $a_{n}(u) \to 0$ if and only if $u$ is compact.
By putting integrability conditions on the sequence $a_{\bullet}(u)$, we obtain useful classes of compact operators.

\begin{defn}
  For a Hilbert space $H$ and $p \in [1,\infty)$, we define the \emph{Schatten class} $\mc{C}^{p}(H) \subset \Lin(H)$ by
  \begin{equation*}
    \mc{C}^{p}(H) := \{u \in \Lin(H) : a_{\bullet}(u) \in \ell^{p}\}.
  \end{equation*}
  Equipped with the norm $\|u\|_{\mc{C}^{p}(H)} := \|a_{\bullet}(u)\|_{\ell^{p}}$, $\mc{C}^{p}(H)$ is a Banach space.
\end{defn}

One can equivalently write
\begin{equation*}
  \|u\|_{\mc{C}^{p}(H)} = \tr(|u|^{p})^{1/p},
\end{equation*}
where $|u| = (u^{*}u)^{1/2}$ (the square root of the positive operator $u^{*} u$) and $|u|^{p}$ is defined via the spectral theorem.
From this representation it is clear that the Schatten class $\mc{C}^{1}(H)$ is exactly the set of trace class operators.
On the other hand, $\mc{C}^{2}(H)$ is the set of Hilbert--Schmidt operators---this is a Hilbert space with inner product
\begin{equation*}
  \langle u, v \rangle := \tr(uv^*).
\end{equation*}
For general $p \in [1,\infty)$ we have
\begin{equation*}
  \|u\|_{\mc{C}^{p}(H)} = \|u^* \|_{\mc{C}^{p}(H)},
\end{equation*}
as well as the useful identity
\begin{equation}\label{eq:Schatten-convexity}
  \|u\|_{\mc{C}^{2p}(H)} = \tr(((u^*u)^{1/2})^{2p})^{1/2p} = \tr((u^* u)^{p})^{1/2p} = \|u^* u\|_{\mc{C}^{p}}^{1/2}
\end{equation}
which follows from $u^{*} u = |u^{*} u|$.
This will let us translate properties of the Banach spaces $\mc{C}^{p}(H)$, most importantly the UMD property, between each other.



We have not defined $\mc{C}^\infty(H)$ (and we will not need to), but there are two reasonable definitions: it is either the subspace $\mc{K}(H) \subset \Lin(H)$ of all compact operators, or $\Lin(H)$ itself.
In general, the Schatten class $\mc{C}^{p}(H)$ behaves like a non-commutative version of the sequence space $\ell^{p}$.
Indeed, since the norm of $\mc{C}^{p}(H)$ is directly modeled on that of $\ell^{p}$, we immediatley have the continuous inclusion
\begin{equation*}
  \mc{C}^{p_{0}}(H) \hookrightarrow \mc{C}^{p_{1}}(H) \qquad 1 \leq p_0 \leq p_1 < \infty.
\end{equation*}
We will not prove the basic properties of Schatten classes here; these can be found in \cite[Appendix D]{HNVW16}.
Below we collect the fundamental properties of Schatten classes that we will use without proof.

\begin{prop}
  For any Hilbert space $H$ and $p \in [1,\infty)$, the finite rank operators form a dense subspace of $\mc{C}^{p}(H)$.
\end{prop}

\begin{prop}[Duality]\label{prop:Schatten-duality}
  For any Hilbert space $H$ and $p \in (1,\infty)$, the map $\map{\phi}{\mc{C}^{p'}(H)}{\mc{C}^{p}(H)^*}$ given by
  \begin{equation*}
    \phi(u)(v) := \tr(uv)
  \end{equation*}
  is an isometric isomorphism.
  Thus the dual of $\mc{C}^{p}(H)$ is naturally identified with $\mc{C}^{p'}(H)$.
  Using this along with \ref{eq:Schatten-convexity}, we can deduce the `non-commmutative H\"older inequality'
  \begin{equation}\label{eq:NC-holder}
    \|uv\|_{\mc{C}^{r}(H)} \leq \|u\|_{\mc{C}^{p}(H)} \|v\|_{\mc{C}^{q}(H)}
  \end{equation}
  whenever $p,q,r \in [1,\infty)$ satisfy $\frac{1}{p} + \frac{1}{q} = \frac{1}{r}$.
\end{prop}

\begin{prop}[Interpolation]\label{prop:Schatten-interpolation}
  Fix a Hilbert space $H$, let $1 \leq p_0 < p_1 < \infty$.
  \begin{itemize}
  \item Suppose that $T \in \Lin(\mc{C}^{p_{1}}(H))$ is a bounded linear operator from $\mc{C}^{p_{1}}(H)$ to itself such that for all $u \in \mc{C}^{p_{0}}(H)$,
  \begin{equation*}
    \|Tu\|_{\mc{C}^{p_{0}}(H)} \lesssim \|u\|_{\mc{C}^{p_{0}}(H)}
  \end{equation*}
  (i.e. $T$ is bounded on $\mc{C}^{p_0}(H)$ as well as on the larger space $\mc{C}^{p_1}(H)$).
  Then for all $p \in [p_0,p_1]$, $T$ extends to a bounded linear operator on $\mc{C}^{p}(H)$, i.e.
  \begin{equation*}
    \|Tu\|_{\mc{C}^{p}(H)} \lesssim \|u\|_{\mc{C}^{p}(H)} \qquad \forall u \in \mc{C}^{p}(H).
  \end{equation*}

\item Let $(S,\mc{A},\mu)$ be a $\sigma$-finite measure space.
  Suppose that $T \in \Lin(L^{p_1}(S;\mc{C}^{p_{1}}(H)))$, and that for all $\mb{f} \in L^{p_0}(S;\mc{C}^{p_{0}}(H))$,
  \begin{equation*}
    \|T\mb{f}\|_{L^{p_0}(S;\mc{C}^{p_{0}}(H))} \lesssim \|\mb{f}\|_{L^{p_0}(S;\mc{C}^{p_{0}}(H))}.
  \end{equation*}
  Then for all $p \in [p_0,p_1]$, $T$ extends to a bounded linear operator on $L^{p}(S;\mc{C}^{p}(H)$).
  \end{itemize}
\end{prop}

\begin{rmk}[For those who know about interpolation of Banach spaces]  In the language of complex interpolation spaces (which we have not introduced), we have that $[\mc{C}^{p_0}(H), \mc{C}^{p_1}(H)]_{\theta} = \mc{C}^{p}(H)$, where $\frac{1}{p} = \frac{1-\theta}{p_0} + \frac{\theta}{p_1}$, for all $\theta \in [0,1]$: this implies the first part of Proposition \ref{prop:Schatten-interpolation}.
  The second part is then implied by the identity $[L^{p_0}(S;X_{0}), L^{p_1}(S; X_{1})]_{\theta} = L^{p}(S;[X_0,X_1]_{\theta})$, which holds for general $\sigma$-finite measure spaces $(S,\mc{A},\mu)$ and interpolation couples $(X_0,X_1)$.
\end{rmk}

\section{The UMD property of the Schatten classes}

\begin{thm}\label{thm:Schatten-UMD}
  Let $H$ be a Hilbert space and $p \in (1,\infty)$.
  Then the Schatten class $\mc{C}^{p}(H)$ is UMD, and there exists a constant $C_{p} < \infty$ such that $\beta_{p}(\mc{C}^{p}(H)) \leq C_{p}$ for all Hilbert spaces.
\end{thm}

Of course, we won't prove the UMD property directly: instead, we'll show that the Hilbert transform $H_{\T}$ is bounded on $L^p(\T; \mc{C}^{p}(H))$, which by Remark \ref{rmk:HT-UMD-T} implies that $\mc{C}^{p}(H)$ is UMD.
This argument relies on the \emph{Cotlar identity} for the Hilbert transform on the torus: for all trigonometric polynomials $\map{f,g}{\T}{\C}$,
\begin{equation}\label{eq:Cotlar}
  H_{\T}\big(f \cdot (H_{\T} g) + (H_{\T} f) \cdot g) = (H_{\T} f)(H_{\T} g) - fg.
\end{equation}
(Exercise \ref{ex:Cotlar}).
This can be extended to Banach-valued functions as follows.

\begin{lem}\label{lem:Cotlar-BV}
  Let $X$, $Y$, and $Z$ be complex Banach spaces, and let $\map{B}{X \times Y}{Z}$ be a bilinear operator.
  For all functions $\mb{F} \colon \T \to X$ and $\mb{G} \colon \T \to Y$, define the lifting $\map{\tilde{B}(\mb{F},\mb{G})}{\T}{Z}$ by
  \begin{equation*}
    \tilde{B}(\mb{F},\mb{G})(t) := B(\mb{F}(t), \mb{G}(t)).
  \end{equation*}
  Then for all trigonometric polynomials $\map{\mb{f}}{\T}{X}$ and $\map{\mb{g}}{\T}{Y}$,
  \begin{equation*}
    H_{\T}^{Z}\big(\tilde{B}(\mb{f}, H_{\T}^{Y} \mb{g}) + \tilde{B}(H_{\T}^{X} \mb{f}, \mb{g})) = \tilde{B}(H_{\T}^{X} \mb{f}, H_{\T}^{Y} \mb{g}) - \tilde{B}(\mb{f}, \mb{g})
  \end{equation*}
  using superscripts to make clear which Banach-valued $H_{\T}$ is being used.
\end{lem}

\begin{proof}
  By linearity, it suffices to check this for elementary tensors $\mb{f} = e_{n} \otimes \mb{x}$ and $\mb{g} = e_{m} \otimes \mb{y}$, with $n,m \in \Z$, $\mb{x} \in X$, and $\mb{y} \in Y$.
  Since $H_{\T}^{X}$ acts as the tensor extension $H_{\T} \otimes I$ on $X$-valued trigonometric polynomials, and likewise with $Y$ and $Z$ in place of $X$, we have
  \begin{equation*}
    \begin{aligned}
      &H_{\T}^{Z}\big(\tilde{B}(\mb{f}, H_{\T}^{Y} \mb{g}) + \tilde{B}(H_{\T}^{X} \mb{f}, \mb{g})) \\
      &= H_{\T}^{Z} \big( \tilde{B}(e_{n} \otimes \mb{x}, H_{\T}^{Y} (e_{m} \otimes \mb{y}) ) + \tilde{B}(H_{\T}^{X} (e_{n} \otimes \mb{x}), e_{m} \otimes \mb{y})) \\
      &= H_{\T}^{Z} \big( \tilde{B}(e_{n} \otimes \mb{x}, H_{\T} e_{m} \otimes \mb{y} ) + \tilde{B}(H_{\T} e_{n} \otimes \mb{x}, e_{m} \otimes \mb{y})) \\
      &= H_{\T}^{Z} \big( (e_{n} \cdot (H_{\T} e_{m}) + (H_{\T} e_{n}) \cdot e_{m} )\otimes B(\mb{x},  \mb{y})) \\
      &= H_{\T}(e_{n} \cdot (H_{\T} e_{m}) + (H_{\T} e_{n})) B(\mb{x}, \mb{y})
    \end{aligned}
  \end{equation*}
  and similarly
  \begin{equation*}
    \tilde{B}(H_{\T}^{X} \mb{f}, H_{\T}^{Y} \mb{g}) - \tilde{B}(\mb{f}, \mb{g})
    = ((H_{\T} e_n)(H_{\T} e_m) - e_n e_m) B(\mb{x},\mb{y}).
  \end{equation*}
  Thus the result follows from the complex-valued case.
\end{proof}

We will use this result with the bilinear form $\map{B}{\mc{C}^{2p}(H) \times \mc{C}^{2p}(H)}{\mc{C}^{p}(H)}$ given by $B(u,v) := uv$, whose boundedness follows by the non-commutative H\"older inequality \eqref{eq:NC-holder}.

\begin{proof}[Proof of Theorem \ref{thm:Schatten-UMD}]
  Write $\mc{C}^{p} = \mc{C}^{p}(H)$ to ease notation.
  It will be clear from the proof that the resulting constants we obtain are independent of the choice of Hilbert space.
  By the duality relation $(\mc{C}^{p})^* = \mc{C}^{p'}$ (Proposition \ref{prop:Schatten-duality}) and the fact that a Banach space is UMD if and only if its dual is UMD (Proposition \ref{prop:UMD-duality}), it suffices to prove the result for $p \in [2,\infty)$.
  We will show that the Hilbert transform $H_{\T}$ is bounded on $L^p(\T;\mc{C}^{p})$ for all $p \in [2,\infty)$: let
  \begin{equation*}
    A_{p} := \|H_{\T}\|_{\Lin(L^p(\T;\mc{C}^{p}))}.
  \end{equation*}
  By interpolation (Proposition \ref{prop:Schatten-interpolation}), it suffices to prove that $A_{2^{n}}$ is finite for $n \in \{1,2,\ldots\}$.
  We will prove this by induction, with base case $n=1$ being true since $\mc{C}^{2}(H)$ is a Hilbert space (so in fact $A_{2} = 1$).\footnote{The fact that $A_{2}$ is independent of the choice of $H$ ultimately leads to our constants being $H$-independent throughout.}
  To simply notation further, we will fix $p \geq 2$, assume that $A_{p}$ is finite, and deduce that $A_{2p}$ is also finite.


  For all $N \in \N$, let $P_{N}(\mc{C}^{2p})$ denote the set of $\mc{C}^{2p}$-valued trigonometric polynomials $\mb{f}$ of degree $\leq N$ (i.e. $\hat{\mb{f}}(m) = \mb{0}$ whenever $|m| > N$), and define
  \begin{equation*}
    A_{2p,N} := \sup_{\substack{\mb{f} \in P_{N}(\mc{C}^{2p}) \\ \mb{f} \neq 0}} \frac{\|H_{\T} \mb{f}\|_{L^{2p}(\T;\mc{C}^{2p})}}{\|\mb{f}\|_{L^{2p}(\T;\mc{C}^{2p})}},
  \end{equation*}
  i.e. $A_{2p,N}$ is the norm of the Hilbert transform $H_{\T}$ restricted to the subspace $P_{N} \subset L^{2p}(\T;\mc{C}^{2p})$.
  Then by density of the trigonometric polynomials
  \begin{equation*}
    A_{2p} = \lim_{N \to \infty} A_{2p,N},
  \end{equation*}
  and by Exercise \ref{ex:HT-FD}, the numbers $A_{2p,N}$ are all finite.\footnote{This could be thought of as a `noncommutative truncation argument'.}

  For every function $\mb{F} \colon \T \to \Lin(H)$, let $\mb{F}^*$ denote the pointwise adjoint, given by $\mb{F}^{*}(t) := \mb{F}(t)^*$.
  Fix a trigonometric polynomial $\mb{f} \in P_{N}(\mc{C}^{2p})$.
  We have the identity
  \begin{equation*}
    (H_{\T} \mb{F})^{*} = H_{\T} (\mb{F}^*)
  \end{equation*}
  (Exercise \ref{ex:HT-adjoint-Schatten}), so by \eqref{eq:Schatten-convexity}, we can write
  \begin{equation*}
    \begin{aligned}
      \|H_{\T} \mb{f}\|_{L^{2p}(\T;\mc{C}^{2p})}^{2}
      &= \Big( \int_{\T} \|H_{\T} \mb{f}(t) \|_{\mc{C}^{2p}}^{2p} \, \dd t \Big)^{2/2p} \\
      &= \Big( \int_{\T} \|H_{\T} \mb{f}(t)^* H_{\T} \mb{f}(t) \|_{\mc{C}^{p}}^{p} \, \dd t \Big)^{1/p} \\
      &= \Big( \int_{\T} \|H_{\T} (\mb{f}^{*})(t) H_{\T} \mb{f}(t) \|_{\mc{C}^{p}}^{p} \, \dd t \Big)^{1/p} \\
      &= \|H_{\T} (\mb{f}^*) H_{\T} \mb{f}\|_{L^p(\T;\mc{C}^{p})}.
    \end{aligned}
  \end{equation*}
  By Cotlar's identity (more precisely, by the Banach-valued extension \eqref{lem:Cotlar-BV}), the definition of $A_{p}$ and $A_{2p,N}$, and the classical and non-commutative H\"older inequalities, we can estimate
  \begin{equation*}
    \begin{aligned}
      &\|H_{\T} (\mb{f}^*) H_{\T} \mb{f}\|_{L^p(\T;\mc{C}^{p})} \\
      &\leq \Big\|H_{\T}\Big(\mb{f}^* (H_{\T} \mb{f}) + (H_{\T} \mb{f}^*)\mb{f} \Big) \Big\|_{L^p(\T;\mc{C}^{p})} + \|\mb{f}^{*} \mb{f}\|_{L^p(\T;\mc{C}^{p})} \\
      &\leq A_{p} \big( \|\mb{f}^* (H_{\T} \mb{f})\|_{L^p(\T;\mc{C}^{p})} + \| (H_{\T} \mb{f}^*)\mb{f} \|_{L^p(\T;\mc{C}^{p})} \big) + \|\mb{f}^{*} \mb{f}\|_{L^p(\T;\mc{C}^{p})} \\
      &\leq A_{p} \big( \|\mb{f}^*\| \|H_{\T} \mb{f}\| + \| H_{\T} \mb{f}^*\| \| \mb{f} \| \big)   + \|\mb{f}^{*}\| \| \mb{f}\| 
      \leq (2A_{p} A_{2p,N} + 1) \|\mb{f}\|^{2}.
    \end{aligned}
  \end{equation*}
  where all the unlabeled norms are in $L^{2p}(\T;\mc{C}^{2p})$.
  This tells us that
  \begin{equation*}
    A_{2p,N}^{2} \leq 2A_{p} A_{2p,N} + 1,
  \end{equation*}
  and thus by solving a quadratic equation
  \begin{equation*}
    A_{2p,N} \leq A_{p} + \sqrt{A_{p}^{2} - 1}.
  \end{equation*}
  Taking the limit as $N \to \infty$ and using that $A_{p} < \infty$, we find that $A_{2p} < \infty$ also.
  This completes the proof.
\end{proof}

\begin{rmk}
  With a small bit of extra work, this argument can be extended to show that
  \begin{equation*}
    \|H_{\T}\|_{L^p(\T;\mc{C}^{p})} \leq \frac{8}{\pi} \max(p,p')
  \end{equation*}
  for all $p \in (1,\infty)$, but we don't need this degree of control.
  See \cite[Proposition 5.4.2]{HNVW16}.
\end{rmk}

\section{Schur multipliers}

Let $H$ be a Hilbert space.
A \emph{countable spectral resolution} of $H$ is a sequence of pairwise orthogonal projections $(e_{\lambda})_{\lambda \in \Lambda}$ on $H$, where $\Lambda \subset \R$ is countable, such that $\mb{h} = \sum_{\lambda \in \Lambda} e_{\lambda} \mb{h}$ for all $\mb{h} \in H$.\footnote{Note that the vectors $e_{\lambda} \mb{h}$ are orthogonal, so convergence of the series is independent of the ordering.}

\begin{defn}
  Given a countable spectral resolution $(e_{\lambda})_{\lambda \in \Lambda}$ on a Hilbert space $H$, and given an infinite matrix $(m_{\lambda,\mu})_{\lambda,\mu \in \Lambda}$ indexed by $\Lambda$, the \emph{Schur multiplier} $M_{m}^{e}$ is the operator acting on $\Lin(H)$ (note that this is an operator \emph{on} $\Lin(H)$, not \emph{in} $\Lin(H)$) defined formally by
  \begin{equation*}
    M_{m}^{e} u := \sum_{\lambda, \mu \in \Lambda} m_{\lambda,\mu} e_{\lambda} u e_{\mu} \qquad \forall u \in \Lin(H).
  \end{equation*}
  Note that when $u$ has finite rank, $M_{m}^{e} u$ is well-defined, as the sum is finite.
\end{defn}

Formally, $M_{m}^{e}$ takes the representation of $u \in \Lin(H)$ as an infinite matrix with respect to the spectral resolution $e_{\bullet}$, and multiplies this \emph{componentwise} by the infinite matrix $m$.
There is no guarantee that this operation is well-defined on all $u \in \Lin(H)$, but when $m_{\lambda,\mu}$ is given by the differences of a scalar-valued Mikhlin symbol (recall Definition \ref{defn:mikhlin-symbol}), we can deduce the boundedness of the Schur multiplier $M_{m}^{e}$ on $\mc{C}^{p}(H)$ from the UMD property.
First we prove a lemma that lets us handle simple diagonal Schur multipliers.

\begin{lem}\label{lem:schur-diag}
  Let $H$ be a Hilbert space with countable spectral resolution $(e_{\lambda})_{\lambda \in \Lambda}$, and fix $p \in [1,\infty)$.
  Let $m_{\lambda,\mu}$ be bounded and diagonal, with $m_{\lambda} := m_{\lambda,\lambda}$ for $\lambda \in \Lambda$.
  Then for all finite rank $u \in \Lin(H)$,
  \begin{equation*}
    \|M_{m}^{e} u\|_{\mc{C}^{p}(H)} \leq \|m_{\bullet}\|_{\ell^\infty(\Lambda)} \|u\|_{\mc{C}^{p}(H)}.
  \end{equation*}
\end{lem}

\begin{proof}
  Let $(\varepsilon_{\lambda})_{\lambda \in \Lambda}$ be a Rademacher sequence indexed by $\Lambda$ on a probability space $\Omega$, and use independence to write
  \begin{equation*}
    M_{m}^{e} u = \sum_{\lambda \in \Lambda} m_{\lambda} e_{\lambda} v e_{\lambda} = \E \Big( \big(\sum_{\lambda \in \Lambda} \varepsilon_{\lambda} m_{\lambda} e_{\lambda} \big) v \big(\sum_{\mu \in \Lambda} \varepsilon_{\mu} e_{\mu} \big) \Big).
  \end{equation*}
  By the ideal property of Schatten classes (Exercise \ref{ex:Schatten-ideal}):
  \begin{equation*}
    \begin{aligned}
    \Big\| \sum_{\lambda \in \Lambda} e_{\lambda} v e_{\lambda} \Big\|_{\mc{C}^{p}(H)}
    &\leq \E \Big\| \big(\sum_{\lambda \in \Lambda} \varepsilon_{\lambda} m_{\lambda} e_{\lambda} \big) v \big(\sum_{\mu \in \Lambda} \varepsilon_{\mu} e_{\mu} \big) \Big\|_{\mc{C}^{p}(H)} \\
    &\E \Big\| \sum_{\lambda \in \Lambda} \varepsilon_{\lambda} m_{\lambda} e_{\lambda} \Big\|_{\Lin(H)} \|v\|_{\mc{C}^{p}(H)} \Big\| \sum_{\mu \in \Lambda} \varepsilon_{\mu} e_{\mu} \Big\|_{\Lin(H)}.
  \end{aligned}
\end{equation*}
For all $\omega \in \Omega$, orthogonality of the projections $e_{\lambda}$ implies that
\begin{equation*}
  \Big\| \sum_{\lambda \in \Lambda} \varepsilon_{\lambda}(\omega) m_{\lambda} e_{\lambda} \Big\|_{\Lin(H)} \leq \|\varepsilon_{\bullet} (\omega) m_{\bullet}\|_{\ell^\infty(\Lambda)} = \| m_{\bullet}\|_{\ell^\infty(\Lambda)};
\end{equation*}
applying this with $m \equiv 1$ to the other Rademacher term and integrating over $\Omega$ completes the proof.
\end{proof}

\begin{thm}\label{thm:Schur-multipliers}
  Let $H$ be a Hilbert space with countable spectral resolution $(e_{\lambda})_{\lambda \in \Lambda}$, and fix $p \in (1,\infty)$.
  Let $m \in C(\R \sm \{0\})$ be a scalar-valued Mikhlin symbol which is defined and continuous on $\R \sm \{0\}$.\footnote{Unlike the Mikhlin symbols we defined earlier, we demand that our symbols are defined at the points $\pm 2^{j}$, $j \in \Z$.}
  Consider the infinite matrix
  \begin{equation*}
    m_{\lambda,\mu} :=
    \begin{cases}
      m(\lambda - \mu) & (\lambda \neq \mu \in \Lambda) \\ c & \lambda = \mu
    \end{cases}
  \end{equation*}
  for some constant $c \in \C$.
  Then for all finite rank operators $v \in \Lin(H)$ we have the estimate
  \begin{equation*}
    \|M_{m}^{e} v\|_{\mc{C}^{p}(H)} \lesssim_{p} (\|m\|_{\mf{M}(\R)} + |c|) \|v\|_{\mc{C}^{p}(H)},
  \end{equation*}
  and thus by density we can extend $M_{m}^{e}$ to a bounded linear operator on $\mc{C}^{p}(H)$.
\end{thm}

\begin{proof}
  Lemma \ref{lem:schur-diag} lets us assume without loss of generality that $c = 0$, by subtracting a constant diagonal Schur multiplier.\footnote{We could replace the constant diagonal with a bounded sequence, and deal with it in the same way.}
  
  For each $t \in \R$, consider the unitary operator
  \begin{equation*}
    u_{t} := \sum_{\lambda \in \Lambda} e^{2\pi i \lambda t} e_{\lambda}, 
  \end{equation*}
  so that $u_{t}^{*} = u_{-t}$.
  Then for all $t \in \R$ we have
  \begin{equation*}
    \| M_{m}^{e} v \|_{\mc{C}^{p}(H)} = 
    \|u_{t}  (M_{m}^{e} v) u_{-t} \|_{\mc{C}^{p}(H)}
  \end{equation*}
  since conjugating by a unitary preserves all approximation numbers, hence also the Schatten norms.
  We compute, using the orthogonality of the projections $e_{\bullet}$,
  \begin{equation*}
    \begin{aligned}
      u_{t}  (M_{m}^{e} v) u_{-t}
      &= \sum_{\lambda \neq \mu \in \Lambda} \Big( \sum_{\lambda_{1} \in \Lambda} e^{2\pi i \lambda_{1} t} e_{\lambda_{1}} \Big)  m(\lambda - \mu) e_{\lambda} v e_{\mu} \Big(\sum_{\lambda_{2} \in \Lambda} e^{-2\pi i \lambda_{2} t} e_{\lambda_{2}} \Big) \\
      &= \sum_{\lambda \neq \mu \in \Lambda} e^{2\pi i (\lambda - \mu) t}  m(\lambda - \mu) e_{\lambda} v e_{\mu} 
      = \sum_{\theta \in \Delta} e^{2\pi i \theta t} m(\theta) v_{\theta},
    \end{aligned}
  \end{equation*}
  where $\Delta$ is the set of nonzero differences $\{\lambda - \mu: \lambda \neq \mu \in \Lambda\}$ (using that $c = 0$)
  and
  \begin{equation*}
    v_{\theta} := \sum_{\substack{\lambda,\mu \in \Lambda \\ \lambda - \mu = \theta}} e_{\lambda} v e_{\mu}.
  \end{equation*}
  Since $v$ has finite rank, we find that $v_{\theta} \neq 0$ for only finitely many $\theta \in \Delta$.
  Note that the same computation with $m \equiv 1$, along with Lemma \ref{lem:schur-diag}, shows that
  \begin{equation*}
    \Big\| \sum_{\theta \in \Delta} e^{2\pi i \theta t} v_{\theta}\Big\|_{\mc{C}^{p}(H)} = \|v - v_{0}\|_{\mc{C}^{p}(H)} \leq 2\|v\|_{\mc{C}^{p}(H)}  \qquad \forall t \in \R.
  \end{equation*}
 
 Consider a function $\phi \in L^2(\R)$ with $\|\phi\|_{2} = 1$.
 Then for all $s > 0$ we can write
 \begin{equation*}
   \begin{aligned}
     \| M_{m}^{e} v \|_{\mc{C}^{p}(H)}^{2}
     &= \int_{\R} \|M_{m}^{e} v \|_{\mc{C}^{p}(H)}^{2} |\Dil^{2}_{s} \phi(t)|^{2} \, \dd t \\
     &= \int_{\R} \Big\| \sum_{\theta \in \Delta} m(\theta) e^{2\pi i \theta t} \Dil_{s}^{2} \phi(t) v_{\theta} \Big\|_{\mc{C}^{p}(H)}^{2} \, \dd t \\
     &\leq \int_{\R} \Big\| \sum_{\theta \in \Delta} T_{m} (e_{\theta} \Dil_{s}^{2} \phi \otimes v_{\theta})(t) \Big\|_{\mc{C}^{p}(H)}^{2}  \, \dd t \\
     &\qquad + \int_{\R} \Big\| \sum_{\theta \in \Delta} (m(\theta) - T_{m}) (e_{\theta} \Dil_{s}^{2} \phi \otimes v_{\theta})(t) \Big\|_{\mc{C}^{p}(H)}^{2}  \, \dd t
   \end{aligned}
 \end{equation*}
  where $T_{m}$ is the Fourier multiplier with symbol $m$.
  The first term in this sum can be bounded by using the Mikhlin theorem on $L^2(\R;\mc{C}^{p}(H))$, valid since $\mc{C}^{p}(H)$ is UMD:
  \begin{equation*}
    \begin{aligned}
      &\int_{\R} \Big\| \sum_{\theta \in \Delta} T_{m} (e_{\theta} \Dil_{s}^{2} \phi \otimes v_{\theta})(t) \Big\|_{\mc{C}^{p}(H)}^{2}  \, \dd t \\
      &\lesssim_{p} \|m\|_{\mf{M}(\R)} \int_{\R} \Big\| \big( \sum_{\theta \in \Delta} e_{\theta} \Dil_{s}^{2} \phi \otimes v_{\theta} \big)(t) \Big\|_{\mc{C}^{p}(H)}^{2}  \, \dd t \\
      &= \|m\|_{\mf{M}(\R)} \int_{\R} \Big\| \sum_{\theta \in \Delta} e^{2\pi i \theta t}  v_{\theta} \Big\|_{\mc{C}^{p}(H)}^{2} |\Dil_{s}^{2} \phi(t)|^{2} \, \dd t \\
      &\lesssim \|m\|_{\mf{M}(\R)} \int_{\R} \|v\|_{\mc{C}^{p}(H)} |\Dil_{s}^{2} \phi(t)|^{2} \, \dd t
      = \|m\|_{\mf{M}(\R)} \|v\|_{\mc{C}^{p}(H)}.
    \end{aligned}
  \end{equation*} 
  As for the second term, we estimate crudely
  \begin{equation*}
    \begin{aligned}
      &\Big(\int_{\R} \Big\| \sum_{\theta \in \Delta} (m(\theta) - T_{m}) (e_{\theta} \Dil_{s}^{2} \phi \otimes v_{\theta})(t) \Big\|_{\mc{C}^{p}(H)}^{2}  \, \dd t \Big)^{1/2} \\
      &\leq \sum_{\theta \in \Delta} \Big( \int_{\R} \Big\| (m(\theta) - T_{m}) (e_{\theta} \Dil_{s}^{2} \phi \otimes v_{\theta})(t) \Big\|_{\mc{C}^{p}(H)}^{2}  \, \dd t \Big)^{1/2} \\
      &\leq \sum_{\theta \in \Delta} \|v_{\theta}\|_{\mc{C}^{p}(H)} \Big( \int_{\R} \big| (m(\theta) - T_{m}) (e_{\theta} \Dil_{s}^{2} \phi)(t) \big|^{2}  \, \dd t \Big)^{1/2} \\
      &= \sum_{\theta \in \Delta} \|v_{\theta}\|_{\mc{C}^{p}(H)} \Big( \int_{\R} \big| (m(\theta) - \tilde{m}(\xi)) \Dil_{1/s}^{2} \hat{\phi})(\xi - \theta) \big|^{2}  \, \dd \xi \Big)^{1/2}
    \end{aligned}
  \end{equation*}
  using Plancherel's theorem for scalar-valued functions and some standard Fourier identities in the last step.
  Notice that for all $\theta \in \Delta$,
  \begin{equation*}
       \int_{\R} \big| (m(\theta) - m(\xi)) \Dil_{1/s}^{2} \hat{\phi})(\xi - \theta) \big|^{2}  \, \dd \xi
      =  \int_{\R} |m(\theta) - m(s^{-1}\xi + \theta)|^{2} |\hat{\phi}(\xi)|^{2}  \, \dd \xi .
  \end{equation*}
  Since $m$ is bounded and continuous away from $0$, and since $\theta \neq 0$, by dominated convergence this tends to $0$ as $s  \to \infty$.
  Furthermore, since $v$ is finite rank, only finitely many differences $\theta \in \Delta$ are relevant, and thus we have proven that
  \begin{equation*}
     \| M_{m}^{e} v \|_{\mc{C}^{p}(H)}^{2} \lesssim_{p} \|m\|_{\mf{M}(\R)} \|v\|_{\mc{C}^{p}(H)}
   \end{equation*}
   as claimed.
 \end{proof}

 \begin{rmk}
   For $v \in \mc{C}^{p}(H)$ for $p \in (1,\infty)$, and with symbol $m$ as in Theorem \ref{thm:Schur-multipliers}, the sum
   \begin{equation*} 
     M_{m}^{e} v = \sum_{\lambda,\mu \in \Lambda} m_{\lambda,\mu} e_{\lambda} v e_{\mu}
   \end{equation*}
   converges in $\mc{C}^{p}(H)$ for any ordering of $\Lambda$.
   See \cite[Theorem 5.4.3]{HNVW16} for a more careful proof of this.
 \end{rmk}

\section{Applications to functional calculus: operator Lipschitz functions}

The result we just proved for Schur multipliers, which followed by the UMD-valued Mikhlin theorem, has deep repercussions in spectral theory.
Let $H$ be a complex Hilbert space.
First recall that for every compact self-adjoint operator $u \in \mc{K}(H)$, the spectral theorem says that there exists a countable spectral resolution $(e_{\lambda})_{\lambda \in \sigma(u)}$ such that
\begin{equation}\label{eq:u-res}
  u = \sum_{\lambda \in \sigma(u)} \lambda e_{\lambda},
\end{equation}
where $\sigma(u) = \{\lambda \in \C : \text{$\lambda - u \in \Lin(H)$ is not invertible}\} \subset \R$ is the spectrum of $u$ (hence the term \emph{spectral resolution}).\footnote{A more complicated version of this theorem holds for self-adjoint but possibly non-compact operators.}
For every continuous function $\map{f}{\sigma(u)}{\C}$ (generally the set $\sigma(u)$ has the cluster point $0$, and continuity is required here) we use the spectral resolution to define an operator
\begin{equation}\label{eq:fu}
  f(u) := \sum_{\lambda \in \sigma(u)} f(\lambda) e_{\lambda}.
\end{equation}
This operator is bounded, with
\begin{equation*}
  \|f(u)\|_{\Lin(H)} \leq \|f\|_{\infty} \|u\|_{\Lin(H)}.
\end{equation*}
This method of defining `functions of $u$' is called the \emph{functional calculus of $u$}.\footnote{For a somewhat na\"ively written introduction to this topic see \cite{AA-FC}.}

We will prove a theorem concering the continuity of the functional calculus with respect to a fixed function as the underlying operator varies.
That is, if two compact self-adjoint operators $u,v \in \mc{K}(H)$ are close, and if $f$ is continuous on $\R$, are $f(u)$ and $f(v)$ also close?
The answer turns out to depend strongly on both the function $f$ (particularly, quantitative continuity properties) and the choice of norm with respect to which we measure the operators.
The Schatten classes $\mc{C}^{p}(H) \subset \mc{K}(H)$ are appropriate choices, and the UMD property of these spaces when $p \in (1,\infty)$ can be used in proving the following deep (and relatively recent) theorem.

\begin{thm}[Potapov--Sukochev]\label{thm:Potapov-Sukochev}
  Let $H$ be a Hilbert space and $p \in (1,\infty)$, and let $u$ and $v$ be compact self-adjoint operators on $H$ such that $u - v \in \mc{C}^{p}$.\footnote{We are following the proof in \cite[\textsection 5.4.c]{HNVW16}, which requires compact $u$ and $v$. The full result, proven in \cite{PS11}, is for general self-adjoint operators (possibly unbounded). Only the difference is assumed to be bounded.}
  Then for all Lipschitz functions $\map{f}{\R}{\R}$,
  \begin{equation*}
    \|f(u) - f(v)\|_{\mc{C}^{p}(H)} \lesssim_{p} \|f\|_{\Lip} \|u - v\|_{\mc{C}^{p}(H)},
  \end{equation*}
  where
  \begin{equation*}
    \|f\|_{\Lip} := \sup_{t \neq s \in \R} \frac{|f(t) - f(s)|}{|t-s|}.
  \end{equation*}
\end{thm}

We prove this as a consequence of the following commutator estimate.

\begin{lem}\label{lem:commutators}
  Let $u \in \mc{K}(H)$ be compact and self adjoint, and let $v \in \Lin(H)$.
  Then for all Lipschitz $\map{f}{\R}{\R}$,
  \begin{equation*}
    \|[f(u),v]\|_{\mc{C}^{p}(H)} \lesssim_{p} \|f\|_{\Lip} \|[u,v]\|_{\mc{C}^{p}(H)}.
  \end{equation*}
\end{lem}

\begin{proof}
  Let $(e_{\lambda})_{\lambda \in \sigma(u)}$ be the spectral resolution of $u$, so that \eqref{eq:u-res} and \eqref{eq:fu} hold.
  Then
  \begin{equation*}
    \begin{aligned}
      [f(u),v]
      = \sum_{\lambda \in \sigma(u)} f(\lambda) [  e_{\lambda},  v ] 
      &= \sum_{\lambda, \mu \in \sigma(u)} f(\lambda) (e_{\lambda} v e_{\mu} - e_{\mu} v e_{\lambda}) \\
      &= \sum_{\lambda, \mu \in \sigma(u)} (f(\lambda) - f(\mu)) e_{\lambda} v e_{\mu}.
    \end{aligned}
  \end{equation*}
  The same computation with $f \equiv 1$ yields
  \begin{equation*}
    [u,v] = \sum_{\lambda, \mu \in \sigma(u)} (\lambda - \mu) e_{\lambda} v e_{\mu}.
  \end{equation*}
  Defining
  \begin{equation*}
    m_{\lambda,\mu}^{f} :=
    \begin{cases}
      \frac{f(\lambda) - f(\mu)}{\lambda - \mu} & \lambda \neq \mu \\
      0 & \lambda = \mu.
    \end{cases}
  \end{equation*}
  This shows that
  \begin{equation*}
    [f(u),v] = M^{e}_{m^{f}}([u,v]),
  \end{equation*}
  where $M^{e}_{m}$ is the Schur multiplier with symbol $m$ with respect to the countable spectral resolution $e_{\bullet}$.
  This Schur multiplier is \emph{almost} of the form covered by Theorem \ref{thm:Schur-multipliers}; we need to do some more reduction to get there.
  
  First note that the function
  \begin{equation*}
    \tilde{f}(t) := \|f\|_{\Lip}^{-1} (f(t) - f(0)) + 2t,
  \end{equation*}
  satisfies
  \begin{equation}\label{eq:f-props}
    \tilde{f}(0) = 0, \qquad  \|\tilde{f}\|_{\Lip} \leq 3, \qquad 1 \leq \frac{f(\lambda) - f(\mu)}{\lambda - \mu} \leq 3 \quad  \forall \lambda \neq \mu.
  \end{equation}
  If we prove the desired estimate for $\tilde{f}$, then we can write
  \begin{equation*}
    [f(u),v] = \|f\|_{\Lip}([\tilde{f}(u),v] - 2[u,v]), 
  \end{equation*}
  and thus deduce it for $f$ also.
  From now on we will assume that $f$ satisfies the properties stated for $\tilde{f}$ in \eqref{eq:f-props}.

  Now fix $x \in [1,3]$ and note that $x = e^{2\pi t}$ for $t = (\log(x))/2\pi \in [0,1]$.
  Choose a Schwartz function $\psi \in \Sch(\R)$ such that $\psi(t) = e^{2\pi i t}$ for all $t \in [0,1]$.
  Then by Fourier inversion we can write
  \begin{equation*}
    x = \psi(\log(x)/2\pi) = \int_{\R} \hat{\psi}(s) e^{2\pi i \log(x)/2\pi} \, \dd s = \int_{\R} \hat{\psi}(s) x^{is} \, \dd s.
  \end{equation*}
  Hence for all $\lambda \neq \mu \in \sigma(u)$, since all slopes of $f$ was arranged to be in the interval $[1,3]$, we get a formula
  \begin{equation*}
    \begin{aligned}
      M^{e}_{m^f} [u,v]
      &= \sum_{\lambda \neq \mu \in \sigma(u)} \frac{|f(\lambda)-f(\mu)|}{|\lambda - \mu|} e_{\lambda} [u,v] e_{\mu} \\
      &= \int_{\R} \hat{\psi}(s) \big( |f(\lambda) - f(\mu)|^{is} |\lambda - \mu|^{-is} e_{\lambda} [u,v] e_{\mu} \big) \, \dd s \\
      &= \int_{\R} \hat{\psi}(s) M^{e,f}_{m_{s}} M^{e}_{m_{-s}} [u,v] \, \dd s,
    \end{aligned}
  \end{equation*}
  where $M_{m_{-s}}^{e}$ is the Schur multiplier with symbol
  \begin{equation*}
    m_{-s,\lambda,\mu} := |\lambda - \mu|^{-is},
  \end{equation*}
  and $M^{e,f}_{m_{s}}$ is the Schur multiplier with the same symbol (with the sign of $s$ flipped), defined with respect to the spectral resolution $(e_{f^{-1}(\alpha)})_{\alpha \in f(\sigma(u))}$.\footnote{By \eqref{eq:f-props}, $f$ is invertible.}
  Now that we have Schur multipliers with symbols depending only on $\lambda - \mu$, we can hope to apply Theorem \ref{thm:Schur-multipliers} with Mikhlin symbols $\map{\phi_{s}}{\R \sm \{0\}}{\C}$ given by $\phi_{s}(\xi) := |\xi|^{is}$.
  For all $\xi \in \R \sm \{0\}$ we have
  \begin{equation*}
    |\phi_{s}(\xi)| = 1, \qquad |\xi| |\phi_{s}'(\xi)| = |\xi| ||\xi|^{is} is/\xi| = |s|, 
  \end{equation*}
  so each $\phi_{s}$ is a Mikhlin symbol with $\|\phi_{s}\|_{\mf{M}(\R)} \leq 1 + |s|$.
  Thus by Theorem \ref{thm:Schur-multipliers} the Schur multipliers in the representation above are well-defined on $\mc{C}^{p}(H)$, and we can estimate
  \begin{equation*}
    \begin{aligned}
      \|M^{e}_{m^f} [u,v]\|_{\mc{C}^{p}(H)}
      &\leq \int_{\R} |\hat{\psi}(s)| \|M^{e,f}_{m_{s}} M^{e}_{m_{-s}} [u,v]\|_{\mc{C}^{p}(H)} \, \dd s \\
      &\lesssim_{p} \Big( \int_{\R} |\hat{\psi}(s)| (1 + |s|)^{2} \, \dd s \Big) \|[u,v]\|_{\mc{C}^{p}(H)}  \\
      &\lesssim  \|[u,v]\|_{\mc{C}^{p}(H)}
    \end{aligned}
  \end{equation*}
  using that $\hat{\psi} \in \Sch(\R)$ to estimate the integral in $s$.
\end{proof}

One final argument completes the proof of the Potapov--Sukochev theorem.

\begin{proof}[Proof of Theorem \ref{thm:Potapov-Sukochev}]
  Consider the operators
  \begin{equation*}
    U := \begin{bmatrix} u & 0 \\ 0 & v
    \end{bmatrix},
    \qquad
    V := \begin{bmatrix} 0 & I \\ I & 0
    \end{bmatrix}
  \end{equation*}
  on the direct sum $H \oplus H$.
  Since $u$ and $v$ are compact and self-adjoint on $H$, $U$ is compact and self-adjoint on $H \oplus H$, and $V$ is clearly self-adjoint.
  Lemma \ref{lem:commutators} (on $H \oplus H$) thus implies that
  \begin{equation}\label{eq:com-bd}
    \|[f(U),V]\|_{\mc{C}^{p}(H \oplus H)} \lesssim_{p} \|f\|_{\Lip} \|[U,V]\|_{\mc{C}^{p}(H \oplus H)}.
  \end{equation}
  We can compute
  \begin{equation*}
    \begin{aligned}
      [f(U),V]
      &= \begin{bmatrix} f(u) & 0 \\ 0 & f(v) \end{bmatrix}
      \begin{bmatrix} 0 & I \\ I & 0 \end{bmatrix}
      -
      \begin{bmatrix} 0 & I \\ I & 0 \end{bmatrix}
      \begin{bmatrix} f(u) & 0 \\ 0 & f(v) \end{bmatrix} \\
      &= \begin{bmatrix}
        0 & f(u) - f(v) \\ f(v) - f(u) & 0
        \end{bmatrix},
    \end{aligned}
  \end{equation*}
  so the estimate \eqref{eq:com-bd} says that
  \begin{equation*}
    \Bigg\| \begin{bmatrix}
      0 & f(u) - f(v) \\ f(v) - f(u) & 0
    \end{bmatrix} \Bigg\|_{\mc{C}^{p}(H \oplus H)}
    \lesssim_{p} \|f\|_{\Lip}
    \Bigg\| \begin{bmatrix}
      0 & u - v \\ v - u & 0
    \end{bmatrix} \Bigg\|_{\mc{C}^{p}(H \oplus H)}.
  \end{equation*}
  Finally, for a general element $w \in \mc{C}^{p}(H)$ we compute
  \begin{equation*}
    \begin{aligned}
    \Bigg\| \begin{bmatrix}
      0 & w \\ -w & 0
    \end{bmatrix} \Bigg\|_{\mc{C}^{p}(H \oplus H)}^{p} 
    &= \tr \bigg( \begin{bmatrix}
      0 & w \\ -w & 0
    \end{bmatrix}^* \begin{bmatrix}
      0 & w \\ -w & 0
    \end{bmatrix} \bigg)^{p/2} \\
    &= \tr\bigg(
    \begin{bmatrix}
      w^{*}w  & 0 \\ 0 & w^{*} w
    \end{bmatrix}
    \bigg)^{p/2} \\
    &= \tr
    \begin{bmatrix}
      (w^* w)^{p/2}  & 0 \\ 0 & (w^* w)^{p/2}
    \end{bmatrix} 
    = 2\|w\|_{\mc{C}^{p}(H)}^{p}.
    \end{aligned}
  \end{equation*}
Applying this to $f(u) - f(v)$ and $u-v$ yields
\begin{equation*}
  \|f(v) - f(u)\|_{\mc{C}^{p}(H)} \lesssim_{p} \|f\|_{\Lip} \|v-u\|_{\mc{C}^{p}(H)},
\end{equation*}
completing the proof.
\end{proof}

\section{Exercises}

\begin{exercise}\label{ex:Cotlar}
  Prove the Cotlar identity \eqref{eq:Cotlar} for complex-valued functions.\footnote{See \cite[\textsection 5.4.a]{HNVW16} if you're stuck.}
\end{exercise}

\begin{exercise}\label{ex:HT-FD}
  Let $X$ be a complex Banach space and let $P_{N}(X)$ denote the set of trigonometric polynomials $\map{\mb{f}}{\T}{X}$ of degree $\leq N$ (meaning that $\hat{\mb{f}}$ is supported in $\{-N,-N+1,\cdots,N-1,N\}$).
  Show that 
  \begin{equation*}
    \|H_{\T} \mb{f}\|_{L^p(\T;X)} \lesssim_{p,X,N} \|\mb{f}\|_{L^p(\T;X)} \qquad \forall \mb{f} \in P_{N}(X).
  \end{equation*}
\end{exercise}

\begin{exercise}\label{ex:HT-adjoint-Schatten}
  Let $H$ be a Hilbert space and  $\mb{f} \colon \T \to \Lin(H)$ a $\Lin(H)$-valued trigonometric polynomial.
  Show that $(H_{\T} \mb{f})(t))^* = (H_{\T} (\mb{f}^*))(t)$, where $\mb{f}^*$ is the pointwise adjoint of $\mb{f}$.
\end{exercise}

\begin{exercise}\label{ex:Schatten-ideal}
  Let $H$ be a Hilbert space and $p \in [1,\infty)$.
  Show that the Schatten class $\mc{C}^{p}(H)$ is an operator ideal: that is, for all $u \in \mc{C}^{p}(H)$ and all $a,b \in \Lin(H)$, $aub \in \mc{C}^{p}(H)$ and
  \begin{equation*}
    \|aub\|_{\mc{C}^{p}(H)} \leq \|a\|_{\Lin(H)} \|u\|_{\mc{C}^{p}(H)} \|b\|_{\mc{C}^{p}(H)}.
  \end{equation*}
\end{exercise}

\begin{exercise}
  Let $N \in \N$ and consider the (standard) $N$-dimensional complex Hilbert space $\C^N$.
  For all $N \times N$ matrices $A = (a_{m,n})_{m,n = 1}^{N} \in \Lin(H)$, define the upper triangular truncation $T(A) \in \Lin(H)$ componentwise by
  \begin{equation*}
    (T(A))_{m,n} := \begin{cases} a_{m,n} & (m \geq n) \\ 0 & (m < n). \end{cases}
  \end{equation*}
  Prove that
  \begin{equation*}
    \|T(A)\|_{\mc{C}^{p}(\C^{n})} \lesssim_{p} \|A\|_{\mc{C}^{p}(\C^{n})}.
  \end{equation*}
  for all $p \in (1,\infty)$, uniformly in $N$.
\end{exercise}


%%% Local Variables:
%%% mode: latex
%%% TeX-master: "../main.tex"
%%% End:
