These notes are formally the basis of a one-semester master-level course to be taught (remotely) at the University of Bonn from October 2020 to February 2021, during what I hope turns out to be the peak of the Covid-19 pandemic.
In practise, they are the basis of an \emph{international} online course, open to everybody.

Most of the material in these notes is heavily based off the textbooks of Pisier \cite{gP16} and Hyt\"onen--van Neerven--Veraar--Weis \cite{HNVW16, HNVW17}.
I do not claim any originality in the proofs, or in any of the ideas; at most, I will claim an $\varepsilon^2$ of originality in the presentation (and perhaps a $\sqrt{\varepsilon}$ of effort in putting it all together).
I have put almost no effort into finding original references, or in getting the history correct: the books cited above do a much better job of this than I possibly could.

\vspace{1cm}

I presume these notes are full of typos and mistakes.
If you find any, please contact me at \texttt{amenta@math.uni-bonn.de} and I'll fix them.
Thanks to Timothy Banova, Victor Olmos, Lennart Ronge, Aidan Schumann, and Feng Shao for comments and corrections.\footnote{These notes are still in progress, so if you want to be in this list, just point out some mistakes!}
Thanks also to Jan van Neerven and Mark Veraar for their advice and patience as I fumbled through this topic in Delft, to Christoph Thiele for supporting and `co-lecturing' the course, and to Gennady Uraltsev for pushing me into the role of Teacher of Banach-Valued Analysis.

\vspace{0.5cm}

Keep healthy! 




\vspace{1cm}

\textbf{Warning:} these notes are incomplete.
Empty chapters will be filled in as they are completed (if all goes well, this will be before we reach the material in the lectures).

\textbf{Changes since version v0.52:}
\begin{itemize}
\item Added Section \ref{sec:rademacher} on Rademacher sums.
\item Added Exercise \ref{ex:Lp-RNP} in the Radon--Nikodym chapter.
\item Fixed a typo in Exercise \ref{ex:UI-characterisation}, and consequently had to fix the proof of Proposition \ref{prop:martingale-measure}.
\item Fixed various typos and reworded the occasional sentence.
\end{itemize}

\textbf{Known issues:}
\begin{itemize}
\item The proof of the $L^p$ estimate in Theorem \ref{thm:doob} (Doob's maximal theorem) has an issue - the `sufficiently large $N$' depends on $t$, but we argue as if it doesn't.
  This can be fixed by a truncation argument.
  Currently thinking of the most efficient way of presenting this
\item The notes are not finished!
\end{itemize}

  





%%% Local Variables:
%%% mode: latex
%%% TeX-master: "../main.tex"
%%% End:
