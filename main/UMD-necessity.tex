The goal of this chapter is to prove the converse to Burkholder's theorem (Theorem \ref{thm:Burkholder}).

\begin{thm}[Bourgain]\label{thm:Bourgain}
  Let $X$ be a Banach space, and suppose that for some $p \in (1,\infty)$ the Hilbert transform $H \in \Lin(L^p(\R))$ admits a bounded $X$-valued extension.
  Then $X$ is UMD.
\end{thm}

\section{The dyadic UMD property}


\begin{defn}
  For $p \in (1,\infty)$, a Banach space $X$ is said to have the \emph{dyadic $\UMD_{p}$ property} if there exists a constant $C < \infty$ such that for all $X$-valued $L^p$-bounded martingales $\mb{f}_{\bullet}$ with respect to the dyadic filtration $\mc{F}_{\bullet}$ on $[0,1)$,
  \begin{equation}\label{eq:dyadic-UMD-property}
    \Big\| \sum_{n=0}^{N} \xi_{n} d\mb{f}_{n} \Big\|_{L^p([0,1);X)} \leq C \| \mb{f}_{N}\|_{L^p([0,1);X)} \qquad \forall N \in \N.
  \end{equation}
\end{defn}

This is the same definition as that of the $\UMD_{p}$ property, but with the class of martingales restricted to martingales on the dyadic filtration.
Clearly the dyadic $\UMD_{p}$ property is implied by the standard $\UMD$ property; in this section we will prove that these two properties are equivalent.
The first step in proving this is to show that every $L^p$-bounded martingale can be approximated in $L^p$ by a martingale formed of simple functions.

\begin{lem}\label{lem:mgale-simple-approx}
  Let $X$ be a Banach space, $p \in [1,\infty)$, and $(\Omega,\mc{A},\P)$ a probability space.
  Suppose that $\mb{f}_{\bullet}$ is an $L^p$-bounded martingale on $\Omega$. 
  Then for all $\varepsilon > 0$ there exists an $X$-valued martingale $\mb{g}_{\bullet}$ on $(\Omega,\mc{A},\P)$ such that each $\mb{g}_{n}$ is simple, with
  \begin{equation*}
    \|d\mb{f}_{n} - d\mb{g}_{n}\|_{L^p(\Omega;X)} < 2^{-n-1}\varepsilon
  \end{equation*}
  and
  \begin{equation*}
    \|\mb{f}_{n} - \mb{g}_{n}\|_{L^p(\Omega;X)} < \varepsilon
  \end{equation*}
  for all $n \in \N$.
\end{lem}

\begin{proof}
  Let $\mc{F}_{\bullet}$ be the filtration generated by $\mb{f}_{\bullet}$.
  For each $n \in \N$, choose a simple function $\mb{f}_{n}' \in L^p(\mc{F}_{n};X)$ such that
  \begin{equation*}
    \|d\mb{f}_{n} - \mb{f}_{n}'\|_{L^p(\mc{F}_{n};X)} < 2^{-n-2}\varepsilon.
  \end{equation*}
  Let $\mc{F}'_{\bullet}$ be the filtration generated by $\mb{f}'_{\bullet}$, and define the sequence of functions $\mb{g}_{\bullet}$ via its difference sequence:
  \begin{equation*}
    d\mb{g}_{n} :=
    \begin{cases} \mb{f}'_{n} - \E^{\mc{F}'_{n-1}} \mb{f}'_{n} & n \geq 1 \\
      \mb{f}'_{0} & n = 0.
    \end{cases}
  \end{equation*}
  Each $d\mb{g}_{n}$ is simple since $\mb{f}'_{n}$ is simple and the $\sigma$-algebra $\mc{F}'_{n}$ is finite, and by construction $\E^{\mc{F}'_{n}}(d\mb{g}_{n}) = 0$ for all $n \geq 1$.
  It follows that $\mb{g}_{\bullet}$ is a martingale with respect to the filtration $\mc{F}'_{\bullet}$.
  For all $n \geq 1$, since $\mc{F}_{n-1}' \subset \mc{F}_{n-1}'$, we have
  \begin{equation*}
    \E^{\mc{F}_{n-1}'} d\mb{f}_{n} = \E^{\mc{F}_{n-1}'}\E^{\mc{F}_{n-1}} d\mb{f}_{n} = \mb{0},
  \end{equation*}
  So for each $n \geq 1$ we can estimate
  \begin{equation*}
    \begin{aligned}
      \|d\mb{f}_{n} - d\mb{g}_{n}\|_{L^p(\Omega;X)}
      &= \|d\mb{f}_{n} - \mb{f}_{n}' + \E^{\mc{F}_{n-1}'} \mb{f}_{n}'\|_{L^p(\Omega;X)} \\
      &= \|d\mb{f}_{n} - \mb{f}_{n}' + \E^{\mc{F}_{n-1}'} (\mb{f}_{n}' - d\mb{f}_{n})\|_{L^p(\Omega;X)} \\
      &\leq \|d\mb{f}_{n} - \mb{f}_{n}'\|_{L^p(\Omega;X)} + \|\E^{\mc{F}_{n-1}'} (\mb{f}_{n}' - d\mb{f}_{n})\|_{L^p(\Omega;X)} \\
      &\leq 2(2^{-n-2}\varepsilon) = 2^{-n-1}\varepsilon.
    \end{aligned}
  \end{equation*}
  Note also that
  \begin{equation*}
    \|d\mb{f}_{0} - d\mb{g}_{0}\|_{L^p(\Omega;X)} = \|d\mb{f}_{0} - \mb{f}_{0}'\|_{L^p(\Omega;X)} < 2^{-2}\varepsilon.
  \end{equation*}
  This gives the first claimed estimates, and furthermore  for all $n \in \N$,
  \begin{equation*}
    \|\mb{f}_{n} - \mb{g}_{n}\|_{L^p(\Omega;X)}
    \leq \sum_{m=0}^{n} \|d\mb{f}_{m} - d\mb{g}_{m}\|_{L^p(\Omega;X)} 
    \leq 2^{-2} \varepsilon + \sum_{m=1}^{n} 2^{-m-1}\varepsilon < \varepsilon
  \end{equation*}
  as required.
\end{proof}

Given a stochastic process $\mb{f}_{\bullet}$ consisting of simple functions on a probability space $(\Omega,\P)$, the filtration $\mc{F}_{\bullet}$ generated by $\mb{f}_{\bullet}$ has the special property that each $\sigma$-algebra $\mc{F}_{n}$ is finite and atomic.
Fix $N \in \N$ and consider the finite filtration $(\mc{F}_{n})_{n=0}^{N}$.
Then there exists a finite filtration $(\mc{F}_{n}')_{n=0}^{N}$ on the unit interval $[0,1)$ such that each $\mc{F}_{n}'$ is atomic, with the atoms being intervals, and such that the atoms of $\mc{F}_{n}'$ are in bijective correspondence with those of $\mc{F}_{n}$ for each $n \in \N$.
In what follows we will refer to such a filtration on $[0,1)$ as a \emph{finite interval filtration}.
This bijective correspondence induces an isometric isomorphism $\psi \colon L^p(\Omega,\mc{F}_{N};X) \to L^p([0,1),\mc{F}_{N}';X)$ for all $p \in [1,\infty]$, such that for every $\mb{g} \in L^p(\Omega, \mb{F}_{N};X)$ and all $n \in \{0,1,\ldots,N\}$,
\begin{equation*}
  \psi(\E^{\mc{F}_{n}} \mb{g}) = \E^{\mc{F}_{n}'} \psi(\mb{g}). 
\end{equation*}
This observation lets us make the following reduction of the UMD property.

\begin{prop}\label{prop:UMD-FIF}
  To check that a Banach space $X$ has the UMD property, it suffices to consider martingales on $[0,1)$ with respect to finite interval filtrations.
\end{prop}

\begin{proof}
  Fix $p \in (1,\infty)$ and let $\mb{f}$ be an $L^p$-bounded $X$-valued martingale on a probability space $(\Omega,\P)$.
  Fix $\varepsilon > 0$.
  Then by Lemma \ref{lem:mgale-simple-approx} there exists an $X$-valued martingale $\mb{g}_{\bullet}$ on $(\Omega,\P)$ formed of simple functions well-approximating $\mb{f}_{\bullet}$.
  For all $N \in \N$ we have
  \begin{equation*}
    \begin{aligned}
      \Big\| \sum_{n=0}^{N} \xi_{n} d\mb{f}_{n} \Big\|_{L^p(\Omega;X)}
      &\leq \Big\| \sum_{n=0}^{N} \xi_{n} d\mb{g}_{n} \Big\|_{L^p(\Omega;X)} + \Big\| \sum_{n=0}^{N} \xi_{n} (d\mb{f}_{n} - d\mb{g}_{n}) \Big\|_{L^p(\Omega;X)}.
    \end{aligned}
  \end{equation*}
  The second term is bounded by
  \begin{equation*}
    \Big\| \sum_{n=0}^{N} \xi_{n} (d\mb{f}_{n} - d\mb{g}_{n}) \Big\|_{L^p(\Omega;X)}
    \leq \sum_{n=0}^{N} \|d\mb{f}_{n} - d\mb{g}_{n}\|_{L^p(\Omega;X)} \leq \sum_{n=0}^{N}2^{-n-2}\varepsilon \leq \varepsilon.
  \end{equation*}
  For the first term, we use the observation above concerning the isomorphism $\psi$ to write
  \begin{equation*}
    \begin{aligned}
      \Big\| \sum_{n=0}^{N} \xi_{n} d\mb{g}_{n} \Big\|_{L^p(\Omega;X)}
      &= \Big\| \sum_{n=0}^{N} \xi_{n} (\E^{\mc{F}_{n}} - \E^{\mc{F}_{n-1}}) \mb{g}_{N} \Big\|_{L^p(\Omega;X)} \\
      &= \Big\| \sum_{n=0}^{N} \xi_{n} (\E^{\mc{F}'_{n}} - \E^{\mc{F}'_{n-1}}) \psi^{-1}(\mb{g}_{N}) \Big\|_{L^p([0,1);X)} \\
      &\leq C \| \psi^{-1}(\mb{g}_{N}) \|_{L^p([0,1);X)} 
      = C \|\mb{g}_{N}\|_{L^p(\Omega;X)}
    \end{aligned}
  \end{equation*}
  using the assumed UMD property with respect to finite interval filtrations.
  The result then follows by writing
  \begin{equation*}
    \|\mb{g}_{N}\|_{L^p(\Omega;X)} \leq \|\mb{f}_{N}\|_{L^p(\Omega;X)} + \|\mb{g}_{N} - \mb{f}_{N}\|_{L^p(\Omega;X)}
    \leq \|\mb{f}_{N}\|_{L^p(\Omega;X)} + \varepsilon
  \end{equation*}
  and using that $\varepsilon > 0$ was arbitrary.
\end{proof}

Our goal is to reduce all finite interval filtrations to the dyadic filtration.
This still needs a bit of work.
First, note that every finite interval filtration $(\mc{F}_{n})_{n=0}^{N}$ such that $\mc{F}_{0}$ is trivial (and we can always assume this without loss of generality) corresponds to a \emph{rooted tree} with $N$ levels: the vertices at level $n$ are the atoms of $\mc{F}_{n}$, and two vertices are adjacent if and only if the corresponding atoms are comparable (i.e. one is contained in the other).
Denote this tree by $T_{\mc{F}} = (V(T_{\mc{F}}), E(T_{\mc{F}}))$, where $V(T_{\mc{F}})$ and $E(T_{\mc{F}})$ denote the sets of vertices and edges respectively.
Let $\alpha \colon V(T_{\mc{F}}) \to \mc{F}_{N}$ be the \emph{vertex-to-atom map}, sending a vertex of the tree $T_{\mc{F}}$ to the associated atom of $\mc{F}_{\bullet}$.
We say that two finite interval filtrations $\mc{F}_{\bullet}$, $\mc{F}'_{\bullet}$ are \emph{commonly indexed} if the associated trees $T_{\mc{F}}$ and $T_{\mc{F}'}$ are isomorphic as graphs: that is, the containment structures among the atoms of $\mc{F}_{\bullet}$ and $\mc{F}'_{\bullet}$ are identical.
A simple example of this construction is pictured in Figure TODO\todo{make figure}.

\begin{defn}
  Let $\mc{F}_{\bullet}$ and $\mc{F}_{\bullet}'$ be commonly indexed finite interval filtrations with associated tree $T$ and vertex-to-atom maps $\map{\alpha}{T}{\mc{F}_{N}}$ and $\map{\alpha'}{T}{\mc{F}_{N}'}$.
  For $\varepsilon \in (0,1)$, we say that $\mc{F}_{\bullet}$ and $\mc{F}_{\bullet}'$ are \emph{$\varepsilon$-close} if for all $v \in V(T)$,
  \begin{equation*}
    |\alpha(v) \Delta \alpha'(v)| = |(\alpha(v) \cup \alpha'(v)) \sm (\alpha(v) \cap \alpha'(v))| < \varepsilon,
  \end{equation*}
  and thus also
  \begin{equation*}
    |\alpha(v)| - |\alpha'(v)| < \varepsilon.
  \end{equation*}
\end{defn}

\begin{lem}\label{lem:closeness}
  Fix $N \in \N$ and let $(\mc{F}_{n})_{n=0}^{N}, (\mc{F}'_{n})_{n=0}^{N}$ be commonly indexed finite interval filtrations, with associated tree $T$ and vertex-to-atom maps $\map{\alpha}{T}{\mc{F}_{N}}$ and $\map{\alpha'}{T}{\mc{F}_{N}'}$.
  Then if $\mc{F}_{\bullet}$ and $\mc{F}_{\bullet}$ are $\varepsilon$-close, then for all $\mb{f} \in L^p(\mc{F}_{N};X)$ with $\|\mb{f}\|_{L^p(\mc{F}_{N};X)} = 1$,
  \begin{equation*}
    \sup_{0 \leq n \leq N} \|(\E^{\mc{F}_{n}} - \E^{\mc{F}_{n}'})\mb{f}\|_{L^p([0,1);X)} \lesssim_{\mc{F}_{\bullet}} \varepsilon^{1/p}.
  \end{equation*}
\end{lem}

\begin{proof}
  First write
  \begin{equation*}
    \|(\E^{\mc{F}_{n}} - \E^{\mc{F}_{n}'})\mb{f}\|_{L^p([0,1);X)} \leq \sum_{v \in V(T)} \Big\| \frac{\1_{\alpha(v)}}{|\alpha(v)|} \otimes \E(\mb{f}\1_{\alpha(v)}) - \frac{\1_{\alpha'(v)}}{|\alpha'(v)|} \otimes \E(\mb{f}\1_{\alpha'(v)}) \Big\|_{L^p([0,1);X)}.
  \end{equation*}
  By the $\varepsilon$-closeness condition, for all $v \in V(T)$ we have
  \begin{equation*}
    \begin{aligned}
      &\Big\| \frac{\1_{\alpha(v)}}{|\alpha(v)|} - \frac{\1_{\alpha'(v)}}{|\alpha'(v)|} \Big\|_{p} \\
      &= \frac{1}{|\alpha(v)|} \Big\| \1_{\alpha(v)} - \frac{|\alpha(v)|}{|\alpha'(v)|} \1_{\alpha'(v)} \Big\|_{p} \\
      &\leq \frac{1}{|\alpha(v)|} \Big( \| \1_{\alpha(v)} - \1_{\alpha'(v)} \|_{p} + \Big\| \Big( 1 - \frac{|\alpha(v)|}{|\alpha'(v)|} \Big) \1_{\alpha'(v)} \Big\|_{p} \Big) \\
      &\leq \frac{1}{|\alpha(v)|} \Big( \varepsilon^{1/p} + |\alpha'(v)|^{1/p}\Big(1 - \frac{|\alpha(v)|}{|\alpha'(v)|} \Big) \Big) \\
      &= \frac{1}{|\alpha(v)|} \Big( \varepsilon^{1/p} + |\alpha'(v)|^{\frac{1}{p} - 1} (|\alpha'(v)| - |\alpha(v)| ) \Big) \\
      &\leq \frac{1}{|\alpha(v)|} (\varepsilon^{1/p} + \varepsilon) 
      \lesssim_{\mc{F}_{\bullet}} \varepsilon^{1/p}.
  \end{aligned}
\end{equation*}
We also have
\begin{equation*}
  \begin{aligned}
    \big\| \E(\mb{f}\1_{\alpha(v)}) - \E(\mb{f}\1_{\alpha'(v)}) \big\|_{X}
    \leq \|\mb{f}\|_{L^\infty([0,1);X)} |\alpha(v) \Delta \alpha'(v)| \lesssim_{\mc{F}_{\bullet}} \varepsilon
  \end{aligned}
\end{equation*}
using that the $\sigma$-algebra $\mc{F}_{N}$ is finite to control $\|\mb{f}\|_{L^\infty([0,1);X)}$ by $\|\mb{f}\|_{L^p([0,1);X)} = 1$.
Thus for all $v \in V(T)$ we can estimate
\begin{equation*}
  \begin{aligned}
    &\Big\| \frac{\1_{\alpha(v)}}{|\alpha(v)|} \otimes \E(\mb{f}\1_{\alpha(v)}) - \frac{\1_{\alpha'(v)}}{|\alpha'(v)|} \otimes \E(\mb{f}\1_{\alpha'(v)}) \Big\|_{L^p([0,1);X)} \\
    &\leq \Big\| \Big( \frac{\1_{\alpha(v)}}{|\alpha(v)|} - \frac{\1_{\alpha'(v)}}{|\alpha'(v)|}\Big) \otimes \E(\mb{f}\1_{\alpha(v)}) \Big\|_{L^p([0,1);X)}
    + \Big\| \frac{\1_{\alpha'(v)}}{|\alpha'(v)|} \otimes \big( \E(\mb{f}\1_{\alpha(v)}) -  \E(\mb{f}\1_{\alpha'(v)}) \big)\Big\|_{L^p([0,1);X)}\\
    &\leq  \Big\| \frac{\1_{\alpha(v)}}{|\alpha(v)|} - \frac{\1_{\alpha'(v)}} {|\alpha'(v)|} \Big\|_{p} \| \E(\mb{f}\1_{\alpha(v)}) \|_{X}
    + \Big\| \frac{\1_{\alpha'(v)}}{|\alpha'(v)|} \Big\|_{p} \big\|  \E(\mb{f}\1_{\alpha(v)}) -  \E(\mb{f}\1_{\alpha'(v)}) \big\|_{X} \\
    &\lesssim_{\mc{F}_{\bullet}} \varepsilon^{1/p} \|\mb{f}\|_{L^p([0,1);X)} + \varepsilon \lesssim \varepsilon^{1/p}.
  \end{aligned}
\end{equation*}
This completes the proof.
\end{proof}

% We say that a finite interval filtration $(\mc{F}_n)_{n=0}^{N}}$ is \emph{conditionally dyadic} if $\mc{F}_{N} \subset \mc{D}_{M}$ for some $M \in \N$: that is, if every atom in $\mc{F}_{N}$ can be written as a union of dyadic intervals $I$ of length $2^{-M}$ for some (potentially very large) $M$.
% Equivalently, all the conditional probabilities $|I|/|J|$, where $I$ is an atom of $\mc{F}_{n}$ and $J$ is an atom of $\mc{F}_{n+1}$, are of the form $2^{-j} k$ for some $j,k \in \N$.
% In the next lemma we show that finite interval filtrations can be approximated arbitrarily well by conditionally dyadic filtrations.

In the next lemma we show that finite interval filtrations can be approximated arbitrary well with filtrations that are very closely related to the dyadic filtration.

\begin{lem}\label{lem:dyadic-approximation}
  Let $(\mc{F}_{n})_{n=0}^{N}$ be a finite interval filtration and $\varepsilon \in (0,1)$.
  Then there is filtration $\mc{F}'_{\bullet}$ such that $\mc{F}_{\bullet}$ and $\mc{F}'_{\bullet}$ are commonly indexed and $\varepsilon$-close, with $\mc{F}'_{n} \subset \mc{D}_{m(n)}$ for some increasing sequence $m(0) < m(1) < \ldots < m(N)$, where $\mc{G}_{\bullet}$ is the dyadic filtration.
  Furthermore, for all $n \in \{0,1,\ldots,N\}$ we have
  \begin{equation}\label{eq:approx-CE-ident}
    \E^{\mc{F}'_{n}} = \E^{\mc{G}_{m(n)}} \E^{\mc{F}'_{N}}.
  \end{equation}
\end{lem}

\begin{proof}
  We construct $\mc{F}'_{\bullet}$ inductively, with the trivial base case $\mc{F}'_{0} = \mc{F}_{0} = \{\varnothing, [0,1)\}$.
  Having constructed $\mc{F}'_{n}$ as required, and with the sequence $m(0) < m(1) < \ldots < m(n)$ at hand, choose $m(n+1) > m(n)$ sufficiently large that each left endpoint $\ell(A)$, with $A$ an atom of $\mc{F}_{n+1}$, is within distance $\varepsilon/2$ of the set of dyadic numbers
  \begin{equation*}
    D_{m(n+1)} := \{2^{-m(n+1)}k : k \in \{0,1,\ldots,2^{m(n+1)} - 1\}\}.
  \end{equation*}
  For each atom $A$ of $\mc{F}_{n+1}$, choose a dyadic number $d_{A} \in D_{m(n+1)}$ with $|\ell(A) - d_{A}| < \varepsilon/2$.
  If $m(n+1)$ is sufficiently large we can ensure that the points $d_{A}$ are unique.
  Let $(d_{i})_{i=0}^{K}$ be the sequence of points $d_{A}$ in increasing order, and let
  \begin{equation*}
    \mc{F}'_{n+1} := \sigma([0,d_{0}), [d_{0}, d_{1}), \ldots, [d_{K},1)).
  \end{equation*}
  Note that $\mc{F}'_{n+1} \subset \mc{G}_{m(n+1)}$ by construction, and that
  \begin{equation*}
    |A \Delta [d_{A}, d_{A+1})| < \varepsilon
  \end{equation*}
  for all atoms $A \in \mc{F}_{n+1}$.
  Iterating this construction yields a filtration with the claimed properties, other than the conditional expectation identity \eqref{eq:approx-CE-ident} which remains to be shown.
  To show this, fix $\mb{f} \in L^1(\mc{F}_{N}')$ and $I \in \mc{F}_{n}'$: we will show that
  \begin{equation*}
    \int_{I} \E^{\mc{G}_{m(n)}} \E^{\mc{F}_{N}'} \mb{f} \, \dd t = \int_{I} \mb{f},
  \end{equation*}
  which establishes the identity via the defining property of conditional expectations.
  Since $I \in \mc{F}_{n}' \subset \mc{G}_{m(n)}$, $I$ can be written as a union of dyadic intervals $J \in \mc{D}_{m(n)}$, so
  \begin{equation*}
    \begin{aligned}
      \int_{I} \E^{\mc{G}_{m(n)}} \E^{\mc{F}_{N}'} \mb{f}(t) \, \dd t
      &= \sum_{\substack{J \in \mc{D}_{m(n)} \\ J \subset I}} \int_{J} \E^{\mc{G}_{m(n)}} \E^{\mc{F}_{N}'} \mb{f}(t) \, \dd t \\
      &= \sum_{\substack{J \in \mc{D}_{m(n)} \\ J \subset I}} \int_{J} \E^{\mc{F}_{N}'} \mb{f}(t) \, \dd t
      = \int_{I} \E^{\mc{F}_{N}'} \mb{f}(t) \, \dd t = \int_{I} \mb{f}(t) \, \dd t
    \end{aligned}
  \end{equation*}
  using $J \in \mc{G}_{m(n)}$ and $I \in \mc{F}_{n}' \subset \mc{F}_{N}'$.
\end{proof}

\begin{thm}
  Suppose that $p \in (1,\infty)$ and that $X$ has the dyadic UMD property.
  Then $X$ is UMD.
\end{thm}

\begin{proof}
  By Proposition \ref{prop:UMD-FIF} it suffices to consider martingales $\mb{f}_{\bullet}$ on $[0,1)$ with respect to finite interval filtrations $(\mc{F}_{n})_{n=0}^{N}$.
  By homogeneity we may assume that $\|\mb{f}_{N}\|_{L^p([0,1);X)} = 1$.
  Let $(\xi_{n})_{n=0}^{N}$ be a sequence of signs.
  Fix $\varepsilon \in (0,1)$ and let $\mc{F}'_{\bullet}$ be a filtration as in Lemma \ref{lem:dyadic-approximation}, such that $\mc{F}_{n}' \subset \mc{D}_{m(n)}$ for all $n$ and such that $\mc{F}_{\bullet}$ and $\mc{F}_{\bullet}'$ are commonly indexed and $\varepsilon$-close.
  Let $\mb{f}_{\bullet}$ be a martingale with respect to a finite interval filtration $\mc{F}_{\bullet}$, and let $\mc{F}'_{\bullet}$ be an $\varepsilon$-close filtration as in the previous lemma.
  Then we have
  \begin{equation*}
    \begin{aligned}
      &\Big\| \sum_{n=0}^{N} \xi_{n} (\E^{\mc{F}_{n}} - \E^{\mc{F}_{n-1}}) \mb{f}_{N} \Big\|_{L^p([0,1);X)} \\
      &= \Big\| \sum_{n=0}^{N} \xi_{n} (\E^{\mc{F}'_{n}} - \E^{\mc{F}'_{n-1}}) \mb{f}_{N} \Big\|_{L^p([0,1);X)} \\
      &+ \sum_{n=0}^{N} \Big( \| (\E^{\mc{F}_{n}} - \E^{\mc{F}'_{n}}) \mb{f}_{N}\|_{L^p([0,1);X)}
      +  \| (\E^{\mc{F}'_{n}} - \E^{\mc{F}'_{n-1}}) \mb{f}_{N} \|_{L^p([0,1);X)} \Big).
    \end{aligned}
  \end{equation*}
  By $\varepsilon$-closeness, the second term is bounded by $C_{\mc{F}_{\bullet}} N \varepsilon^{1/p}$ (Lemma \ref{lem:closeness}, here we use $\|\mb{f}_{N}\|_{L^p([0,1);X)} = 1$).
  To control the first term we write
  \begin{equation*}
      \Big\| \sum_{n=0}^{N} \xi_{n} (\E^{\mc{F}'_{n}} - \E^{\mc{F}'_{n-1}}) \mb{f}_{N} \Big\|_{L^p([0,1);X)}
      &= \Big\| \sum_{n=0}^{N} \xi_{n} (\E^{\mc{G}_{m(n)}} - \E^{\mc{G}_{m(n-1)}}) \E^{\mc{F}'_{N}} \mb{f}_{N} \Big\|_{L^p([0,1);X)}
    \end{equation*}
    using the identity \eqref{eq:approx-CE-ident}.
    By a telescoping sum, we have
    \begin{equation*}
      \begin{aligned}
      &\Big\| \sum_{n=0}^{N} \xi_{n} (\E^{\mc{G}_{m(n)}} - \E^{\mc{G}_{m(n-1)}}) \E^{\mc{F}'_{N}} \mb{f}_{N} \Big\|_{L^p([0,1);X)} \\
      &\qquad = \Big\| \sum_{m=0}^{m(N)} \tilde{\xi}_{m} (\E^{\mc{G}_{m}} - \E^{\mc{G}_{m-1}}) \E^{\mc{F}'_{N}} \mb{f}_{N} \Big\|_{L^p([0,1);X)}
    \end{aligned}
    \end{equation*}
    where
    \begin{equation*}
      \tilde{\xi}_{m} :=
      \begin{cases}
        \xi_{n} & \text{if $m = m(n)$} \\
        1 & \text{otherwise.}
      \end{cases}
    \end{equation*}
    Thus the assumed UMD property with respect to the dyadic filtration yields
    \begin{equation*}
      \begin{aligned}
        \Big\| \sum_{n=0}^{N} \xi_{n} (\E^{\mc{G}_{m(n)}} - \E^{\mc{G}_{m(n-1)}}) \E^{\mc{F}'_{N}} \mb{f}_{N} \Big\|_{L^p([0,1);X)}
        &\leq C \|\E^{\mc{F}'_{N}} \mb{f}_{N}\|_{L^p([0,1);X)} \\
        &\leq C\|\mb{f}_{N}\|_{L^p([0,1);X)}.
      \end{aligned}
    \end{equation*}
    All up we have
    \begin{equation*}
      \Big\| \sum_{n=0}^{N} \xi_{n} d\mb{f}_{N} \Big\|_{L^p([0,1);X)} \leq C\|\mb{f}_{N}\|_{L^p([0,1);X)} + C_{\mc{F}_{\bullet}} N \varepsilon^{1/p},
    \end{equation*}
    and since $\varepsilon \in (0,1)$ was arbitrary this completes the proof.
  \end{proof}

  \todo{have to discuss paley--walsh representation}


\section{The Hilbert transform on the torus}


\begin{defn}
  Fourier transform on the torus, IFT on $\Z$, Fourier multipliers
\end{defn}

\begin{defn}
  hilbert transform on the torus
\end{defn}

\begin{thm}
  UMD implies boundedness of HT on torus
\end{thm}

\begin{defn}
  amplifications of HT 
\end{defn}

\section{Bourgain's theorem}

In this section we prove Theorem \ref{thm:Bourgain}: if the Banach space $X$ is such that the Hilbert transform $H \in \Lin(L^p(\R))$ has a bounded $X$-valued extension for some $p \in (1,\infty)$, then $X$ is UMD.
By Theorem \ref{thm:dyadic-UMD} it suffices to show that $X$ has the dyadic $\UMD_{p}$ property.


\section*{Exercises}

\begin{exercise}
  Let $X$ be a complex Banach space.
  Show that the Mikhlin multiplier theorem (Theorem \ref{thm:Mikhlin}, in the case $X = Y$) and the Littlewood--Paley theorem (Theorem \ref{thm:LittlewoodPaley}) hold if and only if $X$ has the UMD property.\footnote{You can and should argue via Bourgain's theorem.}
\end{exercise}




%%% Local Variables:
%%% mode: latex
%%% TeX-master: "../main.tex"
%%% End:
