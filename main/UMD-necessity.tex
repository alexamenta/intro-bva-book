The goal of this chapter is to prove the converse to Burkholder's theorem (Theorem \ref{thm:Burkholder}).

\begin{thm}[Bourgain]\label{thm:Bourgain}
  Let $X$ be a Banach space, and suppose that for some $p \in (1,\infty)$ the Hilbert transform $H \in \Lin(L^p(\R))$ admits a bounded $X$-valued extension.
  Then $X$ is UMD.
\end{thm}





\section{Paley--Walsh martingales and dyadic UMD}

\begin{defn}
  paley--walsh martingales
\end{defn}

\begin{defn}
  For $p \in (1,\infty)$, a Banach space $X$ is said to have the \emph{dyadic $\UMD_{p}$ property} if there exists a constant $C < \infty$ such that for all $X$-valued Paley--Walsh martingales $\mb{f}_{\bullet}$ and all sequences of signs $\xi_n = \pm 1$,
  \begin{equation}\label{eq:dyadic-UMD-property}
    \sup_{N \in \N} \Big\| \sum_{n=0}^{N} \xi_{n} d\mb{f}_{n} \Big\|_{L^p(\Omega;X)} \leq C \sup_{N \in \N} \Big\| \sum_{n=0}^{N} d\mb{f}_{n}\Big\|_{L^p(\Omega;X)}.
  \end{equation}
\end{defn}

This is the same definition as that of the $\UMD_{p}$ property, but with the class of martingales restricted to the Paley--Walsh martingales.

\begin{thm}\label{thm:dyadic-UMD}
  The dyadic $\UMD_{p}$ property is equivalent to the (unrestricted) $\UMD_{p}$ property.
\end{thm}

\begin{proof}
  
\end{proof}

\section{Transference and Fourier multipliers on the torus}

\begin{thm}
  Suppose that 
\end{thm}

\begin{defn}
  Fourier transform on the torus, IFT on $\Z$, Fourier multipliers
\end{defn}

\begin{defn}
  hilbert transform on the torus
\end{defn}

\begin{thm}
  UMD implies boundedness of HT on torus
\end{thm}

\begin{defn}
  amplifications of HT 
\end{defn}

\section{Bourgain's theorem}

In this section we prove Theorem \ref{thm:Bourgain}: if the Banach space $X$ is such that the Hilbert transform $H \in \Lin(L^p(\R))$ has a bounded $X$-valued extension for some $p \in (1,\infty)$, then $X$ is UMD.
By Theorem \ref{thm:dyadic-UMD} it suffices to show that $X$ has the dyadic $\UMD_{p}$ property.


\section*{Exercises}

\begin{exercise}
  Let $X$ be a complex Banach space.
  Show that the Mikhlin multiplier theorem (Theorem \ref{thm:Mikhlin}, in the case $X = Y$) and the Littlewood--Paley theorem (Theorem \ref{thm:LittlewoodPaley}) hold if and only if $X$ has the UMD property.\footnote{You can and should argue via Bourgain's theorem.}
\end{exercise}




%%% Local Variables:
%%% mode: latex
%%% TeX-master: "../main.tex"
%%% End:
