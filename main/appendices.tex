\section{Functional analysis and Banach spaces}

Throughout this section $X$ is a Banach space over the scalar field $\K$ (which is either $\R$ or $\C$).
$X^*$ denotes the dual Banach space, i.e. the Banach space of all bounded $\K$-linear functionals $\map{\mb{x}^*}{X}{\K}$, under the norm
\begin{equation*}
  \|\mb{x}^*\|_{X^*} := \sup_{\mb{x} \in X \sm \{0\}} \frac{|\mb{x}^*(\mb{x})|}{\|\mb{x}\|_{X}} = \sup_{\substack{\mb{x} \in X \\ \|\mb{x}\|_{X} = 1}} |\mb{x}^*(\mb{x})|.
\end{equation*}
Generally we write $\langle \mb{x}, \mb{x}^* \rangle := \mb{x}^*(\mb{x})$.

The following results are all standard, and the suggested references are essentially picked out at random.
First, the real and complex versions of the Hahn--Banach theorem.
See \cite[Section III.3]{RS80}.

\begin{thm}[Hahn--Banach: real case]\index{theorem!Hahn--Banach}
  Let $X$ be a real Banach space.
  Let $\map{p}{X}{\R}$ be a real-valued function satisfying
  \begin{equation*}
    p(\alpha \mb{x}_1 + (1-\alpha) \mb{x}_2) \leq \alpha p(\mb{x}_1) + (1-\alpha)p(\mb{x}_2)
  \end{equation*}
  for all $\mb{x}_1, \mb{x}_2 \in X$ and all $\alpha \in  [0,1]$.
  Let $\lambda$ be an $\R$-linear functional defined on a subspace $Y \subset X$, satisfying
  \begin{equation*}
    \lambda(\mb{y}) \leq p(\mb{y}) \qquad \forall \mb{y} \in Y.
  \end{equation*}
  Then there exists a functional $\Lambda \in X^*$ such that $\Lambda(\mb{x}) \leq p(\mb{x})$ for all $\mb{x} \in X$ and $\Lambda(\mb{y}) = \lambda(\mb{y})$ for all $\mb{y} \in Y$.
\end{thm}

\begin{thm}[Hahn--Banach: complex case]
  Let $X$ be a complex Banach space.
  Let $\map{p}{X}{\R}$ be a real-valued function satisfying
  \begin{equation*}
    p(\alpha \mb{x}_1 + \beta \mb{x}_2) \leq |\alpha|p(\mb{x}_1) + |\beta|p(\mb{x}_2)
  \end{equation*}
  for all $\mb{x}_1, \mb{x}_2 \in X$ and all $\alpha,\beta \in \C$ with $|\alpha| + |\beta| = 1$.
  Let $\lambda$ be a $\C$-linear functional defined on a subspace $Y \subset X$, satisfying
  \begin{equation*}
    |\lambda(\mb{y})| \leq p(\mb{y}) \qquad \forall \mb{y} \in Y.
  \end{equation*}
  Then there exists a functional $\Lambda \in X^*$ such that $|\Lambda(\mb{x})| \leq p(\mb{x})$ for all $\mb{x} \in X$ and $\Lambda(\mb{y}) = \lambda(\mb{y})$ for all $\mb{y} \in Y$.
\end{thm}

Given a Banach space $X$ and a closed subspace $Y \subset X$, the \emph{quotient space} $X / Y$ is the set of \emph{cosets}
\begin{equation*}
  [\mb{x}] := \{\mb{x} + \mb{y} : \mb{y} \in Y\} \subset X
\end{equation*}
equipped with the quotient norm
\begin{equation*}
  \|[\mb{x}]\|_{X / Y} := \inf\{\|\mb{x} + \mb{y}\|_{X} : \mb{y} \in Y\}.
\end{equation*}
The \emph{quotient map} $\map{\pi}{X}{X / Y}$ is given by $\pi(\mb{x}) = [\mb{x}]$.
The Hahn--Banach theorem implies the following duality relations between subspaces and quotient spaces.

\begin{prop}\label{prop:duality-subspace-quotient}
  Let $Y$ be a closed subspace of $X$, and define the annihilator
  \begin{equation*}
    Y^\perp := \{\mb{x}^* \in X^* : \text{$\langle \mb{y}, \mb{x}^* \rangle$ for all $\mb{y} \in Y$}\}. 
  \end{equation*}
  Then $Y^{\perp}$ is a closed subspace of $X^*$, and there are isometric isomorphisms
  \begin{equation*}
    X^* / Y^\perp = Y^* \qquad \text{and} \qquad (X/Y)^* = Y^{\perp}:
  \end{equation*}
  the first is given by restriction to $Y$, and the second is given by precomposition with the quotient map $\map{\pi}{X}{X / Y}$.
\end{prop}

The \emph{weak topology}\index{topology!weak} on $X$ is the weakest topology such that every functional $\mb{x}^* \in X^*$ is continuous.
The weak toplogy is weaker than the usual (norm) topology on $X$, and these topologies are equal if and only if $X$ is finite dimensional.
The \emph{weak-*} topology\index{topology!weak-*} on a dual space $X^*$ is the weakest topology such that all the functions
\begin{equation*}
  \big\{ \mb{x}^* \mapsto \mb{x}^*(\mb{x}) : \mb{x} \in X\big\}
\end{equation*}
are continuous.
The weak-* topology is great because it has the following fundamental compactness property \cite[Theorem IV.21]{RS80}.

\begin{thm}[Banach--Alaoglu]\index{theorem!Banach--Alaoglu}
  The closed unit ball of $X^*$ is compact in the weak-* topology.
\end{thm}

The \emph{double dual} of $X$ is the space $X^{**} = (X^*)^*$.
Each $\mb{x} \in X$ induces an element of the double dual in a notationally confusing way: for each $\mb{x}^* \in X^*$, $\mb{x}$ acts on $\mb{x}^*$ by
\begin{equation*}
  \langle \mb{x}^*, \mb{x} \rangle := \mb{x}^*(\mb{x}) = \langle \mb{x}, \mb{x}^* \rangle.
\end{equation*}
This identification yields a canonical isometric embedding $\map{j}{X}{X^{**}}$, and $X$ is called \emph{reflexive}\index{reflexivity} if this map is surjective (i.e. each functional on $X^*$ is given by an element of $X$, using the above identification).
In this case $X$ and $X^{**}$ are canonically isomorphic.\footnote{It is possible that $X$ and $X^{**}$ are isometrically isomorphic without $X$ being reflexive \cite{rJ51}.}
One can show that $X$ is reflexive if and only if the weak topology and the weak-* topology inherited from $X^{**}$ coincide.
Using Banach--Alaoglu, this can be restated in the following way.

\begin{cor}\label{cor:reflexive-iff-weakcpt}
  $X$ is reflexive if and only if the closed unit ball $B_{X}$ is weakly compact.
\end{cor}

Hahn--Banach can be used to prove the following density result for the closed unit ball $\overline{B_{X}}$ in $\overline{B_{X^{**}}}$.
See \cite[Proposition B.1.17]{HNVW16}.

\begin{thm}[Goldstine]\label{thm:goldstine}\index{theorem!Goldstine}
  The closed unit ball $\overline{B_{X}}$ of $X$ is weak-* dense in the closed unit ball $\overline{B_{X^{**}}}$ of $X^{**}$ (when viewing $X$ as a subset of $X^{**}$ by the canonical isometric embedding).
\end{thm}

There are three notions of compactness in topological spaces that coincide for metric spaces: \emph{compactness} (every open cover has a finite subcover), \emph{sequential compactness} (every sequence has a convergent subsequence), and \emph{countable compactness} (every sequence has a cluster point, or equivalently, every countable open cover has a finite subcover).
Although the weak topology on $X$ may not be metrisable, it behaves as if it were:

\begin{thm}[Eberlein--Smulian]\label{thm:eberlein-smulian}\index{theorem!Eberlein--Smulian}
  Let $A$ be a subset of a Banach space $X$.
  The following are equivalent:
  \begin{itemize}
  \item $A$ is weakly compact,
  \item $A$ is weakly sequentially compact,
  \item $A$ is weakly countably compact.
  \end{itemize}
\end{thm}

And thus Corollary \ref{cor:reflexive-iff-weakcpt} can be restated as follows:

\begin{cor}\label{cor:reflexive-bdd-subsequence}
  $X$ is reflexive if and only if every bounded sequence in $X$ has a weakly convergent subsequence.
\end{cor}

A Banach space is \emph{separable} if it has a countable dense subset.\index{separability}
One can show via the Hahn--Banach theorem that if $X^*$ is separable, then so is $X$ \cite[Theorem III.7]{RS80}.
Separability can be characterised in terms of weak-* metrisability in the dual space.

\begin{thm}\label{thm:sep-met-dual}
  $X$ is separable if and only if the closed unit ball $\overline{B_{X^*}}$ of $X^*$ is weak-* metrisable.
  Thus by Banach--Alaoglu and the equivalence of compactness notions for metrisable spaces, if $X$ is separable, then $\overline{B_{X^*}}$ is weak-* sequentially compact.
\end{thm}

Given subsets $F \subset X$ and $Y \subset X^{*}$, one says that $Y$ \emph{separates points of $F$} if for every distinct pair of vectors $\mb{x} \neq \mb{y} \in F$ there exists a functional $\mb{x}^{*} \in Y$ such that $\langle \mb{x}, \mb{x}^{*} \rangle \neq \langle \mb{y}, \mb{x}^{*} \rangle$.
A topological argument implies the following \cite[Proposition B.1.11]{HNVW16}.

\begin{prop}\label{prop:sep-sep-points}
  If $F$ is a separable subset of a Banach space $X$, and $Y \subset X^{*}$ is weak-* dense in $X^{*}$, then $Y$ contains a countable subset which separates points of $F$.
  In particular, if $X$ is separable, then there exists a sequence $(\mb{x}_{n}^{*})_{n \in \N}$ in $X^{*}$ which separates points of $X$.
\end{prop}

\textbf{\emph{further material to be added as required}}

\section{Probability theory}\label{sec:probability}

This course uses \emph{basic} probabilistic concepts quite heavily, but no advanced probability theory or stochastic analysis.
Here we collect the basic concepts and results that we will need.
A good reference for probability theory from the viewpoint of mathematical analysis is \cite{rD04} (from which I have taken most of this material).

Let $(\Omega,\mc{A})$ be a measurable space.
A \emph{probability measure} on $(\Omega,\mc{A})$ is a measure $\P$ with $\P(\Omega) = 1$.
Probabilists like to refer to measurable sets $A \in \mc{A}$ as \emph{events}.
If $S$ is a separable completely metrizable topological space (also called a \emph{Polish space}), an \emph{$S$-valued random variable} is a Borel measurable function $\map{f}{\Omega}{S}$.\index{random variable}

\begin{rmk}
  In Definition \ref{defn:RV}, we define $X$-valued random variables $\map{\mb{f}}{\Omega}{X}$, where $X$ is a Banach space which is possibly not separable.
  However, we demand that our random variables are \emph{strongly} measurable, which by the Pettis measurability theorem (Theorem \ref{thm:Pettis-measurability}) implies that there is a separable closed subspace $X_{0} \subset X$ such that $\mb{f}$ is valued in $X_{0}$.
  Thus, by replacing $X$ with $X_{0}$, our Banach-valued random variables are valid random variables in the sense of the previous definition.
\end{rmk}

\begin{defn}
  Let $(\Omega,\mc{A},\P)$ be a probability space, let $\Lambda$ be an indexing set, and let $(E_{\lambda})_{\lambda \in \Lambda}$ be Polish spaces.
  \begin{itemize}
  \item
    A sequence of random variables $\map{\xi_{\lambda}}{\Omega}{E_{\lambda}}$ is called \emph{mutually independent}\index{independence} if for all finite subsets of indices $\{\lambda_{n}\}_{n=1}^{N}$ in $\Lambda$ and all Borel sets $B_{n} \subset E_{\lambda_{n}}$,
    \begin{equation*}
      \P\Big(\bigcap_{n=1}^{N} \{\omega \in \Omega : \xi_{\lambda_{n}}(\omega) \in B_{n}\} \Big) = \prod_{n=1}^{N} \P(\xi_{\lambda_{n}} \in B_{n}).
    \end{equation*}

  \item
    A sequence of $\sigma$-algebras $\mc{A}_{\lambda}$ is called \emph{mutually independent} if for all finite subsets of indices $\{\lambda_{n}\}_{n=1}^{N}$ in $\Lambda$ and all sets $A_{n} \in \mc{A}_{\lambda_{n}}$,
    \begin{equation*}
      \P\Big(\bigcap_{n=1}^{N} A_{n} \Big) = \prod_{n=1}^{N} \P(A_{n}).
    \end{equation*}

  \end{itemize}
  Note that the random variables $(\xi_{\lambda})_{\lambda \in \Lambda}$ are independent if and only if the $\sigma$-algebras $(\sigma(\xi_{\lambda}))_{\lambda \in \Lambda}$ are independent.
\end{defn}

\begin{defn}
  Let $(\Omega,\mc{A},\P)$ be a probability space, $E$ a Polish space, and $\map{\xi}{\Omega}{E}$ a random variable.
  The \emph{distribution} of $\xi$ is the pushforward measure $\mu_{\xi} := \xi_{*}(\P)$ on $E$, which is defined by
  \begin{equation*}
    \mu_{\xi}(B) := \P(\xi \in B) = \P(\xi^{-1}(B))
  \end{equation*}
  for all Borel sets $B \in E$.
\end{defn}

For Banach-valued random variables the notion of the \emph{characteristic function} is also quite useful.

\begin{defn}
  Let $X$ be a Banach space and $(\Omega,\mc{A},\P)$ a probability space, and suppose $\map{\mb{f}}{\Omega}{X}$ is an $X$-valued random variable.
  The \emph{characteristic function} of $\mb{f}$ is the function $\map{\phi_{\mb{f}}}{X^{*}}{\C}$ given by
  \begin{equation*}
    \phi_{\mb{f}}(\mb{x}^{*}) := \E(e^{i\Re\langle \mb{f}, \mb{x}^{*} \rangle}) = \int_{\Omega} e^{i\Re\langle\mb{f}(\omega), \mb{x}^{*} \rangle} \, \dd\P(\omega) = \int_{X} e^{i\Re \langle\mb{x}, \mb{x}^{*} \rangle} \, \dd\mu_{\mb{f}}(\mb{x})
  \end{equation*}
  where $\mu_{\mb{f}}$ is the distribution of $\mb{f}$.
\end{defn}

In fact, the characteristic function of $\mb{f}$ is just the Fourier transform of the distribution $\mu_{\mb{f}}$ (with an appropriate definition of the Fourier transform of a measure on a Banach space).
The distribution of a Banach-valued random variable is completely determined by its characteristic function. See \cite[Corollary E.1.17]{HNVW17} for a proof.

\begin{thm}\label{thm:characteristic-function-uniqueness}
  Let $X$ be a Banach space and suppose that $\mb{f}_{1}$ and $\mb{f}_{2}$ are $X$-valued random variables with equal characteristic functions, $\phi_{\mb{f}_{1}} = \phi_{\mb{f}_{2}}$.
  Then $\mb{f}_{1}$ and $\mb{f}_{2}$ have equal distributions, $\mu_{\mb{f}_{1}} = \mu_{\mb{f}_{2}}$.
\end{thm}

\emph{\textbf{further material to be added as required}}

\section{Results from interpolation theory}

\begin{defn}
  Let $(S,\mc{A},\mu)$ be a measure space and $f$ a measurable scalar-valued function on $S$.
  For $p \in (0,\infty]$ we define
  \begin{equation*}
    \|f\|_{L^{p,\infty}(S)} := \sup_{t > 0} t^{1/p} \mu\big( \{s \in S : |f(s)| > t\} \big).
  \end{equation*}
  The space $L^{p,\infty}(S)$ of functions $f$ with $\|f\|_{L^{p,\infty}(S)}$ is called the \emph{Lorentz space}, or \emph{weak $L^p$ space}.
\end{defn}

Chebyshev's inequality implies that $\|f\|_{L^{p,\infty}(S)} \leq \|f\|_{L^p(S)}$, so the Lebesgue space $L^p(S)$ is contained in the Lorentz space $L^{p,\infty}(S)$.
This containment is generally strict: for example, the function
\begin{equation*}
  t \mapsto \1_{[0,1)}(t) t^{-1/p}
\end{equation*}
is in $L^{p,\infty}(\R)$ but not in $L^p(\R)$.
It also possible to define Lorentz spaces $L^{p,q}(S)$ for general $q$, but we will not use these.

The main application of weak Lebesgue spaces is in the \emph{Marcinkiewicz interpolation theorem}, which lets one deduce the boundedness of an operator on a range of $L^p$ spaces from `weak endpoint bounds'.
See \cite[Theorem 1.3.2]{grafakos} for the proof.

\begin{thm}[Marcinkiewicz]\label{thm:marcinkiewicz}\index{theorem!Marcinkiewicz (interpolation)}
  Let $(S_1,\mc{A}_1,\mu_1)$ and $(S_2,\mc{A}_2,\mu_2)$ be $\sigma$-finite measure spaces, and let $0 < p_0 < p_1 \leq \infty$.
  Let $T$ be a sublinear operator\footnote{That is, $|T(f + g)| \leq |T(f)| + |T(g)|$ and $|T(\lambda f)| = |\lambda||T(f)|$ for all $f,g$ and $\lambda \in \C$. This holds in particular if $T$ is linear.} defined on all functions in
  \begin{equation*}
    L^{p_0}(S_1) + L^{p_1}(S_1) = \{f_0 + f_1 : f_i \in L^{p_i}(S_1)\},
  \end{equation*}
  mapping into the space of measurable functions on $S_1$.
  Suppose there exist constants $C_0, C_1 < \infty$ such that
  \begin{equation*}
    \begin{aligned}
      \|Tf\|_{L^{p_0,\infty}(S_2)} &\leq C_0 \|f\|_{L^{p_0}(S_1)} \qquad \forall f \in L^{p_0}(S_1), \\
      \|Tf\|_{L^{p_1,\infty}(S_2)} &\leq C_1 \|f\|_{L^{p_1}(S_1)} \qquad \forall f \in L^{p_1}(S_1).
    \end{aligned}
  \end{equation*}
  Then there exists a constant $C$ such that for all $p_0 < p < p_1$ and all $f \in L^p(S_1)$,
  \begin{equation*}
    \|Tf\|_{L^{p}(S_2)} \leq C \|f\|_{L^{p}(S_1)} \qquad \forall f \in L^{p}(S_1).
  \end{equation*}
  That is, if $T$ is bounded from $L^{p_0}$ to $L^{p_0,\infty}$ and from $L^{p_1}$ to $L^{p_1,\infty}$, then $T$ is bounded on $L^p$ for all $p \in (p_0,p_1)$.
  This hypothesis holds in particular if $T$ is bounded on $L^{p_0}$ and $L^{p_1}$.
\end{thm}




\emph{\textbf{further material to be added as required}}



% \begin{thm}[Riesz--Thorin]
%   (Riesz--Thorin theorem for Bochner spaces)
% \end{thm}

% then move to more general interpolation spaces

% \begin{itemize}
% \item defn of interpolation spaces (maybe just complex)
% \item basic structural results
% \item (co)retraction theorem
% \item examples 
% \end{itemize}


%%% Local Variables:
%%% mode: latex
%%% TeX-master: "../main.tex"
%%% End:
