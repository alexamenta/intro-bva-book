\section{Results from functional analysis}

Throughout this section $X$ is a Banach space over the scalar field $\K$ (which is either $\R$ or $\C$).
$X^*$ denotes the dual Banach space, i.e. the Banach space of all bounded $\K$-linear functionals $\map{\mb{x}^*}{X}{\K}$, under the norm
\begin{equation*}
  \|\mb{x}^*\|_{X^*} := \sup_{\mb{x} \in X \sm \{0\}} \frac{|\mb{x}^*(\mb{x})|}{\|\mb{x}\|_{X}} = \sup_{\substack{\mb{x} \in X \\ \|\mb{x}\|_{X} = 1}} |\mb{x}^*(\mb{x})|.
\end{equation*}
Often we write $\langle \mb{x}, \mb{x}^* \rangle := \mb{x}^*(\mb{x})$.

The following results are all standard the suggested references are essentially picked out at random.

\cite[Section III.3]{RS80}

\begin{thm}[Hahn--Banach: real case]
  Let $X$ be a real Banach space.
  Let $\map{p}{X}{\R}$ be a real-valued function satisfying
  \begin{equation*}
    p(\alpha \mb{x}_1 + (1-\alpha) \mb{x}_2) \leq \alpha p(\mb{x}_1) + (1-\alpha)p(\mb{x}_2)
  \end{equation*}
  for all $\mb{x}_1, \mb{x}_2 \in X$ and all $\alpha \in  [0,1]$.
  Let $\lambda$ be an $\R$-linear functional defined on a subspace $Y \subset X$, satisfying
  \begin{equation*}
    \lambda(\mb{y}) \leq p(\mb{y}) \qquad \forall \mb{y} \in Y.
  \end{equation*}
  Then there exists a functional $\Lambda \in X^*$ such that $\Lambda(\mb{x}) \leq p(\mb{x})$ for all $\mb{x} \in X$ and $\Lambda(\mb{y}) = \lambda(\mb{y})$ for all $\mb{y} \in Y$.
\end{thm}

\begin{thm}[Hahn--Banach: complex case]
  Let $X$ be a complex Banach space.
  Let $\map{p}{X}{\R}$ be a real-valued function satisfying
  \begin{equation*}
    p(\alpha \mb{x}_1 + \beta \mb{x}_2) \leq |\alpha|p(\mb{x}_1) + |\beta|p(\mb{x}_2)
  \end{equation*}
  for all $\mb{x}_1, \mb{x}_2 \in X$ and all $\alpha,\beta \in \C$ with $|\alpha| + |\beta| = 1$.
  Let $\lambda$ be a $\C$-linear functional defined on a subspace $Y \subset X$, satisfying
  \begin{equation*}
    |\lambda(\mb{y})| \leq p(\mb{y}) \qquad \forall \mb{y} \in Y.
  \end{equation*}
  Then there exists a functional $\Lambda \in X^*$ such that $|\Lambda(\mb{x})| \leq p(\mb{x})$ for all $\mb{x} \in X$ and $\Lambda(\mb{y}) = \lambda(\mb{y})$ for all $\mb{y} \in Y$.
\end{thm}

The Hahn--Banach theorem implies duality relations between subspaces and quotient spaces.

\begin{prop}\label{prop:duality-subspace-quotient}
  Let $Y$ be a closed subspace of $X$, and define the annihilator
  \begin{equation*}
    Y^\perp := \{\mb{x}^* \in X^* : \text{$\langle \mb{y}, \mb{x}^* \rangle$ for all $\mb{y} \in Y$}\}. 
  \end{equation*}
  Then $Y^{\perp}$ is a closed subspace of $X^*$, and we have isometric isomorphisms
  \begin{equation*}
    X^* / Y^\perp = M^* \qquad \text{and} \qquad (X/Y)^* = M^{\perp}:
  \end{equation*}
  the first is given by restriction to $Y$, and the second is given by precomposition with the quotient map $X \to X / Y$.
\end{prop}



The \emph{weak topology} on $X$ the weakest topology on $X$ such that every functional $\mb{x}^* \in X^*$ is continuous.
The weak toplogy is weaker than the usual (norm) topology on $X$, and these topologies are equal if and only if $X$ is finite dimensional.

The \emph{double dual} of $X$ is the space $X^{**} = (X^*)^*$.
Each $\mb{x} \in X$ induces an element of the double dual in a notationally confusing way: for each $\mb{x}^* \in X^*$, $\mb{x}$ acts on $\mb{x}^*$ by
\begin{equation*}
  \langle \mb{x}^*, \mb{x} \rangle := \mb{x}^*(\mb{x}) = \langle \mb{x}, \mb{x}^* \rangle.
\end{equation*}
This identification yields a canonical isometric embedding $X \to X^{**}$, and $X$ is called \emph{reflexive} if this map is surjective (i.e. each functional on $X^*$ is given by an element of $X$, using the above identification).
In this case $X$ and $X^{**}$ are canonically isomorphic.\footnote{It is possible that $X$ and $X^{**}$ are isometrically isomorphic without $X$ being reflexive \cite{rJ51}.}

The \emph{weak-*} topology on a dual space $X^*$ is the weakest topology such that all the functions
\begin{equation*}
  \big\{ \mb{x}^* \mapsto \mb{x}^*(\mb{x}) : \mb{x} \in X\big\}
\end{equation*}
are continuous.
The weak-* topology is great because it has the following fundamental compactness property.

\cite[Theorem IV.21]{RS80}
\begin{thm}[Banach--Alaoglu]
  The closed unit ball of $X^*$ is compact in the weak-* topology.
\end{thm}

One can show that $X$ is reflexive if and only if the weak topology and the weak-* topology inherited from $X^{**}$ coincide.
Using Banach--Alaoglu, this can be restated in the following way.

\begin{cor}\label{cor:reflexive-iff-weakcpt}
  $X$ is reflexive if and only if the closed unit ball $B_{X}$ is weakly compact.
\end{cor}

Hahn--Banach can be used to prove the following density result the closed unit ball $\overline{B_{X}}$ in $\overline{B_{X^{**}}}$.
See \cite[Proposition B.1.17]{HNVW16}.

\begin{thm}[Goldstine]\label{thm:goldstine}
  The closed unit ball $\overline{B_{X}}$ of $X$ is weak-* dense in the closed unit ball $\overline{B_{X^{**}}}$ of $X^{**}$ (when viewing $X$ as a subset of $X^{**}$ by the canonical isometric embedding).
\end{thm}


There are three notions of compactness in topological spaces that coincide for metric spaces: \emph{compactness} (every open cover has a finite subcover), \emph{sequential compactness} (every sequence has a convergent subsequence), and \emph{countable compactness} (every sequence has a cluster point, or equivalently, every countable open cover has a finite subcover).
Although the weak topology on $X$ may not be metrisable, it behaves as if it were:

\begin{thm}[Eberlein--Smulian]\label{thm:eberlein-smulian}
  Let $A$ be a subset of a Banach space $X$.
  The following are equivalent:
  \begin{itemize}
  \item $A$ is weakly compact,
  \item $A$ is weakly sequentially compact,
  \item $A$ is weakly countably compact.
  \end{itemize}
\end{thm}

And thus Corollary \ref{cor:reflexive-iff-weakcpt} can be restated as follows:

\begin{cor}\label{cor:reflexive-bdd-subsequence}
  $X$ is reflexive if and only if every bounded sequence has a weakly convergent subsequence.
\end{cor}

A Banach space is \emph{separable} if it has a countable dense subset.\todo{add examples}
One can show via the Hahn--Banach theorem that if $X^*$ is separable, then so is $X$ \cite[Theorem III.7]{RS80}.
Separability can be characterised in terms of weak-* metrisability in the dual space.

\begin{thm}\label{thm:sep-met-dual}
  $X$ is separable if and only if the closed unit ball $\overline{B_{X^*}}$ of $X^*$ is weak-* metrisable.\todo{need a citation for this theorem}
  Thus by Banach--Alaoglu and the equivalence of compactness notions for metrisable spaces, if $X$ is separable, then $\overline{B_{X^*}}$ is weak-* sequentially compact.
\end{thm}

Given subsets $F \subset X$ and $Y \subset X^{*}$, one says that $Y$ \emph{separates points of $F$} if for every distinct pair of vectors $\mb{x} \neq \mb{y} \in F$ there exists a functional $\mb{x}^{*} \in Y$ such that $\langle \mb{x}, \mb{x}^{*} \rangle \neq \langle \mb{y}, \mb{x}^{*} \rangle$.
A topological argument (see \cite[Proposition B.1.11]{HNVW16}) implies the following.

\begin{prop}\label{prop:sep-sep-points}
  If $F$ is a separable subset of a Banach space $X$, and $Y \subset X^{*}$ is weak-* dense in $X^{*}$, then $Y$ contains a countable subset which separates points of $F$.
  In particular, if $X$ is separable, then there exists a sequence $(\mb{x}_{n}^{*})_{n \in \N}$ in $X^{*}$ which separates points of $X$.
\end{prop}

\section{Results from probability theory}\label{sec:probability}

This course uses \emph{basic} probabilistic concepts quite heavily, but very little of advanced probability theory or stochastic analysis.
Here we collect the basic concepts and results that we will need.
A good reference for probability theory from the viewpoint of mathematical analysis is \cite{rD04} (from which I have taken most of this material).
For a more simple introduction to basic probability, (CITE ROSS)\todo{add ref} is quite good.

Let $(\Omega,\mc{A})$ be a measurable space.
A \emph{probability measure} on $(\Omega,\mc{A})$ is a measure $\P$ with $\P(\Omega) = 1$.
Probabilists like to refer to measurable sets $A \in \mc{A}$ as \emph{events}.
If $S$ is a separable completely metrizable topological space (also called a \emph{Polish space}), an \emph{$S$-valued random variable} is a Borel measurable function $\map{X}{\Omega}{S}$.

\begin{rmk}
  In Definition \ref{defn:RV}, we define $X$-valued random variables $\map{\mb{f}}{\Omega}{X}$, where $X$ is a Banach space which is possibly not separable.
  However, we demand that our random variables are \emph{strongly} measurable, which by the Pettis measurability theorem (Theorem \ref{thm:Pettis-measurability}) implies that there is a separable closed subspace $X_{0} \subset X$ such that $\mb{f}$ is valued in $X_{0}$.
  Thus, by replacing $X$ with $X_{0}$, our Banach-valued random variables are valid random variables in the sense of the previous definition.
\end{rmk}

\begin{defn}
  Let $(\Omega,\mc{A},\P)$ be a probability space, let $\Lambda$ be an indexing set, and let $(E_{\lambda})_{\lambda \in \Lambda}$ be Polish spaces.
  \begin{itemize}
  \item
    A sequence of random variables $\map{\xi_{\lambda}}{\Omega}{E_{\lambda}}$ is called \emph{mutually independent} if for all finite subsets of indices $\{\lambda_{n}\}_{n=1}^{N}$ in $\Lambda$ and all Borel sets $B_{n} \subset E_{\lambda_{n}}$,
    \begin{equation*}
      \P\Big(\bigcap_{n=1}^{N} \{\omega \in \Omega : \xi_{\lambda_{n}}(\omega) \in B_{n}\} \Big) = \prod_{n=1}^{N} \P(\xi_{\lambda_{n}} \in B_{n}).
    \end{equation*}

  \item
    A sequence of $\sigma$-algebras $\mc{A}_{\lambda}$ is called \emph{mutually independent} if for all finite subsets of indices $\{\lambda_{n}\}_{n=1}^{N}$ in $\Lambda$ and all sets $A_{n} \in \mc{A}_{\lambda_{n}}$,
    \begin{equation*}
      \P\Big(\bigcap_{n=1}^{N} A_{n} \Big) = \prod_{n=1}^{N} \P(A_{n}).
    \end{equation*}

  \end{itemize}
  Note that the random variables $(\xi_{\lambda})_{\lambda \in \Lambda}$ are independent if and only if the $\sigma$-algebras $(\sigma(\xi_{\lambda}))_{\lambda \in \Lambda}$ are independent.
\end{defn}

\begin{itemize}
\item distributions
\item characteristic functions and uniqueness
\end{itemize}

\begin{thm}\label{thm:characteristic-function-uniqueness}
  (if two $X$-valued random variables have the same characteristic function then they are equal a.e.)
\end{thm}

\section{Complex interpolation}

\begin{itemize}
\item defn
\item basic structural results
\item (co)retraction theorem
\item examples 
\end{itemize}


%%% Local Variables:
%%% mode: latex
%%% TeX-master: "../main.tex"
%%% End:
