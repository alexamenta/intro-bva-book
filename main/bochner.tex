Rather than considering individual functions, one by one, it is smart to consider \emph{spaces} of functions with certain properties: smooth functions, continuous functions, integrable functions, and so on.
One of the fundamental classes of functions are the Lebesgue spaces, $L^p(S)$, associated with a measure space $S$.
For vector-valued functions, the analogue of Lebesgue spaces are the \emph{Bochner spaces}.
These are spaces of measurable vector-valued functions defined up to mofication on subsets of measure zero, just like Lebesgue spaces.
In order to define them we need to make precise what we mean by `measurability', as this turns out to be more complicated than usual in the vector-valued setting.

\section{Notions of measurability}

Consider a measurable space $(S,\mc{A})$,\footnote{i.e. $S$ is a set and $\mc{A}$ is a $\sigma$-algebra of subsets of $S$.} and let $X$ be a Banach space over the scalar field $\K$ (either $\R$ or $\C$).
The simplest kind of $X$-valued function arises by taking a \emph{scalar}-valued function $\map{f}{S}{\K}$ and a non-zero vector $\mb{x} \in X$, and `placing $f$ in the direction of $\mb{x}$'.
This function is denoted using the tensor notation $f \otimes \mb{x}$,\index{tensor product!of functions and vectors} and formally defined by
\begin{equation*}
  \map{f \otimes \mb{x}}{S}{X}, \qquad (f \otimes \mb{x})(s) := f(s)\mb{x} \quad \forall s \in S.
\end{equation*}
Note that the range of $f \otimes \mb{x}$ is contained in the linear span of $\mb{x}$.

The second simplest kind of $X$-valued function are the \emph{simple functions}.\index{simple functions}
A function $\map{\mb{g}}{S}{X}$ is \emph{simple} if there exists a finite collection of pairwise disjoint measurable subsets $S_1,\ldots,S_N \subset S$ and non-zero vectors $\mb{x}_1,\ldots,\mb{x}_N \in X$ such that
\begin{equation}\label{eqn:simple-function-standard-form}
  \mb{g} = \sum_{n=1}^N \1_{S_n} \otimes \mb{x}_n ,
\end{equation}
where $\1_{S_n}$ is the indicator function of $S_n$.
We denote the vector space of simple functions $S \to X$ by $\Simp(S;X)$ or $\Simp_{\mc{A}}(S;X)$.
Note that the range of $\mb{g}$ is contained in $\spn(\mb{x}_1,\ldots,\mb{x}_N)$, which is a finite dimensional subspace of $X$.

Of course, we need more than simple functions; we need \emph{measurable functions}.
When considering Banach-valued functions, particularly when our Banach spaces are allowed to be infinite dimensional, there is more than one notion of measurability, and these are generally inequivalent.

\begin{defn}
  Let $X$ be a Banach space.
  We say that a function $\map{\mb{f}}{S}{X}$ is
  \begin{itemize}
  \item
    \emph{measurable} if for every Borel set $B \subset X$, the preimage $\mb{f}^{-1}(B)$ is measurable;
  \item
    \emph{strongly measurable}\index{strong measurability} (or \emph{Bochner measurability})\index{Bochner measurability|see {strong measurability}} if it is the pointwise limit of simple functions; that is, if there exists a sequence $(\mb{f}_n)_{n=1}^\infty$ in $\Simp(S;X)$ such that $\mb{f} = \lim_{n \to \infty} \mb{f}_n$ pointwise on $S$;
  \item
    \emph{weakly measurable}\index{weak measurability} if for every functional $\mb{x}^* \in X^*$, the scalar-valued function $\map{\langle \mb{f}, \mb{x}^*\rangle}{S}{\K}$ given by $s \mapsto \langle \mb{f}(s), \mb{x}^* \rangle$ is measurable.
  \end{itemize}
  All of these notions implicitly refer to the $\sigma$-algebra $\mc{A}$.
\end{defn}

Using the notation
\begin{center}
  \begin{tabular}{r|l}
    $\Meas (S;X)$  & Measurable $\map{\mb{f}}{S}{X}$    \\
    $\SMeas(S;X)$  & Strongly measurable $\map{\mb{f}}{S}{X}$ \\
    $\WMeas(S;X)$  & Weakly measurable $\map{\mb{f}}{S}{X}$
  \end{tabular}
\end{center}
we have the containment
\begin{equation}\label{eq:measurability-inclusions}
  \SMeas(S;X) \subset \Meas(S;X) \subset \WMeas(S;X)
\end{equation}
(Exercise \ref{ex:measurability-containments}).
When $X$ is finite dimensional these notions coincide: the derivation of strong measurability from measurability is a standard result in measure theory,\footnote{See for example \cite[Corollary 4.2.7]{rD04} in the one-dimensional case; extending this to the finite dimensional case can be done by summing up coordinates.} and weak measurability is just a convoluted rewriting of coordinatewise measurability.
But in the general context of Banach spaces the inclusions \eqref{eq:measurability-inclusions} are strict.

\begin{example}[A measurable function which is not strongly measurable]\label{eg:non-sm}
  Let $X$ be a non-separable Banach space (for example, $X = L^\infty(\R)$), and consider the identity map $\map{I}{X}{X}$, which is continuous and hence measurable.
  We prove that $I$ is not strongly measurable by contradiction.
  Assuming $I$ is strongly measurable, we find that there exists a sequence of simple functions $(i_n)_{n \in \N}$ converging pointwise to $I$.
  For each $\mb{x} \in X$ we then have
  \begin{equation*}
    \mb{x} = \lim_{n \to \infty} i_n(\mb{x}),
  \end{equation*}
  so that the union $U := \cup_{n \in \N} i_n(X)$ is dense in $X$.
  Since each $i_n$ is simple, $U$ is countable, which implies that $X$ is separable.
  Thus $I$ is not strongly measurable.
\end{example}

\begin{example}[A weakly measurable function which is not measurable]\label{eg:non-meas}
  Consider the non-separable Hilbert space $\ell^2(\R)$ of all countably supported functions $\map{f}{\R}{\C}$ such that
  \begin{equation*}
    \|f\|_{\ell^{2}(\R)} := \Big( \sum_{t \in \spt f} |f(t)|^2 \Big)^{1/2} < \infty,
  \end{equation*}
  equipped with the inner product
  \begin{equation*}
    (f_1, f_2 ) := \sum_{t \in \spt f_1 \cap \spt f_2} f_1(t) \overline{f_2(t)}.
  \end{equation*} 
  For all $t \in \R$, let $\mb{e}_{t} \in \ell^2(\R)$ be the function which is equal to $1$ at $t$, and equal to $0$ everywhere else.
  Note that $\|\mb{e}_{t} - \mb{e}_{s}\|_{\ell^2(\R)} = \sqrt{2}$ whenever $s \neq t$.
  Now define a function $\map{\mb{F}}{\R}{\ell^2(\R)}$ by
  \begin{equation*}
    \mb{F}(t) := \mb{e}_{t}.
  \end{equation*}
  (We consider the domain $\R$ as being equipped with the Lebesgue measure).
  To show that $\mb{F}$ is weakly measurable we use the Riesz representation theorem: all functionals on $\ell^2(\R)$ are of the form $f \mapsto ( f, g )$ for some $g \in \ell^2(\R)$.
  Such a $g$ is countably supported, and can therefore be approximated in norm by finite linear combinations of vectors $\mb{e}_{t}$, and so weak measurability follows from the fact that the function
  \begin{equation*}
    s \mapsto ( \mb{F}(s), \mb{e}_{t} )  = (\mb{e}_{s}, \mb{e}_{t}) = \mb{e}_{t}(s)
  \end{equation*}
  is measurable for all $t \in \R$.
  On the other hand, $\mb{F}$ is not measurable: to see this, fix a non-measurable set $E \subset \R$ and let
  \begin{equation*}
    E' := \bigcup_{t \in E} B_{1}(\mb{e}_{t}).
  \end{equation*}
  Then $E'$ is open, but $\mb{F}^{-1}( E' ) = E$ is not measurable.
\end{example}

We used non-separability to construct a function which is measurable but not strongly measurable, and there is a very good reason for this.

\begin{thm}[Pettis measurability theorem]\label{thm:Pettis-measurability}\index{theorem!Pettis (measurability)}\index{strong measurability!relation with weak measurability}
  Let $(S,\mc{A})$ be a measurable space and $X$ a Banach space.
  Then a function $\map{\mb{f}}{S}{X}$ is strongly measurable if and only if it is weakly measurable and \emph{separably valued} (i.e. there exists a separable subspace $X' \subset X$ such that $\mb{f}(S) \subset X'$).
  In particular, if $X$ is separable, then
  \begin{equation*}
    \SMeas(S;X) = \Meas(S;X) = \WMeas(S;X).
  \end{equation*}
\end{thm}

\begin{proof}
  First suppose that $\mb{f}$ is strongly measurable.
  Then $\mb{f}$ is automatically weakly measurable, and we just need to show that it is separably valued.
  This essentially follows the argument from Example \ref{eg:non-sm}.
  Let $(\mb{f}_n)_{n \in \N}$ be a sequence of simple functions converging to $\mb{f}$ pointwise, and let $X_n \subset X$ denote the span of the range of $\mb{f}_n$.
  Each $X_n$ is finite dimensional, hence separable, and thus the closed subspace $X' \subset X$ generated by the collection $(X_n)_{n \in \N}$ is also separable.
  Since $\mb{f}_n \to \mb{f}$ pointwise, the range of $\mb{f}$ is contained in $X'$, so $\mb{f}$ is separably valued.

  Now assume that $\mb{f}$ is weakly measurable and separably valued.
  By replacing $X$ with the closure of the range of $\mb{f}$, we may assume without loss of generality that $X$ is separable.
  Let $(\mb{x}_n)_{n \in \N}$ be a dense sequence in $X$, and for each $n \in \N$ define a function $\map{\phi_n}{X}{\{\mb{x}_1,\ldots,\mb{x}_n\}}$ such that for all $\mb{x} \in X$,
  \begin{equation*}
    \|\mb{x} - \phi_n(\mb{x})\|_X = \min_{1 \leq j \leq n} \|\mb{x} - \mb{x}_j\|_X,
  \end{equation*}
  and such that if $\phi_{n}(\mb{x}) = \mb{x}_{k}$,
  \begin{equation*}
    \|\mb{x} - \phi_n(\mb{x})\|_X <  \|\mb{x} - \mb{x}_j\|_X \qquad \forall j = 1,\ldots,k-1
  \end{equation*}
  (i.e. $\phi_{n}(\mb{x})$ is the first element in the sequence $\mb{x}_{1}, \ldots, \mb{x}_{n}$ for which the distance to $\mb{x}$ is minimised).
  By density of $(\mb{x}_n)_{n \in \N}$ in $X$ we thus have that $\phi_n(\mb{x}) \to \mb{x}$ for all $\mb{x} \in X$.
  Now define functions $\map{\mb{f}_n}{S}{X}$ by
  \begin{equation*}
    \mb{f}_n(s) := \phi_n(\mb{f}(s)) \qquad \forall s \in S,
  \end{equation*}
  so that $\mb{f}_n \to \mb{f}$ pointwise.
  Each $\mb{f}_n$ has finite range, so to show that $\mb{f}_n$ is simple we need only show that the preimages $\mb{f}_n^{-1}(\mb{x}_k)$ are measurable.
  For all $n \in \N$ and all $1 \leq k \leq n$ we have
  \begin{equation*}
    \begin{aligned}
    \mb{f}_n^{-1}(\mb{x}_k)
    &= \{s \in S : \phi_n(\mb{f}(s)) = \mb{x}_k \} \\
    &= \{s \in S : \|\mb{f}(s) - \mb{x}_k\|_X = \min_{1 \leq j \leq n} \|\mb{f}(s) - \mb{x}_j\|_X \} \cap \\
    &\hspace{2cm} \bigcap_{j=1}^{k-1} \{s \in S : \|\mb{f}(s) - \mb{x}_{k}\|_{X} < \|\mb{f}(s) - \mb{x}_{j}\|_{X} \}.
  \end{aligned}
  \end{equation*}
  Let $(\mb{x}_m^*)_{m \in \N}$ be a norming sequence of unit vectors in $X^*$.
  Since $\mb{f}$ is weakly measurable, for each $j \in \{1,\ldots,n\}$ the function
  \begin{equation*}
    s \mapsto \|\mb{f}(s) - \mb{x}_j\|_X = \sup_{m \in \N} |\langle \mb{f}(s) - \mb{x}_j, \mb{x}_m^* \rangle|
  \end{equation*}
  is measurable (being the countable supremum of measurable functions).
  Thus the function
  \begin{equation*}
    s \mapsto \min_{1 \leq j \leq n} \|\mb{f}(s) - \mb{x}_j\|_X
  \end{equation*}
  is also measurable, and the representation above shows that $\mb{f}_n^{-1}(\mb{x}_k)$ is measurable (being constructed in terms of level sets and sub-level sets of measurable functions).
  Hence each $\mb{f}_n$ is simple, and consequently $\mb{f}$ is strongly measurable.
\end{proof}

The Pettis measurability theorem is incredibly useful, mostly because the conditions of weak measurability and separable-valuedness are easier to work with than the existence of pointwise approximating sequences of simple functions.
As a quick application, we can show that strong measurability is preserved under multiplication with measurable scalar-valued functions.\footnote{This can of course be proven using pointwise approximation with simple functions, but proof is not as clear.}

\begin{cor}\label{cor:strong-meas-meas-mult}
  Let $(S,\mc{A})$ be a measurable space and $X$ a Banach space.
  Suppose that $\map{\mb{f}}{S}{X}$ is strongly measurable and $\map{\phi}{S}{\K}$ is measurable.
  Then the pointwise product $\map{\phi \mb{f}}{S}{X}$ is strongly measurable.
\end{cor}

\begin{proof}
  By the Pettis measurability theorem, it suffices to show that $\phi \mb{f}$ is weakly measurable and separably-valued. 
  First we show weak measurability: for each functional $\mb{x}^* \in X^*$ write for $s \in S$
  \begin{equation*}
    \langle \phi \mb{f}, \mb{x}^* \rangle(s) = \phi(s) \langle \mb{f}(s) , \mb{x}^* \rangle = \phi \langle \mb{f}, \mb{x}^* \rangle.
  \end{equation*}
  Since $f$ is weakly measurable, the product $\phi \langle \mb{f}, \mb{x}^* \rangle$ is measurable for all $\mb{x}^* \in X^*$, so $\phi \mb{f}$ is weakly measurable.
  To show that $\phi \mb{f}$ is separably-valued, first note that since $f$ is separably-valued there exists a separable closed subspace $X' \subset X$ such that $\mb{f}(s) \in X'$ for all $s \in S$.
  But then $\phi(s)\mb{f}(s) \in X'$ too, so $\phi \mb{f}$ is separably-valued.
\end{proof}

In what follows we will generally deal with equivalence classes of functions modulo almost-everywhere equivalence, i.e. given a measure space $(S,\mc{A},\mu)$ we will consider two measurable functions $\map{\mb{f},\mb{g}}{S}{X}$ as being equal if the set
\begin{equation*}
  \{s \in S : \mb{f}(s) \neq \mb{g}(s)\}
\end{equation*}
has measure zero.
In this case we will write $\mb{f} \aeeq \mb{g}$.
For strongly measurable functions, almost-everywhere equality is equivalent to `weak' almost-everywhere equality.
This is a surprisingly useful observation, which can be used to deduce identities for vector-valued functions from corresponding identities for scalar-valued functions.

\begin{lem}\label{lem:coordinatewise-equality-test}
  Let $(S,\mc{A},\mu)$ be a measure space and $X$ a Banach space.
  Suppose that $\map{\mb{f},\mb{g}}{S}{X}$ are strongly measurable.
  Then $\mb{f} \aeeq \mb{g}$ if and only if for all functionals $\mb{x}^* \in X^*$, $\langle \mb{f}, \mb{x}^*\rangle \aeeq \langle \mb{g}, \mb{x}^* \rangle$.
\end{lem}

\begin{proof}
  The `only if' direction is straightforward, so we omit the proof.
  The `if' direction is harder: each of the sets
  \begin{equation*}
    N_{\mb{x}^*} := \{s \in S : \langle \mb{f}(s), \mb{x}^* \rangle \neq \langle \mb{g}(s), \mb{x}^* \rangle\} \qquad \mb{x}^* \in X^*
  \end{equation*}
  has measure zero, but the (uncountable!) union of these sets over all $\mb{x}^* \in X^*$ need not.
  This is where strong measurability comes into play, via the Pettis theorem.
  Since $\mb{f}$ and $\mb{g}$ are separably-valued, there exists a separable closed subspace $X' \subset X$ containing the ranges of both $\mb{f}$ and $\mb{g}$.
  Since $X'$ is separable, there is a (countable!) sequence $(\mb{x}_n^*)_{n \in \N}$ in $X^*$ which separates points of $X'$.\footnote{That is, if $\mb{x} \neq \mb{y} \in X'$, then there exists $n \in \N$ such that $\langle \mb{x}, \mb{x}_n^* \rangle \neq \langle \mb{y}, \mb{x}_n^* \rangle$. See Proposition \ref{prop:sep-sep-points} in the appendices.}
  Now define
  \begin{equation*}
    N := \bigcup_{n \in \N} N_{\mb{x}_n^*}.
  \end{equation*}
  This set has measure zero since it is the countable union of sets with measure zero.
  For all $s \notin N$ we then have $\langle \mb{f}(s), \mb{x}_n^* \rangle = \langle \mb{g}(s), \mb{x}_n^* \rangle$ for all $n \in \N$, and since $(\mb{x}_n^*)_{n \in \N}$ separates points of $X'$, it follows that $\mb{f}(s) = \mb{g}(s)$.
  Thus $\mb{f} \stackrel{\ae}{=} \mb{g}$.  
\end{proof}

\begin{rmk}\label{rmk:eqtest-cex}
  The weakly measurable function $\map{\mb{F}}{\R}{\ell^2(\R)}$ constructed in Example \ref{eg:non-meas} shows that Lemma \ref{lem:coordinatewise-equality-test} fails without the assumption of strong measurability: we have $\mb{F}(s) \neq 0$ for all $s \in \R$, but for all functionals $\mb{x}^* \in \ell^2(\R)^*$, $\langle \mb{F} , \mb{x}^{*} \rangle \aeeq 0$.
\end{rmk}

\section{Bochner spaces}\index{Bochner spaces}

We are ready to define Bochner spaces, which generalise Lebesgue spaces to Banach-valued functions.
Given a Banach-valued function $\map{\mb{f}}{S}{X}$, we let $\|\mb{f}\|_X$ denote the non-negative function on $S$ defined by $s \mapsto \|\mb{f}(s)\|_X$.
If $\mb{f}$ is measurable, then $\|\mb{f}\|_X$ is also measurable, since the function $\mb{x} \mapsto \|\mb{x}\|_X$ is continuous.
\textbf{From now on, we make the standing assumption that all measure spaces are $\sigma$-finite.} This is not necessary, but it avoids a few technicalities that I don't want to deal with.

\begin{defn}
  Let $(S,\mc{A},\mu)$ be a ($\sigma$-finite) measure space and $X$ a Banach space.
  For $p \in [1,\infty]$, we let $L^p(S,\mc{A},\mu;X)$ denote the set of \emph{strongly} $\mc{A}$-measurable functions $\mb{f} \in \SMeas(S;X)$ such that $\|\mb{f}\|_X \in L^p(S,\mc{A},\mu)$, and we write
  \begin{equation*}
    \|\mb{f}\|_{L^p(S,\mc{A},\mu;X)} := \| \|\mb{f}\|_X \|_{L^p(S,\mc{A},\mu)}.
  \end{equation*}
  We consider two functions $\mb{f}, \mb{g} \in L^p(S,\mc{A},\mu;X)$ to be equal if $\mb{f} \aeeq \mb{g}$.
  Each $L^p(S,\mc{A},\mu;X)$ is a Banach space: the proof is identical to the classic proof in the scalar-valued case.\footnote{For revision see \cite[Theorem 5.2.1]{rD04}.}
\end{defn}

\begin{rmk}
  If $\mb{f}$ is strongly measurable and $\mb{g} \aeeq \mb{f}$, then it does not automatically follow that $\mb{g}$ is strongly measurable, but $\mb{g}$ nevertheless has a strongly measurable representative ($\mb{f}$).
  To be more precise, we should say that $L^p(S;X)$ consists of \emph{$\mu$-almost everywhere strongly measurable} functions $\mb{f}$.
  I won't be too careful about this distinction.
  This is discussed at length, and properly, in \cite[Section 1.1.b]{HNVW16}.
\end{rmk}

In general we won't use all of $S$, $\mc{A}$, and $\mu$ in the notation for $L^p(S,\mc{A},\mu;X)$; we will only write out the parameters that need to be emphasised (whatever combination of the set, the $\sigma$-algebra, and the measure).
If the parameters are equally unimportant we may even omit all three (as in Remark \ref{rmk:lp-issues} below).
On the other hand, we will never omit $X$ unless $X = \K$ is the scalar field.

\begin{rmk}\label{rmk:lp-issues}
  It is possible for the scalar-valued function $\|\mb{f}\|_X$ to be in $L^p$ without $\mb{f}$ itself being strongly measurable (or even measurable).
  Such a function does not qualify for membership in $L^p(X)$.
  See Exercise \ref{ex:Lp-issues}.
\end{rmk}

In the scalar-valued setting, the simple functions are dense in $L^p$-spaces, and for $p < \infty$ the same holds for Bochner spaces. 

\begin{prop}\label{prop:simple-density}\index{simple functions!density in Bochner spaces}
  Let $(S,\mc{A},\mu)$ be a measure space and $X$ a Banach space.
  Then for $p \in [1,\infty)$, the subspace of simple functions $\Simp(S;X) \cap L^p(S;X)$ is dense in $L^p(S;X)$.
\end{prop}

\begin{proof}
  Fix $\mb{f} \in L^p(S;X)$.
  Since $\mb{f}$ is strongly measurable, there exists a sequence of simple functions $\mb{f}_n \in \Simp(S;X)$ with $\lim_{n \to \infty} \mb{f}_n = \mb{f}$ pointwise almost everywhere.
  Now set
  \begin{equation*}
    \mb{g}_n := \1_{ \{s \in S : \|\mb{f}_n(s)\|_X \leq 2\|\mb{f}(s)\|_X\} } \mb{f}_n.
  \end{equation*}
  The functions $\mb{g}_n$ are simple, and they converge to $\mb{f}$ pointwise almost everywhere.
  Furthermore we have
  \begin{equation*}
    \|\mb{g}_n\|_{L^p(S;X)}^p = \int_{\{s \in S : \|\mb{f}_n(s)\|_X \leq 2\|\mb{f}(s)\|_X \} } \|\mb{f}_n(s)\|_X^p \, \dd\mu(s) \leq 2^p \|\mb{f}\|_{L^p(S;X)}^p,
  \end{equation*}
  so each $\mb{g}_n$ is in $L^p(S;X)$.
  Since $\|\mb{f}(s) - \mb{g}_n(s)\|_X \leq 3\|\mb{f}(s)\|_X$ for almost all $s$, dominated convergence yields
  \begin{equation*}
    \begin{aligned}
      \lim_{n \to \infty} \|\mb{f} - \mb{g}_n\|_{L^p(S;X)}^p &= \lim_{n \to \infty} \int_S \|\mb{f}(s) - \mb{g}_n(s)\|_{X}^p \, \dd\mu(s) \\
      &= \int_S \lim_{n \to \infty}  \|\mb{f}(s) - \mb{g}_n(s)\|_{X}^p \, \dd\mu(s) = 0,
    \end{aligned}
  \end{equation*}
  so $\mb{g}_n \to \mb{f}$ in $L^p(S;X)$, completing the proof.
\end{proof}


\begin{rmk}
  Proposition \ref{prop:simple-density} can be extended to more general dense subspaces of $L^p(S)$; see Exercise \ref{ex:general-density}.
\end{rmk}

Note that the case $p = \infty$ is not included in the proposition above, even though the simple functions are dense in $L^\infty$.
This is because the density of simple functions in $L^\infty(S;X)$ (for a sufficiently rich measure space) is equivalent to the compactness of the unit ball of $X$, which is equivalent to the finite dimensionality of $X$.
We will prove this in the special case where $S = \N$, but the proof can be extended to any measure space containing infinitely many disjoint measurable sets of positive measure.

\begin{prop}\index{simple functions!are not dense in $\ell^\infty(\N;X)$}
  Let $X$ be a Banach space.
  Then the simple functions are dense in $\ell^\infty(\N;X)$ if and only if $X$ is finite dimensional.
\end{prop}

\begin{proof}
  First suppose $X$ is finite dimensional, and fix $\mb{f} \in \ell^\infty(\N;X)$ and $\varepsilon > 0$.
  Let $C = \|\mb{f}\|_{\ell^\infty(\N;X)}$.
  By compactness of the closed ball $\overline{B_C(0)} \subset X$, there exists a finite collection of vectors $(\mb{x}_i)_{i=1}^N$ in $\overline{B_C(0)}$ such that the open balls $B_{\varepsilon}(\mb{x}_i)$ cover $\overline{B_C(0)}$.
  For each $n \in \N$ we thus have that $\mb{f}(n) \in B_{\varepsilon}(\mb{x}_{i(n)})$ for some $i(n) \in \{1,\ldots,N\}$.
  Define a function $\map{\mb{f}_{\varepsilon}}{\N}{X}$ by
  \begin{equation*}
    \mb{f}_{\varepsilon}(n) := \mb{x}_{i(n)}.
  \end{equation*}
  Since the range of $\mb{f}_{\varepsilon}$ is finite, $\mb{f}_{\varepsilon}$ is simple.
  Furthermore, since $\mb{f}(n) \in B_{\varepsilon}(\mb{x}_{i(n)})$ for each $n \in \N$ we have
  \begin{equation*}
    \|\mb{f} - \mb{f}_{\varepsilon}\|_{\ell^\infty(\N;X)} = \sup_{n \in \N} \|\mb{f}(n) - \mb{x}_{i(n)}\|_{X} \leq \varepsilon.
  \end{equation*}
  Since $\varepsilon > 0$ was arbitrary, we have established density of the simple functions in $\ell^\infty(\N;X)$ when $X$ is finite dimensional.

  Now we prove the converse.
  Aiming for a contradiction, suppose that $X$ is infinite dimensional.
  Then there exists a sequence $(\mb{a}_n)_{n \in \N}$ of unit vectors in $X$ such that
  \begin{equation*}
    \|\mb{a}_n - \mb{a}_m\|_X > 1/2 \qquad \text{for all $n \neq m$.}
  \end{equation*}
  Now let $\mb{f}(n) = \mb{a}_n$ for all $n \in \N$, so that $\mb{f} \in \ell^\infty(\N;X)$, and suppose that there exists a simple function $\mb{g} \in \Simp(\N;X)$ with $\|\mb{f} - \mb{g}\|_{\ell^\infty(\N;X)} < 1/4$.
  Then for all $n \neq m$ we must have
  \begin{equation*}
    \begin{aligned}
      \|\mb{a}_n - \mb{a}_m\|_X &\leq \|\mb{f}(n) - \mb{g}(n)\|_X + \|\mb{g}(n) - \mb{g}(m)\|_X + \|\mb{g}(m) - \mb{f}(m)\|_X \\
      &\leq \frac{1}{2} + \|\mb{g}(n) - \mb{g}(m)\|_X,
    \end{aligned}
  \end{equation*}
  so that
  \begin{equation*}
    \|\mb{g}(n) - \mb{g}(m)\|_X \geq \|\mb{a}_n - \mb{a}_m\|_X - \frac{1}{2} > 0.
  \end{equation*}
  It follows that $\mb{g}$ has infinite range, contradicting the assumption that $\mb{g}$ is simple.
\end{proof}

Now we present some elementary duality results.\index{Bochner spaces!duality}
Fix an exponent $p \in [1,\infty]$, and recall the definition of the H\"older conjugate exponent $p' = p/(p-1)$, with $1' = \infty$ and $\infty' = 1$.
Given a measure space $(S,\mc{A},\mu)$ and a Banach space $X$, every function $\mb{g} \in L^{p'}(S;X^*)$ induces a bounded linear functional $\Phi \mb{g} \in L^p(S;X)^*$ by integration of the duality pairing between $X$ and $X^*$:
\begin{equation*}
  \Phi \mb{g}(\mb{f}) := \int_S \langle \mb{f}(s), \mb{g}(s) \rangle \, \dd\mu(s) \in \K \qquad \forall \mb{f} \in L^p(S;X).
\end{equation*}
H\"older's inequality implies that $\|\Phi \mb{g}\|_{L^p(S;X)^*} \leq \|\mb{g}\|_{L^{p'}(S;X^*)}$.
In the scalar case $X = \K$, for $p \in [1,\infty)$, $\Phi$ is an isometric isomorphism $L^{p'}(S) \cong L^p(S)^*$: that is, every functional $\phi \in L^p(S)^*$ is of the form $\phi = \Phi g$ for some $g \in L^{p'}(S)$, and furthermore $\|\phi\|_{L^p(S)^*} = \|g\|_{L^{p'}(S)}$.
We will see in Chapter \ref{sec:RNP} that for general Banach spaces $X$, $\Phi$ is an isometric isomorphism if and only if $X$ has the \emph{Radon--Nikodym property}\index{Radon--Nikodym property!and duality of Bochner spaces} with respect to the measure space $(S,\mc{A},\mu)$.
For now we will establish a duality result that holds with no additional assumptions on the Banach space.

\begin{prop}\label{prop:bochner-preduality}
  Let $(S,\mc{A},\mu)$ be a measure space and $X$ a Banach space.
  Then for all $1 \leq p \leq \infty$, the map $\map{\Phi}{L^{p'}(S;X^*)}{L^p(S;X)^*}$ is an isometry onto a closed subspace of $L^p(S;X)^*$ which is norming for $L^p(S;X)$: that is, for every $\mb{f} \in L^p(S;X)$,
  \begin{equation*}
    \|\mb{f}\|_{L^p(S;X)} = \sup_{\substack{\mb{g} \in L^{p'}(S;X^*) \\ \|\mb{g}\| = 1}} \Big| \int_S \langle \mb{f}(s), \mb{g}(s) \rangle \, \dd\mu(s) \Big|.
  \end{equation*}
\end{prop}

\begin{proof}
  To show that $\Phi$ is an isometry, it suffices to show that $\|\Phi \mb{g}\|_{L^p(S;X)^*} \geq 1$ whenever $\mb{g} \in L^{p'}(S;X^*)$ with $\|\mb{g}\|_{L^{p'}(S;X^*)} = 1$ (we've already discussed the reverse estimate).
  
  \textbf{Mild case: $p > 1$.}
  In this case we have $p' < \infty$, so by density of the simple functions in $L^{p'}(S;X^*)$ and continuity of $\Phi$ we may assume that $\mb{g}$ is simple, i.e.
  \begin{equation*}
    \mb{g} = \sum_{n=1}^N \1_{S_n} \otimes \mb{x}_n^*
  \end{equation*}
  for some pairwise disjoint sets $S_n \in \mc{A}$ with $\mu(S_n) < \infty$ and some nonzero vectors $\mb{x}_n^* \in X^*$.
  Let $\varepsilon > 0$, and choose unit vectors $\mb{x}_n \in X$ (depending on $\varepsilon$) such that
  \begin{equation*}
    \langle \mb{x}_n, \mb{x}_n^* \rangle \geq (1-\varepsilon)\|\mb{x}_n^*\|_{X^*} \qquad \forall n \in \{1,\ldots,N\}.
  \end{equation*}
  Use these vectors to define a test function
  \begin{equation*}
    \mb{f}_\varepsilon := \sum_{n=1}^N \1_{S_n} \otimes \|\mb{x}_n^*\|_{X^*}^{p' - 1} \mb{x}_n.
  \end{equation*}
  This function satisfies
  \begin{equation*}
    \begin{aligned}
      \|\mb{f}_\varepsilon\|_{L^p(S;X)}^p &= \sum_{n=1}^N \mu(S_n) \|\mb{x}_n^*\|_{X^*}^{p(p' - 1)} \|\mb{x}_n\|_{X}^p \\
      &= \sum_{n=1}^N \mu(S_n) \|\mb{x}_n^*\|_{X^*}^{p'}
      = \|\mb{g}\|_{L^{p'}(S;X^*)}^{p'} = 1
    \end{aligned}
  \end{equation*}
  as the $\mb{x}_n$ are unit vectors and $p(p' - 1) = 1$.
  Testing $\Phi \mb{g}$ against $\mb{f}_{\varepsilon}$ yields
  \begin{equation*}
    \begin{aligned}
      \Phi \mb{g}(\mb{f}_\varepsilon) = \sum_{n=1}^N \mu(S_n) \|\mb{x}_n^*\|_{X^*}^{p' - 1} \langle \mb{x}_n, \mb{x}_n^* \rangle
      &\geq (1-\varepsilon) \sum_{n=1}^N \mu(S_n) \|\mb{x}_n^*\|_{X^*}^{p'} \\
      &= (1-\varepsilon) \|\mb{g}\|_{L^{p'}(S;X^*)} = 1-\varepsilon.
    \end{aligned}
  \end{equation*}
  This proves that $\|\Phi \mb{g}\|_{L^p(S;X)^*} \geq 1-\varepsilon$.
  Since $\varepsilon > 0$ was arbitrary, we find that $\|\Phi \mb{g}\|_{L^p(S;X)^*} \geq 1$ as intended.

  \textbf{Spicy case: $p=1$.}
  Fix $\varepsilon > 0$ and define
  \begin{equation}
    A_{\varepsilon} := \{s \in S : \|\mb{g}(s)\|_{X^*} > 1-\varepsilon\},
  \end{equation}
  recalling that we assumed the normalisation $\|\mb{g}\|_{L^\infty(S;X^{*})} = 1$.
  Then $\mu(A_\varepsilon) > 0$, but we could run into the problem that $\mu(A_\varepsilon) = \infty$.
  Since $S$ is $\sigma$-finite,\footnote{Recall that we always assume this without mention.} we can write $S$ as an increasing union of sets of finite measure
  \begin{equation*}
    S = \bigcup_{n \in \N} S_n, \qquad S_n \subset S_{n+1}, \, \mu(S_n) < \infty \quad \forall n \in \N,
  \end{equation*}
  and thus for sufficiently large $n$ the set
  \begin{equation*}
    A_{\varepsilon}^{n} := \{s \in S_n : \|\mb{g}(s)\|_{X^*} > 1 - \varepsilon\}
  \end{equation*}
  satisfies $0 < \mu(A_\varepsilon) < \infty$.\footnote{The main issue is that $A_{\varepsilon}^{n}$ could have zero measure, but if this were true for all $n$, then $\mu(A_{\varepsilon}) = \sup_{n \in \N} \mu(A_{\varepsilon}^n) = 0$, which is a contradiction.}
  Let $B_\varepsilon = A_{\varepsilon}^{n}$ for such a large $n$.

  Since $\mb{g}$ is strongly measurable, the Pettis measurability theorem says that $\mb{g}(B_\varepsilon)$ is separable, and thus there exists a sequence $(\mb{x}^*_{k})_{k \in \N}$ in $X^*$ such that
  \begin{equation*}
    \mb{g}(B_\varepsilon) \subset \bigcup_{k \in \N} B_{\varepsilon}(\mb{x}^{*}_{k})
  \end{equation*}
  and thus
  \begin{equation*}
    B_{\varepsilon} \subset \bigcup_{k \in \N} \mb{g}^{-1}(B_{\varepsilon}(\mb{x}^{*}_{k})).
  \end{equation*}
  Since $\mu(B_{\varepsilon}) > 0$, there exists a vector $\mb{x}^* \in X^*$ (i.e. one of the vectors $\mb{x}_k^{*}$) such that the set
  \begin{equation*}
     B_{\varepsilon, \mb{x}^*} := B_\varepsilon \cap \mb{g}^{-1}(B_{\varepsilon}(\mb{x}^{*})) = \{s \in B_\varepsilon : \|\mb{g}(s) - \mb{x}^{*}\|_{X^*} < \varepsilon\}
  \end{equation*}
  has positive measure.
  Picking a point $s_0 \in B_{\varepsilon, \mb{x}^*}$ and using the definition of $B_{\varepsilon}$, we see that
  \begin{equation*}
    \|\mb{x}^*\|_{X^*} \geq \|\mb{g}(s_0)\|_{X^*} - \|\mb{g}(s_0) - \mb{x}^*\|_{X^*} > 1 - 2\varepsilon.
  \end{equation*}

  Now fix a unit vector $\mb{x} \in X$ such that $\langle \mb{x}, \mb{x}^* \rangle \geq \|\mb{x}^*\|_{X^*} - \varepsilon$, and consider the test function
  \begin{equation*}
    \mb{f}_{\varepsilon} := \1_{B_{\varepsilon, \mb{x}^*}} \otimes \mu(B_{\varepsilon, \mb{x}^*})^{-1} \mb{x}.
  \end{equation*}
  Then $\|\mb{f}_{\varepsilon}\|_{L^1(S;X)} = 1$, and
  \begin{equation*}
    \begin{aligned}
      |\Phi \mb{g}(\mb{f})| &=  \Big| \fint_{B_{\varepsilon, \mb{x}^*}} \langle \mb{x}, \mb{g}(s) \rangle \, \dd\mu(s) \Big| \\
      &\geq \Big| \fint_{B_{\varepsilon, \mb{x}^*}} \langle \mb{x}, \mb{x}^* \rangle \, \dd\mu(s) \Big| - \Big| \fint_{B_{\varepsilon, \mb{x}^*}} \langle \mb{x}, \mb{g}(s) - \mb{x} \rangle  \, \dd\mu(s) \Big| \\
      &\geq (\|\mb{x}^*\|_{X^*} - \varepsilon) - \varepsilon \geq 1 - 4\varepsilon.
    \end{aligned}
  \end{equation*}
  Since $\varepsilon > 0$ was arbitrary, we get $\|\Phi \mb{g}\|_{L^1(S;X)^*} \geq 1$, as we wanted.

  \textbf{Norming property:}
  This follows from the fact that $\Phi$ is an isometry, arguing via the double dual of $X$.
  Let $\map{j}{X}{X^{**}}$ denote the canonical isometric embedding of $X$ into its double dual.
  Then we can write
  \begin{equation*}
    \begin{aligned}
      \|\mb{f}\|_{L^p(S;X)} = \|j \circ \mb{f}\|_{L^p(S;(X^*)^*)} &= \|\Phi (j \circ \mb{f})\|_{L^{p'}(S; X^*)^*} \\
      &=  \sup_{\substack{\mb{g} \in L^{p'}(S;X^*) \\ \|\mb{g}\| = 1}} \Big| \int_{S} \langle \mb{g}(s) , j(\mb{f}(s)) \rangle \, \dd\mu(s) \Big| \\
      &= \sup_{\mb{g}} \Big| \int_{S} \langle \mb{f}(s) , \mb{g}(s) \rangle \, \dd\mu(s) \Big|,
    \end{aligned}
  \end{equation*}
  completing the proof.
\end{proof}

\section{The Bochner integral}

In the previous section we defined spaces of vector-valued functions satisfying integrability conditions.
It still remains to actually define \emph{integrals} of vector-valued functions.
In the introduction we gave a simple definition for finite dimensonal vector spaces by using basis expansions and linearity, but the world of infinite dimensional spaces is too complicated for such a simple method.
The key to our definition will be the density of the simple functions in $L^1(X)$: this will let us define our integral on simple functions (which are simple) and then extend by continuity.

Let $(S,\mc{A},\mu)$ be a measure space and $X$ a Banach space.
If $\mb{f} \in \Simp(S;X)$ is a simple function represented as
\begin{equation}\label{eq:simple-f-bi}
  \mb{f} = \sum_{n=1}^N \1_{S_n} \otimes \mb{x}_n ,
\end{equation}
and if in addition $\mb{f} \in L^1(\mu;X)$, we define the \emph{Bochner integral}\index{Bochner integral}
\begin{equation*}
  \int_S \mb{f} \, \dd\mu = \int_S \mb{f}(s) \, \dd \mu(s) := \sum_{n=1}^N \mu(S_n) \mb{x}_n \in X.
\end{equation*}
Note that a simple function as in \eqref{eq:simple-f-bi} is in $L^1(\mu;X)$ if and only if $\mu(S_n) < \infty$ for all $n$.
The Bochner integral is a linear map $\Simp(S;X) \cap L^1(\mu;X) \to X$, and for $\mb{f}$ as above it satisfies
\begin{equation*}
  \Big\| \int_S \mb{f} \, \dd \mu \Big\|_X \leq \sum_{n=1}^{N} |\mu(S_n)| \|\mb{x}_n\|_X = \|\mb{f}\|_{L^1(\mu;X)}.
\end{equation*}
Thus by density of $\Simp(S;X) \cap L^1(\mu;X)$ in $L^1(\mu;X)$ (Proposition \ref{prop:simple-density}), the Bochner integral extends to a bounded linear map $L^1(\mu;X) \to X$ which we continue to call the Bochner integral and denote by the same symbol.
That is, the Bochner integral of a general function $\mb{g} \in L^1(\mu;X)$ is given by 
\begin{equation*}
  \int_S \mb{g} \, \dd\mu := \lim_{n \to \infty} \int_S \mb{f}_n \, \dd\mu \in X
\end{equation*}
where $\mb{f}_n \in \Simp(S;X) \cap L^1(\mu;X)$ for all $n \in \N$ and $\mb{f}_n \to \mb{g}$ in $L^1(\mu;X)$.
We refer to functions in $L^1(\mu,X)$ as \emph{Bochner integrable}.

The Bochner integral satisfies various familiar and useful properties.
\begin{prop}\index{Bochner integral!basic properties}
  Let $(S,\mc{A},\mu)$ be a measure space and $X$ a Banach space.
  \begin{description}
  \item[Commutation with linear maps] If $\mb{f} \in L^1(\mu;X)$ and $T \in \Lin(X,Y)$ is a bounded linear map into a Banach space $Y$,
    \begin{equation*}
      T\Big( \int_S \mb{f} \, \dd\mu \Big) = \int_S T\mb{f} \, \dd\mu \in Y,
    \end{equation*}
    where $T\mb{f} \in L^1(\mu;Y)$ is given by $(T\mb{f})(s) = T(\mb{f}(s))$ for almost all $s \in S$.
    In particular, if $\mb{x}^* \in X^* = \Lin(X,\K)$, then
    \begin{equation*}
      \Big\langle \int_S \mb{f} \, \dd\mu, \mb{x}^* \Big\rangle = \int_S \langle \mb{f}(s), \mb{x}^* \rangle \, \dd\mu(s) \in \K.
    \end{equation*}
  \item[Closure] If $\mb{f} \in L^1(\mu;X)$ and $X_0$ is a closed subspace of $X$ such that $f(s) \in X_0$ for almost all $s \in S$, then $\int_S \mb{f} \, \dd\mu \in X_0$.
  \item[Dominated convergence] Let $(\mb{f}_n)_{n \in \N}$ be a sequence in $L^1(\mu;X)$, $\map{\mb{f}}{S}{X}$, and suppose that $\lim_{n \to \infty} \mb{f}_n \aeeq \mb{f}$.
    Suppose that there exists a non-negative $g \in L^1(\mu)$ such that $\|\mb{f}_n\|_X \leq g$ almost everywhere.
    Then $\mb{f} \in L^1(\mu;X)$ and
    \begin{equation*}
      \int_S \mb{f} \, \dd\mu = \lim_{n \to \infty} \int_S \mb{f}_n \, \dd\mu.
    \end{equation*}
  \item[Substitution / Change of Variables]
    Let $(S',\mc{A}')$ be a measurable space and $\map{\phi}{S}{S'}$ a measurable function, and let $\nu = \mu \circ \phi^{-1}$ denote the pushforward measure.
    Suppose $\mb{g} \in L^1(\nu;X)$.
    Then $\mb{g} \circ \phi \in L^1(\mu;X)$, and
    \begin{equation*}
      \int_S \mb{g} \circ \phi \, \dd\mu = \int_{S'} \mb{g} \, \dd\nu.
    \end{equation*}
    
  \end{description}
\end{prop}

\begin{proof}
  \textbf{Commutation with linear maps:} by continuity it suffices to prove this for simple $\mb{f} \in \Simp(S;X) \cap L^1(\mu;X)$.
    Writing $\mb{f}$ as in \eqref{eqn:simple-function-standard-form} we have 
    \begin{equation*}
      \begin{aligned}
        T\Big(\int_S \sum_{n=1}^N \1_{S_n} \otimes \mb{x}_n \, \dd\mu \Big)
        = \sum_{n=1}^N \mu(S_n) T(\mb{x}_n)
        = \int_{S} T\Big(\sum_{n=1}^{N} \1_{S_n}(s) \mb{x}_{n} \Big) \, \dd\mu(s) \\
        = \int_S T\mb{f} \, \dd\mu.
      \end{aligned}
    \end{equation*}
        
  \textbf{Closure:} We may assume that $X_0$ is a proper subspace of $X$, otherwise there is nothing to show.
    Let $\mb{y} \in X \sm X_0$, and by Hahn--Banach\footnote{If you are philosophically opposed to Hahn--Banach, then see \cite[Corollary 1.1.22]{HNVW16} for a proof that avoids it, and promptly stop reading these notes to avoid further frustration.} choose a functional $\mb{x}^* \in X^*$ such that $\langle \mb{y}, \mb{x}^* \rangle = 1$ and $X_0 \subset \ker \mb{x}^*$.
    Then by the commutation property above we have
    \begin{equation*}
      \Big\langle \int_S \mb{f} \, \dd\mu, \mb{x}^* \Big\rangle = \int_S \langle \mb{f}(s), \mb{x}^* \rangle \, \dd\mu(s) = 0
    \end{equation*}
    since $\mb{f}(s) \in X_0$ for almost all $s \in S$.
    Thus $\int_S \mb{f} \, \dd\mu \neq y$.
    Since $y \in X \sm X_0$ was arbitrary, we conclude that $\int_S \mb{f} \, \dd\mu \in X_0$.

  \textbf{Dominated convergence:}
    By continuity of the Bochner integral it suffices to show that $\mb{f} \in L^1(\mu;X)$ and $\mb{f}_n \to \mb{f}$ in $L^1(\mu;X)$.
    The first fact follows from $\|\|\mb{f}\|_{X}\|_{L^1(\mu)} \leq \|g\|_{L^1(\mu)} < \infty$ and the almost-everywhere strong measurability of almost-everywhere limits of strongly measurable functions (Exercise \ref{ex:sm-limits}).
    For the second, since $\|(\mb{f}_n - \mb{f})(s)\|_X \leq 2g(s)$ almost everywhere, we have
    \begin{equation*}
      \lim_{n \to \infty} \int_S \|(\mb{f}_n - \mb{f})(s)\|_X \, \dd\mu(s) = 0
    \end{equation*}
    by dominated convergence for scalar-valued functions.
    
  \textbf{Substitution:}
  First we need to show that $\mb{g} \circ \phi$ is strongly measurable.
  This follows from Pettis' theorem: $\mb{g}$ is separably valued, and therefore so is $\mb{g} \circ \phi$; likewise weak measurability of $\mb{g} \circ \phi$ follows from weak measurability of $\mb{g}$ and measurability of $\phi$.
    The identity for scalar-valued functions
    \begin{equation*}
      \int_{S} \|\mb{g} \circ \phi(s)\|_X \, \dd\mu(s) = \int_{S'} \|\mb{g}(s)\|_X \, \dd\nu(t)
    \end{equation*}
    implies that $\mb{g} \circ \phi \in L^1(\mu;X)$.
    Finally, for all $\mb{x}^* \in X^*$ we have by the commutation property and the substitution identity for scalar-valued functions
    \begin{equation*}
      \begin{aligned}
        \Big\langle \int_S \mb{g} \circ \phi \, \dd\mu , \mb{x}^* \Big\rangle
        = \int_S \langle \mb{g}(\phi(s)), \mb{x}^* \rangle \, \dd\mu(s)
        &= \int_{S'} \langle \mb{g}(s), \mb{x}^* \rangle \, \dd\nu(t) \\
        &= \Big\langle \int_{S'} \mb{g} \, \dd\nu, \mb{x}^* \Big\rangle,
      \end{aligned}
    \end{equation*}
    which proves the result.
\end{proof}

There is also a Fubini theorem for Banach-valued functions (but no Tonelli theorem, as we have no concept of a non-negative vector-valued function at this level of generality).\footnote{Recall that we assume every measure space is $\sigma$-finite. Fubini's theorem fails for non-$\sigma$-finite measure spaces, even for scalar-valued functions.}

\begin{prop}[Fubini]\index{theorem!Fubini}
  Let $(S,\mc{A},\mu)$ and $(S',\mc{A}',\mu')$ be measure spaces, and consider the product measure space $(S \times S', \mc{A} \times \mc{A}', \mu \times \mu')$.
  Let $\mb{f} \in L^1(S \times S'; X)$.
  Then
  \begin{itemize}
  \item for almost all $s \in S$ the function $\mb{f}(s,\cdot)$ is in $L^1(S';X)$,
  \item for almost all $s' \in S'$ the function $\mb{f}(\cdot,s')$ is in $L^1(S;X)$,
  \item the functions $\int_{S'} \mb{f}(\cdot,s') \, \dd\mu'(s')$ and $\int_{S} \mb{f}(s,\cdot) \, \dd\mu(s)$ are in $L^1(S;X)$ and $L^1(S';X)$ respectively, and
    \begin{equation}\label{eq:fubini}
      \int_{S \times S'} \mb{f} \, \dd(\mu \times \mu') = \int_{S'} \Big(  \int_S \mb{f}(s,s') \, \dd\mu(s) \Big) \, \dd\mu'(s') = \int_S \Big(\int_{S'} \mb{f}(s,s') \, \dd\mu'(s') \Big) \, \dd\mu(s).
    \end{equation}
  \end{itemize}
\end{prop}

\begin{proof}
  Consider an everywhere-defined representative of $\mb{f}$.
  Since $\mb{f}$ is strongly measurable, by the Pettis measurability theorem (Theorem \ref{thm:Pettis-measurability}), it is weakly measurable and separably valued.
  Thus the functions $\mb{f}(s,\cdot)$ and $\mb{f}(\cdot,s')$ are separably valued for all $s \in S$ and $s' \in S'$, and by the scalar-valued Fubini theorem, they are both weakly measurable.
  Thus $\mb{f}(s,\cdot)$ and $\mb{f}(\cdot,s')$ are strongly measurable.
  Now since the function $(s,s') \mapsto \|\mb{f}(s,s')\|_X$ is integrable, the scalar-valued Fubini theorem implies all of the integrability claims.
  The equalities \eqref{eq:fubini} are proven by scalarisation: for $\mb{x}^* \in X^*$ we have
  \begin{equation*}
    \begin{aligned}
      \Big\langle \int_{S \times S'} \mb{f} \, \dd(\mu \times \mu') , \mb{x}^* \Big\rangle
      &= \int_{S \times S'} \langle \mb{f}(s,s'), \mb{x}^* \rangle \, \dd(\mu \times \mu') \\
      &= \int_{S} \int_{S'} \langle \mb{f}(s,s'), \mb{x}^* \rangle \, \dd\mu'(s') \, \dd\mu(s) \\
      &= \int_{S} \Big\langle \int_{S'} \mb{f}(s,s') \, \dd\mu'(s') , \mb{x}^* \Big\rangle \, \dd\mu(s) \\
      &= \Big\langle \int_S \Big(\int_{S'} \mb{f}(s,s') \, \dd\mu'(s') \Big)\, \dd\mu(s) , \mb{x}^* \Big\rangle
  \end{aligned}
  \end{equation*}
  by the scalar-valued Fubini theorem, and likewise with the roles of $S$ and $S'$ reversed.
\end{proof}

Let's move away from the abstract stuff for a moment and define vector-valued Fourier transforms.
As we described in the introduction, these are defined just like scalar-valued Fourier transforms, but with Bochner integrals replacing Lebesgue integrals.

\begin{defn}\label{defn:FT}
  Let $X$ be a complex Banach space.
  For a Bochner integrable function $\mb{f} \in L^1(\R^d;X)$ we define the \emph{Fourier transform}\index{Fourier transform} as the Bochner integral
  \begin{equation*}
    \hat{\mb{f}}(\xi) = \mc{F}(\mb{f})(\xi) := \int_{\R^d} e^{-2\pi i t \cdot \xi} \mb{f}(t)  \, \dd t \in X \qquad \forall \xi \in \R^d. 
  \end{equation*}
  We also define the \emph{inverse Fourier transform} on $\mb{g} \in L^1(\R^d;X)$:
  \begin{equation*}
    \mb{g}^{\vee}(x) = \mc{F}^{-1}(\mb{g})(x) := \int_{\R^d} e^{2\pi i x \cdot \xi} \mb{g}(\xi) \, \dd \xi \in X \qquad \forall x \in \R^d.
  \end{equation*}
  For functions $\mb{f} \in L^1(\T^d ; X)$ on the $d$-torus $\T^d = [0,1]^d$, we use the same notation for the Fourier transform (and its inverse on $\mb{g} \in L^1(\Z^d; X)$)
  \begin{equation*}
    \begin{aligned}
      \hat{\mb{f}}(n) &= \mc{F}(\mb{f})(n) := \int_{\T^d} e^{-2\pi i t \cdot n} \mb{f}(t)  \, \dd t \in X \qquad \forall n \in \Z^d, \\
      \mb{g}^{\vee}(t) &= \mc{F}^{-1}(\mb{g})(t) := \sum_{n \in \Z^d} e^{2\pi i t \cdot n}  \mb{g}(n) \in X \qquad \forall t \in \T^d.
    \end{aligned}
  \end{equation*}
\end{defn}

Note that if $\mb{f} \in L^1(\R^d;X)$, then the function $x \mapsto \mb{f}(x)e^{-2\pi i x \cdot \xi}$ is Bochner integrable for each $\xi \in \R^d$ (see Lemma \ref{cor:strong-meas-meas-mult}), so the definitions above make sense.\footnote{Analogous statements hold for $\T^d$ and $\Z^d$.}
Furthermore
\begin{equation*}
  \|\hat{\mb{f}}(\xi)\|_X \leq \int_{\R^d} \| \mb{f}(x) e^{-2\pi i x \cdot \xi} \|_X \, \dd x = \|\mb{f}\|_{L^1(\R^d;X)}.
\end{equation*}
In fact, $\hat{\mb{f}}$ is strongly measurable, and so the Fourier transform is bounded from $L^1(\R^d;X)$ to $L^\infty(\R^d;X)$ (see Exercise \ref{ex:FT-bounded-1-infty}).
Formally, the Fourier transform and inverse Fourier transform are mutually inverse operators, but to make this statement rigourous we have to restrict to appropriate classes of functions or distributions, which for now we will not do.

\section{Extensions of operators to Bochner spaces}

In the introduction we showed that extending bounded operators on scalar-valued functions to bounded operators on vector-valued functions is a potentially difficult task, and depends strongly on the operators and the Banach spaces under consideration.
Before we can talk about the boundedness of such extensions we need to define the extensions themselves.
This can be done by basis expansions (particularly in the finite dimensional setting, where this suffices), but the `right' definition is through tensor extensions.

\begin{defn}
  For a measurable space $(S,\mc{A})$ and a set $V \subset \Meas(S;\K)$ of $\mc{A}$-measurable scalar-valued functions on $S$, we define the \emph{algebraic tensor product}\index{tensor product!algebraic}
  \begin{equation*}
    V \otimes X := \spn\{f \otimes \mb{x} : f \in V, \mb{x} \in X\} \subset \SMeas(S;X).
  \end{equation*}
  That is, $V \otimes X$ is the set of \emph{finite} linear combinations of $X$-valued functions of the form $f \otimes \mb{x}$, where $f$ is a scalar-valued function in $V$ and $\mb{x} \in X$.
  The function $f \otimes \mb{x}$ is called an \emph{elementary tensor}.\index{tensor product!of functions and vectors}
  Functions in $V \otimes X$, having finite dimensional range, are automatically strongly measurable.
\end{defn}

For example, when $V$ is the set of characteristic functions of measurable sets, $V \otimes X = \Simp(S;X)$ is the set of $X$-valued simple functions.
Another fundamental example is $V = L^p(S)$ for some $p \in [1,\infty]$.

\begin{prop}\label{prop:ATP-density}
  Let $(S,\mc{A},\mu)$ be a measure space, $X$ a Banach space, and $p \in [1,\infty)$.
  Then $L^p(S) \otimes X$ is a dense subspace of $L^p(S;X)$.
\end{prop}

\begin{proof}
  For $f \in L^p(S)$ and $\mb{x} \in X$ we compute
  \begin{equation*}
    \|f \otimes \mb{x}\|_{L^p(S;X)}^p = \int_S \|f(s)\mb{x}\|_{X}^p \, \dd\mu(s) = \|\mb{x}\|_X^p \|f\|_{L^p(S)}^p,
  \end{equation*}
  so that $f \otimes \mb{x} \in L^p(S;X)$.
  By linearity, this implies $L^p(S) \otimes X$ is contained in $L^p(S;X)$.
  For density, note that $L^p(S) \otimes X$ contains $(\Sigma(S;\K) \cap L^p(S)) \otimes X$, and that
  \begin{equation*}
    (\Sigma(S;\K) \cap L^p(S)) \otimes X = \Sigma(S;X) \cap L^p(S;X),
  \end{equation*}
  as both spaces are equal to the set of simple functions supported on sets of finite measure.
  By Proposition \ref{prop:simple-density}, this space is dense in $L^p(S;X)$, and thus the same is true of $L^p(S) \otimes X$.
\end{proof}

\begin{defn}\label{defn:tensor-exts}
  Let $(S_i,\mc{A}_i,\mu_i)$ ($i \in \{1,2\}$) be measure spaces, $p_1 \in [1,\infty)$ and $p_2 \in [1,\infty]$ (note that $p_{2} = \infty$ is allowed), and consider a bounded linear operator
  \begin{equation*}
    \map{T}{L^{p_1}(S_{1})}{L^{p_2}(S_{2})}
  \end{equation*}
  acting on scalar-valued functions.
  Let $X$ be a Banach space.
  The \emph{tensor extension of $T$ by the identity map $\map{I}{X}{X}$}\index{tensor extension} is the linear map between algebraic tensor products
  \begin{equation*}
    \map{T \otimes I}{L^{p_1}(S_1) \otimes X}{L^{p_2}(S_2) \otimes X}
  \end{equation*}
  satisfying $(T \otimes I)(f \otimes \mb{x}) = (Tf) \otimes \mb{x}$ for all $f \in L^{p_1}(S_1)$ and $\mb{x} \in X$.
\end{defn}

The tensor extension is a well-defined map between algebraic tensor products $L^{p_1}(S_1) \otimes X \to L^{p_2}(S_2) \otimes X$.
By Proposition \ref{prop:ATP-density}, $L^{p_1}(S_1) \otimes X$ is a dense subspace of $L^{p_1}(S_1;X)$, while $L^{p_2}(S_2) \otimes X$ is a subspace of $L^{p_2}(S_2;X)$ (possibly non-dense if $p_2 = \infty$), so if there exists $C < \infty$ such that
\begin{equation}\label{eq:tensor-ext-estimate}
  \|(T \otimes I)\mb{f}\|_{L^{p_2}(S_2;X)} \leq C \|\mb{f}\|_{L^{p_1}(S_1;X)} \qquad \forall \mb{f} \in L^{p_1}(S_1) \otimes X,
\end{equation}
then $T \otimes I$ may be extended to a bounded linear operator $L^{p_1}(S_1;X) \to L^{p_2}(S_2;X)$.

\begin{defn}\label{defn:X-val-extn}
  With the notation above, if the estimate \eqref{eq:tensor-ext-estimate} holds, we say that $T$ \emph{admits a bounded $X$-valued extension},\index{bounded Banach-valued extension} and we denote the continuous extension of $T \otimes I$ by $\td{T}_X$, $\td{T}$, or even just $T$.
\end{defn}

Writing out a general element $\mb{f} \in L^p(S) \otimes X$ as a linear combination of elementary tensors, we see that $T$ admits a bounded $X$-valued extension if and only if there exists a constant $C < \infty$ such that
\begin{equation}\label{eq:tensor-ext-estimate-full}
  \Big\|\sum_{n=1}^N (Tf_n) \otimes \mb{x}_n\Big\|_{L^{p_2}(S_2;X)} \leq C \Big\|\sum_{n=1}^N f_n \otimes \mb{x}_n\Big\|_{L^{p_1}(S_1;X)}
\end{equation}
for all functions $f_n \in L^{p_1}(S_1)$ and vectors $\mb{x}_n \in X$.
\emph{This estimate does not simply follow from boundedness of $T$.}
It turns out to rely on potentially subtle interactions between the operator $T$ and the Banach space $X$.

\begin{example}
  Fix a measure space $(S,\mc{A},\mu)$ and let $\map{T}{L^1(S)}{\K}$ denote the Lebesgue integral.\footnote{This fits in the scope of Definition \ref{defn:tensor-exts} by considering $\K$ as a Lebesgue space $L^1(\mathrm{pt})$ over a single point. Then $X$ may be identified with the Bochner space $L^1(\mathrm{pt};X)$.}
  Let $X$ be a Banach space.
  Then for all $\mb{f} \in \Simp(S;\K) \otimes X$ we have
  \begin{equation*}
    (T \otimes I)\mb{f} = (T \otimes I)\Big(\sum_{n=1}^N \1_{S_n} \otimes \mb{x}_n \Big) = \sum_{n=1}^N T(\1_{S_n}) \otimes \mb{x}_n = \sum_{n=1}^N \mu(S_n) \otimes \mb{x}_n = \int_S \mb{f} \, \dd\mu,
  \end{equation*}
  so that the tensor extension of the Lebesgue integral agrees with the Bochner integral, which we have already shown maps $L^1(S;X)$ to $X$.
  Thus the Lebesgue integral admits a bounded $X$-valued extension, namely the Bochner integral.
\end{example}

In the example of the Lebesgue integral the Banach space $X$ plays no real role; we will see in Theorem \ref{thm:positive-extensions} that this phenomenon occurs for all \emph{positive} operators.
Before that we record a simple observation: bounds for a tensor extension can be no better than bounds for the original operator.

\begin{prop}\label{prop:lb-ext}
  Fix measure spaces $(S_i,\mc{A}_i,\mu_i)$ ($i \in \{1,2\}$) and exponents $p_1 \in [1,\infty)$, $p_2 \in [1,\infty]$.
  Let $T \in \Lin(L^{p_1}(S),L^{p_2}(S))$ be a bounded linear operator, and let $X$ be a Banach space.
  Then the tensor extension $T \otimes I$ satisfies
  \begin{equation*}
    \|T \otimes I\|_{L^{p_1}(S_1;X) \to L^{p_2}(S_2;X)} \geq \|T\|_{L^{p_1}(S_1) \to L^{p_2}(S_2)}.
  \end{equation*}
\end{prop}

\begin{proof}
  Fix a nonzero vector $\mb{x} \in X$.
  Then for all nonzero $f \in L^{p_1}(S_1)$ we have
  \begin{equation*}
    \begin{aligned}
      \|(T \otimes I)(f \otimes \mb{x})\|_{L^{p_2}(S_2;X)} = \|Tf \otimes \mb{x}\|_{L^{p_2}(S_2;X)} &= \|Tf\|_{L^{p_2}(S_2)} \|\mb{x}\|_X \\
      &= \frac{\|Tf\|_{L^{p_2}(S_2)}}{\|f\|_{L^{p_1}(S_1)}} \|f \otimes \mb{x}\|_{L^{p_1}(S_1;X)}.
    \end{aligned}
  \end{equation*}
  Taking the supremum over all nonzero $f \in L^{p_1}(S_1)$ completes the proof.
\end{proof}

\begin{thm}\label{thm:positive-extensions}\index{tensor extension!of positive operators}
  Fix measure spaces $(S_i,\mc{A}_i,\mu_i)$ ($i \in \{1,2\}$), $p_1 \in [1,\infty)$, and $p_2 \in [1,\infty]$.
  Let $T \in \Lin(L^{p_1}(S_1),L^{p_2}(S_2))$ be a bounded linear operator which is \emph{positive}, i.e. for all non-negative $f \in L^{p_1}(S_1)$, $Tf \in L^{p_2}(S_2)$ is also non-negative.\footnote{When the scalar field $\K$ is $\C$, `non-negative' simply means `real-valued and non-negative'.}
  Then $T$ admits a bounded $X$-valued extension for every Banach space $X$: in fact, we have
  \begin{equation*}
    \|\td{T}\|_{L^{p_1}(S_1;X) \to L^{p_2}(S_2;X)} = \|T\|_{L^{p_1}(S_1) \to L^{p_2}(S_2)}
  \end{equation*}
  and the pointwise estimate
  \begin{equation}\label{eq:positive-pw-est}
    \|\td{T}\mb{f}\|_X \stackrel{\mathrm{a.e.}}{\leq} T(\|\mb{f}\|_X) \qquad \forall \mb{f} \in L^{p_1}(S_1;X).
  \end{equation}
  
\end{thm}

\begin{proof}
  To ease notation we will assume $(S_1, \mc{A}_1, \mu_1) = (S_2, \mc{A}_2, \mu_2)$ and $p_1 = p_2$, and omit the subscripts. The same proof holds in general.
  
  We will show the pointwise estimate in \eqref{eq:positive-pw-est} for all simple functions $\mb{f} \in \Simp(S;X) \cap L^{p}(S;X)$.
  This will imply
  \begin{equation*}
    \begin{aligned}
      \|\td{T}\mb{f}\|_{L^{p}(S;X)} &= \Big( \int_{S} \|\td{T}\mb{f}(s)\|_X^{p} \, \dd\mu(s) \Big)^{1/p} \\
      &\leq \Big( \int_{S} T(\|\mb{f}(s)\|_X)^{p} \, \dd\mu(s) \Big)^{1/p}\\
      &\leq \|T\|_{\Lin(L^{p}(S))} \Big( \int_{S} \|\mb{f}(s)\|_X^{p} \, \dd\mu(s) \Big)^{1/p} \\
      &= \|T\|_{\Lin(L^{p}(S))} \|\mb{f}\|_{L^{p}(S;X)}
    \end{aligned}
  \end{equation*}
  which implies the result by density of $\Simp(S;X) \cap L^{p}(S;X)$ in $L^{p}(S;X)$ (noting that the reverse estimate is shown in Proposition \ref{prop:lb-ext}).

  Now let's prove \eqref{eq:positive-pw-est}.
  Consider a simple function
  \begin{equation*}
    \mb{f} = \sum_{n=1}^N \1_{E_n} \otimes \mb{x}_n
  \end{equation*}
  and note that $|T(\1_{E_n})| \aeeq T(\1_{E_n})$ by positivity of $T$.
  Then
  \begin{equation*}
    \begin{aligned}
      \|\td{T}\mb{f}(s)\|_X &= \Big\| \sum_{n=1}^N T(\1_{E_n})(s) \mb{x}_n \Big\|_X \\
      &\leq \sum_{n=1}^N |T(\1_{E_n})(s)| \|\mb{x}_n\|_X \\
      &\aeeq \sum_{n=1}^N T(\1_{E_n})(s) \|\mb{x}_n\|_X 
      = T\Big( \sum_{n=1}^N \1_{E_n} \|\mb{x}_n\|_X \Big)(s) 
      = T(\|\mb{f}\|_X)(s),
    \end{aligned}
  \end{equation*}
  proving \eqref{eq:positive-pw-est} and completing the proof.
\end{proof}

Theorem \ref{thm:positive-extensions} shows that the `extension problem' for positive operators is not much of a problem: positive operators extend automatically.
Of course, most interesting operators are not positive.

\begin{example}\index{type!Fourier}
  The \emph{Hausdorff--Young inequality}\index{Hausdorff--Young inequality|see {Fourier type}} says that the Fourier transform $\mc{F}$ on scalar-valued functions is bounded from $L^p(\R)$ to $L^{p'}(\R)$ for all $p \in [1,2]$.\footnote{For $p=2$ this is Plancherel's theorem, and for $p=1$ it is straightforward, and as shown above it even holds for Banach-valued functions. The intermediate result can be proven by interpolation, e.g. by the Riesz--Thorin theorem.}
  Fix $p \in [1,2)$, and consider the Banach space $\ell^p := \ell^{p}(\N)$.
  We will show that for all $r \in (p,2]$, the bound
  \begin{equation}\label{eq:FT-bound-lp}
    \map{\mc{F}}{L^r(\R;\ell^p)}{L^{r'}(\R;\ell^p)}
  \end{equation}
  does not hold.
  Thus the Fourier transform \emph{from $L^r(\R)$ to $L^{r'}(\R)$} does not have a bounded extension to $\ell^p$ for any $1 \leq p < r$.\footnote{On the other hand, it has a bounded extension to $\ell^p$ for $r \leq p \leq 2$. What we are saying is that \emph{$\ell^p$ has Fourier type $r$ for all $r \in [1,p]$}: we will discuss this in more depth in Chapter \ref{sec:fouriertype}.}

  Fix a Schwartz function $f \in \mc{S}(\R)$ supported in the unit interval $(0,1)$, normalised such that $\|f\|_r = 1$.
  For each $N \in \{1,2,\ldots\}$ define a function $\map{\mb{f}_N}{\R}{\ell^p}$ by
  \begin{equation*}
      \mb{f}_N := \sum_{n=0}^{N-1} \Tr_{n}(f) \otimes \mb{e}_n
      = (f, \Tr_{1}(f), \Tr_{2}(f), \ldots, \Tr_{N-1}(f), 0, 0, \ldots).
  \end{equation*}
  where $(\mb{e}_{n})_{n \in \N}$ are the standard basis elements of $\ell^p$ and $\Tr_{n}f(x) = f(x-n)$ denotes the operator of translation by $n$.
  We can write
  \begin{equation*}
    \begin{aligned}
      \|\mb{f}_N\|_{L^r(\R;\ell^p)}^{r}
      &= \int_{\R} \|\mb{f}_N(x)\|_{\ell^p}^r \, \dd x  \\
      &= \sum_{m=0}^{N-1} \int_{m}^{m+1} \Big(\sum_{n=0}^{N-1} |f(x-n)|^p \Big)^{r/p} \, \dd x \\
      &= \sum_{m=0}^{N-1} \int_{m}^{m+1} |f(x-m)|^r \, \dd x 
      = \sum_{m=0}^{N-1} \int_0^1 |f(x)|^r \, \dd x = N
    \end{aligned}
  \end{equation*}
  using that for $x \in (m,m+1)$ we have $f(x-n) = 0$ unless $n=m$.
  Now using that the Fourier transform of a translated function is a modulation of the Fourier transform, for all $\xi \in \R$ we have
  \begin{equation*}
      \widehat{\mb{f}_N}(\xi) = \sum_{n=0}^{N-1} (\widehat{\Tr_{n}(f)} \otimes \mb{e}_n)(\xi) 
      = \sum_{n=0}^{N-1} e^{-in\xi} \hat{f}(\xi) \mb{e}_n 
      = \hat{f}(\xi) \sum_{n=0}^{N-1} e^{-in\xi} \mb{e}_n,
  \end{equation*}
  so
  \begin{equation*}
    \|\widehat{\mb{f}_N}\|_{L^{r^\prime}(\R;\ell^p)}^{r'}
    = \int_\R |\hat{f}(\xi)|^{r^\prime} \bigg( \sum_{n=0}^{N-1} |e^{-in\xi}|^p \bigg)^{r^\prime/p} \, d\xi 
    = N^{r'/p} \|\hat{f}\|_{L^{r^\prime}(\R)}^{r'}.
  \end{equation*}
  If the bound \eqref{eq:FT-bound-lp} holds, there is a constant $C$ (independent of $N$) such that
  \begin{equation*}
    \|\widehat{\mb{f}_N}\|_{L^{r^\prime}(\R;\ell^p)} \leq C \|\mb{f}_N\|_{L^r(\R;\ell^p)}
  \end{equation*}
  for all $N \geq 1$, but our estimates then imply
  \begin{equation*}
    N^{1/p} \|\hat{f}\|_{r^\prime} \leq C N^{1/r},
  \end{equation*}
  or equivalently $N^{\frac{1}{p} - \frac{1}{r}} \leq C/\|\hat{f}\|_{r^\prime}$ (using that $\|\hat{f}\|_{r^\prime} \neq 0$).
  But since $r > p$, the left hand side is unbounded, while the right hand side is constant.
  This contradiction shows that the bound \eqref{eq:FT-bound-lp} does not hold.
\end{example}


\section*{Exercises}

\begin{exercise}\label{ex:measurability-containments}
  Let $X$ be a Banach space and $(S,\mc{A})$ a measurable space.
  Prove the containments
  \begin{equation*}
    \SMeas(S;X) \subset \Meas(S;X) \subset \WMeas(S;X).
  \end{equation*}
\end{exercise}

\begin{exercise}\label{ex:continuous-Linfty}
  Let $X$ be a Banach space and let $A$ be a topological space.
  Let $C(A;X)$ denote the Banach space of bounded continuous functions from $A$ to $X$ (with the sup norm).
  \begin{itemize}
  \item If $X$ is separable or $A$ is separable, show that $C(A;X)$ is contained in $L^\infty(A, \mu;X)$ for every Borel measure $\mu$ on $A$. 
  \item Give an example of a topological space $A$, a Banach space $X$, and a Borel measure $\mu$ on $A$ such that $C(A;X)$ is not contained in $L^\infty(A,\mu;X)$.
  \item Given an example of non-separable $A$ and $X$ and a Borel measure $\mu$ such that $C(A;X)$ is contained in $L^\infty(A,\mu;X)$. 
  \end{itemize}
\end{exercise}

\begin{exercise}\label{ex:Lp-issues}
  Give an example of a Banach space $X$, a measure space $(S,\mc{A},\mu)$, and a function $\map{\mb{f}}{S}{X}$ such that $\|\mb{f}\|_{X} \in L^p(S,\mu)$, but $\mb{f} \notin L^p(S,\mu;X)$.
\end{exercise}

\begin{exercise}\label{ex:general-density}\index{tensor product!algebraic!density in Bochner spaces}
  Let $X$ be a Banach space and $(S,\mc{A},\mu)$ a measure space, and let $p \in [1,\infty)$.
  Let $V$ be a dense subspace of $L^p(S)$.
  Show that $V \otimes X$ is dense in $L^p(S;X)$.
\end{exercise}

\begin{exercise}\label{ex:finite-sigma-alg-duality}\index{Bochner spaces!duality!for finite $\sigma$-algebras}
  Let $X$ be a Banach space and $(S,\mc{A},\mu)$ a measure space such that the $\sigma$-algebra $\mc{A}$ is finite.
  Show that the isometric embedding
  \begin{equation*}
    \map{\Phi}{L^{p'}(S,\mc{A},\mu;X^*)}{L^p(S,\mc{A},\mu;X)^*}, \qquad \Phi \mb{g}(\mb{f}) = \int_S \langle \mb{f}(x), \mb{g}(x) \rangle \, \dd\mu(x)
  \end{equation*}
  is an isomorphism for all $p \in [1,\infty]$.
\end{exercise}

\begin{exercise}\label{ex:sm-limits}
  Let $(S,\mc{A},\mu)$ be a measure space and $X$ a Banach space.
  Let $(\mb{f}_{n})_{n \in \N}$ be a sequence of $X$-valued functions and $\map{\mb{f}}{S}{X}$, and suppose that $\mb{f}_{n} \to \mb{f}$ almost everywhere (with respect to $\mu$).
  Suppose that for each $n \in \N$ there exists a set $N_{n} \subset S$ with $\mu(N_{n}) = 0$ such that $\mb{f}_{n}$ is strongly measurable on $S \sm N_{n}$.
  Show that there exists a set $N \subset S$ with $\mu(N) = 0$ such that $\mb{f}$ is strongly measurable on $S \sm N$.
  (That is: show that the a.e. limit of a.e. strongly measurable functions is a.e. strongly measurable.)
\end{exercise}

\begin{exercise}\label{ex:tensor-extension-basic-props}
  Let $(S_i, \mc{A}_i, \mu_i)$ ($i \in \{1,2\}$) be measure spaces, let $p_1 \in [1,\infty)$, and let $p_2 \in [1,\infty]$.
  Suppose that $T \in \Lin(L^{p_1}(S_1), L^{p_2}(S_2))$ is a bounded linear operator.
  \begin{itemize}
  \item\index{bounded Banach-valued extension!for finite dimensional spaces}
    Show that $T$ admits a bounded $X$-valued extension for all finite dimensional Banach spaces $X$.
  \item\index{tensor extension!adjoints}
    Let $X$ be any Banach space and suppose $\mb{x}^* \in X^*$.
    Show that for all $f \in L^{p_1}(S_1) \otimes X$,
    \begin{equation*}
      \langle (T \otimes I)f, \mb{x}^* \rangle = T(\langle f, \mb{x}^* \rangle).
    \end{equation*}
  \end{itemize}

\end{exercise}

\begin{exercise}\label{ex:tensor-adjoint}\index{bounded Banach-valued extension!adjoints}
  Let $(S,\mc{A},\mu)$ be a measure space and $p \in (1,\infty)$, let $T$ be a bounded linear operator on $L^p(S,\mu)$, and let $X$ be a Banach space.
  Let $T^*  \in \Lin(L^{p'}(S,\mu))$ denote the adjoint of $T$.
  Show that $T$ admits a bounded $X$-valued extension if and only if $T^*$ admits a bounded $X^*$-valued extension, and show that
  \begin{equation*}
    (\td{T}_X)^* \Phi \mb{g} = \td{(T^*)}_{X^*} \mb{g}
  \end{equation*}
  for all $\mb{g} \in L^{p'}(S;X^*)$, where $\map{\Phi}{L^{p'}(S;X^*)}{L^p(S;X)^*}$ is as in Proposition \ref{prop:bochner-preduality}.
  Assuming $T$ admits a bounded $X$-valued extension, conclude that for all $\mb{f} \in L^p(S;X)$ and $\mb{g} \in L^{p'}(S;X)$,
  \begin{equation}\label{eq:tensor-adjoint-identity}
    \langle \td{T}_X \mb{f}, \mb{g} \rangle = \langle \mb{f}, \td{(T^*)}_{X^*} \mb{g} \rangle.
  \end{equation}
\end{exercise}

\begin{exercise}\label{ex:FT-bounded-1-infty}
  Let $X$ be a complex Banach space.
  Show that $\hat{\mb{f}} \in C(\R^d;X)$ for all $\mb{f} \in L^1(\R^d;X)$.
  Conclude that the Fourier transform is bounded from $L^1(\R^d;X)$ to $L^\infty(\R^d;X)$.
\end{exercise}

\begin{exercise}\index{Fourier multiplier!with operator-valued symbol}
  Let $X$ and $Y$ be Banach spaces and consider an operator-valued function $\map{M}{\R^d}{\Lin(X,Y)}$, where $\Lin(X,Y)$ is the Banach space of bounded linear operators from $X$ to $Y$.
  Suppose that $M$ is continuous with respect to the strong operator topology on $\Lin(X,Y)$: that is, suppose that for all vectors $\mb{x} \in X$, the map
  \begin{equation*}
    \map{M(\cdot)\mb{x}}{\R^d}{Y}, \qquad \xi \mapsto M(\xi)\mb{x}
  \end{equation*}
  is continuous.
  \begin{itemize}
  \item
    Let $\map{\mb{g}}{\R^d}{X}$ be strongly measurable.
    Show that the function $\map{M\mb{g}}{\R^d}{Y}$ defined by $(M\mb{g})(\xi) := M(\xi)\mb{g}(\xi)$ is strongly measurable.
  \item
    Suppose in addition that the function $\xi \mapsto \|M(\xi)\|_{\Lin(X,Y)}$ is measurable, and that
    \begin{equation*}
      \int_{\R^d} \|M(\xi)\|_{\Lin(X,Y)} \, \dd \xi < \infty.
    \end{equation*}
    Show that the operator $T_M\mb{f} := (M\hat{\mb{f}})^\vee$ is well-defined and bounded from $L^1(\R^d;X)$ to $C(\R^d;Y)$.
  \end{itemize}
\end{exercise}

\begin{exercise}\index{Fourier transform!on Hilbert-valued functions}
  Let $H$ be an infinite dimensional separable Hilbert space with inner product $(\cdot , \cdot)$, and let $(S,\mc{A},\mu)$ be a measure space.
  \begin{itemize}
  \item Show that $L^2(\mu;H)$ is a Hilbert space with respect to the inner product
    \begin{equation*}
      (\mb{f}, \mb{g}) := \int_{S} (\mb{f}(s), \mb{g}(s)) \, \dd\mu(s) \qquad (\mb{f},\mb{g} \in L^2(\mu;H)).
    \end{equation*}
  \item Let $(\mb{e}_n)_{n \in \N}$ be an orthonormal basis of $H$ and $(f_n)_{n \in \N}$ an orthonormal basis of $L^2(\mu)$.
    Show that the elementary tensors $\{f_n \otimes \mb{e}_m : n,m \in \N\}$ are an orthonormal basis of $L^2(\mu;H)$.
  \item Suppose that the Hilbert space $H$ is complex.
    Show that the Fourier transform on the torus, initially defined as a bounded operator $\map{\mc{F}}{L^2(\T^d;H)}{\ell^2(\Z^d;H)}$, extends to an isometry from $L^2(\T^d;H)$ to $\ell^2(\Z^d;H)$.
  \end{itemize}
\end{exercise}


%%% Local Variables:
%%% mode: latex
%%% TeX-master: "../main.tex"
%%% End:
