\subsection{Banach-valued functions and Bochner spaces}

Consider a measure space $(S,\mc{A},\mu)$ (so $\mc{A}$ is a $\sigma$-algebra of subsets of $S$, and $\mu$ is a measure on $\mc{A}$), and a Banach space $X$.
The topic of these notes is the analysis of functions $\map{f}{S}{X}$, and of operators acting on such functions.
Thus it will be useful for us to gain some familiarity with the idea of $X$-valued functions.

The simplest kind of $X$-valued function arises by taking a \emph{scalar}-valued function $\map{f}{S}{\C}$ and a vector $\mb{x} \in X$, and `placing $f$ in the direction of $\mb{x}$'.
This function is denoted by $f \otimes \mb{x}$ and formally defined by
\begin{equation*}
  \map{f \otimes \mb{x}}{S}{X}, \qquad (f \otimes \mb{x})(s) := f(s)\mb{x} \quad \text{for all $s \in S$.}
\end{equation*}
The range of $f \otimes \mb{x}$ is contained in the linear span of $\mb{x}$, and is thus `one-dimensional'.

The second simplest kind of $X$-valued function are the \emph{simple functions}.\footnote{Sorry.}
A function $\map{f}{S}{X}$ is \emph{($\mc{A}$-)simple} if there exists a finite collection of $\mc{A}$-measurable subsets $S_1,\ldots,S_N \subset S$ and vectors $\mb{x}_1,\ldots,\mb{x}_N \in X$ such that
\begin{equation*}
  f = \sum_{n=1}^N \1_{S_n} \otimes \mb{x}_n ,
\end{equation*}
where $\1_{S_n}$ is the indicator function (or characteristic function) of $S_n$.
We denote the vector space of $\mc{A}$-simple functions $S \to X$ by $\Simp_{\mc{A}}(S;X)$ (or by $\Simp(S;X)$ when the $\sigma$-algebra $\mc{A}$ is understood).
Note that the range of $f$ is contained in $\spn(\mb{x}_1,\ldots,\mb{x}_N)$, so $f$ can be thought of as `finite-dimensional'.

Of course, we need more than simple functions; measure theory tells us that the most useful class of functions are the \emph{measurable functions}.
When considering Banach-valued functions, particularly when our Banach spaces are allowed to be infinite-dimensional, there is more than one notion of measurability, and these are generally inequivalent.

\begin{defn}
  Consider a function $\map{f}{S}{X}$.
  We say that $f$ is
  \begin{itemize}
  \item
    \emph{($\mc{A}$-)measurable} if for every Borel set $B \subset X$, the preimage $f^{-1}(B)$ is in $\mc{A}$;
  \item
    \emph{strongly ($\mc{A}$-)measurable} if it is the pointwise limit of simple functions; that is, if there exists a sequence $(f_n)_{n=1}^\infty$ in $\Simp_{\mc{A}}(S;X)$ such that $f = \lim_{n \to \infty} f_n$ pointwise on $S$;
  \item
    \emph{weakly ($\mc{A}$)-measurable} if for every continuous linear functional $\mb{x}^* \in X^*$, the function
    \begin{equation*}
      \map{\langle f, \mb{x}^* \rangle}{S}{\C}, \qquad \langle f, \mb{x}^* \rangle(s) := \langle f(s), \mb{x}^* \rangle \quad \text{for all $s \in S$}
    \end{equation*}
    if measurable.
  \end{itemize}

  We use the following notation:
  \begin{center}
    \begin{tabular}{r|l}
      $\Meas (S;X)$  & Measurable $\map{f}{S}{X}$    \\
      $\SMeas(S;X)$  & Strongly measurable $\map{f}{S}{X}$    \\
      $\WMeas(S;X)$  & Weakly measurable $\map{f}{S}{X}$.    
    \end{tabular}
  \end{center}
  
\end{defn}


Simple functions are measurable, as $f^{-1}(B)$ is measurable for \emph{every} $B \subset X$.
Strongly measurable functions, being pointwise limits of measurable functions, are also measurable.
That is, we have the containments
\begin{equation*}
  \Simp(S;X) \subset \SMeas(S;X) \subset \Meas(S;X).
\end{equation*}

\todo{example showing that strong measurability is stronger than measurability for non-separable spaces?}

Given $\map{f}{S}{X}$, we abuse notation and let $\|f\|_X$ denote the function $S \to [0,\infty)$ defined by $s \mapsto \|f(s)\|_X$.
If $f$ is measurable, then $\|f\|_X$ is also measurable, since the function $\mb{x} \mapsto \|\mb{x}\|_X$ is continuous.

\begin{defn}
  For $p \in [1,\infty]$, we let $L^p(S,\mc{A},\mu;X)$ denote the set of \emph{strongly} $\mc{A}$-measurable functions $f \in \SMeas_{\mc{A}}(S;X)$ such that $\|f\|_X \in L^p(S,\mc{A},\mu)$, and we write
  \begin{equation*}
    \|f\|_{L^p(S,\mc{A},\mu;X)} := \| \|f\|_X \|_{L^p(S,\mc{A},\mu)}.
  \end{equation*}
  When the choice of $\sigma$-algebra $\mc{A}$ and measure $\mu$ is clear, we write $L^p(S;X)$ in place of $L^p(S,\mc{A},\mu;X)$.
\end{defn}

\begin{rmk}
  It is possible for $\|f\|_X$ to be in $L^p(S)$ without $f$ itself being strongly measurable (or even measurable).
  Such an $f$ does not qualify for membership in $L^p(S;X)$.
\end{rmk}

When $V \subset \Meas(S;\C)$ is a set of measurable scalar-valued functions on $S$, we define the \emph{algebraic tensor product}
\begin{equation*}
  V \otimes X := \spn\{f \otimes \mb{x} : f \in V, \mb{x} \in X\}.
\end{equation*}
That is, $V \otimes X$ is the set of finite linear combinations of $X$-valued functions of the form $f \otimes \mb{x}$, where $f$ is a scalar-valued function in $V$ and $\mb{x} \in X$.
For example, when $V$ is the set of characteristic functions of measurable sets, $V \otimes X = \Simp(S;X)$ is the set of $X$-valued simple functions.

\begin{prop}
  Let $p \in [1,\infty)$, and suppose that $V \subset L^p(S)$ is dense.
  Then $V \otimes X$ is dense in $L^p(S;X)$.
\end{prop}

\begin{proof}
  
\end{proof}

\begin{itemize}
\item defn of the integral for $f \in L^1(S;X)$
\item weak measurability; Pettis meas. theorem
\end{itemize}






\begin{itemize}
\item defn
\item density of simple functions on $L^p(S;X)$ for $p \in [1,\infty)$, non-density for $p = \infty$
\item Radon--Nikodym property, duality
\item tensor extensions + extension of positive operators
\end{itemize}


%%% Local Variables:
%%% mode: latex
%%% TeX-master: "../main.tex"
%%% End:
