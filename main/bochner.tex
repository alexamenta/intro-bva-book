\subsection{Strong measurability and Bochner spaces}

Consider a measurable space $(S,\mc{A})$,\footnote{i.e. $S$ is a set and $\mc{A}$ is a $\sigma$-algebra of subsets of $S$} and a Banach space $X$.
The topic of these notes is the analysis of functions $\map{f}{S}{X}$, and of operators acting on such functions.
Thus it will be useful for us to gain some familiarity with the idea of $X$-valued functions.

The simplest kind of $X$-valued function arises by taking a \emph{scalar}-valued function $\map{f}{S}{\C}$ and a non-zero vector $\mb{x} \in X$, and `placing $f$ in the direction of $\mb{x}$'.
This function is denoted by $f \otimes \mb{x}$ and formally defined by
\begin{equation*}
  \map{f \otimes \mb{x}}{S}{X}, \qquad (f \otimes \mb{x})(s) := f(s)\mb{x} \quad \text{for all $s \in S$.}
\end{equation*}
The range of $f \otimes \mb{x}$ is contained in the linear span of $\mb{x}$, and is thus `one-dimensional'.

The second simplest kind of $X$-valued function are the \emph{simple functions}.
A function $\map{f}{S}{X}$ is \emph{simple} if there exists a finite collection of pairwise disjoint measurable subsets $S_1,\ldots,S_N \subset S$ and non-zero vectors $\mb{x}_1,\ldots,\mb{x}_N \in X$ such that
\begin{equation*}
  f = \sum_{n=1}^N \1_{S_n} \otimes \mb{x}_n ,
\end{equation*}
where $\1_{S_n}$ is the indicator function of $S_n$.
We denote the vector space of simple functions $S \to X$ by $\Simp(S;X)$ or $\Simp_{\mc{A}}(S;X)$.
Note that the range of $f$ is contained in $\spn(\mb{x}_1,\ldots,\mb{x}_N)$, so $f$ can be thought of as `finite-dimensional'.

Of course, we need more than simple functions; measure theory tells us that the most useful class of functions are the \emph{measurable functions}.
When considering Banach-valued functions, particularly when our Banach spaces are allowed to be infinite-dimensional, there is more than one notion of measurability, and these are generally inequivalent.

\begin{defn}
  Consider a function $\map{f}{S}{X}$.
  We say that $f$ is
  \begin{itemize}
  \item
    \emph{measurable} if for every Borel set $B \subset X$, the preimage $f^{-1}(B)$ is measurable;
  \item
    \emph{strongly measurable} (or \emph{Bochner measurable}) if it is the pointwise limit of simple functions; that is, if there exists a sequence $(f_n)_{n=1}^\infty$ in $\Simp(S;X)$ such that $f = \lim_{n \to \infty} f_n$ pointwise on $S$;
  \end{itemize}
  Both of these notions implicitly refer to the $\sigma$-algebra $\mc{A}$.

  
\end{defn}

With the convenient notation
\begin{center}
  \begin{tabular}{r|l}
    $\Meas (S;X)$  & Measurable $\map{f}{S}{X}$    \\
    $\SMeas(S;X)$  & Strongly measurable $\map{f}{S}{X}$   
  \end{tabular}
\end{center}
we have the containment
\begin{equation}\label{eq:measurability-inclusions}
  \SMeas(S;X) \subset \Meas(S;X)
\end{equation}
When $X$ is finite-dimensional these notions coincide: the derivation of strong measurability from measurability is a standard result in measure theory.\footnote{See for example \cite[Corollary 4.2.7]{rD04} in the one-dimensional case; extending this to the finite-dimensional case can be done by summing up coordinates.} %mk
But in the general context of Banach spaces this inclusion \eqref{eq:measurability-inclusions} is strict.

\begin{example}[A measurable function which is not strongly measurable]
  Let $X$ be a Banach space, and consider the identity map $\map{I}{X}{X}$, which is continuous and hence measurable.
  If $I$ is strongly measurable, then there exists a sequence of simple functions $(i_n)_{n \in \N}$ converging pointwise to $I$.
  For each $\mb{x} \in X$ we then have
  \begin{equation*}
    \mb{x} = \lim_{n \to \infty} i_n(\mb{x}),
  \end{equation*}
  so that the union $U := \cup_{n \in \N} i_n(X)$ is dense in $X$.
  Since each $i_n$ is simple, $U$ is countable, which implies that $X$ is separable.
  Thus if $X$ is not separable (e.g. if $X = L^\infty(\R)$), the identity map $\map{I_X}{X}{X}$ is measurable (even continuous) but not strongly measurable.\footnote{In fact, if $X$ is separable, then $\SMeas(S;X) = \Meas(S;X)$; this is one part of the \emph{Pettis measurability theorem}, which we will not cover in these notes. See for example \cite[Theorem 1.1.6]{HNVW16}.}
\end{example}


Given $\map{f}{S}{X}$, we abuse notation and let $\|f\|_X$ denote the function $S \to [0,\infty)$ defined by $s \mapsto \|f(s)\|_X$.
If $f$ is measurable, then $\|f\|_X$ is also measurable, since the function $\mb{x} \mapsto \|\mb{x}\|_X$ is continuous.

\begin{defn}
  Let $(S,\mc{A},\mu)$ be a measure space.
  For $p \in [1,\infty]$, we let $L^p(S,\mu;X)$ denote the set of \emph{strongly} measurable functions $f \in \SMeas(S;X)$ such that $\|f\|_X \in L^p(S,\mu)$, modulo $\mu$-almost everywhere equality, and we write
  \begin{equation*}
    \|f\|_{L^p(S,\mu;X)} := \| \|f\|_X \|_{L^p(S,\mu)}.
  \end{equation*}
  Each $L^p(S,\mu;X)$ is a Banach space: the proof is identical to the classic proof in the scalar-valued case.\footnote{For revision see \cite[Theorem 5.2.1]{rD04}.}
\end{defn}

\begin{rmk}
  It is possible for $\|f\|_X$ to be in $L^p(S)$ without $f$ itself being strongly measurable (or even measurable).
  Such a function does not qualify for membership in $L^p(S;X)$.
\end{rmk}


\begin{prop}
  Let $X$ be a Banach space and $p \in [1,\infty)$.
  Then the subspace of simple functions $\Simp(S;X) \cap L^p(S;X)$ is dense in $L^p(S;X)$.
\end{prop}

\begin{proof}
  % L^infty, if needed
  % Let $f \in L^p(S;X)$ and consider the truncations
  % \begin{equation*}
  %   f_\lambda := \1_{\{s \in S : \|f_\lambda(s)\|_X \leq \lambda\}} f \qquad \forall \lambda > 0.
  % \end{equation*}
  % Then $\|(f - f_{\lambda})(s)\|_X^p \leq \|f(s)\|_X^p$ for almost all $s \in S$, so by dominated convergence
  % \begin{equation*}
  %   \begin{aligned}
  %     \lim_{\lambda \to \infty} \|f - f_{\lambda}\|_{L^p(S;X)}^p
  %     &= \lim_{\lambda \to \infty} \int_{S} \|(f - f_{\lambda})(s)\|_X^p \, \dd\mu(s) \\
  %     &= \int_{S} \lim_{\lambda \to \infty}  \|(f - f_{\lambda})(s)\|_X^p \, \dd\mu(s) = 0.
  %   \end{aligned}
  % \end{equation*}
  % Thus $f_{\lambda} \to f$ in $L^p(S;X)$ as $\lambda \to \infty$, and since each $f_{\lambda}$ is in $L^\infty(S;X) \cap L^p(S;X)$, thus subspace is dense in $L^p(S;X)$.

  Fix $f \in L^p(S;X)$.
  Since $f$ is strongly measurable, there exists a sequence of simple functions $f_n \in \Simp(S;X)$ with $\lim_{n \to \infty} f_n = f$ pointwise almost everywhere.
  Now set
  \begin{equation*}
    g_n := \1_{ \{s \in S : \|f_n(s)\|_X \leq 2\|f\|_X \} } f_n;
  \end{equation*}
  the functions $g_n$ are simple and they also converge to $f$ pointwise almost everywhere.
  Furthermore we have
  \begin{equation*}
    \|g_n\|_{L^p(S;X)}^p = \int_{\{s \in S : \|f_n(s)\|_X \leq 2\|f\|_X \} } \|f_n(s)\|_X^p \, \dd\mu(s) \leq 2^p \|f\|_{L^p(S;X)}^p,
  \end{equation*}
  so each $g_n$ is in $L^p(S;X)$.
  Since $\|f(s) - g_n(s)\|_X \leq 3\|f(s)\|_X$ for almost all $s$, dominated convergence yields
  \begin{equation*}
    \begin{aligned}
      \lim_{n \to \infty} \|f - g_n\|_{L^p(S;X)}^p &= \lim_{n \to \infty} \int_S \|f(s) - g_n(s)\|_{X}^p \, \dd\mu(s) \\
      &= \int_S \lim_{n \to \infty}  \|f(s) - g_n(s)\|_{X}^p \, \dd\mu(s) = 0,
    \end{aligned}
  \end{equation*}
  so that $g_n \to f$ in $L^p(S;X)$, completing the proof.
\end{proof}

Note that the case $p = \infty$ is not included in the proposition above, even though the simple functions are dense in $L^\infty(S;\C)$.

\begin{prop}
  Let $X$ be a Banach space.
  Then the simple functions are dense in $\ell^\infty(\N;X)$ if and only if $X$ is finite dimensional.\footnote{This proposition can be extended to more general measure space $S$ in place of $\N$, provided $S$ contains infinitely many disjoint measurable sets of positive measure.}
\end{prop}

\begin{proof}
  First suppose $X$ is finite dimensional, and fix $f \in \ell^\infty(\N;X)$ and $\varepsilon > 0$.
  Let $C = \|f\|_{\ell^\infty(\N;X)}$, and note that the closed ball $\overline{B_C(0)} \subset X$ is compact (this uses finite dimensionality of $X$).
  Thus there exists a finite collection of vectors $(\mb{x}_i)_{i=1}^N$ in $\overline{B_C(0)}$ such that the open balls $B_{\varepsilon}(\mb{x}_i)$ cover $\overline{B_C(0)}$.
  For each $n \in \N$ we thus have that $f(n) \in B_{\varepsilon}(\mb{x}_{i(n)})$ for some $i(n) \in \{1,\ldots,N\}$.
  Define a function $\map{f_{\varepsilon}}{\N}{X}$ by
  \begin{equation*}
    f_{\varepsilon}(n) := \mb{x}_{i(n)};
  \end{equation*}
  since the range of $f_{\varepsilon}$ is finite, $f_{\varepsilon}$ is simple.
  Furthermore since $f(n) \in B_{\varepsilon}(\mb{x}_{i(n)})$ for each $n \in \N$ we have
  \begin{equation*}
    \|f - f_{\varepsilon}\|_{\ell^\infty(\N;X)} = \sup_{n \in \N} \|f(n) - \mb{x}_{i(n)}\|_{X} \leq \varepsilon.
  \end{equation*}
  Since $\varepsilon > 0$ was arbitrary, we have established density of the simple functions in $\ell^\infty(\N;X)$ when $X$ is finite dimensional.

  Now we prove the converse.
  Aiming for a contradiction, suppose that $X$ is infinite dimensional.
  Then there exists a sequence $(a_n)_{n \in \N}$ in the unit ball $B_1(0) \subset X$ such that
  \begin{equation*}
    \|a_n - a_m\|_X > 1/2 \qquad \text{for all $n \neq m$.}
  \end{equation*}
  Now let $f(n) = a_n$ for all $n \in \N$, so that $f \in \ell^\infty(\N;X)$, and suppose that there exists a simple function $g \in \Simp(\N;X)$ with $\|f - g\|_{\ell^\infty(\N;X)} < 1/4$.
  Then for all $n \neq m$ we must have
  \begin{equation*}
    \begin{aligned}
      \|a_n - a_m\|_X &\leq \|f(n) - g(n)\|_X + \|g(n) - g(m)\|_X + \|g(m) - f(m)\|_X \\
      &\leq \frac{1}{2} + \|g(n) - g(m)\|_X,
    \end{aligned}
  \end{equation*}
  so that
  \begin{equation*}
    \|g(n) - g(m)\|_X \geq \|a_n - a_m\|_X - \frac{1}{2} > 0.
  \end{equation*}
  It follows that $g$ has infinite range, contradicting the assumption that $g$ is simple.
\end{proof}

\subsection{The Bochner integral}

\begin{itemize}
\item defn of the integral for $f \in L^1(S;X)$
\item evaluation
\item basic stuff re: linearity. don't dwell too long
\end{itemize}




\subsection{Duality and the Radon--Nikodym property}



\subsection{Extensions of operators to Bochner spaces}

{\color{blue} pasted from earlier - rewrite
When $V \subset \Meas(S;\C)$ is a set of measurable scalar-valued functions on $S$, we define the \emph{algebraic tensor product}
\begin{equation*}
  V \otimes X := \spn\{f \otimes \mb{x} : f \in V, \mb{x} \in X\}.
\end{equation*}
That is, $V \otimes X$ is the set of finite linear combinations of $X$-valued functions of the form $f \otimes \mb{x}$, where $f$ is a scalar-valued function in $V$ and $\mb{x} \in X$.
For example, when $V$ is the set of characteristic functions of measurable sets, $V \otimes X = \Simp(S;X)$ is the set of $X$-valued simple functions.
}


\begin{itemize}
\item defn of tensor extensions
\item result on positive operators
\item nonexample: Fourier transform. counterexamples
\end{itemize}



%%% Local Variables:
%%% mode: latex
%%% TeX-master: "../main.tex"
%%% End:
