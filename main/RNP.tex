We now move from Banach-valued analysis and probability to Banach-valued \emph{measure theory}, and finally to the \emph{geometry} of Banach spaces.
We will tie these concepts together via the Radon--Nikodym property, which is ostensibly a measure-theoretic property but has equivalent characterisations in terms of Bochner spaces, martingales, and convex sets.

\section{Vector measures and the Radon--Nikodym property}

\begin{defn}\index{vector measures}
  Let $X$ be a Banach space and $(S,\mc{A})$ a measurable space.
  An $X$-valued \emph{vector measure} is a function $\map{\mb{\mu}}{\mc{A}}{X}$ which is countably additive, in the sense that for all sequences $(E_n)_{n \in \N}$ of pairwise disjoint sets in $\mc{A}$,
  \begin{equation*}
    \mb{\mu}\Big( \bigcup_{n \in \N} E_n \Big) = \sum_{n \in \N} \mb{\mu}(E_n).
  \end{equation*}
  Note that this condition includes the convergence in $X$ of the series on the right hand side.
\end{defn}

Vector measures are just like measures, except the measure of a set $E \subset S$ is a vector $\mb{\mu}(E) \in X$ rather than a scalar.
We are most interested in vector measures with the following boundedness condition.

\begin{defn}\index{vector measures!variation}
  Let $X$ be a Banach space and $\mb{\mu}$ an $X$-valued vector measure on a measurable space $(S,\mc{A})$.
  The \emph{variation} of $\mb{\mu}$ is the scalar-valued measure $\map{|\mb{\mu}|}{\mc{A}}{[0,\infty]}$ defined by
  \begin{equation*}
    |\mb{\mu}|(E) := \sup_{\pi} \sum_{A \in \pi} \|\mb{\mu}(A)\|_X,
  \end{equation*}
  where the supremum ranges over all partitions $\pi$ of $S$ into $\mc{A}$-measurable sets.
  We define the \emph{total variation norm} $\|\mb{\mu}\|_{\var} := |\mb{\mu}|(S)$, and we say that $\mu$ has \emph{bounded variation} if $\|\mb{\mu}\|_{\var} < \infty$.
  Equivalently, $\mu$ has bounded variation if there exists a finite scalar-valued measure $\nu$ on $\mc{A}$ such that $\|\mb{\mu}(A)\|_X \leq \nu(A)$ for all $A \in \mc{A}$ (the minimal measure with this property is $|\mb{\mu}|$).
  We let $M(S,\mc{A};X)$ denote the Banach space of all $X$-valued vector measures $\mb{\nu}$ on $\mc{A}$ with bounded variation, under the total variation norm.
\end{defn}

It is not particularly difficult to define integrals of scalar-valued functions with respect to vector measures.

\begin{prop}
  Let $X$ be a Banach space and $\mb{\mu}$ an $X$-valued vector measure of bounded variation on a measurable space $(S,\mc{A})$.
  Then there is a unique continuous linear map $\map{[\mb{\mu}]}{L^1(S,\mc{A},|\mb{\mu}|)}{X}$ such that $[\mb{\mu}](\1_{A}) = \mb{\mu}(A)$ for all $A \in \mc{A}$.
  We use integral notation to denote this map, i.e. we write
  \begin{equation*}
    \int_{S} f(s) \, \dd\mb{\mu}(s) := [\mb{\mu}](f) \qquad \forall f \in L^1(S,\mc{A},|\mb{\mu}|).
  \end{equation*}
\end{prop}

\begin{proof}
  We skip the verification that the definition $[\mb{\mu}](\1_{A}) := \mb{\mu}(A)$ extends by linearity to a well-defined map on integrable simple functions.\footnote{``It is dreadfully boring to show that this formula defines a linear map... from the space of simple functions of the above form into $X$ and we leave this as an exercise for masochists.'' \cite[pp5-6]{DU77}}
  We just need to show boundedness, and the conclusion will follow by density.
  Consider a simple function $g \in L^1(S,\mc{A},|\mb{\mu}|)$ of the form
  \begin{equation*}
    g = \sum_{n=1}^{N} c_n \1_{S_n} 
  \end{equation*}
  with scalars $c_n \in \K$.
  Then
  \begin{equation*}
    \begin{aligned}
      \|[\mb{\mu}](g)\|_X \leq \sum_{n=1}^{N} |c_n| \|\mb{\mu}(S_n)\|_X \leq \sum_{n=1}^{N} |c_n| |\mb{\mu}|(S_n) = \|g\|_{L^1(|\mb{\mu}|)}.
    \end{aligned}
  \end{equation*}
  That's all.
\end{proof}

Fundamental examples of vector measures are given by integrating vector-valued functions against scalar measures.

\begin{example}\label{eg:RN-density}
  Let $(S,\mc{A})$ be a measurable space and $X$ a Banach space.
  Suppose $\nu$ is a finite scalar-valued measure on $(S,\mc{A})$ and $\mb{f} \in L^1(S,\mc{A},\nu;X)$.
  Then we can define an $X$-valued vector measure $\mb{\mu}$ (sometimes denoted $\mb{f}\nu$) by Bochner integration:
  \begin{equation*}
    \mb{\mu}(A) = \int_A \mb{f} \, \dd\nu.
  \end{equation*}
  This vector measure has bounded variation: given a partition $S = \bigcup_{n \in \N} S_n$, we compute
  \begin{equation*}
    \sum_{n \in \N} \|\mb{\mu}(S_n)\|_{X} = \sum_{n \in \N} \Big\| \int_{S_n} \mb{f} \, \dd\nu \Big\|_X
    \leq \int_{S} \|\mb{f}\|_{X} \, \dd\nu
  \end{equation*}
  so that $\|\mb{\mu}\|_{\var} \leq \|\mb{f}\|_{L^1(\nu;X)}$.\footnote{In fact, this is an equality. See \cite[pp43]{gP16}.}
\end{example}

Now let's revise some measure theory.
Recall that if $\mu$ and $\nu$ are two scalar-valued signed measures on a measurable space $(S,\mc{A})$, then \emph{$\nu$ is absolutely continuous with respect to $\mu$},\index{absolute continuity} written $\nu \ll \mu$, if $A \in \mc{A}$ and $\mu(A) = 0$ implies $\nu(A) = 0$.

\begin{thm}[Radon--Nikodym]\index{theorem!Radon--Nikodym}
  Let $(S,\mc{A})$ be a measurable space, and let $\mu$ be a $\sigma$-finite measure on $\mc{A}$.
  Let $\nu$ be a finite signed measure on $\mc{A}$ such that $\nu \ll \mu$.
  Then there exists a unique $h \in L^1(\mu)$ such that
  \begin{equation*}
    \nu(A) = \int_{A} h(s) \, \dd\mu(s) \qquad \forall A \in \mc{A}.
  \end{equation*}
  The function $h$ is called the \emph{Radon--Nikodym derivative}\index{Radon--Nikodym derivative} of $\nu$ with respect to $\mu$, and denoted by
  \begin{equation*}
    h = \frac{\dd\nu}{\dd\mu}.
  \end{equation*}
\end{thm}

See \cite[Theorem 5.5.4]{rD04} for a proof.
One might expect that an analogous theorem holds for vector measures, but it turns out to depend on the target Banach space, and thus its validity becomes a definition.

\begin{defn}\index{Radon--Nikodym property}\index{RNP|see {Radon--Nikodym property}}
  Let $(S,\mc{A},\mu)$ be a $\sigma$-finite measure space.
  A Banach space $X$ is said to have the \emph{Radon--Nikodym property (RNP) with respect to $(S,\mc{A},\mu)$} if for every $X$-valued vector measure $\mb{\nu}$ on $(S,\mc{A})$ such that $\|\mb{\nu}\|_{\var} < \infty$ and $|\mb{\nu}| \ll \mu$, there is a function $\mb{f} \in L^1(\mu;X)$ such that $\mb{\nu} = \mb{f}\mu$ (as defined in Example \ref{eg:RN-density}).
  We say $X$ has the \emph{Radon--Nikodym property} if it has the property above with respect to every $\sigma$-finite measure space $(S,\mc{A},\mu)$.
\end{defn}

The classical Radon--Nikodym theorem says that the scalar fields $\R$ and $\C$ have the RNP.
We will investigate this property for other Banach spaces by considering its relationship with martingales and with properties of convex sets.
We will also connect it with the duality of Bochner spaces $L^p(X)$, answering a question left open in Chapter \ref{sec:Bochner-spaces}.
Before moving on we record a simple reduction.

\begin{prop}\label{prop:RNP-finite-sufficient}
  A Banach space $X$ has the Radon--Nikodym property if and only if for every finite measure space $(S,\mc{A},\mu)$ and every $X$-valued vector measure $\mb{\nu}$ on $(S,\mc{A})$ such that $\|\mb{\nu}(A)\|_{X} \leq \mu(A)$ for every $A \in \mc{A}$, there is a function $\mb{f} \in L^1(\mu;X)$ such that $\mb{\nu} = \mb{f}\mu$.
\end{prop}

\begin{proof}
  If $X$ has the RNP and $\mu$ and $\mb{\nu}$ are as hypothesised, then we have in particular that
  \begin{equation*}
    \|\mb{\nu}\|_{\var} < \|\mu\|_{\var} = \mu(S) < \infty
  \end{equation*}
  and $|\mb{\nu}| \ll \mu$, so the RNP gives us the required function $\mb{f} \in L^1(\mu;X)$.

  Conversely, suppose that $X$ has the hypothesised property, and let $(S,\mc{A},\mu)$ be a $\sigma$-finite measure space.
  Let $\mb{\nu}$ be an $X$-valued vector measure of bounded variation on $(S,\mc{A})$ such that $|\mb{\nu}| \ll \mu$.
  The measure $|\mb{\nu}|$ is finite, and for all $A \in \mc{A}$ we have
  \begin{equation*}
    \|\mb{\nu}(A)\|_{X} \leq |\mb{\nu}|(A)
  \end{equation*}
  by definition.
  Thus by hypothesis there exists a function $\mb{f} \in L^1(|\mb{\nu}|;X)$ such that $\mb{\nu} = \mb{f}|\mb{\nu}|$.
  By the scalar Radon--Nikodym theorem, there also exists a function $g \in L^1(\mu)$ such that $|\mb{\nu}| = g\mu$.
  Since $|\mb{\nu}|$ is a non-negative measure, $g$ is non-negative.
  Thus we have $\mb{\nu} = (\mb{f}g)\mu$.
  It remains to show that $\mb{f}g \in L^1(\mu;X)$: this is established by
  \begin{equation*}
    \int_{S} \|\mb{f}g\|_{X} \, \dd\mu
    \leq \int_{S} \|\mb{f}(s)\|_{X} g(s) \, \dd\mu(s) = \|\mb{f}\|_{L^1(g\mu;X)} = \|\mb{f}\|_{L^1(|\mb{\nu}|)} < \infty. 
  \end{equation*}
\end{proof}

\section{The RNP and martingale convergence}

We establish the following connection between the Radon--Nikodym property and the martingale convergence properties.

\begin{thm}\label{thm:RNP-MCP}\index{Radon--Nikodym property!implies $1$-MCP}\index{Martingale Convergence Property!implied by RNP}
  Let $X$ be a Banach space which has the Radon--Nikodym property with respect to a probability space $(\Omega,\mc{A},\P)$.
  Then $X$ has the $1$-martingale convergence property with respect to $(\Omega,\mc{A},\P)$.
\end{thm}

Applications of the Radon--Nikodym property generally involve the construction of an appropriate vector measure, from which a magical function is extracted as a Radon--Nikodym derivative.
In the setting of Theorem \ref{thm:RNP-MCP}, we are given an $X$-valued martingale, and we construct its almost-everywhere limit as the Radon--Nikodym derivative of a certain vector measure.
In the following proposition we construct the vector measure.

\begin{prop}\label{prop:martingale-measure} 
  Let $(\Omega,\mc{A},\P)$ be a probability space and $X$ a Banach space.
  Let $\mb{f}_{\bullet}$ be an $X$-valued $L^1$-bounded uniformly integrable martingale with respect to a filtration $\mc{A}_{\bullet}$.
  Then there exists an $X$-valued vector measure $\mb{\mu}$ on $\mc{A}$ with the following properties:
  \begin{itemize}
  \item $\mb{\mu}(A) = \int_{A} \mb{f}_n \, \dd\P$ for all $n \in \N$ and $A \in \mc{A}_{n}$,
  \item $\|\mb{\mu}\|_{\var} \leq \sup_{n} \|\mb{f}_n\|_{L^1(\Omega;X)}$,
  \item $|\mb{\mu}|$ is absolutely continuous with respect to $\P$.
  \end{itemize}
\end{prop}

\begin{proof}
  For all $A \in \mc{A}$ we would like to define
  \begin{equation}\label{eq:mu-stationary-limit}
    \mb{\mu}(A) := \lim_{k \to \infty} \int_A \mb{f}_k \, \dd\P,
  \end{equation}
  but it is not immediate that this limit exists.
  If $n \in \N$ and $A \in \mc{A}_{n}$, then for $k \geq n$ we have
  \begin{equation*}
    \int_A \mb{f}_k \, \dd\P = \int_A \mb{f}_n \, \dd \P
  \end{equation*}
  by the martingale property, so at least for $A \in \mc{A}_{n}$ the limit exists and equals $\int_A \mb{f}_n \, \dd\P$, establishing the first desired property.
  For a general $A \in \mc{A}$, for each $k \in \N$ we have
  \begin{equation*}
     \int_A \mb{f}_k \, \dd\P = \E( \1_{A}\mb{f}_k) = \E(\E^{\mc{A}_k}( \1_{A} \mb{f}_k)) = \E(\E^{\mc{A}_k}(\1_{A}) \mb{f}_k ).
   \end{equation*}
   Thus to show that the limit in \eqref{eq:mu-stationary-limit} exists, we need to show that the sequence $(\E(\E^{\mc{A}_k}( \1_{A}) \mb{f}_k ))_{k \in \N}$ is Cauchy.
   Let $\phi_{k,\ell} = \E^{\mc{A}_k}(\1_{A}) - \E^{\mc{A}_\ell}( \1_{A})$.
   For $k < \ell$ we have, using conditional expectation magic,
   \begin{equation*}
     \begin{aligned}
       \E(\E^{\mc{A}_k}(\1_{A}) \mb{f}_k  ) - \E(\E^{\mc{A}_\ell}( \1_{A}) \mb{f}_\ell )
       &= \E \Big( \E^{\mc{A}_k}(\1_{A}) \mb{f}_k  - \E^{\mc{A}_\ell}( \1_{A}) \mb{f}_{\ell} \Big) \\
       &= \E \Big(  \E^{\mc{A}_{k}}(\E^{\mc{A}_k}(\1_{A})\mb{f}_\ell ) - \E^{\mc{A}_{k}}( \E^{\mc{A}_\ell}( \1_{A}) \mb{f}_{\ell} )\Big) \\
       &= \E \Big(  \E^{\mc{A}_k}(\1_{A}) \mb{f}_\ell  - \E^{\mc{A}_\ell}( \1_{A}) \mb{f}_{\ell}  \Big) 
       = \E ( \phi_{k,\ell} \mb{f}_\ell ).
     \end{aligned}
   \end{equation*}
   Thus we get
   \begin{equation*}
     \|\E(\E^{\mc{A}_k}(\1_{A}) \mb{f}_k ) - \E( \E^{\mc{A}_\ell}( \1_{A}) \mb{f}_\ell)\|_{X}
     \leq \|\phi_{k,\ell} \mb{f}_\ell \|_{L^1(\Omega;X)}.
   \end{equation*}
   Since $\|\phi_{k,\ell}\|_\infty \leq 2$ for all $k$ and $\ell$, for all $t > 0$ we have
   \begin{equation*}
     \begin{aligned}
       \|  \phi_{k,\ell}  \mb{f}_\ell \|_{L^1(\Omega;X)}
       &\leq \Big( \int_{\|\mb{f}_{\ell}\|_{X} > t} + \int_{\|\mb{f}_{\ell}\|_{X} \leq t} \Big) |\phi_{k,\ell}(\omega)| \|\mb{f}_{\ell}(\omega)\|_{X} \, \dd\P(\omega) \\
       &\leq  2\int_{\|\mb{f}_{\ell}\|_{X} > t} \|\mb{f}_{\ell}(\omega)\|_{X} \dd\P(\omega) + t\E|\phi_{k,\ell}| 
     \end{aligned}
   \end{equation*}
   so that
   \begin{equation*}
     \limsup_{k,\ell \to \infty} \|  \phi_{k,\ell} \mb{f}_\ell  \|_{L^1(\Omega;X)}
     \leq 2\limsup_{\ell \to \infty} \int_{\|\mb{f}_{\ell}\|_{X} > t} \|\mb{f}_{\ell}(\omega)\|_{X} \dd\P(\omega)
   \end{equation*}
   for all $t > 0$, using that $\limsup_{k,\ell \to \infty} \|\phi_{k,\ell}\|_{1} = 0$ (i.e. $(\E^{\mc{A}_{k}} \1_{A})_{k \in \N}$ is convergent in $L^1$).
   Taking the limit as $t \to \infty$, uniform integrability of $\mb{f}_{\bullet}$ says that
   \begin{equation*}
     \limsup_{k,\ell \to \infty} \|  \phi_{k,\ell} \mb{f}_\ell  \|_{L^1(\Omega;X)}
     \leq 2\lim_{t \to \infty} \sup_{\ell \in \N} \int_{\|\mb{f}_{\ell}\|_{X} > t} \|\mb{f}_{\ell}(\omega)\|_{X} \dd\P(\omega) = 0
   \end{equation*}
   (see Exercise \ref{ex:UI-characterisation}), which establishes that the limit in \eqref{eq:mu-stationary-limit} exists.

 We still need to show that $\mb{\mu}$ is actually a vector measure.
 It is clear from the definition that it is finitely additive, but we need \emph{countable} additivity.
 Consider the submartingale $(\|\mb{f}_n\|_{X})_{n \in \N}$: this is $L^1$-bounded and uniformly integrable, so by Theorem \ref{thm:submartingale-convergence} it has an $L^1$-limit $g \in L^1(\Omega)$.
 Thus for all $A \in \mc{A}$
 \begin{equation*}
   \|\mb{\mu}(A)\|_{X} \leq \lim_{n \to \infty} \int_{A} \|\mb{f}_n(\omega)\| \, \dd\P(\omega) = \int_{A} g(\omega) \, \dd\P(\omega).
 \end{equation*}
 By Exercise \ref{ex:fa-meas-ca}, this implies that $\mb{\mu}$ is countably additive with
 \begin{equation*}
   \|\mb{\mu}\|_{\var} \leq \|g\P\|_{\var} = \|g\|_{L^1(\Omega)} = \sup_{n \in \N} \|\mb{f}_n\|_{L^1(\Omega;X)},
 \end{equation*}
 and that
 \begin{equation*}
   |\mb{\mu}| \ll g\P \ll \P,
 \end{equation*}
 as required.
\end{proof}


\begin{proof}[Proof of Theorem \ref{thm:RNP-MCP}: RNP implies $1$-MCP]
  Let $\mb{f}_{\bullet}$ be an $L^1$-bounded uniformly integrable $X$-valued martingale with respect to a filtration $\mc{A}_{\bullet}$.
  By Proposition \ref{prop:martingale-measure}, there exists an $X$-valued vector measure $\mb{\mu}$ on $\mc{A}$ of bounded variation such that
  \begin{equation*}
    \mb{\mu}(A) = \int_{A} \mb{f}_n \, \dd\P \qquad \forall A \in \mc{A}_n
  \end{equation*}
  and $\mb{\mu} \ll \P$.
  Since $X$ has the RNP with respect to $(\Omega,\mc{A},\P)$, there exists a function $\mb{f} \in L^1(\Omega;X)$ such that
  \begin{equation*}
    \int_{A} \mb{f} \, \dd\P = \mb{\mu}(A) = \int_A \mb{f}_n \, \dd\P
  \end{equation*}
  for all $A \in \mc{A}_n$.
  Equivalently stated, we have
  \begin{equation*}
    \E^{\mc{A}_n} \mb{f} = \mb{f}_n
  \end{equation*}
  for all $n \in \N$, and thus by Theorem \ref{thm:mgale-pw-conv} $\mb{f}_n$ is almost everywhere convergent to $\E^{\mc{A}_{\infty}}\mb{f}$.
  Thus $X$ has the $1$-martingale convergence property, and the proof is complete.
\end{proof}

We already have some examples of spaces which do not satisfy the $1$-MCP: it follows that these spaces cannot have the RNP either.

\begin{cor}\index{Radon--Nikodym property!failure of $c_{0}$ and $L^1$}
  The spaces $c_0$ and $L^1([0,1])$ do not have the RNP.
\end{cor}

\begin{proof}
  In Examples \ref{eg:c0-noMCP} and \ref{eg:L1-noMCP} we showed that these spaces do not have the $\infty$-MCP, and hence they do not have the $1$-MCP. 
\end{proof} 

In summary, for all $p \in (1,\infty]$ we currently have the implications
\begin{equation*}
  \mathrm{RNP} \Longrightarrow 1-\mathrm{MCP} \Longrightarrow p-\mathrm{MCP} \Longrightarrow \infty-\mathrm{MCP}
\end{equation*}
where these properties are taken either universally or with respect to a given probability space.
In the next section we will add more properties and `complete the loop'.

\section{Trees and dentability}

To connect the Radon--Nikodym property with the geometry of Banach spaces, we will use the notion of a \emph{separated tree}.

\begin{defn}
  Let $(\Omega,\mc{A},\P)$ be a probability space and $X$ a Banach space.
  Given $\delta > 0$, an $X$-valued $L^1$-bounded martingale $\mb{f}_\bullet$ on $\Omega$ is called \emph{$\delta$-separated}\index{martingales!$\delta$-separated} if the following properties hold:
  \begin{itemize}
  \item $\mb{f}_0$ is constant,
  \item each $\mb{f}_n$ has finitely many values (i.e. $\mb{f}_{n}$ is simple),
  \item for all $n \in \N$ and $\omega \in \Omega$, $\|\mb{f}_n(\omega) - \mb{f}_{n+1}(\omega)\|_X \geq \delta$.
  \end{itemize}
  The set of values $S := \{\mb{f}_n(\omega) : n \in \N, \omega \in \Omega\}$ is called a \emph{$\delta$-separated tree}.  \index{separated trees}
\end{defn}

The martingales described in Examples \ref{eg:c0-noMCP} and \ref{eg:L1-noMCP} (see also Exercise \ref{ex:L1-noMCP-var}) are $1$-separated, and thus yield $1$-separated trees in $c_0$ and $L_1$.
But when one tries to draw a $\delta$-separated tree on a piece of paper, one quickly starts to run out of space.
This is because pieces of paper model finite dimensional Banach spaces, which have good martingale convergence properties.

\begin{prop}\index{separated trees!relation with MCP} 
  If a Banach space $X$ has the $\infty$-MCP, then for all $\delta > 0$, $X$ does not contain a bounded $\delta$-separated tree.
\end{prop}

\begin{proof}
  Bounded $\delta$-separated trees correspond to $L^\infty$-bounded martingales $\mb{f}_{\bullet}$ such that
  \begin{equation*}
    \|\mb{f}_n(\omega) - \mb{f}_{n+1}(\omega)\|_{X} \geq \delta
  \end{equation*}
  for all $\omega \in \Omega$ and $n \in \N$,
  which directly obstructs convergence of $\mb{f}_\bullet$ everywhere in $\Omega$.
\end{proof}

Thus our chain of implications now has a new member:
\begin{equation*}
  \mathrm{RNP} \Longrightarrow 1-\mathrm{MCP} \Longrightarrow p-\mathrm{MCP} \Longrightarrow \infty-\mathrm{MCP}
  \Longrightarrow \mathrm{NBST}
\end{equation*}
where NBST stands for `no bounded separated trees'.\footnote{As with MCP, this is not standard terminology: ultimately it's just equivalent to RNP.}
The connection between (nonexistence of) bounded separated trees and the Radon--Nikodym property will go through the concept of \emph{dentable sets}.

\begin{defn}\index{dentable sets}
  A subset $D \subset X$ of a Banach space $X$ is called \emph{dentable} if for all $\varepsilon > 0$ there exists $\mb{x} \in D$ such that
  \begin{equation*}
    \mb{x} \notin \overline{\conv}(D \sm B_{\varepsilon}(\mb{x}))
  \end{equation*}
  where $\overline{\conv}$ denotes the closure of the convex hull.
\end{defn}

An example of a dentable set in $\R^{2}$ is shown in Figure \ref{fig:dentable}.
It is impossible to draw a bounded non-dentable set due to Theorem \ref{thm:dent-RNP}. 

\begin{figure}
    \centering
    \def\svgwidth{\columnwidth}
    \input{dentable-set.pdf_tex}
    
    \caption{A dentable set $D$ in $\R^{2}$, being `dented' at the point $\mb{x} \in D$.}
    \label{fig:dentable}
\end{figure}

Before going further we'll need a lemma which relates non-dentability at scale $\varepsilon$ to a corresponding property of an enlarged set which does not involve closures.
This will let us work directly with convex hulls rather than their closures.

\begin{lem}\label{lem:dentlem}
  Let $X$ be a Banach space.
  Fix $\varepsilon > 0$ and let $D \subset X$ be a subset such that for all $\mb{x} \in D$,
  \begin{equation}\label{eq:dentlem-ass}
    \mb{x} \in \overline{\conv}(D \sm B_{\varepsilon}(\mb{x})).
  \end{equation}
  Then for all $\mb{x} \in \tilde{D} := D + B_{\varepsilon/2}(\mb{0})$,
  \begin{equation}
    \mb{x} \in \conv(\tilde{D} \sm B_{\varepsilon/2}(\mb{x}))
  \end{equation}
  (note that no closure is taken here).
\end{lem}

\begin{proof}
  Fix $\mb{x} = \mb{x}' + \mb{y} \in \tilde{D}$, where $\mb{x}' \in D$ and $\|\mb{y}\|_{X} < \varepsilon/2$.
  Choose $\delta > 0$ so small that $\delta + \|\mb{y}\|_{X} < \varepsilon/2$.
  By \eqref{eq:dentlem-ass} we have $\mb{x}' \in \overline{\conv}(D \sm B_{\varepsilon}(\mb{x}'))$, so there exists $n \in \N$, scalars $\alpha_i \in [0,1]$ and vectors $\mb{x}_1,\ldots,\mb{x}_n \in D \sm B_{\varepsilon}(\mb{x}')$, $i = 1,\ldots,n$, with $\sum_{i} \alpha_{i} = 1$, such that
  \begin{equation*}
    \mb{x}' = \mb{z} + \sum_{i=1}^{n} \alpha_i \mb{x}_{i}
  \end{equation*}
  for some $\mb{z} \in B_{\delta}(\mb{0})$.
  We then have
  \begin{equation*}
    \mb{x} = \mb{z} + \mb{y} + \sum_{i=1}^{n} \alpha_i \mb{x}_{i} = \sum_{i=1}^{n}\alpha_i(\mb{z} + \mb{y} + \mb{x}_{i}).
  \end{equation*}
  The points $\mb{z} + \mb{y} + \mb{x}_{i}$ are in $\tilde{D} \sm B_{\varepsilon/2}(\mb{x})$: indeed, we have
  \begin{equation*}
    \|\mb{z} + \mb{y}\|_{X} \leq \delta + \|\mb{y}\|_{X} < \varepsilon/2
  \end{equation*}
  by the choice of $\delta$, and
  \begin{equation*}
    \|\mb{x} - (\mb{z} + \mb{y} + \mb{x}_i) \|_{X}
    = \|\mb{x}' - \mb{z} - \mb{x}_i\|_{X}
    \geq \|\mb{x}' - \mb{x}_i\|_{X} - \|\mb{z}\|_{X}
    > \varepsilon - \delta > \varepsilon/2,
  \end{equation*}
  finishing the job.
\end{proof}

The concept of dentability is connected with (non-existence of) bounded separated trees as follows.

\begin{thm}\index{dentable sets!relation with separated trees}\index{separated trees!relation with dentable sets}
  Let $X$ be a Banach space, and suppose that for all $\delta > 0$, $X$ does not contain a bounded $\delta$-separated tree.
  Then every bounded subset of $X$ is dentable.
\end{thm}
  
\begin{proof}
  We prove the contrapositive: we suppose that there exists a bounded non-dentable set $D \subset X$, and given $\delta > 0$ we will construct a bounded $\delta$-separated tree.
  Since $D$ is non-dentable, there exists $\varepsilon > 0$ such that for all $\mb{x} \in D$,
  \begin{equation*}
    \mb{x} \in \overline{\conv}(D \sm B_{2\varepsilon}(\mb{x})).
  \end{equation*}
  By Lemma \ref{lem:dentlem}, for all $\mb{x} \in \tilde{D} := D + B_{\varepsilon}(\mb{0})$,
  \begin{equation*}
    \mb{x} \in \conv(\tilde{D} \sm B_{\varepsilon}(\mb{x})).
  \end{equation*}
  We will construct a $\varepsilon$-separated tree in the bounded set $\tilde{D}$, and by rescaling this will yield the result.

  Let $(\Omega,\mc{A},\P)$ be the interval $[0,1]$ with Borel $\sigma$-algebra and Lebesgue measure.
  We construct a $\varepsilon$-separated martingale inductively.
  Let $\mb{x}_{0} \in \tilde{D}$ be arbitrary and $\mb{f}_0 \equiv \mb{x}_0$.
  Since $\mb{x}_0 \in \conv(\tilde{D} \sm B_{\varepsilon}(\mb{x}_0))$, there exist numbers $\alpha_1,\ldots, \alpha_n \in (0,1)$ summing to $1$ and vectors $\mb{x}_{1}, \ldots, \mb{x}_{n} \in \tilde{D}$ such that
  \begin{equation*}
    \mb{x}_0 = \sum_{i=1}^{n} \alpha_{i} \mb{x}_{i} \qquad \text{and} \qquad \|\mb{x}_{i} - \mb{x}_{0}\|_X \geq \varepsilon.
  \end{equation*}
  Partition the unit interval $[0,1]$ into intervals $(I_i)_{i=1}^{n}$ with length $|I_{i}| = \alpha_{i}$, let $\mc{A}_{0}$ be the trivial $\sigma$-algebra on $[0,1]$, and let $\mc{A}_{1}$ be the $\sigma$-algebra generated by the intervals $(I_i)_{i=1}^{n}$.
  Define
  \begin{equation*}
    \mb{f}_1 = \sum_{i=1}^{n} \1_{I_i} \otimes \mb{x}_{i}.
  \end{equation*}
  Then $\E^{\mc{A}_0} \mb{f}_1 = \mb{f}_0$, and $\|\mb{f}_1(\omega) - \mb{f}_0(\omega)\|_{X} \geq \varepsilon$ for all $\omega \in [0,1]$.
  Since each point $\mb{x}_{i}$ is in $\tilde{D}$, we can repeat this process inductively, representing each $\mb{x}_{i}$ as a convex combination of vectors in $\tilde{D} \sm B_{\varepsilon}(\mb{x}_i)$, using these vectors to define $\mb{f}_2$ on $I_{i}$, and so on, to construct a $\varepsilon$-separated martingale valued in $\tilde{D}$, and thus a bounded $\varepsilon$-separated tree.
\end{proof}

Finally we will `complete the loop' in our discussion of the Radon--Nikodym property, martingale convergence, and dentability.

\begin{thm}\label{thm:dent-RNP}\index{Radon--Nikodym property!relation with dentable sets}\index{dentable sets!relation with RNP}
  Let $X$ be a Banach space such that every bounded subset of $X$ is dentable.
  Then $X$ has the Radon--Nikodym property.
\end{thm}

\begin{proof}
  Let $(S,\mc{A},\mu)$ be a finite measure space, and let $\map{\mb{\nu}}{\mc{A}}{X}$ be an $X$-valued vector measure with $\|\mb{\nu}(A)\|_{X} \leq \mu(A)$ for all $A \in \mc{A}$.
  Our task is to find a function $\mb{f} \in L^1(\mu;X)$ such that $\mb{\nu} = \mb{f}\mu$.
  By Proposition \ref{prop:RNP-finite-sufficient} this is sufficient to prove that $X$ has the RNP.

  For all sets $A \in \mc{A}$ let
  \begin{equation*}
    \begin{aligned}
      \mc{A}_{+}(A) &:= \{B \in \mc{A} : B \subset A, \mu(B) > 0\}, \\
      \mc{A}_{+} &:= \mc{A}_{+}(S)
    \end{aligned}
  \end{equation*}
  and for all $A \in \mc{A}_{+}$ define
  \begin{equation*}
    \mb{x}_{A} := \mu(A)^{-1} \mb{\nu}(A)\in X, \qquad 
    C_{A} := \{ \mb{x}_{B} : B \in \mc{A}_{+}(A)\} \subset X.
  \end{equation*}
  Note that $\|\mb{x}_{A}\|_X \leq 1$ for all $A \in \mc{A}_{+}$ by the assumptions on $\mb{\nu}$, so every $C_{A}$ is bounded and hence (by assumption) dentable.

  \textbf{We make the following claim:} \emph{for all $\varepsilon > 0$ and $A \in \mc{A}_{+}$, there exists $A' \in \mc{A}_+(A)$ such that $\diam(C_{A'}) \leq 2\varepsilon$.}
  
  We assume this is not the case and establish a contradiction.
  Thus there exist $\varepsilon > 0$ and $A \in \mc{A}_{+}$ such that $\diam(C_{A'}) > 2\varepsilon$ for all $A' \in \mc{A}_+(A)$.
  In particular, for every $\mb{x} \in X$ and $A' \in \mc{A}_{+}(A)$, there is a subset $B \in \mc{A}_{+}(A')$ such that $\|\mb{x} - \mb{x}_{B}\|_{X} > \varepsilon$.\footnote{Otherwise there would exist a vector $\tilde{\mb{x}} \in X$ with $\|\tilde{\mb{x}} - \mb{x}_{B}\|_{X} \leq \varepsilon$ for all $B \in \mc{A}_{+}(A')$, which implies $\|\mb{x}_{B} - \mb{x}_{B'}\|_{X} \leq 2\varepsilon$ for all $B,B' \in \mc{A}_+(A')$ and hence $\diam(C_{A'}) \leq 2\varepsilon$. Contradiction.}
  
  Now consider a fixed $A' \in \mc{A}_{+}(A)$ and let $\{B_{\lambda}\}_{\lambda \in \Lambda}$ be a maximal collection of disjoint elements of $\mc{A}_+(A')$ such that $\|\mb{x}_{A'} - \mb{x}_{B_\lambda}\|_{X} > \varepsilon$, where $\Lambda$ is some indexing set.
  Since the sets $B_{\lambda}$ are disjoint and have positive measure, and since
  \begin{equation*}
    \sum_{\lambda \in \Lambda} \mu(B_{\lambda}) \leq \mu(A') < \infty,
  \end{equation*}
  the indexing set $\Lambda$ is at most countable.
  By construction we must have
  \begin{equation}\label{eq:mu-sum}
    \mu\Big(A' \sm \bigcup_{\lambda \in \Lambda} B_\lambda\Big) = 0;
  \end{equation}
  otherwise we could find a set $B_{!} \in \mc{A}_{+}(A' \sm \cup_{\lambda} B_{\lambda}) \subset \mc{A}_{+}(A')$ such that $\|\mb{x}_{A'} - \mb{x}_{B_{!}}\|_{X} > \varepsilon$, contradicting the maximality of the set $
  \{B_{\lambda}\}$.
  The assumption on $\mb{\nu}$ yields
  \begin{equation*}
    \mb{\nu}\Big(A' \sm \bigcup_{\lambda \in \Lambda} B_\lambda\Big) = 0,
  \end{equation*}
  or equivalently (using countable additivity)
  \begin{equation*}
    \mb{\nu}(A') = \sum_{\lambda \in \Lambda} \mb{\nu}(B_\lambda).
  \end{equation*}
  This lets us write
  \begin{equation*}
      \mb{x}_{A'}
      = \mu(A')^{-1}\mb{\nu}(A') 
      = \sum_{\lambda \in \Lambda} \mu(A')^{-1}\mb{\nu}(B_\lambda) 
      = \sum_{\lambda \in \Lambda} \frac{\mu(B_\lambda)}{\mu(A')} \mb{x}_{B_\lambda}.
  \end{equation*}
  By \eqref{eq:mu-sum} we have that the coefficients of this series sum to $1$, and the vectors in the series satisfy
  \begin{equation*}
    \mb{x}_{B_{\lambda}} \in C_{A'} \qquad \text{and} \qquad \| \mb{x}_{A'}- \mb{x}_{B_\lambda} \|_X > \varepsilon,
  \end{equation*}
  which tells us that
  \begin{equation*}
    \mb{x}_{A'} \in \overline{\conv}(C_{A'} \sm B_{\varepsilon}(\mb{x}_{A'})).
  \end{equation*}
  Since this is true for all $A' \in \mc{A}_{+}(A)$, we find that $C_{A}$ is not dentable.
  This is a contradiction, which implies that our claim above is true.

  Now we return to the construction of a Radon--Nikodym derivative $\mb{f}$ of $\mb{\nu}$ with respect to $\mu$.
  Fix $\varepsilon > 0$.
  Using the claim we just established, let $(A_{\lambda})_{\lambda \in \Lambda}$ be a maximal disjoint collection of sets in $\mc{A}_+$ such that $\diam(C_{A_{\lambda}}) \leq 2\varepsilon$.
  Then $\Lambda$ is at most countable (by the same argument used in the last paragraph) and
  \begin{equation*}
    \mu(S \sm \bigcup_{\lambda \in \Lambda} A_{\lambda}) = 0;
  \end{equation*}
  if this were not the case then we could select $A' \in \mc{A}_+(S \sm \cup_{\lambda \in \Lambda} A_{\lambda}) \subset \mc{A}_+$ with $\diam(C_{A'}) \leq 2\varepsilon$ (using the claim) and contradict maximality.
  Define
  \begin{equation*}
    \mb{g}_{\varepsilon} := \sum_{\lambda \in \Lambda} \1_{A_{\lambda}} \otimes \mb{x}_{A_{\lambda}}.
  \end{equation*}
  Then $\mb{g}_{\varepsilon} \in L^1(\mu;X)$ (since it is bounded and the measure is finite).
  We will show that
  \begin{equation}\label{eq:g-computation}
    \|\mb{\nu} - \mb{g}_{\varepsilon}\mu\|_{\var} \leq 2\mu(S)\varepsilon;
  \end{equation}
  since this holds for all $\varepsilon > 0$, we find that $\mb{\nu}$ is in the closure in $M(S,\mc{A};X)$ of the set of measures of the form $\mb{g}\mu$ with $\mb{g} \in L^1(\mu;X)$.
  But this set is closed in $M(S,\mc{A};X)$ (Exercise \ref{ex:closure-M}), so there exists $\mb{g} \in L^1(\mu;X)$ with $\mb{\nu} = \mb{g}\mu$, as desired.

  It remains to show \eqref{eq:g-computation}.
  To see this first note that for all $A \in \mc{A}_+$
  \begin{equation*}
    \begin{aligned}
      \mb{\nu}(A) - \mb{g}_\varepsilon\mu(A)
      &= \sum_{\lambda \in \Lambda} \Big(  \mb{\nu}(A \cap A_{\lambda}) - \int_{A \cap A_{\lambda}} \mb{g}_{\varepsilon} \, \dd\mu \Big) \\
      &= \sum_{\lambda \in \Lambda} \Big(  \mb{\nu}(A \cap A_{\lambda}) - \mu(A \cap A_{\lambda}) \mb{x}_{A_{\lambda}} \Big) \\
      &= \sum_{\lambda \in \Lambda} \mu(A \cap A_{\lambda})(\mb{x}_{A \cap A_{\lambda}} - \mb{x}_{A_{\lambda}}),
    \end{aligned}
  \end{equation*}
  and so
  \begin{equation*}
    \|\mb{\nu}(A) - \mb{g}_\varepsilon\mu(A)\|_{X}
    \leq \sum_{\lambda \in \Lambda} \mu(A \cap A_{\lambda}) \|\mb{x}_{A \cap A_{\lambda}} - \mb{x}_{A_{\lambda}}\|_{X} \leq 2\mu(A)\varepsilon
  \end{equation*}
  using that $\diam(C_{A_{\lambda}}) \leq 2\varepsilon$.
  Taking the supremum over all partitions of $S$ proves \eqref{eq:g-computation} and completes the proof.
\end{proof}

Combining this with everything else we know, we have proven the following theorem.
\begin{thm}\label{thm:RNP-characterisations}\index{Radon--Nikodym property!equivalent characterisations}
  The following properties of a Banach space $X$ are equivalent:
  \begin{itemize}
  \item $X$ has the Radon--Nikodym property;
  \item $X$ has the $p$-martingale convergence property for all $p \in [1,\infty]$;
  \item $X$ has the $\infty$-martingale convergence property;
  \item for all $\delta > 0$, $X$ does not contain a bounded $\delta$-separated tree;
  \item every bounded subset of $X$ is dentable.
  \end{itemize}
\end{thm}

\begin{rmk}
  It is possible to prove directly that the $\infty$-MCP implies the RNP, but I don't think this is as nice as arguing via dentability.
  See \cite[Proof of Theorem 2.9]{gP16}.
\end{rmk}

This set of equivalences says quite a bit.
First, it says that almost everywhere convergence of $L^p$-bounded martingales holds for some $p \in [1,\infty]$ if and only if it holds for all $p \in [1,\infty]$.
This $p$-independence of martingale-based Banach space properties turns out to be fairly typical; martingales satisfy miraculous extrapolation properties of this kind.
Second, note that the first four properties are `extrinsic': the RNP makes reference to all $\sigma$-finite measure spaces, and the MCP properties and the nonexistence of bounded separated trees make reference to martingales valued in $X$.
In contrast, the last property is an intrinsic geometric property of $X$.
It is always satisfying to find an intrinsic geometric characterisation of what seems to be an extrinsic property.
One more point: by carefully looking at the proofs of these implications, one can show that it suffices to have the RNP with respect to the unit interval $[0,1]$ in order to show that every bounded subset $X$ is dentable.
This argument shows that a Banach space has the RNP if and only if it has the RNP with respect to the unit interval (see Exercise \ref{ex:RNP-unitinterval}).

Let's rattle off some consequences of this theorem.

\begin{cor}\index{Radon--Nikodym property!of reflexive spaces and separable duals}
  Reflexive spaces and separable dual spaces have the RNP.
\end{cor}

\begin{proof}
  By Theorem \ref{thm:MCP-sepdual} and Corollary \ref{cor:MCP-reflexive}, these spaces have the $\infty$-MCP.
  Thus they also have the RNP.
\end{proof}

\begin{cor}\index{Radon--Nikodym property!is separably determined}
  The RNP is separably determined, i.e. it holds for a Banach space $X$ if and only if it holds for all separable subspaces $Y \subset X$.
\end{cor}

\begin{proof}
  By Lemma \ref{lem:MCP-sepdet}, this is true for the $p$-MCP (for all $p \in [1,\infty]$, but these properties are now known to be equivalent anyway). 
\end{proof}

Finally, we return to Bochner spaces.
Recall Proposition \ref{prop:bochner-preduality}: for every $\sigma$-finite measure space $(S,\mc{A},\mu)$ and every Banach space $X$, for all $p \in [1,\infty]$, the map $\map{\Phi}{L^{p'}(S;X^*)}{L^p(S;X)^*}$ given by
\begin{equation*}
  \Phi \mb{g}(\mb{f}) = \int_{S} \langle \mb{f}(s), \mb{g}(s) \rangle \, \dd\mu(s) \qquad \forall \mb{f} \in L^p(S;X)
\end{equation*}
is an isometry onto a closed norming subspace of $L^p(S;X)^*$.
We will now complete this result with the help of the Radon--Nikodym property.

\begin{thm}\label{thm:bochner-duality-RNP}\index{Radon--Nikodym property!and duality of Bochner spaces}
  Let $X$ be a Banach space, and let $(S,\mc{A},\mu)$ be a countably generated measure space.\footnote{This says that there is a countable (or finite) collection of sets $(E_{\lambda})_{\lambda \in \Lambda}$ which generates $\mc{A}$.}
  Then the dual space $X^*$ has the Radon--Nikodym property with respect to $(S,\mc{A},\mu)$ if and only if for all $p \in [1,\infty)$ the isometric embedding $\map{\Phi}{L^{p'}(\mu;X^*)}{L^p(\mu;X)^*}$ is an isomorphism.
\end{thm}

We leave the proof as an exercise (Exercise \ref{ex:RNP-bochner}). 

  
\begin{rmk}\index{Radon--Nikodym property!equivalent characterisations}
  Not satisfied? See \cite[\textsection VII.6]{DU77} for 29 characterisations of the Radon--Nikodym property.
\end{rmk}


  


\section*{Exercises}

\begin{exercise}
  Let $X$ be a Banach space with the Radon--Nikodym property, and suppose that $Y$ is a Banach space which is isomorphic to $X$.
  Show that $Y$ also has the Radon--Nikodym property.
\end{exercise}

\begin{exercise}
  Let $X$ be a Banach space with the Radon--Nikodym property.
  Show that every bounded linear operator $\map{T}{L^1([0,1])}{X}$ is of the form
  \begin{equation*}
    Tf = \int_0^1 f(t) \mb{g}(t) \, \dd t
  \end{equation*}
  for some $\mb{g} \in L^1([0,1];X)$.
\end{exercise}

\begin{exercise}\index{conditional expectations!existence}
  Let $X$ be a Banach space with the Radon--Nikodym property.
  Use this property to show that for every measure space $(S,\mc{A},\mu)$ and every sub-$\sigma$-algebra $\mc{B} \subset \mc{A}$, every $\mb{f} \in L^1(\mc{A};X)$ has a conditional expectation with respect to $\mc{B}$.
  (Of course this holds for \emph{every} Banach space, but your job here is to derive it in a simpler way under the RNP assumption.)
\end{exercise}

\begin{exercise}\label{ex:fa-meas-ca}
  Let $X$ be a Banach space and $(S,\mc{A})$ a measure space.
  Let $\mb{\nu}$ be a \emph{finitely additive} $X$-valued vector measure on $(S,\mc{A})$, i.e. a function $\map{\mb{\nu}}{\mc{A}}{X}$ such that
  \begin{equation*}
    \mb{\nu}\Big(\sum_{n=1}^{N} E_n\Big) = \sum_{n=1}^{N} \mb{\nu}(E_n)
  \end{equation*}
  for all $N \in \N$ and $E_1,\ldots,E_n \in \mc{A}$.
  Suppose that there is a \emph{countably additive} (scalar) measure $\mu$ on $(S,\mc{A})$ such that
  \begin{equation*}
    \|\mb{\nu}(A)\|_{X} \leq \mu(A) \qquad \forall A \in \mc{A}.
  \end{equation*}
  Show that $\mb{\nu}$ is countably additive, $\|\mb{\nu}\|_{\var} \leq \|\mu\|_{\var}$, and $|\mb{\nu}| \ll \mu$.
\end{exercise}

\begin{exercise}\label{ex:closure-M}
  Let $(S,\mc{A})$ be a measurable space and $X$ a Banach space.
  Let $\mu$ be a finite measure on $(S,\mc{A})$.
  Show that the set of $X$-valued vector measures with Radon--Nikodym derivatives with respect to $\mu$, i.e. the set
  \begin{equation*}
    \{\mb{g}\mu : \mb{g} \in L^1(\mu;X)\} \subset M(S,\mc{A};X),
  \end{equation*}
  is closed in $M(S,\mc{A};X)$.
\end{exercise}

\begin{exercise}[Rademacher's theorem and the RNP]\index{theorem!Rademacher}\index{Radon--Nikodym property!equivalence with Rademacher's theorem}
  Show that a Banach space $X$ has the Radon--Nikodym property if and only if every $X$-valued Lipschitz function on $[0,1]$ is differentiable almost everywhere.
\end{exercise}

\begin{exercise}\label{ex:RNP-unitinterval}
  Suppose that $X$ has the Radon--Nikodym property with respect to the unit interval $[0,1]$ with Borel $\sigma$-algebra and Lebesgue measure.
  Show that $X$ has the Radon--Nikodym property with respect to all $\sigma$-finite measure spaces.
\end{exercise}

\begin{exercise}\label{ex:RNP-bochner}
  Prove Theorem \ref{thm:bochner-duality-RNP} in two different ways: first, using vector measures and the Radon--Nikodym property, then using martingales and the martingale convergence properties.\footnote{You can always search for references if you get stuck!}
\end{exercise}

\begin{exercise}\label{ex:Lp-RNP}\index{Radon--Nikodym property!of Bochner spaces}\index{Bochner spaces!have RNP}
  Let $X$ be a Banach space and $(\Omega,\mc{A},\mu)$ a measure space, and $p \in (1,\infty)$.
  Show that $X$ has the Radon--Nikodym property if and only if $L^p(\Omega;X)$ has the Radon--Nikodym property.
\end{exercise}



%%% Local Variables:
%%% mode: latex
%%% TeX-master: "../main.tex"
%%% End:
