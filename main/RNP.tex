We now move from Banach-valued analysis and probability to Banach-valued \emph{measure theory}, and finally to the \emph{geometry} of Banach spaces.
We will tie these concepts together via the Radon--Nikodym property, which is ostensibly a measure-theoretic property but has equivalent characterisations in terms of Bochner spaces, martingales, and convex sets.

\subsection{Vector measures and the Radon--Nikodym property}

\begin{defn}
  Let $X$ be a Banach space and $(S,\mc{A})$ a measurable space (recall: $\mc{A}$ is a $\sigma$-algebra on $S$).
  An $X$-valued \emph{vector measure} is a function $\map{\mu}{\mc{A}}{X}$ which is countably additive, in the sense that for all sequences $(E_n)_{n \in \N}$ of pairwise disjoint sets in $\mc{A}$,
  \begin{equation*}
    \mu\big( \bigcup_{n \in \N} E_n \big) = \sum_{n \in \N} \mu(E_n).
  \end{equation*}
  Note that this condition includes convergence of the series on the right hand side in $X$.
\end{defn}

Vector measures are just like measures, except the measure of a set $E \subset S$ is now a vector $\mu(E) \in X$ rather than a scalar.
We are most interested in vector measures with the following boundedness condition.

\begin{defn}
  Let $X$ be a Banach space and $\mu$ an $X$-valued vector measure on a maesurable space $(S,\mc{A})$.
  The \emph{variation} of $\mu$ is the scalar-valued measure $\map{|\mu|}{\mc{A}}{[0,\infty]}$ defined by
  \begin{equation*}
    |\mu|(E) := \sup_{\pi} \sum_{A \in \pi} \|\mu(A)\|_X,
  \end{equation*}
  where the supremum ranges over all partitions $\pi$ of $S$ into $\mc{A}$-measurable sets.
  We define the \emph{total variation norm} $\|\mu\|_{\var} := |\mu|(S)$, and we say that $\mu$ has \emph{bounded variation} if $\|\mu\|_{\var} < \infty$.
  Equivalently, $\mu$ has bounded variation if there exists a finite scalar-valued measure $\nu$ on $\mc{A}$ such that $\|\mu(A)\|_X \leq \nu(A)$ for all $A \in \mc{A}$ (as the minimal measure with this property is $|\mu|$).
\end{defn}

It is not particularly difficult to define integrals of scalar-valued functions with respect to vector measures.

\begin{prop}
  Let $X$ be a Banach space and $\mu$ an $X$-valued vector measure of bounded variation on a measurable space $(S,\mc{A})$.
  Then there is a unique continuous linear map $\map{[\mu]}{L^1(S,\mc{A},|\mu|)}{X}$ such that $[\mu](\1_{A}) = \mu(A)$ for all $A \in \mc{A}$.
  We use integral notation to denote this map, i.e. we write
  \begin{equation*}
    \int_{S} f(s) \, \dd\mu(s) := [\mu](f) \qquad \forall f \in L^1(S,\mc{A},|\mu|).
  \end{equation*}
\end{prop}

\begin{proof}
  We skip the verification that the prescription $[\mu](\1_{A})$ extends by linearity to a consistent map on integrable simple functions.\footnote{``It is dreadfully boring to show that this formula defines a linear map... from the space of simple functions of the above form into $X$ and we leave this as an exercise for masochists.'' \cite[pp5-6]{DU77}}
  We just need to show boundedness, and the conclusion will follow by density.
  Consider a simple function $g \in L^1(S,\mc{A},|\mu|)$ of the form
  \begin{equation*}
    g = \sum_{n=1}^{N} c_n \1_{S_n} 
  \end{equation*}
  with scalars $c_n \in \K$.
  Then
  \begin{equation*}
    \begin{aligned}
      \|[\mu](g)\|_X \leq \sum_{n=1}^{N} |c_n| \|\mu(S_n)\|_X \leq \sum_{n=1}^{N} |c_n| |\mu|(S_n) = \|g\|_{L^1(|\mu|)}.
    \end{aligned}
  \end{equation*}
  That's all.
\end{proof}


\begin{example}\label{eg:RN-density}
  Let $(S,\mc{A})$ be a measurable space and $X$ a Banach space.
  Suppose $\nu$ is a finite scalar-valued measure on $(S,\mc{A})$ and $f \in L^1(S,\mc{A},\nu;X)$.
  Then we can define an $X$-valued vector measure $\mu$ (sometimes denoted $\mu = f\nu$) by Bochner integration:
  \begin{equation*}
    \mu(A) = \int_A f(s) \, \dd\nu.
  \end{equation*}
  This vector measure has bounded variation: given a partition $S = \bigcup_{n \in \N} S_n$, we compute
  \begin{equation*}
    \sum_{n \in \N} \|\mu(S_n)\|_{X} = \sum_{n \in \N} \Big\| \int_{S_n} f(s) \, \dd\nu \Big\|_X
    \leq \int_{S} \|f(s)\|_{X} \, \dd\nu
  \end{equation*}
  so that $\|\mu\|_{\var} \leq \|f\|_{L^1(\nu;X)}$.\footnote{In fact, this is an equality. See \cite[pp43]{gP16}.}
\end{example}

Now let's revise some measure theory.
Recall that if $\mu$ and $\nu$ are two scalar-valued signed measures on a measurable space $(S,\mc{A})$, then we say \emph{$\nu$ is absolutely continuous with respect to $\mu$}, written $\nu \ll \mu$, if $A \in \mc{A}$ and $\mu(A) = 0$ implies $\nu(A) = 0$.

\begin{thm}[Radon--Nikodym]
  Let $(S,\mc{A})$ be a measurable space, and let $\mu$ be a $\sigma$-finite measure on $\mc{A}$.
  Let $\nu$ be a finite signed measure on $\mc{A}$ such that $\nu \ll \mu$.
  Then there exists a unique $h \in L^1(S,\mc{A},\mu)$ such that
  \begin{equation*}
    \nu(A) = \int_{A} h(s) \, \dd\mu(s) \qquad \forall A \in \mc{A}.
  \end{equation*}
\end{thm}

See \cite[Theorem 5.5.4]{rD04} for a proof.
One might expect that an analogous theorem holds for vector measures, but it turns out to depend on the geometry of the target Banach space.

\begin{defn}
  Let $(S,\mc{A},\mu)$ be a $\sigma$-finite measure space.
  A Banach space $X$ is said to have the \emph{Radon--Nikodym property (RNP) with respect to $(S,\mc{A},\mu)$} if for every $X$-valued vector measure $\nu$ on $(S,\mc{A})$ such that $\|\nu\|_{\var} < \infty$ and $|\nu| \ll \mu$, there is a function $f \in L^1(S,\mc{A},\mu;X)$ such that $\nu = f\mu$ (as defined in Example \ref{eg:RN-density}).
  We say $X$ has the \emph{Radon--Nikodym property} if it has the property above with respect to every $\sigma$-finite measure space $(S,\mc{A},\mu)$.
\end{defn}

The classical Radon--Nikodym theorem says that the scalar fields $\R$ and $\C$ have the RNP.
We will investigate this property for other Banach spaces by considering its relationship with martingales and with properties of convex sets.
We will also connect it with the duality of Bochner spaces $L^p(S;X)$, answering a question left open in Chapter \ref{sec:Bochner-spaces}.

\subsection{The RNP and martingale convergence}

\begin{thm}
  Let $X$ be a Banach space which has the Radon--Nikodym property with respect to a probability space $(\Omega,\mc{A},\P)$.
  Let $(f_n)_{n \in \N}$ be a martingale on $\Omega$ such that the set $\{\|f_n\|_X : n \in \N\} \subset L^1(\Omega)$ is bounded and uniformly integrable.
  Then there exists $f_\infty \in L^1(\Omega;X)$ such that $f_n = \E^{\mc{F}_n} f_\infty$ for all $n \in \N$.
\end{thm}

\begin{proof}
  Let the martingale be adapted to a filtration $(\mc{A}_n)_{n \in \N}$, and for simplicity suppose $\mc{A} = \mc{A}_\infty$.
  Since this is an RNP proof, our goal is to use $(f_n)_{n \in \N}$ to construct a vector measure on $\mc{A}$ of bounded variation which is absolutely continuous with respect to $\P$, from which we will extract the function $f_\infty$.

  For all $A \in \mc{A}$ we wish to define
  \begin{equation}\label{eq:mu-stationary-limit}
    \mu(A) := \lim_{k \to \infty} \int_A f_k \, \dd\P,
  \end{equation}
  but we will have to work to show that this definition makes sense.
  First observe that the definition makes sense for all $A \in \bigcup_{n \in \N} \mc{A}_{n}$; indeed, if $A \in \mc{A}_{n}$ then for $k \geq n$ we have
  \begin{equation*}
    \int_A f_k \, \dd\P = \int_A f_n \, \dd \P
  \end{equation*}
  and the limit in \eqref{eq:mu-stationary-limit} is equal to $\int_A f_n \, \dd\P$.
  \todo{UP TO HERE}
\end{proof}

\begin{cor}
  (spaces without RNP)
\end{cor}

\subsection{Duality of Bochner spaces}

\subsection*{Exercises}

% \begin{exercise}
%   Let $\map{T}{L^1([0,1])}{X}$ be a bounded linear operator, and for all Borel sets $E \subset [0,1]$ define
%   \begin{equation*}
%     \mu_T(E) := T(\1_{E}).
%   \end{equation*}
%   Show that $\mu_T$ is an $X$-valued vector measure with bounded variation, and show furthermore that $\mu_T$ is absolutely continuous with respect to the Lebesgue measure.
%   What does this say about bounded linear operators $L^1([0,1]) \to X$ when $X$ has the Radon--Nikodym property?
% \end{exercise}

\begin{exercise}
  Let $X$ be a Banach space with the Radon--Nikodym property.
  Show that every bounded linear operator $\map{T}{L^1([0,1])}{X}$ is of the form
  \begin{equation*}
    Tf = \int_0^1 f(t) g(t) \, \dd t
  \end{equation*}
  for some $g \in L^1([0,1];X)$.
\end{exercise}

\begin{exercise}
  Let $X$ be a Banach space with the Radon--Nikodym property.
  Use this property to show that for every measure space $(S,\mc{A},\mu)$ and every sub-$\sigma$-algebra $\mc{B} \subset \mc{A}$, every $f \in L^1(\mc{A};X)$ has a conditional expectation with respect to $\mc{B}$.
  (Of course this holds for \emph{every} Banach space, but your job here is to derive it in a simpler way under the RNP assumption.)
\end{exercise}


%%% Local Variables:
%%% mode: latex
%%% TeX-master: "../main.tex"
%%% End:
