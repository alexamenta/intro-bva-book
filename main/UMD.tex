% i think we will lose too much time doing unconditionality...
%
% \subsection{Unconditional summability}

% Recall the following counterintuitive theorem from basic real analysis (cite RUDIN PMA Theorem 3.54).\todo{add ref}

% \begin{thm}[Riemann]
%   Let $\sum_{n \in \N} a_n$ be a series of real numbers which is convergent but not absolutely convergent.
%   Then for all $-\infty \leq \alpha \leq \beta \leq \infty$ there exists a bijective function $\map{r}{\N}{\N}$ such that
%   \begin{equation*}
%     \liminf_{N \to \infty} \sum_{n=0}^{N} a_{r(n)} = \alpha, \quad \limsup_{N \to \infty} \sum_{n=0}^{N} a_{r(n)} = \beta.
%   \end{equation*}
%   In particular, the numbers $(a_n)_{n \in \N}$ can be rearranged so that the resulting series converge to any extended real number.  
% \end{thm}

% Such a series is called \emph{conditionally convergent}, and the key point to understand is that the value of the series $\sum_{n \in \N} a_n$ depends on the order of summation.
% This phenomenon relies on the possibility of negative entries of the series, as if all $a_n$ are nonnegative, the series is either absolutely convergent or divergent.

% Consider the following restatement of the contrapositive.

% \begin{cor}
%   Let $\sum_{n \in \N} a_n$ be a series of real numbers such that for every bijective function $\map{r}{\N}{\N}$, the rearranged series $\sum_{n \in \N} a_{r(n)}$ converges to a \emph{finite} limit.
%   Then the series $\sum_{n \in \N} a_n$ is absolutely convergent (and thus has a unique limit).  
% \end{cor}

% \begin{proof}
%   If the series were not absolutely convergent, then Riemann's theorem would give a rearrangement that converges to $+\infty$.
% \end{proof}

% In infinite-dimensional Banach spaces, the situation is, as always, more complicated.

% %i'm working from Lindenstrauss-Tzafriri for this

% \begin{prop}
%   For a sequence $(\mb{x}_{n})_{n \in \N}$ in a Banach space $X$, the followinf properties are equivalent.
%   \begin{enumerate}[(a)]
%   \item The series $\sum_{n \in \N} \mb{x}_{r(n)}$ converges for every bijective $\map{r}{\N}{\N}$.
%   \item There exists $\mb{x} = X$ such that for every bijective $\map{r}{\N}{\N}$,
%     \begin{equation*}
%       \mb{x} = \sum_{n \in \N} \mb{x}_{r(n)}.
%     \end{equation*}
%   \item The series $\sum_{i=1}^{\infty} \mb{x}_{n_{i}}$ converges for every increasing sequence $n_1 < n_2 < \cdots$.
%   \item The series $\sum_{n \in \N} \epsilon_{n} \mb{x}_{n}$ converges for every sequence of signs $\epsilon_{n} = \pm 1$.
%   \item There exists a constant $C < \infty$ such that for all $(a_{n})_{n \in \N} \in \ell^\infty(\N)$ and all $N \in \N$,
%     \begin{equation*}
%       \Big\| \sum_{n = 0}^{N} a_n \mb{x}_n \Big\|_{X} \leq C\|a\|_{\ell^\infty(\N)}.
%     \end{equation*}
%   \item For every $\varepsilon > 0$, there exists an $n \in \N$ such that
%     \begin{equation*}
%       \Big\| \sum_{i \in S} \mb{x}_{i} \Big\|_{X} < \varepsilon
%     \end{equation*}
%     for every finite set $S \subset \{n, n+1, \cdots\}$.
%   \end{enumerate}
%   If these conditions hold, we say that the sequence $(\mb{x}_{n})_{n \in \N}$ is \emph{unconditionally summable}, and the infimum of the possible constants $C$ is called the \emph{unconditionality constant} of the sequence.
% \end{prop}

% \begin{proof}
%   (b) directly implies (a), and
% \end{proof}


  
\subsection{The UMD property}

\begin{defn}
  For $p \in (1,\infty)$ a Banach space $X$ is said to have the \emph{$\UMD_{p}$ property} (Unconditionality of Martingale Differences in $L^p$) if there exists a constant $C < \infty$ such that for all probability spaces $(\Omega,\mc{A},\P)$, all $X$-valued $L^p$-bounded martingales $(f_n)_{n \in \N}$, and all sequences of signs $\xi_n = \pm 1$,
  \begin{equation*}
    \sup_{N \in \N} \Big\| \sum_{n=0}^{N} \xi_{n} df_{n} \Big\|_{L^p(\Omega;X)} \leq C \sup_{n \in \N} \|f_n\|_{L^p(\Omega;X)}.
  \end{equation*}
  The best possible constant $C$ in this inequality is denoted $\beta_{p}(X)$.
  We say that $X$ has the \emph{UMD property} if it has the $\UMD_{p}$ property for all $p \in (1,\infty)$.
\end{defn}

\begin{rmk}
  We will see in Section \ref{sec:UMD-p-independence} that the $\UMD_{p}$ property for some $p \in (1,\infty)$ implies the $\UMD$ property (i.e. $\UMD_{q}$ for all $q \in (1,\infty)$).
\end{rmk}

Recall that $df_{n} := f_n - f_{n-1}$ (with $df_0 := f_0$), so this property can be rewritten as having
\begin{equation*}
  \Big\| \sum_{n=0}^{N} \xi_{n} df_{n} \Big\|_{L^p(\Omega;X)} \leq C \Big\| \sum_{n=0}^{N} df_{n}\Big\|_{L^p(\Omega;X)}
\end{equation*}
for all $N \in \N$ and all sign sequences $\xi$.
In Banach space lingo, thus says that the sequence $(df_n)_{k \in \N}$ in $L^p(\Omega;X)$ is \emph{unconditional}, or equivalently that the series $\sum_{n} df_n$ converges in $L^p(\Omega;X)$ independently of the order of summation.

Put very bluntly, the UMD property says that martingale differences behave vaguely like orthogonal sequences in $L^p(\Omega;X)$.

\begin{example}
  Every Hilbert space $H$ has $\UMD_{2}$, with $\beta_{2}(H) = 1$.
  To show this let $(f_n)_{n \in \N}$ be an $H$-valued martingale on a probability space $(\Omega,\mc{A},\P)$.
  Then for all sign sequences $\xi$ and all $N \in \N$, since the martingale differences $df_{n}$ are independent and hence pairwise orthogonal, we have
  \begin{equation*}
      \Big\| \sum_{n=0}^N \xi_{n} df_{n} \Big\|_{L^2(\Omega;H)}^{2}
      = \sum_{n = 0}^{N} \xi_{n}^{2} \|df_{n}\|_{L^2(\Omega)}^{2}
      = \sum_{n = 0}^{N} \|df_{n}\|_{L^2(\Omega)}^{2}
      = \Big\| \sum_{n=0}^N df_{n} \Big\|_{L^2(\Omega;H)}^{2}.
  \end{equation*}
\end{example}

Before going into a more in-depth analysis of the UMD property, we present a few straightforward consequences of it.

\todo{up to here}

\begin{prop}
  stable under duals, subspaces
\end{prop}



\subsection{$p$-independence of the UMD property}\label{sec:UMD-p-independence}

\begin{cor}
  Hilbert spaces and $L^p$-spaces are UMD
\end{cor}

\subsection{Sufficiency of dyadic martingales}

\subsection{UMD and reflexivity}

UMD implies reflexive, hence also RNP

\subsection{Examples and counterexamples}

Bunch of counterexamples from non-reflexivity
State Qiu's counterexample
give a few more if possible...

\subsection*{Exercises}

\begin{exercise}
  (boundedness of martingale transforms on UMD spaces)
\end{exercise}


%%% Local Variables:
%%% mode: latex
%%% TeX-master: "../main.tex"
%%% End:
