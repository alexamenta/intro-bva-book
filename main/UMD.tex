\emph{under construction}


% \section{The UMD property}

% \begin{defn}
%   For $p \in (1,\infty)$ a Banach space $X$ is said to have the \emph{$\UMD_{p}$ property} (Unconditionality of Martingale Differences in $L^p$) if there exists a constant $C < \infty$ such that for all probability spaces $(\Omega,\mc{A},\P)$, all $X$-valued $L^p$-bounded martingales $(f_n)_{n \in \N}$, and all sequences of signs $\xi_n = \pm 1$,
%   \begin{equation*}
%     \sup_{N \in \N} \Big\| \sum_{n=0}^{N} \xi_{n} df_{n} \Big\|_{L^p(\Omega;X)} \leq C \sup_{n \in \N} \|f_n\|_{L^p(\Omega;X)}.
%   \end{equation*}
%   The best possible constant $C$ in this inequality is denoted $\beta_{p}(X)$.
%   We say that $X$ has the \emph{UMD property} if it has the $\UMD_{p}$ property for all $p \in (1,\infty)$.
% \end{defn}

% \begin{rmk}
%   We will see in Section \ref{sec:UMD-p-independence} that the $\UMD_{p}$ property for some $p \in (1,\infty)$ implies the $\UMD$ property (i.e. $\UMD_{q}$ for all $q \in (1,\infty)$).
% \end{rmk}

% Recall that $df_{n} := f_n - f_{n-1}$ (with $df_0 := f_0$), so this property can be rewritten as having
% \begin{equation*}
%   \Big\| \sum_{n=0}^{N} \xi_{n} df_{n} \Big\|_{L^p(\Omega;X)} \leq C \Big\| \sum_{n=0}^{N} df_{n}\Big\|_{L^p(\Omega;X)}
% \end{equation*}
% for all $N \in \N$ and all sign sequences $\xi$.
% In Banach space lingo, thus says that the sequence $(df_n)_{k \in \N}$ in $L^p(\Omega;X)$ is \emph{unconditional}, or equivalently that the series $\sum_{n} df_n$ converges in $L^p(\Omega;X)$ independently of the order of summation.

% Put very bluntly, the UMD property says that martingale differences behave vaguely like orthogonal sequences in $L^p(\Omega;X)$.

% \begin{example}
%   Every Hilbert space $H$ has $\UMD_{2}$, with $\beta_{2}(H) = 1$.
%   To show this let $(f_n)_{n \in \N}$ be an $H$-valued martingale on a probability space $(\Omega,\mc{A},\P)$.
%   Then for all sign sequences $\xi$ and all $N \in \N$, since the martingale differences $df_{n}$ are independent and hence pairwise orthogonal, we have
%   \begin{equation*}
%       \Big\| \sum_{n=0}^N \xi_{n} df_{n} \Big\|_{L^2(\Omega;H)}^{2}
%       = \sum_{n = 0}^{N} \xi_{n}^{2} \|df_{n}\|_{L^2(\Omega)}^{2}
%       = \sum_{n = 0}^{N} \|df_{n}\|_{L^2(\Omega)}^{2}
%       = \Big\| \sum_{n=0}^N df_{n} \Big\|_{L^2(\Omega;H)}^{2}.
%   \end{equation*}
% \end{example}

% Before going into a more in-depth analysis of the UMD property, we present a few straightforward consequences of it.

% \todo{up to here}

% \begin{prop}
%   stable under duals, subspaces
% \end{prop}



% \section{$p$-independence of the UMD property}\label{sec:UMD-p-independence}

% \begin{cor}
%   Hilbert spaces and $L^p$-spaces are UMD
% \end{cor}

% \section{Sufficiency of dyadic martingales}

% \section{UMD and reflexivity}

% UMD implies reflexive, hence also RNP

% \section{Examples and counterexamples}

% Bunch of counterexamples from non-reflexivity
% State Qiu's counterexample
% give a few more if possible...

% \section*{Exercises}

% \begin{exercise}
%   (boundedness of martingale transforms on UMD spaces)
% \end{exercise}


%%% Local Variables:
%%% mode: latex
%%% TeX-master: "../main.tex"
%%% End:
