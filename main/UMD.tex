In many applications of Banach-valued analysis, the most natural assumption to make on the target Banach space is the \emph{UMD property}.
This can be characterised in various ways, but the characterisation which gives it its name is the unconditionality of difference sequences of martingales valued in the Banach space.
In this chapter we will consider `basic' characterisations and implications of this property.
In the next chapter, which is probably the most important in terms of applications, we will discuss consequences for Fourier multiplier theorems and Littlewood--Paley theory.

\section{Definition and first examples}

Recall that the difference sequence of a stochastic process $\mb{f}_{\bullet}$ is the stochastic process $d\mb{f}_{\bullet}$ defined by $d\mb{f}_{n} := \mb{f}_{n} - \mb{f}_{n-1}$, with $\mb{f}_{-1} := \mb{0}$.

\begin{defn}
  For $p \in (1,\infty)$ a Banach space $X$ is said to have the \emph{$\UMD_{p}$ property} (Unconditionality of Martingale Differences in $L^p$) if there exists a constant $C < \infty$ such that for all probability spaces $(\Omega,\mc{A},\P)$, all $X$-valued $L^p$-bounded martingales $\mb{f}_{\bullet}$, and all sequences of signs $\xi_n = \pm 1$,
  \begin{equation}\label{eq:UMD-property}
    \sup_{N \in \N} \Big\| \sum_{n=0}^{N} \xi_{n} d\mb{f}_{n} \Big\|_{L^p(\Omega;X)} \leq C \sup_{N \in \N} \Big\| \sum_{n=0}^{N} d\mb{f}_{n}\Big\|_{L^p(\Omega;X)}.
  \end{equation}
  The best possible constant $C$ in this inequality is denoted $\beta_{p}(X)$.
  We say that $X$ has the \emph{UMD property} if it has the $\UMD_{p}$ property for all $p \in (1,\infty)$.
\end{defn}

\begin{rmk}
  We will see in Section \ref{sec:UMD-p-independence} that the $\UMD_{p}$ property for some $p \in (1,\infty)$ implies the $\UMD$ property, so (as with the $p$-martingale convergence properties) there is no need to isolate $\UMD_{p}$ as a separate property other than for pedagogical reasons.
\end{rmk}

In Banach space lingo, \eqref{eq:UMD-property} says that the sequence $d\mb{f}_\bullet$ in $L^p(\Omega;X)$ is \emph{unconditional}, or equivalently, that the convergence (or divergence) in $L^p(\Omega;X)$ of the series $\sum_{n \in \N} d\mb{f}_n$ is independent of the order of summation.
Put very coarsely, the UMD property says that martingale differences behave somewhat like orthogonal sequences in $L^p(\Omega;X)$.
In fact, when $X$ is a Hilbert space, the $\UMD_{2}$ property follows directly from orthogonality considerations.

\begin{prop}\label{prop:Hilbert-UMD2}
  Every Hilbert space $H$ has the $\UMD_{2}$ property, with best constant $\beta_{2}(H) = 1$.
\end{prop}

\begin{proof}
  Let $\mb{f}_{\bullet}$ be an $H$-valued martingale on a probability space $(\Omega,\mc{A},\P)$.
  Then for all sign sequences $\xi_{\bullet}$ and all $N \in \N$, since the differences $d\mb{f}_{n}$ are independent and hence pairwise orthogonal, we have
  \begin{equation*}
    \Big\| \sum_{n=0}^N \xi_{n} d\mb{f}_{n} \Big\|_{L^2(\Omega;H)}^{2}
    = \sum_{n = 0}^{N} \xi_{n}^{2} \|d\mb{f}_{n}\|_{L^2(\Omega;H)}^{2}
    = \sum_{n = 0}^{N} \|d\mb{f}_{n}\|_{L^2(\Omega;H)}^{2}
    = \Big\| \sum_{n=0}^N d\mb{f}_{n} \Big\|_{L^2(\Omega;H)}^{2}.
  \end{equation*}
  Thus we have equality in \eqref{eq:UMD-property}, with constant $C = 1$.
\end{proof}

The UMD property can be rephrased succinctly in terms of \emph{martingale sign transforms}.
Given a finite martingale $(\mb{f}_{n})_{n=0}^{N}$ valued in a Banach space $X$, and given a finite sequence of signs $(\xi_{n})_{n=0}^{N}$, recall that we defined the transformed martingale $(\xi \cdot \mb{f})_{\bullet}$ in terms of its difference sequence,
\begin{equation*}
  d(\xi \cdot \mb{f})_{n} := \xi_{n} d\mb{f}_{n} \qquad \text{(or equivalently, $(\xi \cdot \mb{f})_{k} = \sum_{n=0}^{k} \xi_{n} d\mb{f}_{n}$.)}
\end{equation*}
The $\UMD_{p}$ property can then be restated as follows: for all finite martingales $(\mb{f}_{n})_{n=0}^{N}$ and all finite sign sequences $(\xi_{n})_{n=1}^{N}$, we have a bound
\begin{equation*}
  \|(\xi \cdot \mb{f})_{N}\|_{L^p(\Omega;X)} \lesssim \|\mb{f}_{N}\|_{L^p(\Omega;X)}.
\end{equation*}
This can be restated in terms of filtrations.
Given a finite filtration $(\mc{A}_{n})_{n=0}^{N}$ on a probability space $(\Omega,\mc{A},\P)$ and a finite sequence of signs $(\xi_{n})_{n=0}^{N}$, let $T_{\mc{A}_{\bullet}, \xi_{\bullet}}$ denote the bounded operator acting on $L^1(\Omega;X)$ by
\begin{equation*}
  T_{\mc{A}_{\bullet}, \xi_{\bullet}} \mb{f} := \sum_{n=0}^{N} \xi_{n} d\mb{f}_{n} = (\xi \cdot \mb{f})_{N},
\end{equation*}
where $\mb{f}_{\bullet}$ is the finite martingale associated with $\mb{f}$ and $\mc{A}_{\bullet}$.
At this point, we have already given all the necessary details to prove the following convenient characterisation of the UMD property.

\begin{prop}\label{prop:UMD-signtransforms}
  Let $X$ be a Banach space.
  Then the $\UMD_{p}$ constant of $X$ is given by
  \begin{equation*}
    \beta_{p}(X) = \sup_{\Omega,\mc{A}_{\bullet}, \xi_{\bullet}}  \|T_{\mc{A}_{\bullet}, \xi_{\bullet}}\|_{\Lin(L^p(\Omega;X))},
  \end{equation*}
  where the supremum is taken over all probability spaces $(\Omega,\mc{A},\P)$, all finite filtrations $\mc{A}_{\bullet}$, and all finite sign sequences $\xi_{\bullet}$.
\end{prop}

This characterisation of the $\UMD_{p}$ constant, with a short duality argument, implies the following.

\begin{prop}\label{prop:UMD-duality}
  Let $p \in (1,\infty)$, and let $X$ be a Banach space.
  Then $X$ has the $\UMD_{p}$ property if and only if the dual space $X^{*}$ has the $\UMD_{p'}$ property, and $\beta_{p}(X) = \beta_{p'}(X^{*})$.
\end{prop}

\begin{proof}
  Fix a finite filtration $\mc{A}_{\bullet}$ on a probability space $(\Omega,\mc{A},\P)$, and a finite sign sequence $\xi_{\bullet}$.
  Suppose $\mb{f} \in L^{p}(\Omega;X)$ and $\mb{g} \in L^{p'}(\Omega;X^{*})$.
  Since the (scalar) conditional expectation $\E^{\mc{A}_{n}}$ on $L^p(\Omega)$ is adjoint to the corresponding conditional expectation on $L^{p'}(\Omega)$ (Proposition \ref{prop:CE-adjoint}), and since
  \begin{equation*}
    \langle \E^{\mc{A}_{n}} \mb{f}, \mb{g} \rangle = \langle \mb{f}, \E^{\mc{A}_{n}} \mb{g} \rangle 
  \end{equation*}
  by Exercise \ref{ex:tensor-adjoint}, we get
  \begin{equation*}
    \langle T_{\mc{A}_{\bullet}, \xi_{\bullet}} \mb{f}, \mb{g} \rangle =  \langle \mb{f}, T_{\mc{A}_{\bullet}, \xi_{\bullet}} \mb{g} \rangle.
  \end{equation*}
  Thus by the norming property of $L^{p'}(\Omega;X^{*})$ as a subset of $L^{p}(\Omega;X)^{*}$ (Proposition \ref{prop:bochner-preduality}), and of $L^{p}(\Omega;X)$ as a subset of $L^{p'}(\Omega;X^{*})^{*}$,\footnote{Technically we haven't proven this. The proof is a small modification of that of Proposition \ref{prop:bochner-preduality}, using that $X$ is a norming subspace of $X^{**} = (X^{*})^{*}$. See \cite[Proposition 1.3.1]{HNVW16} for the needed duality result in full generality.}  we get
  \begin{equation*}
    \begin{aligned}
      \|T_{\mc{A}_{\bullet}, \xi_{\bullet}}\|_{\Lin(L^{p}(\Omega;X))}
      &= \sup_{\mb{f}} \| T_{\mc{A}_{\bullet}, \xi_{\bullet}} \mb{f} \|_{L^{p}(\Omega;X)} \\
      &= \sup_{\mb{f}, \mb{g}} \langle T_{\mc{A}_{\bullet}, \xi_{\bullet}} \mb{f}, \mb{g} \rangle \\
      &= \sup_{\mb{f}, \mb{g}} \langle  \mb{f}, T_{\mc{A}_{\bullet}, \xi_{\bullet}} \mb{g} \rangle \\
      &= \sup_{\mb{g}} \| T_{\mc{A}_{\bullet}, \xi_{\bullet}} \mb{g} \|_{L^{p'}(\Omega;X^{*})}
      &= \|T_{\mc{A}_{\bullet}, \xi_{\bullet}}\|_{\Lin(L^{p'}(\Omega;X^{*}))},
  \end{aligned}
\end{equation*}
with suprema taken over all normalised $\mb{f} \in L^{p}(\Omega;X)$ and $\mb{g} \in L^{p'}(\Omega;X^{*})$.
Taking the supremum over all $\Omega$, $\mc{A}_{\bullet}$, and $\xi_{\bullet}$ then shows that
\begin{equation*}
  \beta_{p}(X) = \beta_{p'}(X^{*}), 
\end{equation*}
completing the proof.
\end{proof}

Before moving on to more subtle properties, we show that the $\UMD_{p}$ property extends to Bochner spaces.
This will be more useful once we know the $p$-independence of the UMD property.

\begin{prop}\label{prop:Bochner-UMDp}
  Let $p \in (1,\infty)$, and suppose $X$ is a Banach space with the $\UMD_{p}$ property.
  Let $(S,\mc{A},\mu)$ be a measure space with $\mu(S) > 0$.
  Then the Bochner space $L^p(\mu;X)$ has $\UMD_{p}$, with $\beta_{p}(L^p(\mu;X)) = \beta_{p}(X)$.
\end{prop}

\begin{proof}
  To avoid confusion let $Y = L^p(\mu;X)$, and let $\mb{f}_{\bullet}$ be a $Y$-valued martingale on a probability space $(\Omega,\mc{A},\P)$.
  Then for all $N \in \N$, by wrapping the $\UMD_{p}$ property of $X$ in two applications of Fubini's theorem, we have
  \begin{equation*}
    \begin{aligned}
      \Big\| \sum_{n=0}^N \xi_{n} d\mb{f}_{n} \Big\|_{L^p(\Omega;Y)}^{p}
      &= \int_{\Omega} \int_{S} \Big\| \sum_{n=0}^N \xi_{n} d\mb{f}_{n}(\omega) \Big\|_{X}^{p} \, \dd\mu(s) \, \dd\P(\omega) \\
      &=  \int_{S} \Big( \int_{\Omega} \Big\| \sum_{n=0}^N \xi_{n} d\mb{f}_{n}(\omega) \Big\|_{X}^{p}  \, \dd\P(\omega) \Big) \, \dd\mu(s) \\
      &\leq \beta_{p}(X)^{p} \int_{S} \Big( \int_{\Omega} \Big\| \sum_{n=0}^N d\mb{f}_{n}(\omega) \Big\|_{X}^{p}  \, \dd\P(\omega) \Big) \, \dd\mu(s) \\
      &= \beta_{p}(X)^{p} \int_{\Omega}  \int_{S} \Big\| \sum_{n=0}^N d\mb{f}_{n}(\omega) \Big\|_{X}^{p}   \, \dd\mu(s) \, \dd\P(\omega) \\
      &= \beta_{p}(X)^{p} \Big\| \sum_{n=0}^N d\mb{f}_{n} \Big\|_{L^p(\Omega;Y)}^{p},
    \end{aligned}
  \end{equation*}
  which establishes $\beta_{p}(Y) \leq \beta_{p}(X)$.
  The reverse estimate is Exercise \ref{ex:UMD-Lp-reverse}.
\end{proof}

\section{$p$-independence of the UMD property}\label{sec:UMD-p-independence}

Our goal now is to prove that the $\UMD_{p}$ property is independent of $p$.
For this, we will use the characterisation from Proposition \ref{prop:UMD-signtransforms} of the $\UMD_{p}$ property in terms of finite martingale sign transforms $T_{\mc{A}_{\bullet}, \xi_{\bullet}}$.
The basic strategy is as follows:
\begin{itemize}
\item Assuming that the finite sign transforms $T_{\mc{A}_{\bullet},\xi_{\bullet}}$ are uniformly bounded on $L^p(\Omega;X)$, prove that they have weak-type $(1,1)$ (i.e. that they are bounded from $L^1(\Omega;X)$ to the Lorentz space $L^{1,\infty}(\Omega;X)$ uniformly over all finite filtrations and sign sequences.
\item Use Marcinkiewicz's interpolation theorem to deduce that the finite sign transforms are uniformly bounded on $L^q(\Omega;X)$ for all $1 < q < p$, and thus that $X$ has the $\UMD_{q}$ property for all $1 < q < p$.
\item Argue by duality to show that $X$ has the $\UMD_{r}$ property for all $p < r < \infty$, and thus that $X$ has the full $\UMD$ property.
\end{itemize}

The first two steps are similar to how we proved Doob's maximal inequality (Theorem \ref{thm:doob}), except there we implicitly interpolated between a weak-type $(1,1)$ estimate and an $L^\infty$ bound that such maximal operators automatically satisfy.
The weak-type $(1,1)$ estimate for sign transforms will be proven relies on \emph{Gundy's decomposition} for martingales, which is analogous to the Calder\'on--Zygmund decomposition of a function.\footnote{
Students of harmonic analysis may recognise this procedure as essentially being the same as that used in Calder\'on--Zygmund theory: start with an assumed $L^p$ estimate, use properties of the operators under consideration (here we use the martingale property rather than kernel estimates) to deduce weak-type $(1,1)$, and finally use interpolation and duality to extend the $L^p$-boundedness to $L^q$ for all $q \in (1,\infty)$.}

\begin{thm}[Gundy's decomposition]\label{thm:gundy}
  Let $X$ be a Banach space and $(\Omega,\mc{A},\P)$ a probability space.
  Suppose $\mb{f} \in L^1(\Omega;X)$, and let $\mc{A}_{\bullet}$ be a filtration with $\mc{A}_{\infty} = \mc{A}$.
  Then given $\lambda > 0$, there is a decomposition
  \begin{equation*}
    \mb{f} = \mb{a} + \mb{b} + \mb{c}
  \end{equation*}
  with $\mb{a}, \mb{b}, \mb{c} \in L^1(\Omega;X)$ satisfying
  \begin{itemize}
  \item $\|\mb{a}\|_{L^{1}(\Omega;X)} \lesssim 1$, $\P(\sup_{n \in \N} \|d\mb{a}_{n}\|_{X} \neq 0) \lesssim \lambda^{-1}$,
  \item $\| \sum_{n \in \N} \|d\mb{b}_{n}\|_{X} \|_{L^1(\Omega)} \lesssim 1$,
  \item $\| \mb{c} \|_{L^\infty(\Omega;X)} \lesssim \lambda$ and $\|\mb{c}\|_{L^1(\Omega;X)} \lesssim 1$.
  \end{itemize}
  where $\mb{a}_{n} = \E^{\mc{A}_{n}} \mb{a}$ and $\mb{b}_{n} = \E^{\mc{A}_{n}} \mb{b}$.
\end{thm}

We will prove this in the next section.
For now we will assume it, and deduce the weak-type $(1,1)$ estimate for finite sign transforms in $\UMD_{p}$ spaces.

\begin{prop}\label{prop:martingale-w11}
  Suppose that $X$ is a Banach space with the $\UMD_{p}$ property.
  Then for all probability spaces $(\Omega,\mc{A},\P)$ and all finite filtrations $\mc{A}_{\bullet}$ and sign sequences $\xi_{\bullet}$,
  \begin{equation*}
    \|T_{\mc{A}_{\bullet}, \xi_{\bullet}} \mb{f}\|_{L^{1,\infty}(\Omega;X)} := \sup_{t > 0} t\P(\|T_{\mc{A}_{\bullet}, \xi_{\bullet}} \mb{f}\|_{X} > t) \lesssim_{p,X} \|\mb{f}\|_{L^1(\Omega;X)}
  \end{equation*}
  for all $\mb{f} \in L^1(\Omega;X)$.
\end{prop}

\begin{proof}[Proof, assuming Theorem \ref{thm:gundy}]
  Fix a finite filtration $(\mc{A}_{n})_{n=0}^{N}$ on a probability space $(\Omega,\mc{A},\P)$, and a sequence of signs $(\xi_{n})_{n=0}^{N}$.
  Let $\mb{f} \in L^1(\Omega;X)$; by rescaling we may assume $\|\mb{f}\|_{L^1(\Omega;X)} = 1$.\footnote{If $\mb{f} = 0$ there is nothing to show anyway.}
  Fix $t > 0$, and let
  \begin{equation*}
    \mb{f} = \mb{a} + \mb{b} + \mb{c}
  \end{equation*}
  be the Gundy decomposition of $\mb{f}$ at level $t$ (Theorem \ref{thm:gundy}).
  Writing
  \begin{equation*}
    \td{\mb{f}} := T_{\mb{A}_{\bullet}, \xi_{\bullet}}\mb{f},
    \qquad \td{\mb{a}} := T_{\mb{A}_{\bullet}, \xi_{\bullet}}\mb{a},
    \qquad \td{\mb{b}} := T_{\mb{A}_{\bullet}, \xi_{\bullet}}\mb{b},
    \qquad \td{\mb{c}} := T_{\mb{A}_{\bullet}, \xi_{\bullet}}\mb{c},
  \end{equation*}
  We have by linearity of the martingale transform
  \begin{equation*}
    \td{\mb{f}} = \td{\mb{a}} + \td{\mb{b}} + \td{\mb{c}}
  \end{equation*}
  and thus
  \begin{equation*}
    \P(\|\td{\mb{f}}\|_{X} > 3t)
    \leq \P(\|\td{\mb{a}}\|_{X} > t) + \P(\|\td{\mb{b}}\|_{X} > t) + \P(\|\td{\mb{c}}\|_{X} > t).
  \end{equation*}

  We estimate these three summands separately.
  First, we have
  \begin{equation*}
    \begin{aligned}
      \P(\|\td{ \mb{a} }\|_{X} > t)
      \leq \P(\sup_{n} \|d \mb{a}_{n}\|_{X} \neq 0) \lesssim t^{-1},
    \end{aligned}
  \end{equation*}
  as we cannot have $\| \td{ \mb{a} }  \|_{X} = \|T_{ \mc{A}_{\bullet}, \xi_{\bullet} } \mb{a}\|_{X}> t$ without one of the differences $d\mb{a}_{n}$ being nonzero.
  For the second term, since
  \begin{equation*}
    \|\td{\mb{b}}(\omega)\|_{X} = \Big\| \sum_{n=0}^{N} \xi_{n} d\mb{b}_{n}(\omega) \Big\|_{X} \leq \sum_{n = 0}^{N} \|d\mb{b}_{n}(\omega)\|_{X} \qquad \forall \omega \in \Omega,
  \end{equation*}
  we have
  \begin{equation*}
    \P(\|\td{\mb{b}}\|_{X} > t) \leq \P\Big( \sum_{n = 0}^{N} \|d\mb{b}_{n}\|_{X} > t \Big)
    \leq t^{-1} \Big\|  \sum_{n = 0}^{N} \|d\mb{b}_{n}\|_{X} \Big\|_{L^1(\Omega)} \lesssim t^{-1}
  \end{equation*}
  by Chebyshev's inequality.
  Finally, using Chebyshev again, the $\UMD_{p}$ property, and log-convexity of $L^p$-norms,
  \begin{equation*}
    \begin{aligned}
      \P(\|\td{\mb{c}}\|_{X} > t)
      &\leq t^{-p}\|T_{\mc{A}_{\bullet, \xi_{\bullet}}} \mb{c} \|_{L^p(\Omega;X)}^{p} \\
      &\leq t^{-p} \beta_{p}(X)^{p} \|\mb{c} \|_{L^p(\Omega;X)}^{p} \\
      &\leq t^{-p} \beta_{p}(X)^{p} \Big( \|\mb{c} \|_{L^1(\Omega;X)}^{\frac{1}{p}} \|\mb{c}\|_{L^\infty(\Omega;X)}^{1-\frac{1}{p}} \Big)^{p} \\
      &\lesssim t^{-p} \beta_{p}(X)^{p} t^{p-1} = t^{-1}\beta_{p}(X)^{p} .
    \end{aligned}
  \end{equation*}
  Therefore we have
  \begin{equation*}
    3t\P(\|T_{\mc{A}_{\bullet}, \xi_{\bullet}} \mb{f}\|_{X} > 3t)
    \lesssim 3t\big(2t^{-1} + t^{-1} \beta_{p}(X)^{p}\big) \lesssim \beta_{p}(X)^{p}.
  \end{equation*}
  Taking the supremum over $t > 0$ completes the proof.
\end{proof}

Now we can prove the $p$-independence of the UMD property, still implicitly assuming the validity of Gundy's decomposition.

\begin{thm}\label{thm:UMD-p-independent}
  Let $X$ be a Banach space which has the $\UMD_{p}$ property for some $p \in (1,\infty)$.
  Then $X$ has $\UMD_{q}$ for all $q \in (1,\infty)$ (i.e. $X$ is UMD).
\end{thm}

\begin{proof}
  By Proposition \ref{prop:martingale-w11} we have that all finite martingale sign transforms $T_{\mc{A}_{\bullet}, \xi_{\bullet}}$ are uniformly bounded from $L^1(\Omega;X)$ to $L^{1,\infty}(\Omega;X)$.
  By the Marcinkiewicz interpolation theorem for Bochner spaces (Theorem \ref{thm:marcinkiewicz} in the appendix), since these operators are also uniformly bounded on $L^p(\Omega;X)$, we find that they are uniformly bounded on $L^q(\Omega;X)$ for all $q \in (1,p)$, and thus $X$ has the $\UMD_{q}$ property for all such $q$.

  On the other hand, by Proposition \ref{prop:UMD-duality}, we know that $X^{*}$ has the $\UMD_{p'}$ property, and thus by the argument above $X^{*}$ has the $\UMD_{r}$ property for all $r \in (1,p')$.
  Hence $X$ itself has the $\UMD_{r'}$ property for all $r \in (1,p')$, which (messing with H\"older conjugates) says that $X$ has the $\UMD_{s}$ property for all $s \in (p,\infty)$.
  Thus $X$ is UMD.
\end{proof}

\begin{cor}
  The following Banach spaces are UMD:
  \begin{itemize}
  \item every Hilbert space,
  \item every finite-dimensional space,
  \item every Lebesgue space $L^p(S)$ over a measure space $(S,\mc{A},\mu)$, with $p \in (1,\infty)$,
  \item every Bochner space $L^p(S;X)$ with $S$ and $p$ as above, provided that $X$ is UMD.
  \end{itemize}
\end{cor}

\begin{proof}
  We proved in Proposition \ref{prop:Hilbert-UMD2} that every Hilbert space has the $\UMD_{2}$ property, so by $p$-independence, every Hilbert space is $\UMD$.
  Every finite-dimensional space is isomorphic to a Hilbert space, hence also UMD (see Exercise \ref{ex:UMD-isomorphism}).
  If $X$ is a UMD space, then in particular $X$ is $\UMD_{p}$, and thus by Proposition \ref{prop:Bochner-UMDp} the Bochner space $L^p(S;X)$ is also $\UMD_{p}$ (hence $\UMD$).
  Taking $X$ to be the scalar field, which is a Hilbert space, proves that $L^p(S)$ is $\UMD$.
\end{proof}

\begin{rmk}
  Theorem \ref{thm:UMD-p-independent} says that if the constant $\beta_{p}(X)$ is finite for some $p \in (1,\infty)$, then $\beta_{q}(X) < \infty$ for all $q \in (1,\infty)$, but it does not give sharp control on how $\beta_{q}(X)$ changes.
  It is possible (but difficult) to prove that
  \begin{equation*}
    \beta_{p}(H) = \max(p,p') - 1
  \end{equation*}
  for every Hilbert space $H$ and all $p \in (1,\infty)$ \cite[Corollary 4.5.15]{HNVW16}.
  As far as I know, this is the only case where the UMD constant of a space is known exactly (excluding the cases $\beta_{p}(X) = \infty$).
\end{rmk}


\section{The proof of Gundy's decomposition}

\todo{we need a bit more on stopping times that we forgot!} ($\sigma$-algebra associated with a stopping time...)

\begin{proof}[Proof of Theorem \ref{thm:gundy}]
  Define a stopping time
  \begin{equation*}
    r(\omega) := \inf\big\{ n \in \N : \|\mb{f}_{n}(\omega)\|_{X} > \lambda \big\},
  \end{equation*}
  where $\mb{f}$ is the martingale associated with $\mb{f}$ and $\mc{A}_{\bullet}$.
  For $n \in \N$ define the scalar-valued function $v_{n} = \|d\mb{f}_{n}\|_{X} \1_{\{r = n\}}$, and define two more stopping times
  \begin{equation*}
    \begin{aligned}
      s(\omega) &:= \inf\Big\{ n \in \N : \sum_{k=0}^{n} \E^{\mc{A}_{k}} v_{k+1} > \lambda \Big\} \\
      T(\omega) &:= \min(r(\omega), s(\omega)).
    \end{aligned}
  \end{equation*}
  
\end{proof}

\section{Sufficiency of dyadic martingales}

\section{UMD implies reflexivity}

UMD implies reflexive, hence also RNP

\section{Examples and counterexamples}

Bunch of counterexamples from non-reflexivity
State Qiu's counterexample
give a few more if possible...



\section{Applications to martingale transforms and Haar decompositions}

-martingale transforms are bounded
-Haar decompositions are unconditional

\section{Stein's inequality: $R$-boundedness of conditional expectations}




\section*{Exercises}

\begin{exercise}\label{ex:UMD-isomorphism}
  Let $X$ be a Banach space and $Y$ a closed subspace of $X$.
  Suppose that $Z$ is a Banach space and $\map{\phi}{Z}{Y}$ is an isomorphism.
  Show that 
  \begin{equation*}
    \beta_{p}(Z) \leq \|\phi\|_{\Lin(Z,Y)} \|\phi^{-1}\|_{\Lin(Y,Z)} \beta_{p}(X) \qquad \forall p \in (1,\infty).
  \end{equation*}
\end{exercise}

\begin{exercise}\label{ex:UMD-Lp-reverse}
  Let $X$ be a Banach space and $(S,\mc{A},\mu)$ be a measure space with $\mu(S) > 0$.
  For $p \in (1,\infty)$, show that $\beta_{p}(L^p(\mu;X)) \geq \beta_{p}(X)$.
\end{exercise}

\begin{exercise}
  (boundedness of martingale transforms on UMD spaces with $R$-bounded coefficients)
\end{exercise}

\begin{exercise}
  $\UMD_{p}$ implies randomised $\UMD_{p}$.
  Randomised $\UMD_{p}$ are all equivalent.
  $L^1$ is 'UMD-'.
\end{exercise}



%%% Local Variables:
%%% mode: latex
%%% TeX-master: "../main.tex"
%%% End:
