
\section{The UMD property}

\begin{defn}
  For $p \in (1,\infty)$ a Banach space $X$ is said to have the \emph{$\UMD_{p}$ property} (Unconditionality of Martingale Differences in $L^p$) if there exists a constant $C < \infty$ such that for all probability spaces $(\Omega,\mc{A},\P)$, all $X$-valued $L^p$-bounded martingales $\mb{f}_{\bullet}$, and all sequences of signs $\xi_n = \pm 1$,
  \begin{equation}\label{eq:UMD-property}
    \sup_{N \in \N} \Big\| \sum_{n=0}^{N} \xi_{n} d\mb{f}_{n} \Big\|_{L^p(\Omega;X)} \leq C \sup_{N \in \N} \|\mb{f}_N\|_{L^p(\Omega;X)}.
  \end{equation}
  The best possible constant $C$ in this inequality is denoted $\beta_{p}(X)$.
  We say that $X$ has the \emph{UMD property} if it has the $\UMD_{p}$ property for all $p \in (1,\infty)$.
\end{defn}

\begin{rmk}
  We will see in Section \ref{sec:UMD-p-independence} that the $\UMD_{p}$ property for some $p \in (1,\infty)$ implies the $\UMD$ property, so there is no need to isolate $\UMD_{p}$ as a separate property other than for pedagogical reasons.
\end{rmk}

Recall that the difference process $d\mb{f}_{\bullet}$ is given by $d\mb{f}_{n} := \mb{f}_n - \mb{f}_{n-1}$ (with $\mb{f}_{-1} := \mb{0}$), so \eqref{eq:UMD-property} can be rewritten as
\begin{equation*}
  \sup_{N \in \N} \Big\| \sum_{n=0}^{N} \xi_{n} d\mb{f}_{n} \Big\|_{L^p(\Omega;X)} \leq C \sup_{N \in \N} \Big\| \sum_{n=0}^{N} d\mb{f}_{n}\Big\|_{L^p(\Omega;X)}.
\end{equation*}
In Banach space lingo, thus says that the sequence $d\mb{f}_\bullet$ in $L^p(\Omega;X)$ is \emph{unconditional}, or equivalently, that the convergence (or divergence) in $L^p(\Omega;X)$ of the series $\sum_{n} d\mb{f}_n$ is independent of the order of summation.

Put very coarsely, the UMD property says that martingale differences behave vaguely like orthogonal sequences in $L^p(\Omega;X)$.
In fact, when $X$ is a Hilbert space, the $\UMD_{2}$ property follows directly from orthogonality considerations.

\begin{example}
  Every Hilbert space $H$ has the $\UMD_{2}$ property, with $\beta_{2}(H) = 1$.
  To show this let $\mb{f}_{\bullet}$ be an $H$-valued martingale on a probability space $(\Omega,\mc{A},\P)$.
  Then for all sign sequences $\xi_{\bullet}$ and all $N \in \N$, since the differences $d\mb{f}_{n}$ are independent and hence pairwise orthogonal, we have
  \begin{equation*}
      \Big\| \sum_{n=0}^N \xi_{n} d\mb{f}_{n} \Big\|_{L^2(\Omega;H)}^{2}
      = \sum_{n = 0}^{N} \xi_{n}^{2} \|d\mb{f}_{n}\|_{L^2(\Omega;H)}^{2}
      = \sum_{n = 0}^{N} \|d\mb{f}_{n}\|_{L^2(\Omega;H)}^{2}
      = \Big\| \sum_{n=0}^N d\mb{f}_{n} \Big\|_{L^2(\Omega;H)}^{2}.
  \end{equation*}
\end{example}

We can't say very much about the UMD property by looking directly at its definition.\todo{we do need Rademacher sums first... square functions are important}

\section{$p$-independence of the UMD property}\label{sec:UMD-p-independence}

\begin{cor}
  Hilbert spaces and $L^p$-spaces are UMD
\end{cor}

\section{Sufficiency of dyadic martingales}

\section{UMD and reflexivity}

UMD implies reflexive, hence also RNP

\section{Examples and counterexamples}

Bunch of counterexamples from non-reflexivity
State Qiu's counterexample
give a few more if possible...

\section*{Exercises}

\begin{exercise}\label{ex:UMD-isomorphism}
  Let $X$ be a Banach space and $Y$ a closed subspace of $X$.
  Suppose that $Z$ is a Banach space and $\map{\phi}{Z}{Y}$ is an isomorphism.
  Show that 
  \begin{equation*}
    \beta_{p}(Z) \leq \|\phi\|_{\Lin(Z,Y)} \|\phi^{-1}\|_{\Lin(Y,Z)} \beta_{p}(X) \qquad \forall p \in (1,\infty).
  \end{equation*}
\end{exercise}

\begin{exercise}
  (boundedness of martingale transforms on UMD spaces)
\end{exercise}


%%% Local Variables:
%%% mode: latex
%%% TeX-master: "../main.tex"
%%% End:
