A vast portion of analysis, probably the majority, is concerned with properties of scalar-valued functions
\begin{equation*}
  \map{f}{\R^{d}}{\K}, \qquad \K = \text{$\R$ or $\C$}
\end{equation*}
and with operators acting on such functions.
The one-line goal of this course is to study what happens when scalar-valued functions are replaced by \emph{Banach-valued} functions: that is, we are going to study properties of functions
\begin{equation*}
  \map{\mb{f}}{\R^{d}}{X},
\end{equation*}
where $X$ is a Banach space.

If $X$ is finite-dimensional, then we can choose a basis $\mb{e}_{1}, \ldots, \mb{e}_{n}$ for $X$, and with respect to this basis the function $\mb{f}$ may be written as
\begin{equation*}
  \mb{f}(x) = \sum_{i=1}^{n} f_{i}(x) \mb{e}_{i}
\end{equation*}
for some scalar-valued functions $\map{f_{i}}{\R^{d}}{\K}$.
In this way, properties of the function $\mb{f}$ can be deduced from properties of the scalar-valued functions $f_{i}$.

Properties of operators acting on $X$-valued functions can also be reduced to scalar-valued properties when $X$ is finite-dimensional.
Consider for example the Fourier transform, defined on integrable scalar-valued functions $f \in L^1(\R)$ by
\begin{equation*}
  \hat{f}(\xi) := \int_{\R} f(x) e^{-2\pi i x \xi} \, \dd x
\end{equation*}
The (debatably) most important theorem in Fourier analysis is the Plancherel theorem, which says that the Fourier transform is an isometry on $L^2$:
\begin{equation*}
  \|\hat{f}\|_{L^2(\R)} = \|f\|_{L^2(\R)}.
\end{equation*}
For a vector-valued function $\map{\mb{f}}{\R}{X}$ as above, the Fourier transform $\map{\hat{\mb{f}}}{\R}{X}$ can be defined in the same way: we have
\begin{equation*}
  \begin{aligned}
    \hat{\mb{f}}(\xi)
    &:= \int_{\R} \mb{f}(x) e^{-2\pi i x \xi} \, \dd x \\
    &= \int_{\R} \Big( \sum_{i=1}^{n} f_{i}(x) \mb{e}_{i} \Big) e^{-2\pi i x \xi} \, \dd x \\
    &= \sum_{i=1}^{n} \Big(\int_{\R} f_{i}(x) e^{-2\pi i x \xi} \, \dd x \Big) \mb{e}_{i}
    &= \sum_{i=1}^{n} \hat{f_{i}}(\xi) \mb{e}_{i}
  \end{aligned}
\end{equation*}
where the Lebesgue integral is extended to $X$-valued functions by linearity.\footnote{This construction requires that $X$ is finite-dimensional. In general, the Lebesgue integral has to be replaced by a \emph{Bochner integral}. This is covered in Chapter \ref{sec:Bochner-spaces}.}



\subsection{Conventions}\label{sec:conventions}
Throughout these notes we will deal with both real and complex Banach spaces.
If we do not explicitly specify `real' or `complex', then either choice can be made, and we will write $\K$ to denote the scalar field (i.e. $\K = \R$ or $\K = \C$).
Every complex Banach space can be seen as a real Banach space by restricting scalar multiplication to the reals.
On the other hand, every real Banach space can be `complexified', a process which doubles the dimension over $\R$ and does what it should: for example the complexification of $\R^n$ is $\C^n$, the complexification of $L^p(S;\R)$ is $L^p(S;\C)$, and so on.
It's a good idea not to think too hard about this.

For an exponent $p \in [1,\infty]$ we let $p'$ denote the \emph{H\"older conjugate}
\begin{equation*}
  p' := \begin{cases}
    \frac{p}{p-1} & p \in (1,\infty) \\
    \infty & p = 1 \\
    1 & p = \infty,
  \end{cases}
\end{equation*}
so that $p^{-1} + (p')^{-1} = 1$ (interpreting $1/\infty$ as $0$).




 % {\footnotesize
 %   \subsection{Acknowledgements}
 % }


%%% Local Variables:
%%% mode: latex
%%% TeX-master: "../main"
%%% End:

