A good deal of real analysis is concerned with properties of scalar-valued functions
\begin{equation*}
  \map{f}{\R^{d}}{\K}, \qquad \K = \text{$\R$ or $\C$}
\end{equation*}
and with operators acting on such functions.
The one-line goal of this course is to study what happens when scalar-valued functions are replaced by \emph{Banach-valued} functions: that is, we are going to study functions
\begin{equation*}
  \map{\mb{f}}{\R^{d}}{X},
\end{equation*}
where $X$ is a Banach space (typically infinite dimensional), and we will study properties of operators acting on $X$-valued functions.
This will reveal interesting relationships between Fourier analysis, probability, measure theory, operator theory, and the geometry of Banach spaces.

\section{Case study: the Fourier transform and Plancherel's theorem}

To keep things simple, consider a finite dimensional complex Banach space $X$ with a basis $\mb{e}_{1}, \ldots, \mb{e}_{N}$.
Every vector $\mb{x} \in X$ has a basis expansion
\begin{equation*}
  \mb{x} = \sum_{n=1}^{N} x_{n} \mb{e}_{n}
\end{equation*}
for some scalars $x_{n} \in \C$, so every $X$-valued function $\map{\mb{f}}{\R}{X}$ may be written as
\begin{equation*}
  \mb{f}(x) = \sum_{n=1}^{N} f_{n}(x) \mb{e}_{n}
\end{equation*}
for some scalar-valued functions $\map{f_{n}}{\R}{\C}$.

We will investigate the Fourier transform on $X$-valued functions.
For integrable scalar-valued functions $f \in L^1(\R)$, the Fourier transform is defined by
\begin{equation*}
  \hat{f}(\xi) := \int_{\R} f(x) e^{-2\pi i x \xi} \, \dd x.
\end{equation*}
Plancherel's theorem, arguably the most important result in Fourier analysis, says that the Fourier transform is an isometry on $L^2$:
\begin{equation*}
  \|\hat{f}\|_{L^2(\R)} = \|f\|_{L^2(\R)}.
\end{equation*}
Does this still hold for $X$-valued functions?
Before we can answer the question we need to say what we mean by the Fourier transform and the $L^2$-norm for $X$-valued functions.
Given $\map{\mb{f}}{\R}{X}$ as above, the Fourier transform $\map{\hat{\mb{f}}}{\R}{X}$ can be defined as in the scalar-valued case: we have
\begin{equation*}
  \begin{aligned}
    \hat{\mb{f}}(\xi)
    &:= \int_{\R} \mb{f}(x) e^{-2\pi i x \xi} \, \dd x \\
    &= \int_{\R} \Big( \sum_{n=1}^{N} f_{n}(x) \mb{e}_{n} \Big) e^{-2\pi n x \xi} \, \dd x \\
    &= \sum_{n=1}^{N} \Big(\int_{\R} f_{n}(x) e^{-2\pi i x \xi} \, \dd x \Big) \mb{e}_{n}
    &= \sum_{n=1}^{N} \hat{f_{n}}(\xi) \mb{e}_{n}
  \end{aligned}
\end{equation*}
where the Lebesgue integral is extended to $X$-valued functions by linearity.\footnote{This requires finite dimensionality of $X$. In general, the Lebesgue integral is replaced by a \emph{Bochner integral}. This is covered in Chapter \ref{sec:Bochner-spaces}.}
The $L^2$-norm on $X$-valued functions is defined by
\begin{equation*}
  \|\mb{f}\|_{L^2(\R;X)} := \Big( \int_{\R} \|\mb{f}(x)\|_{X}^{2} \, \dd x \Big)^{1/2}.
\end{equation*}
This yields a Banach space $L^2(\R;X)$, but already we start to see that something suspicious is going on: \emph{there is no natural way to turn $L^2(\R;X)$ into a Hilbert space unless $X$ itself is a Hilbert space.}
Of course, since any two norms on a finite dimensional vector space are equivalent, there exists a constant $1 \leq C_{X} < \infty$ such that
\begin{equation}\label{eq:fd-norm-equivalence}
  C_{X}^{-1} \|(x_{n})_{n=1}^{N}\|_{\ell^2_{N}} \leq \|\mb{x}\|_{X} \leq C_{X} \|(x_{n})_{n=1}^{N}\|_{\ell^2_{N}},
\end{equation}
where
\begin{equation*}
  \|(x_{n})_{i=1}^{N}\|_{\ell^2_{N}} := \Big( \sum_{n=1}^{N} |x_{n}|^{2} \Big)^{1/2}
\end{equation*}
is the familiar Euclidean norm on $\C^{N}$.
Since $\ell^{2}_{N}$ is a Hilbert space, \eqref{eq:fd-norm-equivalence} lets us treat $X$ as if it were a Hilbert space; the constant $C_{X}$ measures how close $X$ is to the Hilbert space $\ell^{2}_{N}$, with $C_{X} = 1$ if and only if $X$ is isometric to $\ell^{2}_{N}$.

We proceed with our investigation of the Plancherel theorem.
Using the basis expansion and the equivalence of norms in \eqref{eq:fd-norm-equivalence}, we can compute
\begin{equation*}
  \begin{aligned}
    \|\hat{\mb{f}}\|_{L^2(\R;X)}
    &= \Big( \int_{\R} \Big\| \sum_{n=1}^{N} \hat{f_{n}}(\xi)  \mb{e}_{n} \Big\|_{X}^{2} \, \dd \xi \Big)^{1/2} \\
    &\leq C_{X} \Big( \sum_{n=1}^{N} \int_{\R} | \hat{f_{n}}(\xi) |^{2} \, \dd \xi \Big)^{1/2} \\
    &\stackrel{(*)}{=} C_{X} \Big( \sum_{n=1}^{N} \int_{\R} | f_{n}(x) |^{2} \, \dd x \Big)^{1/2} \\
    &\leq C_{X}^{2} \Big( \int_{\R} \Big\| \sum_{n=1}^{N} f_{n}(x)  \mb{e}_{n} \Big\|_{X}^{2} \Big)^{1/2} = C_{X}^{2} \|\mb{f}\|_{L^2(\R;X)},
  \end{aligned}
\end{equation*}
using the (scalar-valued) Plancherel theorem to deduce the starred equality.
Thus we do have a kind of $X$-valued Plancherel theorem, but now instead of being an isometry, the Fourier transform is merely bounded on $L^2(\R;X)$ with norm $\leq C_{X}^{2}$.
So the boundedness of the Fourier transform on $L^2(\R;X)$ seems to have something to do with the proximity of $X$ to a Hilbert space.

Now what if $X$ is infinite dimensonal?
If a constant $C_{X}$ as in \eqref{eq:fd-norm-equivalence} exists (but with $N = \infty$), i.e. if $X$ is isomorphic to $\ell^2(\N)$, then the argument above still works,\footnote{As mentioned in an earlier footnote, the Lebesgue integral has to be replaced by the Bochner integral.} and the norm of the $X$-valued Fourier transform on $L^2(\R;X)$ is $\leq C_{X}^{2}$.
The surprising fact is that the converse is also true: we will prove this in Chapter \ref{sec:fouriertype}.

\begin{thm}[Kwapie\'n, 1972]\label{thm:Kwapien-intro}
  The $X$-valued Fourier transform is bounded on $L^2(\R;X)$ if and only if $X$ is isomorphic to a Hilbert space.
\end{thm}

This is just one (extreme) example of the following general principle: when $T$ is an operator on scalar-valued functions which is bounded on some Lebesgue space $L^p(\R)$, then $T$ can be extended to $X$-valued functions, and the boundedness of this extension on $L^p(\R;X)$ reflects geometric properties of the Banach space $X$.
Different operators can be used to reflect different geometric properties.

\section{Case study: the Hilbert transform}

Another important operator in harmonic analysis is the Hilbert transform, defined on $\map{f}{\R}{\C}$ by
\begin{equation}\label{eq:HT-scalar}
  Hf(x) := \frac{1}{\pi} \mathrm{p. v.} \int_{\R} f(x-y) \, \frac{\dd y}{y} := \frac{1}{\pi} \lim_{\varepsilon \downarrow 0} \int_{|y| > \varepsilon} f(x-y) \, \frac{\dd y}{y}.
\end{equation}
The Hilbert transform is a prototypical \emph{singular integral operator}: the integral kernel $1/y$ is not integrable, so $H$ cannot be defined as a classical integral operator, but cancellation between the positive and negative values of $1/y$ allow $Hf$ to be well-defined as a \emph{principal value integral} when $f$ is sufficiently smooth (Schwartz, for example).
One can show that $H$ is a \emph{Fourier multiplier} with symbol $m(\xi) = -i\sgn(\xi)$: that is, for all Schwartz functions $f \in \Sch(\R)$, $Hf = (m\hat{f})^{\vee}$, where $\vee$ denotes the inverse Fourier transform.
Plancherel's theorem implies that the Hilbert transform is an isometry on $L^2(\R)$: indeed,
\begin{equation*}
  \|Hf\|_{L^2(\R)} = \|m\hat{f}\|_{L^2(\R)} \leq \|m\|_{L^\infty(\R)} \|\hat{f}\|_{L^2(\R)} = \|f\|_{L^{2}(\R)},
\end{equation*}
since $|m(\xi)| = 1$ for all $\xi \in \R$.
Furthermore, since the singular kernel $1/y$ has relatively nice decay and smoothness estimates (despite its singular nature), Calder\'on--Zygmund theory can be used to extrapolate the $L^2$-boundedness to $L^p$: for $p \in (1,\infty)$ there exists a constant $C_{p} < \infty$ such that
\begin{equation*}
  \|Hf\|_{L^p(\R)} \leq C_{p} \|f\|_{L^p(\R)} \qquad \forall f \in \Sch(\R).
\end{equation*}

What about vector-valued extensions?
If $X$ is a Banach space, then the $X$-valued Hilbert transform can be defined as a singular integral via the formula \eqref{eq:HT-scalar}, and it is an $X$-valued Fourier multiplier with symbol $-i\sgn(\xi)$ (using the $X$-valued Fourier transform from the previous case study).\footnote{Once more: when $X$ is infinite dimensional, Bochner integrals need to replace Lebesgue integrals!}
If $X$ is isomorphic to a Hilbert space, we can invoke the $X$-valued Plancherel theorem to estimate
\begin{equation*}
  \|H\mb{f}\|_{L^2(\R;X)} \leq K_{X} \|m\hat{\mb{f}}\|_{L^2(\R;X)} \leq K_{X} \|m\|_{L^\infty(\R)} \|\hat{\mb{f}}\|_{L^2(\R;X)}= K_{X}^{2} \|\mb{f}\|_{L^{2}(\R;X)},
\end{equation*}
where $1 \leq K_{X} < \infty$ denotes the norm of the Fourier transform on $L^2(\R;X)$.

If $X$ is not isomorphic to a Hilbert space, then we know by Kwapie\'n's theorem that the Fourier transform is \emph{not} bounded on $L^2(\R;X)$, so the previous argument does not apply.
However, it is possible to approach bounds for the Hilbert transform using probabilistic techniques, avoiding use of Plancherel's theorem, and these arguments can be carried out with values in certain (but not all) Banach spaces.
We will prove the following theorem in Chapter \ref{sec:HT}.

\begin{thm}[Burkholder--Bourgain, 1983]
  The $X$-valued Hilbert transform is bounded on $L^p(\R;X)$ for some (equivalently, all) $p \in (1,\infty)$ if and only if $X$ has the \emph{UMD property}.
\end{thm}

UMD stands for \emph{unconditionality of martingale differences}.
Martingales are a fundamental class of stochastic processes, and unconditionality of a sequence in a Banach space says that the elements of the sequence are, in some sense, `independent' or `orthogonal'.
The UMD property is probabilistic in nature; the Burkholder--Bourgain theorem says that it is also a harmonic-analytic property.
Thus the UMD property is a natural and often necessary assumption for Banach-valued analysis.
Hilbert spaces are UMD,\footnote{The arguments above show this as a consequence of the Kwapie\'n and Burkholder--Bourgain theorems, but this is absolute overkill. It can be proven directly from the definition in terms of martingales.} but so are most natural function spaces, in particular $L^p$ spaces and Sobolev spaces $W^{p}_{s}$ (with $p \in (1,\infty)$).
Thus the UMD property appears readily in applications to PDEs, both deterministic and stochastic.
Even spaces of operators can be UMD, for example the Schatten classes $\mc{S}^{p} \subset B(H)$ (studied in Chapter \ref{sec:schatten}), and consequences of the UMD property for these spaces can be used to prove deep results in operator theory.


\section{Conventions and notation}\label{sec:conventions}
Throughout these notes we will deal with both real and complex Banach spaces.
If I do not explicitly specify `real' or `complex', then either choice can be made, and $\K$ denotes the scalar field (i.e. $\K = \R$ or $\K = \C$).
Every complex Banach space can be seen as a real Banach space by restricting scalar multiplication to the reals.
On the other hand, every real Banach space can be `complexified', a process which doubles the dimension over $\R$ and does what it should: for example the complexification of $\R^n$ is $\C^n$, the complexification of $L^p(S;\R)$ is $L^p(S;\C)$, and so on.
It's a good idea not to think too hard about this.

I have tried to maintain the convention of writing vectors and vector-valued functions in bold: so I write $\mb{x} \in X$ for a vector in a Banach space $X$, and $\map{\mb{f}}{\R}{X}$ for an $X$-valued function.
This convention is not standard but I believe it helps in gaining intuition.
The way I see it, vectors and vector-valued functions are intrinsically `heavier' than their scalar analogues.
This leads to some typographical paradoxes: do we consider an element of $L^2(\R)$ as a scalar-valued function (thus writing $f$) or an element of a Banach space (thus writing $\mb{f}$)?
The answer depends on the context.
I just want to point out that nobody makes this distinction in the literature (except myself, in recent papers).
Probably for good reason.

Throughout most of these notes I assume without mention that every measure space is $\sigma$-finite.
To deal with non-$\sigma$-finite measure spaces the notion of strong measurability has to be adjusted: pointwise limits of simple functions have to be replaced by almost everywhere limits of $\mu$-simple functions.
See \cite[Section 1.1.b]{HNVW16}.


For an exponent $p \in [1,\infty]$ we let $p'$ denote the \emph{H\"older conjugate}
\begin{equation*}
  p' := \begin{cases}
    \frac{p}{p-1} & p \in (1,\infty) \\
    \infty & p = 1 \\
    1 & p = \infty,
  \end{cases}
\end{equation*}
so that $p^{-1} + (p')^{-1} = 1$ (interpreting $1/\infty$ as $0$).

The natural numbers are $\N = \{0,1,2,\ldots\}$, but sometimes I will mess up and they will be $\N = \{1,2,3,\ldots\}$.


 % {\footnotesize
 %   \subsection{Acknowledgements}
 % }


%%% Local Variables:
%%% mode: latex
%%% TeX-master: "../main"
%%% End:

