

\subsection{(Rademacher) type and cotype}

\begin{itemize}
\item definitions and examples
\item K-convexity: implied by UMD; implies type/cotype duality
\item STATE that K-convexity iff nontrivial type (Maurey--Pisier theorem is too hard for this course)
\item gaussian sums, orthonormal invariance/covariance domination. STATE the equivalence with rademacher for finite cotype
\item detecting Hilbert spaces through type and cotype $2$ (have to do the Lindenstrauss reduction to f.d. subspaces, and the LIndenstrauss--Pelczynski Theorem 7.3)
\end{itemize}

\subsection{Fourier type}

\begin{itemize}
\item Fourier type with respect to $\R$, $\T$, and $\Z$; examples (e.g. by interpolation or Fubini)
\item UMD implies nontrivial Fourier type (can't find an elementary proof)
\item FT with respect to arbitrary lca group, in particular $\Z/N\Z$ and Cantor/Walsh group
\item Fourier type $2$ implies type and cotype $2$
\item STATE relations with type and cotype (some too subtle to prove)
\end{itemize}


%%% Local Variables:
%%% mode: latex
%%% TeX-master: "../main.tex"
%%% End:
